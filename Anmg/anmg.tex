\chapter{Angular multigrid}
\section{Introduction}
The discrete ordinates method has been shown to be quite accurate for electron
and coupled electron-photon transport \cite{morel_81,accuracy_1,accuracy_2}, 
which is required in the development of radiation therapy protocols, satellite 
electronics shielding, flash x-ray machine design, and a wide variety of other 
applications. Charged particles interact through Coulomb interactions with the 
background medium. Such interactions predominately result in extremely small changes 
in particle direction and energy. These interactions are well characterized by the
Fokker-Planck limit of the transport equation \cite{fp_limit,morel_96}. In this limit,
the directional and energy changes are decoupled with the former modeled by the
continuous scattering operator and the latter modeled by the continuous-slowing-down 
operator. In this chapter, we consider the discrete-ordinate ($S_n$) angular 
discretization of the transport equation with a focus upon iterative solution methods 
for problems with highly forward-peaked scattering characteristic of the 
Fokker-Planck limit. In this limit, the mean-free-path and the directional 
change per scattering interaction goes to zero while the momentum transfer 
(also called the transport-corrected scattering cross section) remains fixed.

When the scattering is highly forward-peaked, solving the $S_n$ transport
equation can be challenging due to the slow convergence of standard iterative
algorithm, such as Source Iteration (SI). To speed up iterative convergence,
acceleration schemes such as Diffusion Synthetic Acceleration (DSA) are used.
With isotropic or weakly anisotropic scattering, DSA is generally highly
effective \cite{dsa_ref}, but it becomes completely ineffective in the
Fokker-Planck limit \cite{multigrid_1d}.

To address this deficiency, a 1-D angular multigrid method for the $S_n$
equations was developed by Morel and Manteuffel (MM) \cite{multigrid_1d}. This
method was extremely efficient yielding a maximum spectral radius for 
a model infinite-medium problem of approximately 0.6 at a cost of
approximatively twice that of DSA. This maximum spectral radius is approached
in the Fokker-Planck limit. In this same limit, the spectral radius of DSA
approaches 1. 

Pautz, Adams, and Morel (PAM) \cite{multigrid_2d} generalized the MM method to
2-D, but it was found to be stable only for weakly forward-peaked scattering.
The instability arose from high-frequency spatial error amplification that
occurred in the transfer of error estimates between angular grids (a sequence
of different $S_n$ orders). Stabilization was achieved by filtering the error
estimates via diffusion operators. However, this filtering was expensive and
significantly degraded the effectiveness of the method such that the spectral
radius approaches $1$ in the Fokker-Planck limit.  Nonetheless, the method was 
always more efficient than the DSA method for the test problems considered.

In this chapter, the PAM method with no filtering (PAMNF) is recast as a 
preconditioner and used in conjunction with the GMRES Krylov method. In this form, 
stability of the iteration scheme is guaranteed. Krylov subspace methods have been 
developed to solve large sparse linear systems. Their application to the transport 
equation has been extensively studied in the past \cite{faber,oliveira,patton,warsa} 
where the importance of preconditioning was highlighted. In \cite{oliveira}, the 
authors used successfully the 1D MM angular method as a preconditioner for GMRES and 
CGS. These preconditioned Krylov methods were significantly faster than MM. In this 
work, we compute the eigenvalues of preconditioned system for a model problem and 
compare the spectrum with those of preconditioners based upon transport sweeps and 
DSA. It is found that relative to these preconditioners, PAMNF preconditioning moves 
the eigenvalues away from zero while leaving them constrained to a reasonably small 
portion of the complex plane. These are desirable properties for a preconditioner 
because the convergence rate of GMRES is proportional to the size of the eigenvalue 
cluster and/or the distance between the clusters \cite{warsa,campbell}. The 
eigenvalues close to zero slow down the convergence of GMRES because they can be 
viewed as single values that are processed one at the time \cite{warsa,campbell}.
We also compare the convergence rates and efficiency of these preconditioners for 
various test problems with forward-peaked scattering. We find that PAMNF 
preconditioning is significantly more efficient than DSA preconditioning, and 
become increasingly so as the Fokker-Planck limit is approached. However, the 
number of iterations required for convergence nonetheless increases as this limit 
is approached. Thus, optimal multigrid performance is not achieved. In spite of 
this fact, the PAMNF-preconditioned Krylov method achieves good efficiency without 
the costly filtering associated with the original PAM fixed-point iteration scheme, 
and appears to be more effective than other existing algorithms for solving the 
$S_n$ equations with highly forward-peaked scattering.

One key features of the angular multigrid method is that transport sweeps can
strongly damp the high frequency error modes (upper half of the flux moments)
with the use of an ``optimal'' transport correction \cite{multigrid_1d}. This
``optimal'' transport correction is a variant of the well-known extended
transport correction \cite{lathrop,morel_79}.

\section{Iterative schemes for highly forward-peaked scattering}
\red{Citer le papier quand il sera publier}
\subsection{Source Iteration and DSA}
\Cref{transport_operator} can be solved using the Source Iteration (SI)
method, which is a Richardson iteration or a Krylov method. The Source
Iteration method at the $k^{th}$ iteration is given by:
\begin{equation}
\Phi^{(k+1)} = \bs{DL}^{-1} \bs{M\Sigma}\Phi^{(k)}+\bs{DL}^{-1}Q
\end{equation}
When the scattering ratio $c=\max\(\frac{\Sigma_{s,l}}{\Sigma_t}\)$ is close
to one, the spectral radius of SI can become arbitrary close to one. Since
most physical forward-peaked scattering produces $\Sigma_{s,0} >
\Sigma_{s,1}>\hdots$, the flat modes are the ones which should be accelerated.
The Diffusion Synthetic Acceleration scheme \cite{dsa_ref} is commonly used to
accelerate these flat modes. The SI+DSA scheme is given by a transport sweep:
\begin{equation}
\Phi^{(k+1/2)} = \bs{DL}^{-1}\bs{M\Sigma}\Phi^{(k)} +\bs{DL}^{-1}Q,
\label{dsa_sweep}
\end{equation}
followed by a diffusion synthetic acceleration for the correction:
\begin{equation}
\delta \Phi^{(k)} = \bs{\mc{T}}_0^{-1} \bs{R}_{n\rightarrow0}
\(\Phi^{(k+1/2)}-\Phi^{(k)}\),
\label{correction}
\end{equation}
yielding the next iterate for the flux moments:
\begin{equation}
\Phi^{(k+1)} = \Phi^{(k+1/2)} + \bs{P}_{0/1\rightarrow n} \delta \Phi^{(k)}
\label{si_dsa_it}
\end{equation}
Finally, using \crefrange{dsa_sweep}{si_dsa_it}, we obtain:
\begin{equation}
\begin{split}
\Phi^{(k+1)} =& \((\bs{I}+\bs{P}_{0/1\rightarrow n}\bs{\mc{T}}_0^{-1} \bs{R}_{n
\rightarrow 0}\bs{DL}^{-1} \bs{M\Sigma} - \bs{P}_{0/1\rightarrow n}
\bs{\mc{T}}_0^{-1} \bs{R}_{n\rightarrow 0}\) \Phi^{(k)}\\
&+ (\bs{I} + bs{P}_{0/1\rightarrow n}\bs{\mc{T}}_0^{-1} \bs{R}_{n\rightarrow
0}) \bs{DL}^{-1} Q
\end{split}
\label{dsa_it}
\end{equation}
where $\bs{\mc{T}}_0$ is the matrix associated to the DSA operator $\mc{T}_0$,
$\bs{R}_{n\rightarrow 0}$ is the restriction matrix of $\Phi_n$ (all moments)
to $\Phi_0$ (only $0^{th}$ moment) and $\bs{P}_{0/1\rightarrow n}$ the
projection matrix of $\Phi_0$ of $\Phi_1$, depending whether only the zeroth
or the zeroth and the first moment are accelerated, onto $\Phi_n$. When only
the zeroth moment is accelerated, the scheme is always stable and the spectral
radius is max $\(\rho_{iso},\frac{\Sigma_{s,1}}{\Sigma_t}\)$ where
$\rho_{iso}$ is the spectral radius when the scattering is isotropic. In
multidimensional geometry, when both the zeroth and the first moments are
accelerated, the scheme is not always stable and the spectral radius is given
by $\(\rho_{iso}.\frac{\Sigma_{s,1}}{\Sigma_{t}-\Sigma_{s,1}}\)$
\cite{multisweep}. For highly forward peaked scattering, accelerating the
zeroth moment is ineffective $\(\frac{\Sigma_{s,1}}{\Sigma_t}\rightarrow 1\)$,
whereas accelerating both moments can be unstable
$\(\frac{\Sigma_{s,1}}{\Sigma_t-\Sigma_{s,1}}>1\)$.

\section{Review of previous angular multigrid work}
\subsection{One dimensional geometry: the Morel and Manteuffel (MM) method}
As mentioned previously, only the zeroth and the first flux moments can be
accelerated with DSA. To accelerate higher moments, other methods have to be
used. Morel and Manteuffel proposed an angular multigrid method to accelerate
the SI calculation of the one-dimension $S_n$ equations with highly
anisotropic scattering \cite{multigrid_1d}. They use a variation of the
extended transport correction \cite{lathrop} to attenuate the ``upper half''
of the flux moments (higher frequencies) thanks to transport sweeps. The
``lower half'' of the flux moments (lower frequencies) is accelerated using
the $S_{n/2}$ equations. These $S_{n/2}$ equations are themselves accelerated
using $S_{n/4}$ equations. The order of the transport operator is divided by
two until the $S_4$ equations. The order of the transport operator is divided
by two until the $S_4$ level. At this point, the $P_1$ equations are used to
accelerate the $S_4$ equations. Following Morel and Mateuffel, we define:
\begin{equation}
Half(n) = \left\{
\begin{aligned}
&\frac{n}{2}, &\textrm{if }\frac{n}{2}\textrm{ is even}\\
&\frac{n}{2}+1, &\textrm{if }\frac{n}{2}\textrm{ is odd}
\end{aligned}
\right.
\end{equation}
Using this definition of ``Half'' to coarsen the angular grid, the sequence of
sweeps for an $S_{16}$ base level is $(S_{16}-S_8-S_4)$ and for a $S_{18}$
base level, the sequence is $(S_{18}-S_{10}-S_6-S_4)$. Morel and Manteuffel's
scheme works as follows:
\begin{enumerate}
\item Perform a transport sweep for the $S_n$ equations.
\item Perform a transport sweep for the $S_{n_2}$ equations with a $P_{n_2-1}$
expansions using the $S_n$ residual as the inhomogeneous source, where
$n_2=Half(n)$.
\item Continue coarsening the angular grid by a factor two (i.e., according to
the definition of ``Half'') until a sweep has been performed for the $S_4$
equations.
\item Solve the $P_1$ equations ($P1$ synthetic acceleration, $P1SA$) with a
$P_1$ expansion of the $S_4$ residual as the inhomogeneous source.
\item Add the Legendre moments of the $P_1$ solution to the Legendre moments
of the $S_4$ iterate to obtain the accelerated $S_4$ iterate.
\item Continue to add the corrections from each coarse grid to the finer grid
above to obtain the accelerated $S_n$ moments.
\end{enumerate}
Every time a transport sweep is performed, the optimal transport correction
needs to be used \cite{multigrid_1d}. For a $P_{n-1}$ expansion of the cross
sections, the corrected cross sections are given by:
\begin{equation}
\Sigma_{s,j}^* = \Sigma_{s,j} - \frac{\Sigma_{s,n/2}+\Sigma_{s,n-1}}{2}
\textrm{with }j-\{t\}\textrm{ or }\{s,l\}
\end{equation}
This correction is said to be optimal because for an infinite homogeneous
medium, it minimizes the ``high-frequency'' angular errors, the smoothing
factor being given by:
\begin{equation}
\rho_s =\max\(|\Sigma_{s,n/2}|/\Sigma_{s,0},|\Sigma_{s,n/2+1}|/\Sigma_{s,0},
\hdots,|\Sigma_{s,n-1}/\Sigma_{s,0}\)
\end{equation}
To compare the effectiveness of the angular multigrid method with DSA,
Fokker-Planck scattering cross sections (\cref{sigma_m_sigma}) can be used. In
one dimension geometry, DSA becomes less efficient as $\Sigma_{s,l}$ $(0<l\leq
L)$ becomes closer to $\Sigma_{s,0}$. Therefore, in the limit as $L\rightarrow
\infty$, DSA no longer accelerates the convergence of SI for Fokker-Planck
scattering (the spectral radius tends to 1.0). However, the spectral radius of
the angular multigrid method has an upper bound of $0.6$ when $L\rightarrow
\infty$.

\subsection{Multidimensional geometry: the Pautz-Adams-Morel (PAM) methods}
In the multidimensional case, DSA becomes unstable when both the zeroth and
the first flux moments are accelerated and $\frac{\Sigma_{s,1}}{\Sigma_t}\geq
0.5$, \cite{multisweep}. In \cite{multigrid_2d}, the authors modified the one
dimensional angular multigrid method by accelerating only the zeroth flux
moment with the DSA and by using $S_2$ as lowest transport sweep instead of an
$S_4$. Even so, the proposed method (PAMNF, with ``NF'' for no-filtering) was
unstable and a filter was needed to stabilize the scheme (PAMF). Therefore,
the angular multigrid method was modified as follows \cite{multigrid_2d}:
\begin{enumerate}
\item Perform a transport sweep for the $S_n$ equations.
\item Perform a transport sweep for the $S_{n_2}$ equations with a $P_{n_2}$
for 2-D problem and a $P_{n_2+1}$ for 3-D problem expansion for the $S_n$
residual as the inhomogeneous source, where $n_2 = Half(n)$.
\item Continue coarsening the angular grid by a factor two (i.e., according to
the definition of ``Half'') until a sweep has been performed for the $S_2$
equations.
\item Solve the diffusion equation with a $P_0$ expansion for the $S_2$
residual as the inhomogeneous source.
\item Apply a diffusive filter to the corrections from steps 2 and 3 (without
this, the method is unstable).
\item Add the corrections from steps 4 and 5 to the Legendre moments of the
$S_n$ iterate to obtain the accelerated $S_n$ moments.
\end{enumerate}
The filter stabilizes the method which otherwise would diverge. Without the
filtering process, the low frequency are well attenuated but instabilities are
introduced in higher frequency modes. Filtering eliminates the high frequency
corrections which are well attenuated by SI alone but it keeps the low
frequency corrections. The filter is given by:
\begin{equation}
\(-\bn\cdot \frac{\beta_f}{3\Sigma_f}\bn + \Sigma_f\) f_{corr} = \Sigma_f
\(\phi)_{n_2}+P_{n_4\rightarrow n_2}\phi_{n_4}+\hdots+P_{2\rightarrow
n_2}\phi_2\)
\end{equation}
where $\Sigma_f$ is the filter cross section and $\beta_f$ is the filter
tuning parameter. A Fourier analysis shows that given an input amplitude $A$,
the ``diffusively filtered'' amplitude is:
\begin{equation}
F=\frac{A}{1+\frac{\beta_f \lambda^2}{3\Sigma_f}}
\end{equation}
It is clear that the modes with large $|\lambda|$ (high frequencies) are
strongly attenuated while low-frequency are not. However, the filtering
process does not prevent the spectral radius from becoming arbitrary close to
1 when $L$ becomes large \cite{multigrid_2d}.

\section{Angular multigrid as preconditioner for Krylov Solvers}
In this research, we propose to abandon SI as the solver for the $S_n$
equations with highly-forward peaked scattering and to use a Krylov solver
instead. When using a Krylov solver, \cref{transport_operator} is rewritten as
follows:
\begin{equation}
(\bs{I}-\bs{DL}^{-1}\bs{M\Sigma}\Phi = \bs{DL}^{-1}Q
\label{krylov_transport_operator}
\end{equation}
\Cref{krylov_transport_operator} is \cref{transport_operator} preconditioned
by $\bs{DL}^{-1}$ (transport sweep preconditioning). DSA can also help to
speed up the convergence of the Krylov solver and the DSA-preconditioned
system of equations solved with a Krylov method is:
\begin{equation}
\begin{split}
&\((\bs{I} + \bs{P}_{0/1\rightarrow n} \bs{\mc{T}}_0^{-1} \bs{R}_{n\rightarrow
0})(\bs{I}-\bs{DL}^{-1}\bs{M\Sigma})\) \Phi=\\
&(\bs{I}+\bs{P}_{0/1\rightarrow n} \bs{\mc{T}}_0^{-1} \bs{R}_{n\rightarrow 0})
\bs{DL}^{-1} Q
\end{split}
\end{equation}
The angular multigrid scheme can also be recast to be used by a Krylov
solver. Here, we have chosen the recast the PAM method without filtering
(PAMNF) as a preconditioner for a Krylov solver. The successive corrections of
the angular multigrid acceleration form now different stages of a
preconditioner used in the Krylov solver. Two variations of the PAMNF
preconditioner will be tested:
\begin{itemize}
\item the coarsest level is DSA (ANMG-DSA) (with the coarsest $S_n$ level
being $S_2$).
\item the coarsest level is $P1SA$ (ANMG-P1SA) (with the coarsest $S_n$ level
being $S_4$).
\end{itemize}
First, we present the angular multigrid using DSA and then, the angular
multigrid using $P1SA$. Later, these two versions are compared.
\subsection{ANMG-DSA}
Using a method similar to the one we used to write the equation for the
preconditioned Krylov solver, we recast the PAMNF for SI as a preconditioner
for a Krylov solver. First, we write the SI sweep equation, the successive
corrections and the new iterate built from the sweep values plus all the
successive corrections:
\begin{align}
& \Phi_n^{(k+1/2)} = \bs{D}_n\bs{L}_n^{-1}\bs{M}_n\bs{\Sigma}_n\Phi_n^{(k)} +
\bs{D}_n \bs{L}_n^{-1} Q\\
& \delta \Phi_{n_2}^{(k)} = \bs{D}_n\bs{L}_{n_2}^{-1} \bs{M}_{n_2}
\bs{\Sigma}_{n_2} \bs{R}_{n\rightarrow n_2}
\(\Phi_n^{(k+1/2)}-\Phi_n^{(k)}\)\\
& \hdots\\
& \delta \Phi_2^{(k)} = \bs{D}_2 \bs{L}_2^{-1} \bs{M}_2 \bs{\Sigma}_2
\bs{R}_{4\rightarrow 2}\delta \Phi_4\\
& \delta \Phi_0^{(k)} = \bs{\mc{T}}_0^{-1} \bs{R}_{2\rightarrow 0}\delta
\Phi_2^{(k)}\\
& \Phi_n^{(k+1)} = \Phi_n^{(k+1/2)} + \bs{P}_{n_2\rightarrow n} \delta
\Phi_{n_2}^{(k)} + \hdots + \bs{P}_{2\rightarrow n} \delta \Phi_2^{(k)} +
\bs{P}_{0\rightarrow n} \delta \Phi_0^{(k)} \label{mom_update}
\end{align}
Now, all the corrections $\delta \Phi_0^{(k)}$ through $\delta \Phi_{n_2}^{k}$
are substituted into the moment update equation, \cref{mom_update}, yielding:
\begin{equation}
\begin{split}
\Phi_n^{(k+1)} =& \bs{T}_n \Phi_n^{(k)} + \bs{D}_n \bs{L}_n^{-1} Q +
\bs{P}_{n_2\rightarrow n}\(\bs{T}_{n_2}\bs{R}_{n\rightarrow n_2}
\(\Phi_n^{(k+1/2)}-\Phi_n^{(k)}\)\)+\hdots\\
&+\bs{P}_{2\rightarrow n} \bs{T}_2\bs{R}_{4\rightarrow 2} \delta \Phi_4^{(k)}
+ \bs{P}_{0\rightarrow n} \bs{\mc{T}}_0^{-1} \bs{R}_{2\rightarrow 0} \delta
\Phi_2^{(k)}\\
=& \bs{T}_n\Phi_n^{(k)} + \bs{D}_n \bs{L}_n^{-1} Q + \bs{P}_{n_2\rightarrow n}
\(\bs{T}_{n_2} \bs{R}_{n\rightarrow n_2} \(\bs{T}_n \Phi_n^{(k)} +\bs{D}_n
\bs{L}_n^{-1} Q -\Phi_n^{(k)}\)\)\\
&+\hdots+\bs{P}_{2\rightarrow n} \bs{T})2 \bs{R}_{4\rightarrow 2}\( \bs{T}_4
\bs{R}_{8\rightarrow 4} \(\hdots\(\bs{T}_n \Phi_n^{(k)} + \bs{D}_n
\bs{L}_n^{-1} Q -\Phi_n^{(k)}\)\)\)\\
&+ \bs{P}_{0\rightarrow n} \bs{\mc{T}}_0^{-1} \bs{R}_{2\rightarrow 0}
\(\bs{T}_2 \bs{R}_{4\rightarrow 2}\(\hdots\( \bs{T}_N\Phi_n^{(k)} + \bs{D}_n
\bs{L}_n^{-1}Q -\Phi_n^{(k)}\)\)\)\\
=& \( \bs{T}_n + \bs{P}_{n_2\rightarrow n} \bs{T}_{n_2} \bs{R}_{n\rightarrow
n_2} (\bs{T}_n -\bs{I}) + \hdots + \bs{P}_{2\rightarrow n} \bs{T}_2
\bs{R}_{4\rightarrow 2} \(\bs{T}_4 \bs{R}_{8\rightarrow
4}\(\hdots\(\bs{T}_n-\bs{I}\)\)\)\right.\\
&\left.+\bs{P}_{0\rightarrow n}\bs{\mc{T}}_0^{-1} \bs{R}_{2\rightarrow 0}
\bs{R}_{2\rightarrow 0} \(\bs{T}_2 \bs{R}_{4\rightarrow
2}\(\hdots\(\bs{T}_n-\bs{I}\)\)\)\) \Phi_n^{(k)} +
\(bs{I}+\bs{P}_{n_2\rightarrow n}\bs{T}_{n_2} \bs{R}_{n\rightarrow n_2}+
\hdots +\right.\\
&\bs{P}_{2\rightarrow n} \bs{T}_2 \bs{R}_{4\rightarrow 2} \(\bs{T}_4
\bs{R}_{8\rightarrow 4}\(\hdots \(\bs{T}_{n_2} \bs{R}_{n\rightarrow
n_2}\)\)\)+\\
&\left. \bs{P}_{0\rightarrow n} \bs{\mc{T}}_0^{-1} \bs{R}_{2\rightarrow 0}
\(\bs{T}_2 \bs{R}_{4\rightarrow 2}\(\hdots\(\bs{T}_{n_2} \bs{R}_{n\rightarrow
n_2}\)\)\)\) \bs{D}_n \bs{L}^{-1} Q
\end{split}
\end{equation}
where we defined $\bs{T}_n = \bs{D}_n \bs{L}_n^{-1} \bs{M}_n \bs{\Sigma}_n$
(the subscript $n$ denotes the $S_n$ level). Finally, the linear system to be
solved is given by:
\begin{equation}
\bs{P}^{MG/DSA}(\bs{I}-\bs{T}_n) \Phi_n = \bs{P}^{MG/DSA}\bs{D}_n\bs{L}_n^{-1}
Q
\label{p_mg_dsa}
\end{equation}
where the multigrid preconditioner $P^{MG/DSA}$ is given by:
\begin{equation}
\begin{split}
&P^{MG/DSA} =\\
&\(\bs{I}+\bs{P}_{n_2\rightarrow n} \bs{T}_{n_2}
\(\bs{I}+\bs{P}_{n_4\rightarrow n_2}
\bs{T}_{n_4}\(\hdots\(\bs{I}+\bs{P}_{0\rightarrow 2}
\bs{\mc{T}}_0^{-1}\bs{R}_{2\rightarrow 0}\)\hdots\)\bs{R}_{n_2\rightarrow
n_4}\) \bs{R}_{n\rightarrow n_2}\)
\end{split}
\end{equation}
At this point, it is necessary to choose a DSA for implementation. Various DSA
schemes have been reviewed in \cite{dsa_ref,multisweep,consistent_p1,larsen_91,
wareing,trans_87}. We have chosen to employ the Modified Interior Penalty
(MIP) DSA scheme developed by Wang and Ragusa \cite{mip}. The MIP-DSA scheme
is based on a discretization of the diffusion equation rather than the $P1$
equations. More specifically, MIP uses a bilinear \emph{discontinuous} trial
space, which is the same trial space as the one used for the $S_n$ transport
equations. However, the MIP equations are not fully consistent with the
bilinear-discontinuous spatial discretization of the transport equation. Full
consistency requires discretization of the $P1$ equations. The consistency
discretized $P1$ equations are of a non-symmetric mixed form. The MIP-based
DSA algorithm is always stable for isotropic scattering and the MIP diffusion
matrix is symmetric positive definite (SPD), which makes it much easier to
invert than the mixed $P1SA$ equation. For instance, one can use a conjugate
gradient technique, preconditioned with SSOR to solve the MIP equation.

\subsection{ANMG-P1SA}   
Using $S_4$ as the lowest $S_n$ order followed by a $P1SA$ acceleration
(instead $S_2$ followed by DSA) in \cref{p_mg_dsa} yields the following linear
system:
\begin{equation}
P^{MG/P1DSA}(\bs{I}-\bs{T}_n) \Phi_n = P^{MG/P1SA} \bs{D}_n \bs{L}_n^{-1} Q
\end{equation}
where the multigrid preconditioner $P^{MG/P1SA}$ is now given by:
\begin{equation}
\begin{split}
&P^{MG/P1SA} =\\
&\(\bs{I}+\bs{P}_{n_2\rightarrow n} \bs{T}_{n_2}
\(\bs{I}+\bs{P}_{n_4\rightarrow n_2}
\bs{T}_{n_4}\(\hdots\(\bs{I}+\bs{P}_{1\rightarrow 4}\bs{\mc{T}}_1^{-1}
\bs{R}_{4\rightarrow 1}\)\hdots\)\bs{R}_{n_2\rightarrow n_4}\)
\bs{R}_{n\rightarrow n_2}\) \(\bs{I}+\bs{P}_{n_2\rightarrow n}
\bs{T}_{n_2}\)
\end{split}
\end{equation}
where $\bs{\mc{T}}_1$ is the matrix associated to the $P1SA$ operator
$\mc{T}_1$ operator. The $P1SA$ discretization used here is the $P1C$ method,
defined in \cite{yaqiPhD,P1C_MC2009}. This $P1SA$ preconditioner is positive
definite (PD), but not symmetric. In principle, the analytic $P1$ equations
can be put in a second-order diffusion form and discretized using the MIP
approach. However, the first moments of the angular flux will be treated wit
less accuracy than the zeroth moment, which is undesirable.
