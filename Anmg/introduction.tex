\chapter{\uppercase{Angular Multigrid Preconditioner for $S_n$ equations with Highly
Forward-Peaked Scattering Kernel}} \label{anmg_chapter}
\section{Introduction}
The discrete ordinates method has been shown to be quite accurate for electron
and coupled electron-photon transport \cite{accuracy_2,morel_81,accuracy_1}, 
which is required in the development of radiation therapy protocols, satellite 
electronics shielding, flash x-ray machine design, and a wide variety of other 
applications. Charged particles interact through Coulomb interactions with the 
background medium. Such interactions predominately result in extremely small changes 
in particle direction and energy. These interactions are well characterized by the
Fokker-Planck limit of the transport equation \cite{fp_limit,morel_96}. In this 
limit, the directional and energy changes are decoupled with the former modeled 
by the continuous scattering operator and the latter modeled by the continuous 
slowing-down operator. In this chapter, we consider the discrete-ordinate ($S_n$) 
angular discretization of the transport equation with a focus upon iterative 
solution methods for problems with highly forward-peaked scattering characteristic 
of the Fokker-Planck limit. 

When the scattering is highly forward-peaked, solving the $S_n$ transport
equation can be challenging due to the slow convergence of standard iterative
algorithm, such as Source Iteration (SI). To speed up iterative convergence,
acceleration schemes such as Diffusion Synthetic Acceleration (DSA) are used.
With isotropic or weakly anisotropic scattering, DSA is generally highly
effective \cite{dsa_ref}. This is because, in such case, the quickly varying
error modes are strongly attenuated by the transport sweep whereas the
diffusion operator attenuates the slowly varying error modes. However, DSA is
completely ineffective in the Fokker-Planck limit \cite{multigrid_1d} because
the diffusion operator does not attenuate all the slowly varying anymore.

To address this deficiency, a 1-D angular multigrid method for the $S_n$
equations was developed by Morel and Manteuffel (MM) \cite{multigrid_1d}. This
method was extremely efficient yielding a maximum spectral radius for 
a model infinite-medium problem of approximately 0.6 at a cost of
approximatively twice that of DSA. This maximum spectral radius is approached
in the Fokker-Planck limit. In this same limit, the spectral radius of DSA
approaches 1. 

Pautz, Adams, and Morel (PAM) \cite{multigrid_2d} generalized the MM method to
2-D, but it was found to be stable only for weakly forward-peaked scattering.
The instability arose from high-frequency spatial error amplification that
occurred in the transfer of error estimates between angular grids (a sequence
of different $S_n$ orders). Stabilization was achieved by filtering the error
estimates via diffusion operators. However, this filtering was expensive and
significantly degraded the effectiveness of the method such that the spectral
radius approaches $1$ in the Fokker-Planck limit.  Nonetheless, the method was 
always more efficient than the DSA method for the test problems considered.

In this chapter, the PAM method with no filtering (PAMNF) is recast as a 
preconditioner and used in conjunction with the GMRES Krylov method. In this form, 
stability of the iteration scheme is guaranteed. Krylov subspace methods have been 
developed to solve large sparse linear systems. Their application to the transport 
equation has been extensively studied in the past \cite{faber,oliveira,patton,warsa} 
where the importance of preconditioning was highlighted. In \cite{oliveira}, the 
authors used successfully the 1D MM angular method as a preconditioner for GMRES 
and CGS. These preconditioned Krylov methods were significantly faster than MM. 
In this research, we compute the eigenvalues of preconditioned system for a model 
problem and compare the spectrum with those of preconditioners based upon transport 
sweeps and DSA. It is found that relative to these preconditioners, PAMNF 
preconditioning moves the eigenvalues away from zero while leaving them 
constrained to a reasonably small portion of the complex plane. These are 
desirable properties for a preconditioner because the convergence rate of GMRES 
is proportional to the size of the eigenvalue cluster and/or the distance between 
the clusters \cite{campbell,warsa}. The eigenvalues close to zero slow down the 
convergence of GMRES because they can be viewed as single values that are processed 
one at the time \cite{campbell,warsa}. We also compare the convergence rates and 
efficiency of these preconditioners for various test problems with forward-peaked 
scattering. We find that PAMNF preconditioning is significantly more efficient 
than DSA preconditioning, and become increasingly so as the Fokker-Planck limit 
is approached. However, the number of iterations required for convergence 
nonetheless increases as this limit is approached. In spite of this fact, 
the PAMNF-preconditioned Krylov method achieves good efficiency without the 
costly filtering associated with the original PAM fixed-point iteration scheme, 
and appears to be more effective than other existing algorithms for solving 
the $S_n$ equations with highly forward-peaked scattering.

One key features of the angular multigrid method is that transport sweeps can
strongly damp the high frequency error modes (upper half of the flux moments)
with the use of an ``optimal'' transport correction \cite{multigrid_1d}. This
``optimal'' transport correction is a variant of the well-known extended
transport correction \cite{lathrop,morel_79}.

