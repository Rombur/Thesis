\chapter{\uppercase{Charged particle transport}} \label{cp_transport_chapter}
\section{Introduction}
For charged particle transport, the scattering kernel is very pronounced 
for both an almost zero energy transfer and an almost zero direction 
change. However, because the
scattering mean free path (average distance between two collisions) is very
small, particles can undergo a large number of collisions. Therefore,
significant changes of energy and direction are common in most applications.

In theory, the linear Boltzmann equation could be used for charged particle
transport but the scattering cross sections are so forward peaked that a Legendre 
expansion of the cross section would require too many terms in its expansion. 
Moreover, using a deterministic method to solve the Boltzmann equation often 
requires cells of the same scale as the very small mean free path \cite{pomraning}. 
To avoid these difficulties, a Fokker-Planck operator can
be used to represent the highly forward peaked scattering. Since this operator 
cannot represent the large angle scattering collisions, a Legendre expansion 
is used for them. The addition of the Fokker-Planck operator to the 
Boltzmann equation yields the Boltzmann-Fokker-Planck equation \cite{ligou}.

\section{Boltzmann-Fokker-Planck equation}
In this section, we start by deriving the Boltzmann-Fokker-Planck equation.
Then, we show that the Fokker-Planck equation is an asymptotic limit of the 
Boltzmann equation when both the energy transfer and the direction changes 
during a collision go to zero. We conclude this section with an analysis of 
the validity of the Fokker-Planck and Boltzmann-Fokker-Planck equations.
\subsection{Derivation of the Boltzmann-Fokker-Planck equation}
Following \cite{ligou}, the BFP equation is derived starting from the Boltzmann 
equation:
\begin{equation}
\begin{split}
&\bo\cdot\bn\psi(\br,\bo,E) + \(\Sigma_s(\br,E)+\Sigma_a(\br,E)\)\psi(\br,\bo,E)
=\\ 
&\sum_{l=0}^{\infty}\sum_{m=-l}^l \int_{0}^{\infty} \Sigma_{s,l}\(\br,E'\rightarrow
E\) \phi_{l,m}(\br,E')Y_l^m(\bo)\ dE' + Q(\br,\bo,E).
\label{b_2}
\end{split}
\end{equation}
When the microscopic scattering cross section
$(\sigma_s(\br,\mu_c,E'))$ is known ($\mu_c$ is the scattering angle cosine in
the center of mass system, whereas we will use $\mu_L$ for the scattering 
angle cosine in the laboratory system), the Legendre expansion of the
macroscopic scattering cross section is given by:
\begin{equation}
\Sigma_{s,l}(E'\rightarrow E) = N \frac{4\pi}{\beta E'}
\sigma_s\(\mu_c\(\frac{E'-E}{E'}\),E'\)P_l\(\mu_L\(\frac{E'}{E}\)\),
\label{eq7}
\end{equation}
where $E\leq E' \leq \frac{E}{\alpha}$, $\beta=1-\alpha$,
$\alpha=\(\frac{A-1}{A+1}\)^2$, $A$ is the particle mass ratio, and $N$ is the
atom density.

First, the differential scattering cross section is split into two parts:
\begin{equation}
  \begin{split}
    \Sigma_s\(\mu_c\(\frac{E'-E}{E}\),E'\) =& 
    \Sigma_{s,reg}\(\mu_c\(\frac{E'-E}{E}\),E'\)\\
    &+\Sigma_{s,sing}\(\mu_c\(\frac{E'-E}{E}\),E'\),
  \end{split}
\end{equation}
where:
\begin{itemize}
\item $\Sigma_{s,reg}$ is the ``regular'' cross section that does not vary
  rapidly as $\mu_c$ goes to one. By definition, a Legendre polynomial expansion 
  of this cross section converges quickly (i.e., with a few terms).
\item $\Sigma_{s,sing}$ is the ``singular'' cross section which is highly forward 
peaked and is not negligible only when $\mu_c\approx 1$. 
\end{itemize}
Then, the scattering term in \cref{b_2} 
\begin{equation}
  q\(\bo,E\)=\sum_{l=0}^{\infty} \sum_{m=-l}^l \int_0^{\infty} 
  \Sigma_{s,l}(E'\rightarrow E) \phi_{l,m}(E') Y_l^m(\bo)dE'
\end{equation}
is split
\begin{equation}
q = q_{reg}+q_{sing},
\end{equation}
with
\begin{equation}
\begin{bmatrix}
q_{reg}(\bo,E)\\
q_{sing}(\bo,E)
\end{bmatrix}
= \sum_{l=0}^{\infty} \sum_{m=-l}^l Y_l^m(\bo)
\begin{bmatrix}
q_{l,m,reg}(E)\\
q_{l,m,sing}(E)
\end{bmatrix}
\end{equation}
and
\begin{equation}
\begin{bmatrix}
q_{l,m,reg}(E)\\
q_{l,m,sing}(E)
\end{bmatrix}
=\int_{E}^{E/\alpha} dE' \phi_{l,m}(E')
\begin{bmatrix}
\Sigma_{s,l,reg}(E'\rightarrow E)\\
\Sigma_{s,l,sing}(E' \rightarrow E)
\end{bmatrix},
\label{prec}
\end{equation}
where $\Sigma_{s,l,reg}$ and $\Sigma_{s,l,sing}$ are the Legendre expansion
coefficients corresponding to $\sigma_{s,reg}$ and $\sigma_{s,sing}$ through 
\cref{eq7}. Using $\mu_c$ instead of $E'$ as integration variable in \cref{prec}, 
we obtain:
\begin{equation}
q_{l,m,sing}(E) = 2\pi N\int_{-1}^1\frac{E'}{E}\sigma_{s,sing}(\mu_c,E')
P_l\(\mu_L(\mu_c)\) \phi_{l,m}(E')d\mu_c
\end{equation}
with
\begin{equation}
\mu_L = \frac{1+A\mu_c}{(1+A^2+2A\mu_c)^{1/2}}.
\end{equation}
Assuming that $\phi_{l,m}(E')$ is a smooth function
($\sigma_{s,sing}(\mu_c,E')$ is almost singular in $\mu_c$ but smooth in $E'$), 
we can perform the following Taylor expansions:
\begin{equation}
  \begin{split}
    &E'\sigma_{s,sing}(\mu_c,E')\phi_{l,m} (E') = E\sigma_{s,sing}(\mu_c,E)
    \phi_{l,m}(E) +\\ 
    &(E'-E)\frac{\partial}{\partial E}\(E\sigma_{s,sing}(\mu_c,E)
    \phi_{l,m}(E)\)+\hdots
  \end{split}
\end{equation}
\begin{equation}
  P_l(\mu_L) = P_l(1)-(1-\mu_L)P_l'(1)+\hdots
\end{equation}
with:
\begin{align}
& P_l(1)=1 \label{p_l}\\
& P_l'(1)=\frac{l(l+1)}{2}. \label{p_l_p}
\end{align}
Using:
\begin{align}
\mu_c &= 1-\frac{2}{\beta}\frac{E'-E}{E'} \label{mu_c}\\
\mu_L &= \frac{1}{2}\[(A+1)\sqrt{\frac{E}{E'}}-(A-1)\sqrt{\frac{E'}{E}}\]
\label{mu_l}
\end{align}
and assuming $\mu_c\approx 1$ or $A\ll 1$, we get:
\begin{align}
& \frac{E'-E}{E'} = \frac{2A}{(A+1)^2} (1-\mu_c)\\
& 1-\mu_L \approx \(\frac{A}{A+1}\)^2 (1-\mu_c).
\end{align}
To a first order in $(1-\mu_c)$, we get:
\begin{equation}
q_{l,m,sing}(E) = \Sigma_{s,sing}(E) \phi_{l,m}(E) +\frac{\partial}{\partial E} S(E)
\phi_{l,m}(E) - l(l+1)T(E) \phi_{l,m}(E)
\end{equation}
where:
\begin{align}
&\Sigma_{s,sing}(E) = 2 \pi N \int_{-1}^{1}\sigma_{s,sing}(E,\mu_c)\ d\mu_c\\
&T(E) = N \pi \int_{-1}^1 (1-\mu_L)\sigma_{s,sing}(E,\mu_c)\ d\mu_c\\
&S(E) = \frac{4E}{A}T(E).
\end{align}
$T$ is half of the restricted momentum transfer and $S$ is the
restricted stopping power. Using \cref{ang_flux,eigenvalue}, we obtain:
\begin{equation}
q_{sing} = \Sigma_{s,sing}\psi+\frac{\partial}{\partial E}S\psi + T
\(\frac{\partial}{\partial \mu} (1-\mu^2) \frac{\partial}{\partial \mu}+
\frac{1}{1-\mu^2}\frac{\partial^2}{\partial \varphi^2}\) \psi.
\end{equation}
Finally, we get the BFP equation:
\begin{equation}
\begin{split}
&\bo\cdot\bn\psi+(\Sigma_{s,reg}+\Sigma_a)\psi=\sum_{l=0}^{\infty}\sum_{m=-l}^l
Y_l^m(\bo)\int_E^{E/\alpha}\Sigma_{s,l,reg}(E'\rightarrow E) \times \\
&\phi_{l,m}(E')dE'+ \frac{\partial}{\partial E}S\psi+T\[\frac{\partial}{\partial
\mu}(1-\mu^2) \frac{\partial}{\partial
\mu}+\frac{1}{(1-\mu^2)}\frac{\partial^2}{\partial\varphi^2}\]\psi+Q.
\end{split}
\label{bfp}
\end{equation}
We see that only the regular portion of the scattering, $\Sigma_{s,reg}$,
appears in the BFP equation, and that $\Sigma_{s,sing}$ is hidden in the
restricted stopping power, $S$, and the restricted momentum transfer, $T$.

\subsection{The Fokker-Planck equation as a limit of Boltzmann equation}
In \cite{pomraning}, Pomraning showed that the Fokker-Planck equation is an
asymptotic limit of the Boltzmann equation when the mean free path goes to
zero and $\bar{\mu}_0$ goes to one. Since his development will help us to understand
the limitations of the Boltzmann-Fokker-Planck, we will briefly recall it here.

First, we assume that the unit of distance is chosen such that the characteristic
size of the domain is $O(1)$ and that the
scattering mean free path is small (i.e., $\Sigma_s \gg 1$). Next, we scale
$\Sigma_s$ as:
\begin{equation}
\Sigma_s(E) = \frac{\hat{\Sigma}_s(E)}{\Delta}
\label{sigma_s}
\end{equation}
where $\hat{\Sigma}_s=O(1)$ and $\Delta \ll 1$, $\Delta$ represents the
scattering mean free path. We introduce the fast varying variables:
\begin{align}
&x=\frac{1-\mu_0}{\delta},\  \delta \ll 1 \label{x}\\
&y=\frac{E'-E}{\epsilon},\  \epsilon \ll 1 \label{y}.
\end{align}
$\delta$ and $\epsilon$ measure how peaked is the scattering kernel. $\delta$
represents the deviation of the cosine of a characteristic scattering angle
from one. $\epsilon$ represents a characteristic value of the fraction of
energy change during a single scattering collision. Next, the scattering kernel 
is scaled as:
\begin{equation}
\begin{split}
\Sigma_s(\mu_0,E',E) &= \frac{1}{\Delta}
\hat{\Sigma}_s\(\frac{1-\mu_0}{\delta},E',\frac{E'-E}{\epsilon}\)\\
&=\frac{1}{\Delta}\hat{\Sigma}_s(x,E',y),
\end{split}
\end{equation}                
where $\hat{\Sigma}_s(x,E',y)$ is $O(1)$ and $\frac{\partial
\hat{\Sigma}_s(x,E',y)}{\partial x}$, and $\frac{\partial
\hat{\Sigma}_s(x,E',y)}{\partial y}$ are $O(1)$ when $(\epsilon,\delta)
\rightarrow 0$. 
This scaling implies that the cross section is large and very peaked about
$\mu_0=1$ and $E=E'$. The scaled transport equation is given by:
\begin{equation}
  \begin{split}
    &\bo\cdot\bn\psi+\(\Sigma_a+\frac{\hat{\Sigma}_s}{\Delta}\)\psi=
    \frac{2\pi}{\Delta} \sum_{l=0}^{\infty}\sum_{m=-l}^{l}Y_l^m(\bo)\\
    &\int_0^{\infty}dE'\ \phi_{l,m}(E')\int_{-1}^1d\mu_0\
    P_l(\mu_0)\hat{\Sigma}_s\(\frac{1-\mu_0}{\delta},E',\frac{E'-E}{\epsilon}\)+Q.
  \end{split}
  \label{scaled_transport}
\end{equation}
We are interested in the asymptotic limit when the three parameters
$\epsilon$, $\delta$, and $\Delta$ approach zero. Let us consider the following term:
\begin{equation}
K = \frac{2\pi}{\Delta} \int_0^{\infty}dE'\int_{-1}^1 d\mu_0\ P_n(\mu_0)
\hat{\Sigma}_s\(\frac{1-\mu_0}{\delta},E',\frac{E'-E}{\epsilon}\)
\phi_{l,m}(E').
\end{equation}
Now, we change the integration variables from $(\mu_0,E')$ to $(x,y)$ according to 
\cref{x,y}:
\begin{equation}
K = \frac{2\pi\epsilon\delta}{\Delta}\int_{-E/\epsilon}^{\infty}dy 
\int_0^{2/\delta}dx\  P_l(1-\delta x)\hat{\Sigma}_s(x,E+\epsilon y,y)\phi_{l,m}
(E+\epsilon y).
\label{K_def}
\end{equation}
Next, we perform a Taylor expansion of the integrand about $\epsilon=\delta=0$. We
only keep the linear terms in $\delta$ and the quadratic terms in $\epsilon$
(we will see later that we only need to compute the linear terms in $\delta$
because we assume that the medium is isotropic):
\begin{equation}
\begin{split}
K=& \frac{2\pi \epsilon \delta}{\Delta}\int_{-E/\epsilon}^{\infty}dy\
\int_0^{2/\delta}dx\ \[P_l(1)-\delta x P_l'(1)+O(\delta^2)\]\\
&\[1+\epsilon y \frac{\partial}{\partial
E}+\frac{\epsilon^2y^2}{2}\frac{\partial^2}{\partial E^2}+O(\epsilon^3)\]
\hat{\Sigma}_s(x,E,y)\phi_{l,m}(E).
\end{split}
\label{error}
\end{equation}
Now, $-E/\epsilon$ is replaced by $-\infty$ as the lower limit of the integral 
on $y$. We assume that the error made is at most $O(\epsilon^3)$. This is correct 
if the scattering kernel falls off exponentially in energy from its maximum at 
$y=0$ but if the kernel falls off too slowly, this substitution may increase the 
error above $O(\epsilon^3)$ and the current development becomes wrong. When the 
kernel falls off exponentially in energy, the replacement of the integration limit 
introduce an exponentially small error. We also neglect the cross terms 
in angle and energy in \cref{error}. It is not necessary to neglect them but the 
standard Fokker-Planck operator does not have these terms.

Using \cref{p_l,p_l_p}, \cref{error} becomes:
\begin{equation}
\begin{split}
K=& \frac{2\pi \epsilon \delta}{\Delta}
\int_{-\infty}^{\infty}dy\int_0^{2/\delta}
dx\(1+O(\delta^2+\epsilon \delta+\epsilon^3)\)\hat{\Sigma}_s(x,E,y)\phi_{l,m}(E)\\
&-\frac{l(l+1)\pi\epsilon \delta^2}{\Delta} \int_{-\infty}^{\infty} dy
\int_0^{2/\delta}dx\ x \hat{\Sigma}_s(x,E,y)\phi_{l,m}(E)\\
&+\frac{2\pi \epsilon^2 \delta}{\Delta}\frac{\partial}{\partial E}
\int_{-\infty}^{\infty} dy \int_0^{2/\delta} dx\
y\hat{\Sigma}_s(x,E,y)\phi_{l,m}(E)\\
&+\frac{\pi\epsilon^3\delta}{\Delta}\frac{\partial^2}{\partial E^2}
\int_{-\infty}^{\infty} dy\int_0^{2/\delta}dx\ y^2\hat{\Sigma}_s(x,E,y)
\phi_{l,m}(E).
\end{split}
\label{extended}
\end{equation}
We now go back to the $(\mu_0,E')$ variables using the following
relationships:
\begin{align}
x&= \frac{1-\mu_0}{\delta}\label{x2}\\
y&= \frac{E-E'}{\epsilon}\label{y2}.
\end{align}
\Cref{x,x2} are identical, but \cref{y,y2} are different ($E$ and $E'$
are interchanged). Therefore, we get:
\begin{equation}
  \begin{split}
    K=& \frac{2\pi}{\Delta} \int_{-\infty}^{\infty} dE'
    \int_{-1}^{1}d\mu_0\Bigg(\(1+O(\delta^2+\epsilon\delta+\epsilon^3)\)
    \hat{\Sigma}_s \(\frac{1-\mu_0}{\delta},E,\frac{E-E'}{\epsilon}\) \\
    & \phi_{l,m}(E)\Bigg) - \frac{l(l+1)\pi}{\Delta}
    \int_{-\infty}^{\infty}dE' \int_{-1}^1d\mu_0 \Bigg( (1-\mu_0)
    \hat{\Sigma}_s \(\frac{1-\mu_0}{\delta},E,\right.\\
    &\left.\frac{E-E'}{\epsilon}\)\phi_{l,m}(E)\Bigg) 
    +\frac{2\pi}{\Delta}\frac{\partial}{\partial E}
    \int_{-\infty}^{\infty} dE' \int_{-1}^{1} d\mu_0 \Bigg( \(E-E'\)
    \hat{\Sigma}_s \(\frac{1-\mu_0}{\delta},\right.\\
    &\left.E,\frac{E-E'}{\epsilon}\)
    \phi_{l,m}(E)\Bigg) + \frac{\pi}{\Delta}\frac{\partial^2}{\partial E^2}
    \int_{-\infty}^{\infty}  dE' \int_{-1}^1 d\mu_0\ \Bigg((E-E')^2\\
    & \hat{\Sigma}_s\(\frac{1-\mu_0}{\delta},E,\frac{E-E'}{\epsilon}\)
    \phi_{l,m}(E)\Bigg).
  \end{split}
  \label{K}
\end{equation}
Finally, using:
\begin{equation}
\frac{1}{\Delta}\hat{\Sigma}_s \(\frac{1-\mu_0}{\delta},E,\frac{E-E'}{\epsilon}\)
= \Sigma_s(\mu_0,E,E')
\end{equation}
and replacing the lower limit of integration on the $E'$ integral by 0, since
the probability of scattering to a negative energy is zero, we obtain:
\begin{equation}
  K=\Sigma_{s}\phi_{l,m}-l(l+1)\tilde{T}\phi_{l,m}+\frac{\partial}{\partial E}
  \tilde{S}\phi_{l,m}+\frac{1}{2}\frac{\partial^2}{\partial E^2}\tilde{R}
  \phi_{l,m}+O\(\frac{\delta^2+\epsilon\delta+\epsilon^3}{\Delta}\)
  \label{K2}
\end{equation}
where:
\begin{align}
&\tilde{T}(E) = \pi \int_0^{\infty} dE' \int_{-1}^1d\mu_0\
(1-\mu_0)\Sigma_s(\mu_0,E,E')=
O\(\frac{\delta}{\Delta}\) \label{T}\\
&\tilde{S}(E) = 2\pi \int_0^{\infty}dE' \int_{-1}^1 d\mu_0\ (E-E')
\Sigma_s(\mu_0,E,E') = O\(\frac{\epsilon}{\Delta}\) \label{S}\\
&\tilde{R}(E) = 2 \pi \int_0^{\infty}dE'\int_{-1}^1 d\mu_0 (E-E')^2 \Sigma_s
(\mu_0,E,E') = O\(\frac{\epsilon^2}{\Delta}\). \label{R}
\end{align}
The function $\tilde{\alpha} = 2\tilde{T}$ is known as the momentum transfer while
$\tilde{S}$ and $\tilde{R}$ are the stopping power and the mean square
stopping power, 
respectively. Note that the difference with the restricted parameters
defined earlier is that $\Sigma_s$ is used instead of $\Sigma_{s,sing}$. Using 
\cref{K_def}, substituting \cref{K2} into
the scaled transport \cref{scaled_transport}, and using \cref{sigma_s} for 
$\hat{\Sigma}_s(E)$, we get:
\begin{equation}
  \begin{split}
    &\bo\cdot \bn\psi + \(\Sigma_a+\Sigma_s\) \psi= \sum_{l=0}^{\infty}
    \sum_{m=-l}^l Y_l^m\Big(\Sigma_s \phi_{l,m} - l(l+1) \tilde{T} \phi_{l,m}+\\
    &\left. \frac{\partial}{\partial E}\tilde{S}\phi_{l,m}+\frac{1}{2}
    \frac{\partial^2}{\partial E^2}\tilde{R} \phi_{l,m}\)+Q+
    O\(\frac{\delta^2+\epsilon\delta+\epsilon^3}{\Delta}\).
  \end{split}
  \label{final_m1}
\end{equation}
Using \cref{eigenvalue}, the $l(l+1)$ factor can be eliminated, then we can 
sum over the spherical harmonics according to \cref{ang_flux} to obtain our final 
result:
\begin{equation}
  \begin{split}
    &\bo\cdot \bn \psi(\br,\bo,E) + \Sigma_a(\br,E) \psi(\br,\bo,E) =
    \tilde{T}(\br,E) \(\frac{\partial}{\partial \mu}(1-\mu^2) 
    \frac{\partial}{\partial\mu}+\right.\\
    &\left. \(\frac{1}{1-\mu^2}\)\frac{\partial^2}{\partial \varphi^2}\)
    \psi(\br,\bo,E) +\frac{\partial}{\partial E} \tilde{S}(\br,E)\psi(\br,\bo,E) 
    +\\
    &\frac{\partial^2}{\partial E^2} \tilde{R}(\br,E) \psi(\br,\bo,E)+
    Q(\br,\bo,E)+O\(\frac{\delta^2+\epsilon\delta+\epsilon^3}{\Delta}\).
\end{split}
\label{final}
\end{equation}
We note that $\Sigma_s(\br,E) \psi(\br,\bo,E)$ has canceled out in this equation.

\Cref{final} is the Fokker-Planck equation for linear particle
transport in an isotropic medium. As mentioned previously, the fact that the
scattering kernel is peaked in energy ($E'\approx E$) and angle ($\mu_0\approx
1$) is a necessary but not sufficient condition for \cref{final} to be an
asymptotic limit of \cref{transport_p}. The additional sufficient condition 
is that the fall off is either exponential or strongly algebraic. Assuming that 
\cref{final} is a valid asymptotic limit, it is clear from 
\crefrange{T}{R} that $\epsilon$, $\delta$ and $\Delta$ must tend to zero 
in a correlated way. Looking at \cref{T}, we see that we need
$O(\delta)=O(\Delta)$ to have a finite and nonzero angular term in
\cref{final}. Similarly, by looking at \cref{S}, we must have $O(\epsilon) =
O(\Delta)$ for \cref{final} to have a finite and nonzero energy term.
Therefore, when the scattering becomes more peaked ($\epsilon \rightarrow 0$
and $\delta \rightarrow 0$), the magnitude of the cross section must increase
to have the momentum transfer to stay finite and nonzero.

Since $\tilde{R}$ is $O\(\frac{\epsilon^2}{\Delta}\)$, we do not need $\tilde{R}$ 
to keep the leading order behavior in energy transfer. In many
applications, the BFP equation used does not have the $\tilde{R}$ term. The 
reason for keeping $\tilde{R}$, even though it is a higher order term, 
is that it describes a totally different physics than the stopping power 
($\tilde{S}$) term: the stopping power term is convective 
whereas the $\tilde{R}$ term is diffusive. In certain applications, even a
small diffusion of the particles in the energy variable can have an important
effect and thus, it is important to keep the $\tilde{R}$ term \cite{pomraning}. 
The reason that no convective term in angle appears in
\cref{final} (the $\tilde{T}$ term is diffusive) is that we have assumed 
an isotropic medium. There is the same probability for a particle to be scattered 
to the left or the right, and thus, the mean scattering angle is zero.

\subsection{Limits of the Boltzmann-Fokker-Planck approximation}
\subsubsection{Introduction}
In this section, we recall the work of Larsen in \cite{larsen_fp}. We will show 
the limitation of the Fokker-Planck operator and, therefore, of 
the Boltzmann-Fokker-Planck equation on two well-known scattering
kernels: the Henyey-Greenstein scattering kernel and the Rutherford scattering
kernel \cite{larsen_fp}. The Henyey-Greenstein scattering kernel mimics the angular 
dependence of light scattering by small particles, whereas the screened Rutherford
scattering kernel represents the scattering of electrons by a screened atomic
nucleus. 

First, we need to define several operators \cite{larsen_fp}:
\begin{align}
\mc{L}_B\psi(\bo) &= \int_{4\pi} \(\sum_{l=0}^{\infty} \frac{2l+1}{4\pi} f_l
P_l(\bo\cdot\bo')\) \psi(\bo')d\bo'-\psi(\bo) \label{L_B}\\
\mc{L}_{FP}\psi(\bo) &= \(\frac{\partial}{\partial \mu}
(1-\mu^2)\frac{\partial}{\partial
\mu}+\frac{1}{1-\mu^2}\frac{\partial^2}{\partial \varphi^2}\)\psi(\bo)
\label{L_FP}\\
\mc{L}_1 \psi(\bo) &= \frac{1}{4\pi} \int_{4\pi}
\frac{\psi(\bo')-\psi(\bo)}{1-\bo'\cdot\bo}d\bo'
\label{L_1}\\
\mc{L}_{3/2} \psi(\bo) &= \frac{1}{4\pi\sqrt{2}} \int_{4\pi}
\frac{\psi(\bo')-\psi(\bo)}{(1-\bo'\cdot\bo)^{3/2}} d\bo',\label{L_3/2}
\end{align}
where $f_l$ are the Legendre expansion coefficient of $f(\bo\cdot\bo')d\bo$ 
which is the probability that a particle, entering a scattering event with 
a direction $\bo'$, will emerge from the event with a direction in $d\bo$ 
about $\bo$. If $\psi(\bo)$ is sufficiently smooth, \crefrange{L_B}{L_3/2} become:
\begin{align}
\mc{L}_B\psi(\bo) &= \sum_{l=0}^{\infty} \sum_{m=-l}^l (-(1-f_l))Y_l^m(\bo)
\int_{4\pi}Y_l^{m,*}(\bo')\psi(\bo')d\bo' \label{L_B_smooth}\\
\mc{L}_{FP}\psi(\bo) &= \sum_{l=0}^{\infty} \sum_{m=-l}^l (-l(l+1))
Y_l^m(\bo) \int_{4\pi} Y_l^{m,*}(\bo')\psi(\bo')d\bo' \label{L_FP_smooth}\\
\mc{L}_1 \psi(\bo) &= \sum_{l=0}^{\infty} \sum_{m=-l}^l \(-\sum_{k=1}^l
\frac{1}{k}\) Y_l^m(\bo) \int_{4\pi} Y_l^{m,*}(\bo')\psi(\bo')d\bo'
\label{L_1_smooth}\\
\mc{L}_{3/2} \psi(\bo) &= \sum_{l=0}^{\infty} \sum_{m=-l}^l (-l) Y_l^m(\bo)
\int_{4\pi} Y_l^{m,*}(\bo') \psi(\bo') d\bo'. \label{L_3/2_smooth}
\end{align}
Therefore, the spherical harmonics are eigenfunctions of
\crefrange{L_B_smooth}{L_3/2_smooth}:
\begin{align}
\mc{L}_B Y_l^m(\bo) &= -(1-f_l) Y_l^m(\bo) \label{L_B_eig}\\
\mc{L}_{FP} Y_l^m(\bo) &= -l(l+1) Y_l^m(\bo) \label{L_FP_eig}\\
\mc{L}_1 Y_l^m(\bo) &= \(-\sum_{k=1}^l \frac{1}{k}\) Y_l^m(\bo)
\label{L_1_eig}\\
\mc{L}_{3/2} Y_l^m(\bo) &= -l Y_l^m(\bo). \label{L_3/2_eig}
\end{align}
%
\subsubsection{Henyey-Greenstein scattering kernel}
The Henyey-Greenstein \cite{H-G} differential scattering kernel is defined by
\cite{larsen_fp}:
\begin{equation}
f(\mu_0) = \frac{1-\bar{\mu}_0^2}{4\pi(1-2\bar{\mu}_0\mu
+\bar{\mu}_0^2)^{3/2}}.
\label{H-G}
\end{equation}
This kernel was proposed in \cite{H-G}, and is widely used mainly because of 
its great simplicity. The exact Henyey-Greenstein kernel does not arise from 
a deeper theory of any known physical process \cite{larsen_fp}. The Legendre 
polynomial expansion of the kernel, 
$f(\mu_0)=\sum_{l=0}^{\infty}\frac{2l+1}{2}f_l P_l(\mu_0)$, is very simple
since the expansion coefficients, $f_l$, are given by:
\begin{equation}
f_l = \bar{\mu}_0^l,\ l\geq 0.
\end{equation}
We look at:
\begin{equation}
\Sigma_s\mathcal{L}_B = \Sigma_s\(\int_{4\pi}
f(\bo\cdot\bo')\psi(\br,\bo',E)d\bo'-\psi(\br,\bo,E)\)
\end{equation}
with the scaled Henyey-Greenstein kernel:
\begin{align}
&\Sigma_s = \frac{\hat{\Sigma}_s}{\epsilon}\\
&\bar{\mu}_0 = 1-\epsilon,
\end{align}
where $\hat{\Sigma}_s$ is fixed and $\epsilon \approx 0$. When the
scattering mean free path tends to zero, the mean scattering cosine tends to
one. In the absence of absorption, the transport cross section is independent
of $\epsilon$:
\begin{equation}
  \Sigma_{tr}=\Sigma_s(1-\bar{\mu}_0) = \frac{\hat{\Sigma}_s}{\epsilon}\cdot 
  \epsilon = \hat{\Sigma}_s.
\end{equation}
When $\epsilon$ tends to zero, the eigenvalues of $\Sigma_s\mathcal{L}_s$
become:
\begin{equation}
\begin{split}
-\Sigma_s(1-f_l) &= -\frac{\hat{\Sigma}_s}{\epsilon}\(1-(1-\epsilon)^l\)\\
&=-\frac{\hat{\Sigma}_s}{\epsilon}\(l\epsilon-\frac{l(l-1)}{2}\epsilon^2+
O(\epsilon^3)\)\\
&= \hat{\Sigma}_s(1+\epsilon)(-l)-\frac{\hat{\Sigma}_s \epsilon}{2}(-l(l+1))+
O(\epsilon^2).
\end{split}
\label{eps_eig}
\end{equation}
Using \cref{L_B_eig,L_FP_eig,L_3/2_eig,eps_eig}, we get:
\begin{equation}
\Sigma_s\mc{L}_B = \hat{\Sigma}_s(1+\epsilon) \mc{L}_{3/2} -
\frac{\hat{\Sigma}_s \epsilon}{2} \mc{L}_{FP} + O(\epsilon^2).
\label{HG_approx}
\end{equation}
The leading-order approximation, $\mathcal{O}(1)$, to $\Sigma_s \mc{L}_B$ is:
\begin{equation}
\Sigma_s \mc{L}_B = \Sigma_s(1-\bar{\mu}_0) \mc{L}_{3/2} +O(\epsilon).
\end{equation}
The first-order approximation, $\mathcal{O}(\epsilon)$, to $\mc{L}_B$ is given by:
\begin{equation}
\Sigma_s\mc{L}_B = \Sigma_s(1-\bar{\mu}_0)(2-\bar{\mu}_0) \mc{L}_{3/2} -
\frac{\Sigma_s}{2}(1-\bar{\mu}_0)^2\mc{L}_{FP} + O(\epsilon^2).
\end{equation}
We see that the Fokker-Planck operator is not an asymptotic limit of the
Henyey-Greenstein kernel because $\mc{L}_{FP}$ appears only in the first order
approximation. The zeroth-order approximation contains only $\mc{L}_{3/2}$.

\subsubsection{Screened Rutherford scattering kernel}
The screened Rutherford differential scattering kernel, which is widely used
to model the Coulomb scattering of non-relativistic electrons, is defined by:
\begin{equation}
f(\mu_0) = \frac{\eta(1+\eta)}{\pi(1+2\eta-\mu_0)^2},
\label{scr_rut}
\end{equation}
where $\eta>0$ is the screening parameter. We consider $\eta\approx 0$, for
which $f(\mu_0)$ is the most forward-peaked. The first two Legendre expansion
coefficient of $f(\mu_0)$ are given by:
\begin{align}
&f_0 = 1\\
&f_1 = \bar{\mu}_0 = 1 -2\eta \((1+\eta)\ln\(1+\frac{1}{\eta}\)-1\).
\label{f_1}
\end{align}
Using \cref{scr_rut}, $f_k = 2\pi \int_{-1}^1 P_k(\mu) f(\mu)\ d\mu$,  and:
\begin{equation}
\mu P_k(\mu) = \frac{(k+1)P_{k+1}(\mu)+kP_{k-1}(\mu)}{2k+1},
\end{equation}
we obtain the following recurrence formula:
\begin{equation}
kf_{k+1} - (1+2\eta)(2k+1)f_k+(k+1)f_{k-1} = 0,
\label{recursion}
\end{equation}
or equivalently:
\begin{equation}
\frac{f_{k+1}-f_k}{k+1} - \frac{f_k-f_{k-1}}{k} = 2 \eta \frac{2k+1}{k(k+1)}
f_k.
\label{frac_k}
\end{equation}
Summing \cref{frac_k} from $k=1$ to $m-1$, using \cref{f_1}, and then 
multiplying by $m$ yields:
\begin{equation}
f_m-f_{m-1}-(\bar{\mu}_0-1)m = 2 \eta m \sum_{k=1}^{m-1} \frac{2k+1}{k(k+1)}
f_k.
\label{f_f}
\end{equation}
Summing \cref{f_f} from $m=1$ to $l$ yields:
\begin{equation}
\begin{split}
f_l-1-(\bar{\mu}_0-1)\frac{l(l+1)}{2} &= 2 \eta \sum_{m=1}^l m\sum_{k=1}^{m-1}
\frac{2k+1}{k(k+1)}f_k\\
&= 2\eta \sum_{k=1}^{l-1} \(\sum_{m=k+1}^l m\) \frac{2k+1}{k(k+1)} f_k\\
&=2\eta\sum_{k=1}^{l-1}\(\frac{l(l+1)}{2}-\frac{k(k+1)}{2}\) 
\frac{2k+1}{k(k+1)}f_k,
\end{split}
\end{equation}
where we have used:
\begin{equation}
\sum_{m=1}^l m = \frac{l(l+1)}{2}.
\label{sum_m}
\end{equation}
Finally, we find the $f_l$ coefficient:
\begin{equation}
f_l = 1-(1-\bar{\mu}_0)\frac{l(l+1)}{2}+\eta\sum_{k=1}^{l-1}
\(\frac{l(l+1)}{k(k+1)}-1\) (2k+1) f_k.
\label{f_n_R}
\end{equation}
If the summation term is assumed to be zero for $l=0$ and $l=1$,
\cref{recursion} and \cref{f_n_R} are equivalent for all $l\geq 0$ and $\eta >
0$. Next, we assume the following expansion of $f_l$ for $\eta \ll 1$:
\begin{equation}
f_l = f_{l,0}+\eta f_{l,1} + O(\eta^2).
\end{equation}
Introducing this expansion into \cref{f_n_R} and equating the coefficients of
$\eta^0$ and $\eta^1$ (note that $\bar{\mu}_0$ is not expanded for $\eta \ll 1$), 
we find:
\begin{align}
& f_{l,0} = 1 - (1-\bar{\mu}_0) \frac{l(l+1)}{2}\\
& f_{l,1} =
\sum_{k=1}^{l-1}\(\frac{l(l+1)}{k(k+1)}-1\)(2k+1)\(1-(1-\bar{\mu}_0)
\frac{k(k+1)}{2}\)
\end{align}
From \cref{f_1}, we know that:
\begin{equation}
1-\bar{\mu}_0 = O\(\eta \ln\(\frac{1}{\eta}\)\).
\label{weakly_small}
\end{equation}
Therefore, we get:
\begin{equation}
  \begin{split}
    f_l =& 1+\(\frac{1-\bar{\mu}_0}{2}\) \(-l(l+1)\)+\eta \(\sum_{k=1}^{l-1}
    \(\frac{l(l+1)}{k(k+1)}-1\)(2k+1)\)\\ 
    &+ O\(\eta^2\ln\(\frac{1}{\eta}\)\).
  \end{split}
  \label{eq_58}
\end{equation}
Now using \cref{sum_m}, we get:
\begin{equation}
\sum_{k=1}^{l-1} \(\frac{l(l+1)}{k(k+1)}-1\)(2k+1) = 2 l(l+1) \(\(\sum_{k=1}^l
\frac{1}{k}\)-1\)
\end{equation}
for $l\geq 0$.\\
Thus, \cref{eq_58} becomes:
\begin{equation}
  \begin{split}
    f_l= &1+\(\frac{1-\bar{\mu}_0}{2}\)\big(-l(l+1)+2\eta\(-l(l+1)\)\big)
    \(1-\sum_{k=1}^l \frac{1}{k}\)\\ 
    &+ O\(\eta^2\ln\(\frac{1}{\eta}\)\).
  \end{split}
\end{equation}
Next, we define $\Sigma_s$ as:
\begin{equation}
\Sigma_s = \frac{\hat{\Sigma}_s}{1-\bar{\mu}_0}.
\end{equation}
Then for $\eta \ll 1$, the eigenvalues of $\Sigma_s \mc{L}_B$ are:
\begin{equation}
  -\Sigma_s (1-f_l) = \frac{\hat{\Sigma}_s}{2}\(-l(l+1)\)+
  \frac{2\eta\hat{\Sigma}_s}{1-\bar{\mu}_0} (-l(l+1))\(1-\sum_{k=1}^l 
  \frac{1}{k}\)+O(\eta).
\end{equation}
Using \cref{L_1_eig}, we finally get:
\begin{equation}
\begin{split}
\Sigma_s \mc{L}_B &= \frac{\sigma}{2}\mc{L}_{FP} + \frac{2\eta
\sigma}{1-\bar{\mu}_0} \mc{L}_{FP}(I+\mc{L}_1) + O(\eta)\\
&= \frac{\Sigma_s(1-\bar{\mu}_0)}{2} \mc{L}_{FP} + 2 \eta \Sigma_s \mc{L}_{FP}
(I+\mc{L}_1) + O(\eta).
\label{screen_rut_final}
\end{split}
\end{equation}
The first term in this expansion is the standard Fokker-Planck operator which
is $O(1)$. The second term in \cref{screen_rut_final} is $O(1/\ln(1/\eta))$. 
Therefore, $1/\ln(1/\eta)$ must be very small for the Fokker-Planck description 
to be valid. Because of this logarithmic behavior, realistic values of $\eta$
are typically not small enough \cite{pencil_pomraning}.

In conclusion, we see that the Fokker-Planck operator appears in the asymptotic 
approximations of $\mc{L}_B$ for both Henyey-Greenstein and screened Rutherford 
scattering. However, for the Henyey-Greenstein scattering, $\mc{L}_{FP}$ is 
dominated by a pseudodifferential operator $\mc{L}_{3/2}$, and for the screened 
Rutherford scattering, $\mc{L}_{FP}$ weakly dominates another pseudodifferential 
operator $\mc{L}_{FP}\mc{L}_1$. In neither case, the first-order approximations of 
$\mc{L}_B$ can be written only with $\mc{L}_{FP}$ \cite{larsen_fp}.
