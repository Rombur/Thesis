\chapter{Transport equation}

\section{Derivation of the Boltzmann-Fokker-Planck equation}
\subsection{Ligou's derivation}
The Boltzmann equation cannot be used to describe charged particles transport.
The Boltzmann equation is valid from the theoretical point of view but the
scattering cross sections are so forward peaked that a polynomial expansion
would require too many terms (slow convergence of the expansion). To avoid the
cross section expansion, the Boltzmann equation can be replaced by a
Fokker-Planck equation. The Fokker-PlancK equation can be used to represent
the highly forward peaked scattering but it cannot represent the large angles
scattering. The Boltzmann-Fokker-Planck (BFP) equation uses a polynomial expansion
for the large angle scattering and a Fokker-Planck term for the forward peaked
scattering. Following \cite{ligou}, the BFP equation can be derived starting
from the Boltzmann equation\footnote{\red{L'equation doit etre definie avant,
bien faire attention a la definition de harmoniques spheriques->ca change les
coefficients dans l'equation de transport. On prend les harmoniques
spheriques orthonormees.Il faut verifier que tout a bien ete defini avant.}}:
\begin{equation}
\begin{split}
&\bo\cdot\bn\Psi(\br,\bo,E) + \(\Sigma_s(\br,E)+\Sigma_a(\br,E)\)\Psi(\br,bo,E)
=\\ 
&\sum_{l=0}^{\infty}\sum_{m=-l}^l \int_{0}^{\infty} \Sigma_{s,l}\(\br,E'\rightarrow
E\) \Phi_{l}^m(\br,E')Y_l^m(\bo)\ dE' + Q(\br,\bo,E)
\label{b_2}
\end{split}
\end{equation}
The scattering angle cosines in the center-of-mass ($\mu_c$) or in the
laboratory system ($\mu_L$) are given by:
\begin{align}
\mu_c &= 1-\frac{2}{\beta}\frac{E'-E}{E'} \label{mu_c}\\
\mu_L = \frac{1}{2}\[(A+1)\sqrt{\frac{E}{E'}}-(A-1)\sqrt{E'}{E}\]
\label{mu_l}
\end{align}
where:
\begin{align}
\beta &= 1 - \alpha\\
\alpha &= \(\frac{A-1}{A+1}\)^2
\end{align}
A being the particle mass ratio. When the microscopic scattering cross section
$(\sigma_s(\br,\mu_c,E'))$ are known, the Legendre expansion of the scattering
cross section is given by:
\begin{equation}
\Sigma_{s,l}(\br,E'\rightarrow E) = N \frac{4\pi}{\beta E'}
\sigma_s\[E',\mu_c\(\frac{E'-E}{E'}\)\]P_l\[\mu_L\(\frac{E'}{E}\)\]
\label{eq7}
\end{equation}
for $E\leq E' \leq \frac{E}{\alpha}$

Now the differential scattering cross section is split into two parts:
\begin{equation}
\Sigma_s(\br,\mu_c) = \Sigma_{s,reg}(\br,\mu_c + \Sigma_{s,sing}(\br,\mu_c)
\end{equation}
The first term is a ``regular'' cross section which does not increase too much
when $\mu$ goes to one. By definition, a expansion on the Legendre polynomials
of this term will converge quickly. The second term is ``singular'' cross section 
which is highly forward peaked. This term is only important when $\mu\approx
1$. The scattering term in (\ref{b_2}) can be written as:
\begin{equation}
q = q'+q''
\end{equation}
with\footnote{this shot uses orthonormal spherical harmonics}:
\begin{equation}
\begin{bmatrix}
q'(\bo,E)\\
q''(\bo,E)
\end{bmatrix}
= \sum_{l=0}^{\infty} \frac{2l+1}{4\pi}\sum_{m=-l}^l \frac{(l-|m|)!}{(l+|m|)!}
Y_l^m(\bo)
\begin{bmatrix}
q_{l,m}'(E)\\
q_{l,m}'(E)
\end{bmatrix}
\end{equation}
and:
\begin{equation}
\begin{bmatrix}
q_{l,m}'(E)\\
Q_{l,m}''(E)
\end{bmatrix}
=\int_{E}^{E/\alpha} dE' \phi_{l,m}(E')
\begin{bmatrix}
\Sigma_{s,l}'(E'\rightarrow E)\\
\Sigma_{s,l}''(E' \rightarrow E)
\end{bmatrix}
\end{equation}
where $\Sigma_{s,l}'$ and $\Sigma_{s,l}''$ are the Legendre kernels
corresponding to $\sigma_s'$ and $\sigma_s''$ through (\ref{eq7}).
In the preceding equation, it is more convenient to use $\mu_c$ instead of
$E'$ as a new integration variable:
\begin{equation}
q_{l,m}''(E) = 2\pi N\int_{-1}^1\frac{E'}{E}\sigma_s''(E',\mu_c)
P_l\[\mu_L(\mu_c)\] \phi_{l,m}(E')d\mu_c
\end{equation}
with:
\begin{equation}
\mu_L = \frac{1+A\mu_c}{(1+A^2+2A\mu_c)^{1/2}}
\end{equation}
Assuming that $\phi_{l,m}(E')$ is a smooth function, one can perform the
following Taylor expansions:
\begin{align}
&E'\sigma_s''(E',\mu_c)\phi_{l,m} (E') = E\sigma_s''(E,\mu_c)\phi_{l,m}(E) +
(E'-E)\frac{\partial}{\partial
E}\(E\sigma_s''(E,\mu_c)\phi_{l,m}(E)\)+\hdots\\
& P_l(\mu_L) = P_l(1)-(1-\mu_L)P_l'(1)+\hdots
\end{align}
with $P_l(1)=1$ and $P_l'(1)=l\frac{l+1}{2}$.
Moreover, using equations (\ref{mu_c}) and (\ref{mu_l}) and $\mu_c\approx
1$ or $A\ll 1$, we get:
\begin{align}
& \frac{E'-E}{E'} = \frac{2A}{(A+1)^2} (1-\mu_c)\\
& 1-\mu_L \approx \(\frac{A}{A+1}\)^2 (1-\mu_c)
\end{align}
then to a first order in $(1-\mu_c)$ one obtains:
\begin{equation}
q_{l,m}''(E) = \Sigma_s''(E) \phi_{l,m}(E) +\frac{\partial}{\partial E} S''(E)
\phi_{l,m}(E) - l(l+1)T''(E) \phi_{l,m}(E)
\end{equation}
where:
\begin{align}
&\Sigma_s''(E) N 2 \pi \int_{-1}^{1}\sigma_s''(E,\mu_c)\ d\mu_c\\
&T''(E) = \frac{N}{2} 2 \pi \int_{-1}^1 (1-\mu_L)\sigma_s''(E,\mu_c)\ d\mu_c\\
&S''(E) = \frac{4E}{A}T''(E)
\end{align}
Using:
\begin{align}
& \phi(E,\bo) = \sum_{l=0}^{\infty} \sum_{m=-l}^l \phi_{l,m}Y_l^m(\bo)\\
& \frac{\partial}{\partial \mu}(1-\mu^2)\frac{\partial Y_l^m}{\partial \mu} + 
\frac{1}{1-\mu^2} \frac{\partial^2 Y_l^m}{\partial \varphi^2} = -l(l+1)Y_l^m
\end{align}
We get:
\begin{equation}
q''(E,\bo) = \Sigma_s''(E)\phi(E,\bo)+\frac{\partial}{\partial
E}S''(E)\phi(E,\bo) + T''(E)\[\frac{\partial}{\partial \mu} (1-\mu^2)
\frac{\partial}{\partial \mu}+\frac{1}{1-\mu^2}\frac{\partial^2}{\partial
\varphi^2}\] \phi(E,\bo)
\end{equation}
Finally, we get:
\begin{equation}
\begin{split}
&\bo\cdot\Psi+(\Sigma_s'+\Sigma_s)\Psi=\sum_{l=0}^{\infty}\sum_{m=-l}^l
Y_l^m(\bo)\int_E^{E/\alpha}\Sigma_{s,l}(E'\rightarrow
E)\phi_{l,m}(E')dE'+\\
&\frac{\partial}{\partial E}S''\phi+T''\[\frac{\partial}{\partial
\mu}(1-\mu^2) \frac{\partial}{\partial
\mu}+\frac{1}{(1-\mu^2)}\frac{\partial^2}{\partial\varphi^2}\]\phi+Q
\end{split}
\end{equation}

\subsection{Pomraning's derivation: Fokker-Planck equation as a limit of
Boltzmann equation}
\begin{itemize}
\item $\mu_0 = \bo'\cdot \bo$ is the cosine of the polar angle
\item $\bo$ is a unit vector in the flight direction
\item $\mu = \cos(\theta)$, where $\theta$ is a polar angel
\end{itemize}
\subsubsection{Transport equation}
In charged particle transport, the scattering kernel is very peaked about both
a zero energy transfer and a zero direction change. However, the number of
scattering collisions is very large. The average distance a particle travels
between scattering events (the scattering mean free path) is very small. The
slight effect of a single scattering can loosely be characterized as an
$O(\epsilon \ll 1)$ effect, and the small scattering mean free path can be
characterized as an $O(\Delta \ll 1)$ distance. Thus, in an $O(1)$ distance in
the medium, one can possibly observe a significant change in the energy and
direction of the particle, depending upon the relationship between $\epsilon$
and $\Delta$\cite{pomraning}.

To solve such a scattering transport problem numerically using deterministic
methods is very difficult since the mesh size in such a calculation must be on
the same scale as the mean free path, which in this case is very small. This
implies an unrealistically fine degree of numerical resolution. Likewise, a
Monte-Carlo simulation is very time consuming since a very large number of
scattering interactions must be followed for each particle before its demise
by either absorption or leakage out of the system. To circumvent these
difficulties, it has been suggested to replace the integral scattering
operator in the transport equation with a differential Fokker-Planck operator.
The effect of this replacement is that the dominant (large) in and out
scattering terms cancel, thus effectively increasing the mean free
path\cite{pomraning}.

The transport equation is given by:
\begin{equation}
\bo\cdot \bn \Psi(E,\bo) + \Sigma_t(E)\Psi(E,\bo) =
\int_0^{\infty}dE'\int_{4\pi}
\Sigma_s(E',E,\bo'\cdot\bo)\Psi(E',\bo')+Q(E,\bo)
\label{transport_p}
\end{equation}
Here $\Psi(E,\bo)$ is defined by:
\begin{equation}
\Psi(E,\bo) = vf(E,\bo)
\end{equation}
where $v$ is the particle speed, $\Sigma_t(E)$ is the total cross-section
given by:
\begin{equation}
\Sigma_t(E) = \Sigma_a(E)+\Sigma_s(E)
\end{equation}
and $Q(E,\bo)$ represents any source of particles. We are not concern about
boundary conditions.

The in-scattering term can be represented as a sum over its a surface harmonic
components. To this end, we expand the scattering kernel in Legendre
polynomials according to:
\begin{equation}
\Sigma_s(E',E,\bo'\cdot\bo)=\sum_{l=0}^{\infty}
\Sigma_{s,l}(E',E)P_l(\bo'\cdot\bo)
\end{equation}
The orthogonality of the Legendre polynomials allows us to write the expansion
coefficients as:
\begin{equation}
\Sigma_{s,l}(E',E) = 2\pi\int_{-1}^1 d\mu_0 P_l(\mu_0) \Sigma_s(E',E,\mu_0)
\end{equation}
Further, we expand the solution $\Psi(E,\bo)$ in surface harmonics according
to:
\begin{equation}
\Psi(E,\bo) = \sum_{l=0}^{\infty}\sum_{m=-l}^l \phi_{l,m} Y_l^m(\bo)
\end{equation}
The expansion coefficients $\phi_{l,m}$ are given by:
\begin{equation}
\phi_{l,m}(E) = \int_{4\pi} d\bo Y_{l}^{m,*} \Psi(E,\bo)
\label{moments}
\end{equation}
A property of the surface harmonics is that they satisfy the partial equation
given by:
\begin{equation}
\[\frac{\partial}{\partial\mu}(1-\mu^2)\frac{\partial}{\partial
\mu}+\(\frac{1}{1-\mu^2}\)\frac{\partial^2}{\partial \varphi}+l(l+1)\]Y_l^m(\bo)=0
\label{eigenvalue}
\end{equation}
Using the addition theorem, we get:
\begin{equation}
\bo\cdot\bn \Psi(E,\bo) +
\Sigma_t(E)\Psi(E,\bo)=\sum_{l=0}^{\infty}\sum_{m=-l}^{l} Y_l^m(\bo) 
\int_0^{\infty} dE' \Sigma_{s,l}(E',E)\Psi_{l,m}(E') + Q(E,\bo)
\end{equation}
or equivalently:
\begin{equation}
\begin{split}
&\bo\cdot\bn \Psi(E,\bo) + \(\Sigma_a(E)+\Sigma_s(E)\)\Psi(E,\bo) =\\
\sum_{l=0}^{\infty}\sum_{m=-l}^l Y_l^m(\bo) \int_0^{\infty}dE'\ \Psi_{l,m}(E')
\int_{-1}^1 d\mu_0\ P_l(\mu_0) \Sigma_{s}(E',E,\mu_0)+Q(E,\bo)
\end{split}
\end{equation}

\subsubsection{Fokker-Planck development}
We begin by assuming that the unit of distance is chosen such that the
characteristic size of the system in which the particle transport is $O(1)$.
In such a length unit, we take the scattering mean free path to be small, and
since this mean free path is the reciprocal of the scattering cross-section,
we have $\Sigma_s(E) \gg 1$. Accordingly, we scale $\Sigma_s(E)$ as:
\begin{equation}
\Sigma_s(E) = \frac{\hat{\Sigma}_s(E)}{\Delta}
\label{sigma_s}
\end{equation}
where $\hat{\Sigma}_s=O(1)$ and $\Delta \ll 1$. We apply this same scaling to
the scattering kernel $\Sigma_s(E',E,\mu_0)$, and additionally introduce the
fast variables:
\begin{align}
x=\frac{E'-E}{\epsilon}, & \epsilon \ll 1 \label{x}\\
y=\frac{1-\mu_0}{\delta}, & \delta \ll 1 \label{y}
\end{align}
Thus, we scale the scattering kernel as:
\begin{equation}
\begin{split}
\Sigma_s(E',E,\mu_0) &= \frac{1}{\Delta}
\hat{\Sigma}_s\(E',\frac{E'-E}{\epsilon},\frac{1-\mu_0}{\delta}\)\\
&=\frac{1}{\Delta}\hat{\Sigma}_s(E',x,y)
\end{split}
\end{equation}
where $\hat{\Sigma}_s(E',x,y)$ is $O(1)$ and the derivatives of this kernel
with respect to both $x$ and $y$ are assumed to be $O(1)$ as
$\epsilon,\delta\rightarrow 0$. Physically, the smallness parameter $\delta$
is a measure of the peaking of the scattering kernel in angle, and can be
roughly thought of as the deviation from unity of the cosine of a
characteristic scattering angle. Likewise, the smallness parameter $\epsilon$
is a measure of the peaking of the scattering kernel in energy, and can be
thought of as a characteristic value of the fractional energy change due to a
single scattering. That is, these scalings imply that the scattering
cross-section is large, and that the scattering kernel is very peaked about
$\mu_0=1$ and $E=E'$. Inserting the scalings given, we then have the scaled
transport equation:
\begin{equation}
\begin{split}
&\bo\cdot\bn\Psi(E,\bo)+\(\Sigma_a(E)+\frac{\hat{\Sigma}_s}{\Delta}\)\Psi(E,\bo)\\
&\frac{1}{\Delta}\sum_{l=0}^{\infty}\sum{m=-l}^{l}Y_l^m(\bo)
\int_0^{\infty}dE'\ \phi_{l,m}(E')\int_{-1}^1d\mu_0\
P_l(\mu_0)\hat{\Sigma}_s(E',\frac{E'-E}{\epsilon},\frac{1-\mu_0}{\delta})+Q(E,\bo)
\end{split}
\label{scaled_transport}
\end{equation}
We see the asymptotic limit as the three smallness parameters
$\epsilon,\delta$ and $\Delta$ approach zero.\\
To this end, we consider the term $K$, defined to be:
\begin{equation}
K = \frac{2\pi}{\Delta} \int_0^{\infty}dE'\int_{-1}^1 d\mu_0 P_n(\mu_0)
\hat{\Sigma}_s(E',\frac{E'-E}{\epsilon},\frac{1-\mu_0}{\delta}) \phi_{l,m}(E')
\end{equation}
We change integration variables from $E',\mu_0$ to $x,y$ according to equations
(\ref{x}) and (\ref{y}) to obtain:
\begin{equation}
K = \frac{2\pi\epsilon\delta}{\Delta}\int_{-E/\epsilon}^{2/\delta}dy\
P_l(1-\delta y)\hat{\Sigma}_s(E+\epsilon x,x,y)\phi_{l,m}(E+\epsilon x)
\label{K_def}
\end{equation}
We now Taylor expand the integrand about $\epsilon=\delta=0$. We carry linear
terms in $\delta$ and quadratic terms in $\epsilon$. We will discuss the
reason for this asymmetry in the two expansions at the end of out development.
We then find, indicating the errors in the neglected terms in the Taylor
expansions,
\begin{equation}
\begin{split}
K=& \frac{2\pi \epsilon \delta}{\Delta}\int_{-E/\epsilon}^{\infty}dx\
\int_0^{2/\delta}dy\ \[P_l(1)-\delta y P_l'(1)+O(\delta^2)\]\\
&\[1+\epsilon x \frac{\partial}{\partial
E}+\frac{\epsilon^2x^2}{2}\frac{\partial^2}{\partial E^2}+O(\epsilon^3)\]
\hat{\Sigma}_s(E,x,y)\Phi_{l,m}(E)
\end{split}
\label{error}
\end{equation}
To proceed, we assume that we can replace the lower limit of integration over
$E'$ by $-\infty$, and that the error we make in this doing this is
$O(\epsilon^3)$ or smaller. This is certainly legitimate if the scattering
kernel falls off exponentially in energy from its peak at $x=0$. However, if
the kernels falls off algebraically at too weak a rate, this replacement may
increase the error in the Fokker-Planck treatment over the $O(\epsilon^3)$
error indicated in equation (\ref{error}). We note that even if the kernel
falls off exponentially in energy from its peak, this integration limit
replacement does introduce an error, but this error is exponentially small. It
is this exponential error that makes the present development an asymptotic,
rather than a convergent, procedure. At this point, we also neglect the cross
terms angle and energy in equation (\ref{error}). It is not necessary to
neglect these terms to obtain a clean result, but the standard Fokker-Planck
operator does not have the contributions that arise from these terms.

In view of the above discussion, we then rewrite equation (\ref{error}),
using:
\begin{align}
P_l(1) &= 1\\
P_l'(1) &= \frac{l(l+1)}{2}
\end{align}
as:
\begin{equation}
\begin{split}
K=& \frac{2\pi \epsilon \delta}{\Delta}
\int_{-\infty}^{\infty}dx\int_0^{2/\delta}
dy\(1+O(\delta^2+\epsilon+\epsilon^3)\)\hat{\Sigma}_s(E,x,y)\phi_{l,m}(E)\\
&-\frac{l(l+1)\pi\epsilon \delta^2}{\Delta} \int_{-\infty}^{\infty} dx
\int_0^{2/\delta}dy\ y \hat{\Sigma}_s(E,x,y)\phi_{l,m}(E)\\
&+\frac{2\pi \epsilon^2 \delta}{\Delta}\frac{\partial}{\partial E}
\int_{-\infty}^{\infty} dx \int_0^{2/\delta} dy\
x\hat{\Sigma}_s(E,x,y)\phi_{l,m}(E)\\
&+\frac{\pi\epsilon^3\delta}{\Delta}\frac{\partial^2}{\partial E^2}
\int_{-\infty}^{\infty} dx\int_0^{2/\delta}dy\ x^2\hat{\Sigma}_s(E,x,y)
\phi_{l,m}(E)
\end{split}
\label{extended}
\end{equation}
We now change integration variables in all four double integrals in equation
(\ref{extended}) from $x,y$ to $E',\mu_0$ according to:
\begin{align}
x&= \frac{E-E'}{\epsilon}\label{x2}\\
y&= \frac{1-\mu_0}{\delta}\label{y2}
\end{align}
We note that equation (\ref{y2}) is identical to equation (\ref{y}), but
equation (\ref{x2}) is not identical to equation(\ref{x}) (the $E$ and $E'$
are interchanged). This gives:
\begin{equation}
\begin{split}
K=& \frac{2\pi}{\Delta} \int_{-\infty}^{\infty} dE'
\int_{-1}^{1}d\mu_0\(1+O(\delta^2+\epsilon\delta+\epsilon^3)\)\hat{\Sigma}_s 
\(E,\frac{E-E'}{\epsilon},\frac{1-\mu_0}{\delta}\) \Phi_{l,m}(E)\\
&- \frac{l(l+1)\pi}{\Delta} \int_{-\infty}^{infty}dE'\int_{-1}^1d\mu_0
(1-\mu_0)
\hat{\Sigma}_s \(E,\frac{E-E'}{\epsilon},\frac{1-\mu_0}{\delta}\)\phi_{l,m}(E)\\
&+\frac{2\pi}{\Delta}\frac{\partial}{\partial E} \int_{-\infty}^{\infty} dE'
\int_{-1}^{1} d\mu_0 \(E-E'\)\hat{\Sigma}_s
\(E,\frac{E-E'}{\epsilon},\frac{1-\mu_0}{\delta}\) \phi_{l,m}(E)\\
&+ \frac{\pi}{\Delta}\frac{\partial^2}{\partial E^2} \int_{-\infty}^{\infty}
dE' \int_{-1}^1 d\mu_0\ (E-E')^2
\(E,\frac{E-E'}{\epsilon},\frac{1-\mu_0}{\delta}\) \phi_{l,m}(E)
\end{split}
\label{K}
\end{equation}
We use:
\begin{equation}
\frac{1}{\Delta}\hat{\Sigma}_s \(E,\frac{E-E'}{\epsilon},\frac{1-\mu_0}{\delta}\)
= \Sigma_s(E,E',\mu_0)
\end{equation}
in equation (\ref{K}), and note that the bottom limit of integration on the
$E'$ integral can be replaced by zero since the scattering kernel vanishes for
negative $E'$ (the probability of scattering to a negative energy is zero). We
then have:
\begin{equation}
\begin{split}
K=&\Sigma_{s}(E)\phi_{l,m}(E)-l(l+1)T(E)
\phi_{l,m}(E)+\frac{\partial}{\partial E}S(E)\phi_{l,m}\\
&\frac{\partial^2}{\partial
E^2}R(E)\phi_{l,m}(E)+O\(\frac{\delta^2+\epsilon\delta+\epsilon^3}{\Delta}\)
\end{split}
\label{K2}
\end{equation}
where we have defined:
\begin{align}
&T(E) = \pi \int_0^{\infty} dE' \int_{-1}^1d\mu_0\
(1-\mu_0)\Sigma_s(E,E',\mu_0)=
O\(\frac{\delta}{\Delta}\) \label{T}\\
&S(E) = 2\pi \int_0^{\infty}DE' \int_{-1}^1 d\mu_0\ (E-E')
\Sigma_s(E,E',\mu_0) = O\(\frac{\epsilon}{\Delta}\) \label{S}\\
&R(E) = 2 \pi \int_0^{\infty}DE'\int_{-1}^1 d\mu_0 (E-E')^2 \Sigma_s
(E,E',\mu_0) = O\(\frac{\epsilon^2}{\Delta}\) \label{R}
\end{align}
The order assigned to the terms in equations (\ref{T}) through (\ref{R})
seemingly follows friom the fact that since $x$ and $y$ are $O(1)$ variables,
then according to equations (\ref{x2}) and (\ref{y2}) $(1-\mu_0)$ is
$O(\delta)$, $(E-E')$ is $O(\epsilon)$ , and $(E-E')^2$ is $O(\epsilon^2)$,
with $\Sigma_s(E,E',\mu_0)$ being $O\(\frac{1}{\Delta}\)$. However, these
ordering are integrand, the assumption we have made in writing equations
(\ref{T}) through (\ref{R}) is that the integrations do not change the
ordering. The same holds true for taking the
$O(\delta^2+\epsilon\delta+\epsilon^3)$ out from under the integrals in going
from equation (\ref{K}) to equation (\ref{K2}). It is not obvious that the
integrations do not change the ordering since $\Sigma_s(E,E',\mu_0)$ contains
the smallness parameters $\epsilon$ and $\delta$. The situation again depends
upon the rate fall off of the scattering kernel from its peaks in energy and
angle. For exponential fall off, equations (\ref{K2}) through (\ref{R}) are
corrected ordered, but for algebraic fall off the order of one or more of
these terms may be larger than indicated. This observation again places a
restriction on the scattering kernel for the  Fokker-Planck differential
operator to be an asymptotic limit of the exact integral operator.

With the above caution in mind concerning the validity of the expression for
$K$, we proceed to complete our development. Recalling the definition of $K$
according to equation (\ref{K_def}), substitution of equation (\ref{K2}) into
the scaled transport equation (\ref{scaled_transport}), yields, also making
use of equation (\ref{sigma_s}) for $\hat{\Sigma}_s(E)$:
\begin{equation}
\begin{split}
&\bo\cdot \bn\Psi(E,\bo) + \(\Sigma_a(E)+\Sigma_s(E)\) \Psi(E,\bo)=\\
&\sum_{l=0}^{\infty}\sum_{m=-l}^lY_l^m(\bo)\(\Sigma_s(E) \phi_{l,m}(E) -
l(l+1) T(E) \phi_{l,m}(E)+\right.\\
&\left. \frac{\partial}{\partial
E}S(E)\phi_{l,m}(E)+\frac{\partial^2}{\partial E^2}R(E)\phi_{l,m}(E)\)
+Q(E,\bo)+O\(\frac{\delta^2+\epsilon\delta+\epsilon^3}{\Delta}\)
\end{split}
\label{final_m1}
\end{equation}
The $l(l+1)$ factor in equation (\ref{final_m1}) can be eliminated by making
use of the equation satisfied by the $Y_l^m(\bo)$, namely equation
(\ref{eigenvalue}), and then we can sum over the surface harmonics according
to equation (\ref{moments}) to obtain our final result. This is given by:
\begin{equation}
\begin{split}
&\bo\cdot \bn \Psi(E,\bo) + \Sigma_a(E) \Psi(E,\bo) =\\
&T(E) \(\frac{\partial}{\partial \mu}(1-\mu^2)\frac{\partial}{\partial\mu}+
\(\frac{1}{1-\mu^2}\)\frac{\partial^2}{\partial \varphi^2}\)\Psi(E,\bo)\\
&+\frac{\partial}{\partial E} S(E)\phi(E,\bo) + \frac{\partial^2}{\partial
E^2} R(E) \Psi(E,\bo)\\
& Q(E,\bo)+O\(\frac{\delta^2+\epsilon\delta+\epsilon^3}{\Delta}\)
\end{split}
\label{final}
\end{equation}
We note that the dominant scattering term, $\Sigma_s(E) \Psi(E,\bo)$, has
canceled out in this equation.

Equation (\ref{final}) is the conventional Fokker-Planck equation for linear
particle transport in an isotropic medium. As we noted during the course of
its derivation, a necessary but not sufficient condition for equation
(\ref{final}) to be an asymptotic limit of equation (\ref{transport_p}) is
that the scattering the kernel be peaked in both energy and angle. The
additional sufficient condition is that this peaking be either exponential or
strongly algebraic. Assuming that the scattering kernel is such that equation
(\ref{final}) is a valid asymptotic limit, it is clear, from the order of the
$T$, $S$ and $R$ terms as given by equations (\ref{T}) through (\ref{R}),
that the smallness parameters $\epsilon$, $\delta$ and $\Delta$ must tend to
zero in a correlated way. That is, we must have $O(\delta)=O(\Delta)$ in order
to obtain a meaningful (finite and nonzero) angular term in equation
(\ref{final}), and we must and we must have $O(\epsilon)=O(\Delta)$ to obtain
a meaningful energy term in equation (\ref{final}). The physical meaning of
thos is that as the peaking in the scattering kernel increases
$(\epsilon,\delta \rightarrow 0)$, the magnitude of the scattering
cross-section must increase $(\Delta \rightarrow 0)$ in such a corresponding
way that the momentum transfer remains bounded and nonzero. Only in this case
does equation (\ref{final}) have any meaning.

Finally, we note that since $R=O\(\frac{\delta}{\Delta}\)$, one could in an
asymptotically consistent manner always set $R$ to zero and still maintain the
leading order behavior in energy transfer. In applications, this is often
done, but not always. The reason for retaining $R$, even though it is a higher
order term, ist that the $R$ term in equation (\ref{final}) describes entirely
different physics than doe the $S$ term. The $S$ term, the so-called stopping
power term, is convective in nature, whereas the $R$ term is diffusive. In
certain applications the diffusion of particles in the energy variable,
although small, can be an important effect. To see any such diffusive
behavior, the $R$ term in the Fokker-Planck operator mus be retained. We also
note that the $T$ term in equation (\ref{final}) decribes diffusion in angle ,
and ofr aesthetic reasons it seems reasonable for the Fokker-Planck operator
to descirbe diffusion in both energy and angle. The reason that no convective
term in angle appears in equation (\ref{final}) is that we have restricted our
attention to an isotropic background medium. Upon scattering in such a medium,
it is equally likely for a particle to be scattered to the left or to the
right, and thus the mean scattering angle is zero. The lowest order nonzero
measure of the scattering in such a medium, it is equally likely for a
particle to be scattered to the left or to the right, and thus the mean
scattering angle is zero.

\begin{itemize}
\item montrer le foirage sur Henyey-Greintein et que screened rutherford works
\end{itemize}
\section{Fokker-Planck cross section}
\section{Galerkin}
