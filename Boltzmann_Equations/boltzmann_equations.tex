\chapter{Transport equation}

\section{Derivation of the Boltzmann-Fokker-Planck equation}
\subsection{Ligou's derivation}
The Boltzmann equation cannot be used to describe charged particles transport.
The Boltzmann equation is valid from the theoretical point of view but the
scattering cross sections are so forward peaked that a polynomial expansion
would require too many terms (slow convergence of the expansion). To avoid the
cross section expansion, the Boltzmann equation can be replaced by a
Fokker-Planck equation. The Fokker-PlancK equation can be used to represent
the highly forward peaked scattering but it cannot represent the large angles
scattering. The Boltzmann-Fokker-Planck (BFP) equation uses a polynomial expansion
for the large angle scattering and a Fokker-Planck term for the forward peaked
scattering. Following \cite{ligou}, the BFP equation can be derived starting
from the Boltzmann equation\footnote{\red{L'equation doit etre definie avant,
bien faire attention a la definition de harmoniques spheriques->ca change les
coefficients dans l'equation de transport. On prend les harmoniques
spheriques orthonormees.Il faut verifier que tout a bien ete defini avant.}}:
\begin{equation}
\begin{split}
&\bo\cdot\bn\Psi(\br,\bo,E) + \(\Sigma_s(\br,E)+\Sigma_a(\br,E)\)\Psi(\br,bo,E)
=\\ 
&\sum_{l=0}^{\infty}\sum_{m=-l}^l \int_{0}^{\infty} \Sigma_{s,l}\(\br,E'\rightarrow
E\) \Phi_{l}^m(\br,E')Y_l^m(\bo)\ dE' + Q(\br,\bo,E)
\label{b_2}
\end{split}
\end{equation}
The scattering angle cosines in the center-of-mass ($\mu_c$) or in the
laboratory system ($\mu_L$) are given by:
\begin{align}
\mu_c &= 1-\frac{2}{\beta}\frac{E'-E}{E'} \label{mu_c}\\
\mu_L = \frac{1}{2}\[(A+1)\sqrt{\frac{E}{E'}}-(A-1)\sqrt{E'}{E}\]
\label{mu_l}
\end{align}
where:
\begin{align}
\beta &= 1 - \alpha\\
\alpha &= \(\frac{A-1}{A+1}\)^2
\end{align}
A being the particle mass ratio. When the microscopic scattering cross section
$(\sigma_s(\br,\mu_c,E'))$ are known, the Legendre expansion of the scattering
cross section is given by:
\begin{equation}
\Sigma_{s,l}(\br,E'\rightarrow E) = N \frac{4\pi}{\beta E'}
\sigma_s\[E',\mu_c\(\frac{E'-E}{E'}\)\]P_l\[\mu_L\(\frac{E'}{E}\)\]
\label{eq7}
\end{equation}
for $E\leq E' \leq \frac{E}{\alpha}$

Now the differential scattering cross section is split into two parts:
\begin{equation}
\Sigma_s(\br,\mu_c) = \Sigma_{s,reg}(\br,\mu_c + \Sigma_{s,sing}(\br,\mu_c)
\end{equation}
The first term is a ``regular'' cross section which does not increase too much
when $\mu$ goes to one. By definition, a expansion on the Legendre polynomials
of this term will converge quickly. The second term is ``singular'' cross section 
which is highly forward peaked. This term is only important when $\mu\approx
1$. The scattering term in \cref{b_2} can be written as:
\begin{equation}
q = q'+q''
\end{equation}
with\footnote{this shot uses orthonormal spherical harmonics}:
\begin{equation}
\begin{bmatrix}
q'(\bo,E)\\
q''(\bo,E)
\end{bmatrix}
= \sum_{l=0}^{\infty} \frac{2l+1}{4\pi}\sum_{m=-l}^l \frac{(l-|m|)!}{(l+|m|)!}
Y_l^m(\bo)
\begin{bmatrix}
q_{l,m}'(E)\\
q_{l,m}'(E)
\end{bmatrix}
\end{equation}
and:
\begin{equation}
\begin{bmatrix}
q_{l,m}'(E)\\
Q_{l,m}''(E)
\end{bmatrix}
=\int_{E}^{E/\alpha} dE' \phi_{l,m}(E')
\begin{bmatrix}
\Sigma_{s,l}'(E'\rightarrow E)\\
\Sigma_{s,l}''(E' \rightarrow E)
\end{bmatrix}
\end{equation}
where $\Sigma_{s,l}'$ and $\Sigma_{s,l}''$ are the Legendre kernels
corresponding to $\sigma_s'$ and $\sigma_s''$ through \cref{eq7}.
In the preceding equation, it is more convenient to use $\mu_c$ instead of
$E'$ as a new integration variable:
\begin{equation}
q_{l,m}''(E) = 2\pi N\int_{-1}^1\frac{E'}{E}\sigma_s''(E',\mu_c)
P_l\[\mu_L(\mu_c)\] \phi_{l,m}(E')d\mu_c
\end{equation}
with:
\begin{equation}
\mu_L = \frac{1+A\mu_c}{(1+A^2+2A\mu_c)^{1/2}}
\end{equation}
Assuming that $\phi_{l,m}(E')$ is a smooth function, one can perform the
following Taylor expansions:
\begin{align}
&E'\sigma_s''(E',\mu_c)\phi_{l,m} (E') = E\sigma_s''(E,\mu_c)\phi_{l,m}(E) +
(E'-E)\frac{\partial}{\partial
E}\(E\sigma_s''(E,\mu_c)\phi_{l,m}(E)\)+\hdots\\
& P_l(\mu_L) = P_l(1)-(1-\mu_L)P_l'(1)+\hdots
\end{align}
with $P_l(1)=1$ and $P_l'(1)=l\frac{l+1}{2}$.
Moreover, using \cref{mu_c,mu_l}) and $\mu_c\approx
1$ or $A\ll 1$, we get:
\begin{align}
& \frac{E'-E}{E'} = \frac{2A}{(A+1)^2} (1-\mu_c)\\
& 1-\mu_L \approx \(\frac{A}{A+1}\)^2 (1-\mu_c)
\end{align}
then to a first order in $(1-\mu_c)$ one obtains:
\begin{equation}
q_{l,m}''(E) = \Sigma_s''(E) \phi_{l,m}(E) +\frac{\partial}{\partial E} S''(E)
\phi_{l,m}(E) - l(l+1)T''(E) \phi_{l,m}(E)
\end{equation}
where:
\begin{align}
&\Sigma_s''(E) N 2 \pi \int_{-1}^{1}\sigma_s''(E,\mu_c)\ d\mu_c\\
&T''(E) = \frac{N}{2} 2 \pi \int_{-1}^1 (1-\mu_L)\sigma_s''(E,\mu_c)\ d\mu_c\\
&S''(E) = \frac{4E}{A}T''(E)
\end{align}
Using:
\begin{align}
& \phi(E,\bo) = \sum_{l=0}^{\infty} \sum_{m=-l}^l \phi_{l,m}Y_l^m(\bo)\\
& \frac{\partial}{\partial \mu}(1-\mu^2)\frac{\partial Y_l^m}{\partial \mu} + 
\frac{1}{1-\mu^2} \frac{\partial^2 Y_l^m}{\partial \varphi^2} = -l(l+1)Y_l^m
\end{align}
We get:
\begin{equation}
q''(E,\bo) = \Sigma_s''(E)\phi(E,\bo)+\frac{\partial}{\partial
E}S''(E)\phi(E,\bo) + T''(E)\[\frac{\partial}{\partial \mu} (1-\mu^2)
\frac{\partial}{\partial \mu}+\frac{1}{1-\mu^2}\frac{\partial^2}{\partial
\varphi^2}\] \phi(E,\bo)
\end{equation}
Finally, we get:
\begin{equation}
\begin{split}
&\bo\cdot\Psi+(\Sigma_s'+\Sigma_s)\Psi=\sum_{l=0}^{\infty}\sum_{m=-l}^l
Y_l^m(\bo)\int_E^{E/\alpha}\Sigma_{s,l}(E'\rightarrow
E)\phi_{l,m}(E')dE'+\\
&\frac{\partial}{\partial E}S''\phi+T''\[\frac{\partial}{\partial
\mu}(1-\mu^2) \frac{\partial}{\partial
\mu}+\frac{1}{(1-\mu^2)}\frac{\partial^2}{\partial\varphi^2}\]\phi+Q
\end{split}
\end{equation}

\subsection{Pomraning's derivation: Fokker-Planck equation as a limit of
Boltzmann equation}
\begin{itemize}
\item $\mu_0 = \bo'\cdot \bo$ is the cosine of the polar angle
\item $\bo$ is a unit vector in the flight direction
\item $\mu = \cos(\theta)$, where $\theta$ is a polar angel
\end{itemize}
\subsubsection{Transport equation}
In charged particle transport, the scattering kernel is very peaked about both
a zero energy transfer and a zero direction change. However, the number of
scattering collisions is very large. The average distance a particle travels
between scattering events (the scattering mean free path) is very small. The
slight effect of a single scattering can loosely be characterized as an
$O(\epsilon \ll 1)$ effect, and the small scattering mean free path can be
characterized as an $O(\Delta \ll 1)$ distance. Thus, in an $O(1)$ distance in
the medium, one can possibly observe a significant change in the energy and
direction of the particle, depending upon the relationship between $\epsilon$
and $\Delta$\cite{pomraning}.

To solve such a scattering transport problem numerically using deterministic
methods is very difficult since the mesh size in such a calculation must be on
the same scale as the mean free path, which in this case is very small. This
implies an unrealistically fine degree of numerical resolution. Likewise, a
Monte-Carlo simulation is very time consuming since a very large number of
scattering interactions must be followed for each particle before its demise
by either absorption or leakage out of the system. To circumvent these
difficulties, it has been suggested to replace the integral scattering
operator in the transport equation with a differential Fokker-Planck operator.
The effect of this replacement is that the dominant (large) in and out
scattering terms cancel, thus effectively increasing the mean free
path\cite{pomraning}.

The transport equation is given by:
\begin{equation}
\bo\cdot \bn \Psi(E,\bo) + \Sigma_t(E)\Psi(E,\bo) =
\int_0^{\infty}dE'\int_{4\pi}
\Sigma_s(E',E,\bo'\cdot\bo)\Psi(E',\bo')+Q(E,\bo)
\label{transport_p}
\end{equation}
Here $\Psi(E,\bo)$ is defined by:
\begin{equation}
\Psi(E,\bo) = vf(E,\bo)
\end{equation}
where $v$ is the particle speed, $\Sigma_t(E)$ is the total cross-section
given by:
\begin{equation}
\Sigma_t(E) = \Sigma_a(E)+\Sigma_s(E)
\end{equation}
and $Q(E,\bo)$ represents any source of particles. We are not concern about
boundary conditions.

The in-scattering term can be represented as a sum over its a surface harmonic
components. To this end, we expand the scattering kernel in Legendre
polynomials according to:
\begin{equation}
\Sigma_s(E',E,\bo'\cdot\bo)=\sum_{l=0}^{\infty}
\Sigma_{s,l}(E',E)P_l(\bo'\cdot\bo)
\end{equation}
The orthogonality of the Legendre polynomials allows us to write the expansion
coefficients as:
\begin{equation}
\Sigma_{s,l}(E',E) = 2\pi\int_{-1}^1 d\mu_0 P_l(\mu_0) \Sigma_s(E',E,\mu_0)
\end{equation}
Further, we expand the solution $\Psi(E,\bo)$ in surface harmonics according
to:
\begin{equation}
\Psi(E,\bo) = \sum_{l=0}^{\infty}\sum_{m=-l}^l \phi_{l,m} Y_l^m(\bo)
\end{equation}
The expansion coefficients $\phi_{l,m}$ are given by:
\begin{equation}
\phi_{l,m}(E) = \int_{4\pi} d\bo Y_{l}^{m,*} \Psi(E,\bo)
\label{moments}
\end{equation}
A property of the surface harmonics is that they satisfy the partial equation
given by:
\begin{equation}
\[\frac{\partial}{\partial\mu}(1-\mu^2)\frac{\partial}{\partial
\mu}+\(\frac{1}{1-\mu^2}\)\frac{\partial^2}{\partial \varphi}+l(l+1)\]Y_l^m(\bo)=0
\label{eigenvalue}
\end{equation}
Using the addition theorem, we get:
\begin{equation}
\bo\cdot\bn \Psi(E,\bo) +
\Sigma_t(E)\Psi(E,\bo)=\sum_{l=0}^{\infty}\sum_{m=-l}^{l} Y_l^m(\bo) 
\int_0^{\infty} dE' \Sigma_{s,l}(E',E)\Psi_{l,m}(E') + Q(E,\bo)
\end{equation}
or equivalently:
\begin{equation}
\begin{split}
&\bo\cdot\bn \Psi(E,\bo) + \(\Sigma_a(E)+\Sigma_s(E)\)\Psi(E,\bo) =\\
\sum_{l=0}^{\infty}\sum_{m=-l}^l Y_l^m(\bo) \int_0^{\infty}dE'\ \Psi_{l,m}(E')
\int_{-1}^1 d\mu_0\ P_l(\mu_0) \Sigma_{s}(E',E,\mu_0)+Q(E,\bo)
\end{split}
\end{equation}

\subsubsection{Fokker-Planck development}
We begin by assuming that the unit of distance is chosen such that the
characteristic size of the system in which the particle transport is $O(1)$.
In such a length unit, we take the scattering mean free path to be small, and
since this mean free path is the reciprocal of the scattering cross-section,
we have $\Sigma_s(E) \gg 1$. Accordingly, we scale $\Sigma_s(E)$ as:
\begin{equation}
\Sigma_s(E) = \frac{\hat{\Sigma}_s(E)}{\Delta}
\label{sigma_s}
\end{equation}
where $\hat{\Sigma}_s=O(1)$ and $\Delta \ll 1$. We apply this same scaling to
the scattering kernel $\Sigma_s(E',E,\mu_0)$, and additionally introduce the
fast variables:
\begin{align}
x=\frac{E'-E}{\epsilon}, & \epsilon \ll 1 \label{x}\\
y=\frac{1-\mu_0}{\delta}, & \delta \ll 1 \label{y}
\end{align}
Thus, we scale the scattering kernel as:
\begin{equation}
\begin{split}
\Sigma_s(E',E,\mu_0) &= \frac{1}{\Delta}
\hat{\Sigma}_s\(E',\frac{E'-E}{\epsilon},\frac{1-\mu_0}{\delta}\)\\
&=\frac{1}{\Delta}\hat{\Sigma}_s(E',x,y)
\end{split}
\end{equation}
where $\hat{\Sigma}_s(E',x,y)$ is $O(1)$ and the derivatives of this kernel
with respect to both $x$ and $y$ are assumed to be $O(1)$ as
$\epsilon,\delta\rightarrow 0$. Physically, the smallness parameter $\delta$
is a measure of the peaking of the scattering kernel in angle, and can be
roughly thought of as the deviation from unity of the cosine of a
characteristic scattering angle. Likewise, the smallness parameter $\epsilon$
is a measure of the peaking of the scattering kernel in energy, and can be
thought of as a characteristic value of the fractional energy change due to a
single scattering. That is, these scalings imply that the scattering
cross-section is large, and that the scattering kernel is very peaked about
$\mu_0=1$ and $E=E'$. Inserting the scalings given, we then have the scaled
transport equation \red{(verifier si il ne manque pas un coefficitent (2l+1)/2
ou (2l+1)/4pi)}:
\begin{equation}
\begin{split}
&\bo\cdot\bn\Psi(E,\bo)+\(\Sigma_a(E)+\frac{\hat{\Sigma}_s}{\Delta}\)\Psi(E,\bo)=\\
&\frac{1}{\Delta}\sum_{l=0}^{\infty}\sum_{m=-l}^{l}Y_l^m(\bo)
\int_0^{\infty}dE'\ \phi_{l,m}(E')\int_{-1}^1d\mu_0\
P_l(\mu_0)\hat{\Sigma}_s\(E',\frac{E'-E}{\epsilon},\frac{1-\mu_0}{\delta}\)+Q(E,\bo)
\end{split}
\label{scaled_transport}
\end{equation}
We see the asymptotic limit as the three smallness parameters
$\epsilon,\delta$ and $\Delta$ approach zero.\\
To this end, we consider the term $K$, defined to be:
\begin{equation}
K = \frac{2\pi}{\Delta} \int_0^{\infty}dE'\int_{-1}^1 d\mu_0 P_n(\mu_0)
\hat{\Sigma}_s(E',\frac{E'-E}{\epsilon},\frac{1-\mu_0}{\delta}) \phi_{l,m}(E')
\end{equation}
We change integration variables from $E',\mu_0$ to $x,y$ according to \cref{x,y}) 
to obtain:
\begin{equation}
K = \frac{2\pi\epsilon\delta}{\Delta}\int_{-E/\epsilon}^{2/\delta}dy\
P_l(1-\delta y)\hat{\Sigma}_s(E+\epsilon x,x,y)\phi_{l,m}(E+\epsilon x)
\label{K_def}
\end{equation}
We now Taylor expand the integrand about $\epsilon=\delta=0$. We carry linear
terms in $\delta$ and quadratic terms in $\epsilon$. We will discuss the
reason for this asymmetry in the two expansions at the end of out development.
We then find, indicating the errors in the neglected terms in the Taylor
expansions,
\begin{equation}
\begin{split}
K=& \frac{2\pi \epsilon \delta}{\Delta}\int_{-E/\epsilon}^{\infty}dx\
\int_0^{2/\delta}dy\ \[P_l(1)-\delta y P_l'(1)+O(\delta^2)\]\\
&\[1+\epsilon x \frac{\partial}{\partial
E}+\frac{\epsilon^2x^2}{2}\frac{\partial^2}{\partial E^2}+O(\epsilon^3)\]
\hat{\Sigma}_s(E,x,y)\Phi_{l,m}(E)
\end{split}
\label{error}
\end{equation}
To proceed, we assume that we can replace the lower limit of integration over
$E'$ by $-\infty$, and that the error we make in this doing this is
$O(\epsilon^3)$ or smaller. This is certainly legitimate if the scattering
kernel falls off exponentially in energy from its peak at $x=0$. However, if
the kernels falls off algebraically at too weak a rate, this replacement may
increase the error in the Fokker-Planck treatment over the $O(\epsilon^3)$
error indicated in \cref{error}. We note that even if the kernel
falls off exponentially in energy from its peak, this integration limit
replacement does introduce an error, but this error is exponentially small. It
is this exponential error that makes the present development an asymptotic,
rather than a convergent, procedure. At this point, we also neglect the cross
terms angle and energy in \cref{error}. It is not necessary to
neglect these terms to obtain a clean result, but the standard Fokker-Planck
operator does not have the contributions that arise from these terms.

In view of the above discussion, we then rewrite \cref{error},
using:
\begin{align}
P_l(1) &= 1\\
P_l'(1) &= \frac{l(l+1)}{2}
\end{align}
as:
\begin{equation}
\begin{split}
K=& \frac{2\pi \epsilon \delta}{\Delta}
\int_{-\infty}^{\infty}dx\int_0^{2/\delta}
dy\(1+O(\delta^2+\epsilon+\epsilon^3)\)\hat{\Sigma}_s(E,x,y)\phi_{l,m}(E)\\
&-\frac{l(l+1)\pi\epsilon \delta^2}{\Delta} \int_{-\infty}^{\infty} dx
\int_0^{2/\delta}dy\ y \hat{\Sigma}_s(E,x,y)\phi_{l,m}(E)\\
&+\frac{2\pi \epsilon^2 \delta}{\Delta}\frac{\partial}{\partial E}
\int_{-\infty}^{\infty} dx \int_0^{2/\delta} dy\
x\hat{\Sigma}_s(E,x,y)\phi_{l,m}(E)\\
&+\frac{\pi\epsilon^3\delta}{\Delta}\frac{\partial^2}{\partial E^2}
\int_{-\infty}^{\infty} dx\int_0^{2/\delta}dy\ x^2\hat{\Sigma}_s(E,x,y)
\phi_{l,m}(E)
\end{split}
\label{extended}
\end{equation}
We now change integration variables in all four double integrals in 
\cref{extended} from $x,y$ to $E',\mu_0$ according to:
\begin{align}
x&= \frac{E-E'}{\epsilon}\label{x2}\\
y&= \frac{1-\mu_0}{\delta}\label{y2}
\end{align}
We note that \cref{y2} is identical to \cref{y}, but
equation \cref{x2} is not identical to \cref{x} (the $E$ and $E'$
are interchanged). This gives:
\begin{equation}
\begin{split}
K=& \frac{2\pi}{\Delta} \int_{-\infty}^{\infty} dE'
\int_{-1}^{1}d\mu_0\(1+O(\delta^2+\epsilon\delta+\epsilon^3)\)\hat{\Sigma}_s 
\(E,\frac{E-E'}{\epsilon},\frac{1-\mu_0}{\delta}\) \Phi_{l,m}(E)\\
&- \frac{l(l+1)\pi}{\Delta} \int_{-\infty}^{infty}dE'\int_{-1}^1d\mu_0
(1-\mu_0)
\hat{\Sigma}_s \(E,\frac{E-E'}{\epsilon},\frac{1-\mu_0}{\delta}\)\phi_{l,m}(E)\\
&+\frac{2\pi}{\Delta}\frac{\partial}{\partial E} \int_{-\infty}^{\infty} dE'
\int_{-1}^{1} d\mu_0 \(E-E'\)\hat{\Sigma}_s
\(E,\frac{E-E'}{\epsilon},\frac{1-\mu_0}{\delta}\) \phi_{l,m}(E)\\
&+ \frac{\pi}{\Delta}\frac{\partial^2}{\partial E^2} \int_{-\infty}^{\infty}
dE' \int_{-1}^1 d\mu_0\ (E-E')^2
\(E,\frac{E-E'}{\epsilon},\frac{1-\mu_0}{\delta}\) \phi_{l,m}(E)
\end{split}
\label{K}
\end{equation}
We use:
\begin{equation}
\frac{1}{\Delta}\hat{\Sigma}_s \(E,\frac{E-E'}{\epsilon},\frac{1-\mu_0}{\delta}\)
= \Sigma_s(E,E',\mu_0)
\end{equation}
in \cref{K}, and note that the bottom limit of integration on the
$E'$ integral can be replaced by zero since the scattering kernel vanishes for
negative $E'$ (the probability of scattering to a negative energy is zero). We
then have:
\begin{equation}
\begin{split}
K=&\Sigma_{s}(E)\phi_{l,m}(E)-l(l+1)T(E)
\phi_{l,m}(E)+\frac{\partial}{\partial E}S(E)\phi_{l,m}\\
&\frac{\partial^2}{\partial
E^2}R(E)\phi_{l,m}(E)+O\(\frac{\delta^2+\epsilon\delta+\epsilon^3}{\Delta}\)
\end{split}
\label{K2}
\end{equation}
where we have defined:
\begin{align}
&T(E) = \pi \int_0^{\infty} dE' \int_{-1}^1d\mu_0\
(1-\mu_0)\Sigma_s(E,E',\mu_0)=
O\(\frac{\delta}{\Delta}\) \label{T}\\
&S(E) = 2\pi \int_0^{\infty}DE' \int_{-1}^1 d\mu_0\ (E-E')
\Sigma_s(E,E',\mu_0) = O\(\frac{\epsilon}{\Delta}\) \label{S}\\
&R(E) = 2 \pi \int_0^{\infty}DE'\int_{-1}^1 d\mu_0 (E-E')^2 \Sigma_s
(E,E',\mu_0) = O\(\frac{\epsilon^2}{\Delta}\) \label{R}
\end{align}
The order assigned to the terms in \crefrange{T}{R}
seemingly follows from the fact that since $x$ and $y$ are $O(1)$ variables,
then according to \cref{x2,y2}) $(1-\mu_0)$ is
$O(\delta)$, $(E-E')$ is $O(\epsilon)$ , and $(E-E')^2$ is $O(\epsilon^2)$,
with $\Sigma_s(E,E',\mu_0)$ being $O\(\frac{1}{\Delta}\)$. However, these
ordering are integrand, the assumption we have made in writing \cref{T,R} is 
that the integrations do not change the
ordering. The same holds true for taking the
$O(\delta^2+\epsilon\delta+\epsilon^3)$ out from under the integrals in going
from \crefrange{K}{K2}. It is not obvious that the
integrations do not change the ordering since $\Sigma_s(E,E',\mu_0)$ contains
the smallness parameters $\epsilon$ and $\delta$. The situation again depends
upon the rate fall off of the scattering kernel from its peaks in energy and
angle. For exponential fall off, \crefrange{K2}{R} are
corrected ordered, but for algebraic fall off the order of one or more of
these terms may be larger than indicated. This observation again places a
restriction on the scattering kernel for the  Fokker-Planck differential
operator to be an asymptotic limit of the exact integral operator.

With the above caution in mind concerning the validity of the expression for
$K$, we proceed to complete our development. Recalling the definition of $K$
according to \cref{K_def}, substitution of \cref{K2} into
the scaled transport \cref{scaled_transport}, yields, also making
use of \cref{sigma_s} for $\hat{\Sigma}_s(E)$:
\begin{equation}
\begin{split}
&\bo\cdot \bn\Psi(E,\bo) + \(\Sigma_a(E)+\Sigma_s(E)\) \Psi(E,\bo)=\\
&\sum_{l=0}^{\infty}\sum_{m=-l}^lY_l^m(\bo)\(\Sigma_s(E) \phi_{l,m}(E) -
l(l+1) T(E) \phi_{l,m}(E)+\right.\\
&\left. \frac{\partial}{\partial
E}S(E)\phi_{l,m}(E)+\frac{\partial^2}{\partial E^2}R(E)\phi_{l,m}(E)\)
+Q(E,\bo)+O\(\frac{\delta^2+\epsilon\delta+\epsilon^3}{\Delta}\)
\end{split}
\label{final_m1}
\end{equation}
The $l(l+1)$ factor in \cref{final_m1} can be eliminated by making
use of the equation satisfied by the $Y_l^m(\bo)$, namely \cref{eigenvalue}, 
and then we can sum over the surface harmonics according
to \cref{moments} to obtain our final result. This is given by:
\begin{equation}
\begin{split}
&\bo\cdot \bn \Psi(E,\bo) + \Sigma_a(E) \Psi(E,\bo) =\\
&T(E) \(\frac{\partial}{\partial \mu}(1-\mu^2)\frac{\partial}{\partial\mu}+
\(\frac{1}{1-\mu^2}\)\frac{\partial^2}{\partial \varphi^2}\)\Psi(E,\bo)\\
&+\frac{\partial}{\partial E} S(E)\phi(E,\bo) + \frac{\partial^2}{\partial
E^2} R(E) \Psi(E,\bo)\\
& Q(E,\bo)+O\(\frac{\delta^2+\epsilon\delta+\epsilon^3}{\Delta}\)
\end{split}
\label{final}
\end{equation}
We note that the dominant scattering term, $\Sigma_s(E) \Psi(E,\bo)$, has
canceled out in this equation.

\Cref{final} is the conventional Fokker-Planck equation for linear
particle transport in an isotropic medium. As we noted during the course of
its derivation, a necessary but not sufficient condition for \cref{final} to be 
an asymptotic limit of \cref{transport_p} is that the scattering the kernel be 
peaked in both energy and angle. The additional sufficient condition is that 
this peaking be either exponential or strongly algebraic. Assuming that the 
scattering kernel is such that \cref{final} is a valid asymptotic limit, it 
is clear, from the order of the $T$, $S$ and $R$ terms as given by
\crefrange{T}{R},that the smallness parameters $\epsilon$, $\delta$ and $\Delta$ 
must tend to zero in a correlated way. That is, we must have $O(\delta)=O(\Delta)$ 
in order to obtain a meaningful (finite and nonzero) angular term in 
\cref{final}, and we must and we must have $O(\epsilon)=O(\Delta)$ to obtain
a meaningful energy term in \cref{final}. The physical meaning of
thos is that as the peaking in the scattering kernel increases
$(\epsilon,\delta \rightarrow 0)$, the magnitude of the scattering
cross-section must increase $(\Delta \rightarrow 0)$ in such a corresponding
way that the momentum transfer remains bounded and nonzero. Only in this case
does \cref{final} have any meaning.

Finally, we note that since $R=O\(\frac{\delta}{\Delta}\)$, one could in an
asymptotically consistent manner always set $R$ to zero and still maintain the
leading order behavior in energy transfer. In applications, this is often
done, but not always. The reason for retaining $R$, even though it is a higher
order term, ist that the $R$ term in \cref{final} describes entirely
different physics than doe the $S$ term. The $S$ term, the so-called stopping
power term, is convective in nature, whereas the $R$ term is diffusive. In
certain applications the diffusion of particles in the energy variable,
although small, can be an important effect. To see any such diffusive
behavior, the $R$ term in the Fokker-Planck operator mus be retained. We also
note that the $T$ term in \cref{final} decribes diffusion in angle ,
and for aesthetic reasons it seems reasonable for the Fokker-Planck operator
to describe diffusion in both energy and angle. The reason that no convective
term in angle appears in \cref{final} is that we have restricted our
attention to an isotropic background medium. Upon scattering in such a medium,
it is equally likely for a particle to be scattered to the left or to the
right, and thus the mean scattering angle is zero. The lowest order nonzero
measure of the scattering in such a medium, it is equally likely for a
particle to be scattered to the left or to the right, and thus the mean
scattering angle is zero.

\subsection{Limit of the Boltzmann-Fokker-Planck approximation.}
\subsubsection{Henyey-Greenstein scattering}
The Henyey-Greenstein\cite{H-G} differential scattering kernel is defined by
\cite{larsen_fp}:
\begin{equation}
f(\mu_0) = \frac{1-\bar{\mu}_0^2}{4\pi(1-2\bar{\mu}_0\mu +\bar{\mu}_0^2)^{3/2}}
\label{H-G}
\end{equation}
where the mean scattering cosine $\mu_0$ ranges from $-1<\bar{\mu}_0<1$. An
important feature of this kernel is that its Legendre polynomial expansion,
$f(\mu_0)=\sum_{l=0}^{\infty}\frac{2l+1}{4\pi}f_l P_l(\mu_0)$, is very simple; 
the expansion coefficients $f_n$ are given by:
\begin{equation}
f_l = \bar{\mu}_0^l,\ n\geq 0
\end{equation}
We now consider the operator:
\begin{equation}
\Sigma_s\mathcal{L}_B = \Sigma_s\(\int_{4\pi}
f(\bo\cdot\bo')\Psi(\br,\bo',E)d\bo'-\Psi(\br,\bo,E)\)
\end{equation}
using the Henyey-Greenstein kernel, \cref{H-G}, with:
\begin{align}
&\Sigma_s = \frac{\sigma}{\epsilon}\\
&\mu_0 = 1-\epsilon
\end{align}
where $\sigma$ is fixed and $\epsilon \approx 0$. Then, as desired, the
scattering mean free path tends to zero, the mean scattering cosine tends to
unity, and in the absence of absorption, the transport cross section:
\begin{equation}
\Sigma_{tr}=\Sigma_s(1-\bar{\mu}_0) = \frac{sigma}{\epsilon}\cdot \epsilon =
\epsilon
\end{equation}
is fixed.

In the limit of small $\epsilon$, the eigenvalues of $\Sigma_s\mathcal{L}_s$
become:
\begin{equation}
\begin{split}
-\Sigma_s(1-f_l) &= -\frac{\sigma}{\epsilon}\(1-(1-\epsilon)^l\)\\
&=-\frac{\sigma}{\epsilon}\(l\epsilon-\frac{l(l-1)}{2}\epsilon^2+O(\epsilon^3)\)\\
&= \sigma(1+\epsilon)(-l)-\frac{\sigma \epsilon}{2}(-l(l+1))+O(\epsilon^2)
\end{split}
\label{eps_eig}
\end{equation}
Using \cref{L_B_smooth,L_FP_smooth,L_3/2_smooth,eps_eig}, we get:
\begin{equation}
\Sigma_s\mc{L}_B = \sigma(1+\epsilon) \mc{L}_{3/2} - \frac{\sigma
\epsilon}{2} \mc{L}_{FP} + O(\epsilon^2)
\label{HG_approx}
\end{equation}
Using \cref{HG_approx}, we find that the leading-order approximation
to $\Sigma_s \mc{L}_B$ is defined entirely by $\mc{L}_{3/2}$:
\begin{equation}
\Sigma_s \mc{L}_B = \Sigma_s(1-\bar{\mu}_0) \mc{L}_{3/2} +O(\epsilon)
\end{equation}
The first-order approximation to $\mc{L}_B$ contains both $\mc{L}_{3/2}$ and
$\mc{L}_{FP}$:
\begin{equation}
\Sigma_s\mc{L}_B = \Sigma_s(1-\bar{\mu}_0)(2-\bar{\mu}_0) \mc{L}_{3/2} -
\frac{\Sigma_s}{2}(1-\bar{\mu}_0)^2\mc{L}_{FP} + O(\epsilon^2)
\end{equation}
The Henyey-Greenstein kernel was proposed\cite{H-G}, and has found wide use,
primarily because of its great simplicity. The exact Henyey-Greenstein kernel
does not arise from a deeper theory as the differential scattering kernel for
any known physical process \cite{larsen_fp}. However, as a model for certain
transport processes, it has been successful.


\red{A mettre quelque part}
\begin{equation}
\mathcal{L}_{FP}\psi(\bo) = \(\frac{\partial}{\partial
\mu}(1-\mu^2)\frac{\partial}{\partial
\mu}+\frac{1}{1-\mu^2}\frac{\partial^2}{\partial\varphi^2}\)\psi(\bo)
\end{equation}
\begin{align}
&\mathcal{L}_B Y_l^m(\bo) = -(1-f_l) Y_l^m(\bo)\\
&\mathcal{L}_{FP} Y_l^m(\bo) = -l(l+1)Y_l^m(\bo)
\end{align}
For a sufficiently smooth function $\Psi(\bo)$, $\mathcal{L}_B$ may be
written:
\begin{equation}
\begin{split}
\mathcal{L}_B\Psi(\bo) &= \int_{4\pi}\(\sum_{l=0}^{\infty} \frac{2l+1}{4\pi}
f_l P_l(\bo\cdot\bo')\)\Psi(\bo')d\bo'-\Psi(\bo)\\
&=\sum_{l=0}^{\infty} f_l\int_{4\pi} \(\sum_{m=l}^l Y_l^m(\bo)
Y_l^{m,*}(\bo')\) \Psi(\bo')dbo' - \sum_{l=0}^{\infty}\sum_{m=-l}^l Y_l^m(\bo)
\int_{4\pi}Y_l^{m,*}(\bo') \Psi(\bo') d\bo'\\
&= \sum_{l=0}^{\infty} \sum_{m=-l}^l \(-\(1-f_l\)\)Y_l^m(\bo)\int_{4\pi}
Y_l^{m,*} (\bo')\Psi(\bo') d\bo'
\end{split}
\label{L_B_smooth}
\end{equation}
If $\Psi(\bo)$ is sufficiently smooth, $\mc{L}_{FP}\Psi(\bo)$ may be
written:
\begin{equation}
\mathcal{L}_{FP}\Psi(\bo) = \sum_{l=0}^{\infty} \sum_{m=-l}^l (-l(l+1))
Y_l^m(\bo)\int_{4\pi} Y_l^{m,*} \int_{4\pi} Y_l^{m,*}(\bo')\Psi(\bo') d\bo'
\label{L_FP_smooth}
\end{equation}
Now we need to introduce the operator:
\begin{equation}
\mc{L}_{3/2}\Psi(\bo) = \frac{1}{4\pi\sqrt{2}} \int_{4\pi}
\frac{\Psi(\bo')-\Psi(\bo)}{\(1-\bo\cdot\bo'\)^{3/2}} d\bo'
\label{L_3/2}
\end{equation}
For this integral to exist, it must be interpreted as a principal value.
$\mathcal{L}_{3/2}$ satisfies:
\begin{equation}
\mathcal{L}_{3/2}Y_l^m(\bo) = -lY_l^m(\bo)
\end{equation}
Thus, if $\Psi(\bo)$ is sufficiently smooth, $\mc{L}_{3/2}\Psi(\bo)$ may be
written:
\begin{equation}
\mathcal{L}_{3/2}\Psi(\bo) = \sum_{l=0}^{\infty} \sum_{m=-l}^l -l Y_l^m(\bo)
\int_{4\pi} Y_l^{m,*} (\bo) \int_{4\pi} \int_{4\pi} Y_l^m(\bo') \Psi(\bo')
d\bo'
\label{L_3/2_smooth}
\end{equation}
A simple relationship between $\mathcal{L}_{3/2}$ and $\mathcal{L}_{FP}$ is:
\begin{equation}
\mathcal{L}_{FP} = -\mathcal{L}_{3/2}^2+\mathcal{L}_{3/2}
\end{equation}

\begin{equation}
\begin{split}
\mc{L}_{3/2}\Psi(\bo) &= \frac{1}{\sigma} \lim_{\epsilon \to 0} \Sigma_s
\mc{L}_B \Psi(\bo)\\
&= \lim_{\epsilon \to 0} \frac{\Sigma_S}{\sigma}
\(\frac{1-\bar{\mu}_0^2}{4\pi} \int_{4\pi} \frac{\Psi(\bo')}{(1-2\bar{\mu}_0
\bo\cdot\bo' +\bar{\mu}_0^2)^{3/2}} - \Psi(\bo)\)\\
&=\lim_{\epsilon \to 0} \frac{\Sigma_s}{\sigma} \frac{1-\bar{\mu}_0^2}{4\pi}
\int_{4\pi}\frac{\Psi(\bo')-\Psi(\bo)}{(1-2\bar{\mu}_0 \bo\cdot \bo' +
\bar{\mu}_0^2)^{3/2}}d\bo'\\
&= \lim_{\epsilon \to 0} \frac{2-\epsilon}{4\pi} \int_{4\pi} \frac{\Psi(\bo')
- \Psi(\bo)}{(1-2(1-\epsilon) \bo\cdot\bo'+(1-\epsilon)^2)^{3/2}} d\bo'\\
& = \frac{1}{4\pi\sqrt{2}} \int_{4\pi}
\frac{\Psi(\bo')-\Psi(\bo)}{(1-\bo\cdot\bo')^{3/2}}d\bo'
\end{split}
\end{equation}
For this last integral to exist, it must be interpreted as a principal value:
one takes the limit as $\delta\rightarrow 0$ of the integral over the unit
sphere excluding the small disk $1-\bo'\cdot\bo <\delta$.\\
The operator $\mc{L}_{3/2}$ is an unbounded pseudodifferential operator (its
eigenvalues tend to $-\infty$). If it were possible to split the integral into
two integrals, the first would represent a standard Boltzmann
``out-scattering'' term, while the second would represent an ``in-scattering''
term. However, this split cannot be performed because the two integrals
individually diverge, due to the singularity in the denominator at $\bo'=\bo$.
Thus, $\mc{L}_{3/2}$ only exists when the two integrals are combined. Then,
the numerator and denominator both vanish at $\bo'=\bo$ in such a way that the
single integral exists as a principal value. In effect, $\mc{L}_{3/2}$ defines
a limiting linear transport process in which the total cross section is
infinite (the mean free path is zero).

\subsubsection{Screened Rutherford scattering}
The screened Rutherford differential scattering kernel, which is widely used
for the Coulomb scattering of non-relativistic electrons, is defined by:
\begin{equation}
f(\mu_0) = \frac{\eta(1+\eta)}{\pi(1+2\eta-\mu_0)^2}
\end{equation}
where $\eta>0$ is the screening parameter. We consider $\eta\approx 0$, for
which $f(\mu_0)$ is forward-peaked. In the Legendre expansion of $f(\mu_0)$,
one has:
\begin{align}
&f_0 = 1\\
&f_1 = \bar{\mu}_0 = 1 -2\eta \((1+\eta)\ln\(1+\frac{1}{\eta}\)-1\)
\label{f_1}
\end{align}
Using:
\begin{equation}
\mu P_l(\mu) = \frac{(l+1)P_{l+1}(\mu)+lP_{l-1}(\mu)}{2l+1}
\end{equation}
we get:
\begin{equation}
lf_{l+1} - (1+2\eta)(2l+1)f_l+(l+1)f_{l-1} = 0
\label{recursion}
\end{equation}
\Cref{recursion} may be written:
\begin{equation}
\frac{f_{l+1}-f_l}{l+1} - \frac{f_l-f_{l-1}}{l} = 2 \eta \frac{2l+1}{l(l+1)} f_l
\end{equation}
Summing from $l=1$ to $m-1$ and then multiplying by $m$, we obtain:
\begin{equation}
f_m-f_{m-1}-(\bar{\mu}_0-1)m = 2 \eta m \sum_{k=1}^{m-1} \frac{2k+1}{k(k+1)} f_k
\end{equation}
Summing thus result from $m=1$ to $l$ and using the identity:
\begin{equation}
\sum_{m=1}^l m = \frac{l(l+1)}{2}
\label{sum_m}
\end{equation}
we obtain:
\begin{equation}
\begin{split}
f_l-1-(\bar{\mu}_0-1)\frac{l(l+1)}{2} &= 2 \eta \sum_{m=1}^l m\sum_{k=1}^{m-1}
\frac{2k+1}{k(k+1)}f_k\\
&= 2\eta \sum_{k=1}^{l-1} \(\sum_{m=k+1}^l m\) \frac{2k+1}{k(k+1)} f_k\\
&= 2\eta\sum_{k=1}^{l-1}\(\frac{l(l+1)}{2}-\frac{k(k+1)}{2}\)\frac{2k+1}{k(k+1)}f_k
\end{split}
\end{equation}
Hence:
\begin{equation}
f_n = 1-(1-\bar{\mu}_0)\frac{l(l+1)}{2}+\eta\sum_{k=1}^{l-1}
\(\frac{l(l+1)}{k(k+1)}-1\) (2k+1) f_k
\label{f_n_R}
\end{equation}
This result is algebraically equivalent to \cref{recursion} for all
$l\geq 0$ and $\eta >0$ if the summation term is interpreted as zero for $l=0$
and 1.

To proceed, we assume for $\eta \ll 1$ the expansion:
\begin{equation}
f_l = f_{l,0}+\eta f_{l,1} + O(\eta^2)
\end{equation}
Introducing this into \cref{f_n_R} and equating the coefficients of
$\eta^0$ and $\eta^1$ (note that we do not expand $\bar{\mu}_0$ for small
$\eta$), we obtain:
\begin{align}
& f_{l,0} = 1 - (1-\bar{\mu}_0) \frac{l(l+1)}{2}\\
& f_{l,1} =
\sum_{k=1}^{l-1}\(\frac{l(l+1)}{k(k+1)}-1\)(2k+1)\(1-(1-\bar{\mu}_0)
\frac{k(k+1)}{2}\)
\end{align}
From \cref{f_1}, we have:
\begin{equation}
1-\bar{\mu}_0 = O\(\eta \ln\(\frac{1}{\eta}\)\)
\label{weakly_small}
\end{equation}
Therefore, we get:
\begin{equation}
f_l = 1+\(\frac{1-\bar{\mu}_0}{2}\) \(-l(l+1)\)+\eta \(\sum_{k=1}^{l-1}
\(\frac{l(l+1)}{k(k+1)}-1\)(2k+1)\) + O\(\eta^2\ln\(\frac{1}{\eta}\)\)
\label{eq_58}
\end{equation}
Now using \cref{sum_m} and:
\begin{equation}
\frac{1}{k(k+1)} = \frac{1}{k+1}-\frac{1}{k}
\end{equation}
we get:
\begin{equation}
\sum_{k=1}^{l-1} \(\frac{l(l+1)}{k(k+1)}-1\)(2k+1) = 2 l(l+1) \(\(\sum_{k=1}^l
\frac{1}{k}\)-1\)
\end{equation}
for $l\geq 0$.

Thus, \cref{eq_58} may be written:
\begin{equation}
f_l=1+\(\frac{1-\bar{\mu}_0}{2}\)\(-l(l+1)+2\eta\(-l(l+1)\)\)\(1-\sum_{k=1}^l
\frac{1}{k}\) + O(\eta^2\ln\(\frac{1}{\eta}\))
\end{equation}
We define $\Sigma_s$ as:
\begin{equation}
\Sigma_s = \frac{\sigma}{1-\bar{\mu}_0}
\end{equation}
Then for small $\eta$, the eigenvalues of $\Sigma_s \mc{L}_B$ become:
\begin{equation}
-\Sigma_s (1-f_n) = \frac{\sigma}{2}
\(-l(l+1)\)+\frac{2\eta\sigma}{1-\bar{\mu}_0} (-l(l+1))\(1-\sum_{k=1}^l
\frac{1}{k}\)+O(\eta)
\end{equation}
So we finally get:
\begin{equation}
\begin{split}
\Sigma_s \mc{L}_B &= \frac{\sigma}{2}\mc{L}_{FP} + \frac{2\eta
\sigma}{1-\bar{\mu}_0} \mc{L}_{FP}(I+\mc{L}_1) + O(\eta)\\
&= \frac{\Sigma_s(1-\bar{\mu}_0}{2} \mc{L}_{FP} + 2 \eta \Sigma_s \mc{L}_{FP}
(I+\mc{L}_1) + O(\eta)
\end{split}
\end{equation}
where:
\begin{equation}
\mc{L}_1 \Psi(\bo) = \frac{1}{4\pi} \int_{4\pi}
\frac{\Psi(\bo')-\Psi(\bo)}{1-\bo\cdot\bo'}d\bo'
\end{equation}
and:
\begin{equation}
\mc{L}_1 Y_l^m(\bo) = \(-\sum_{k=1}^l\frac{1}{k}\) Y_l^m(\bo)
\end{equation}
The first term in this expansion is the standard Fokker-Planck operator which
is $O(1)$. The second term in the expansion is weakly small; by \cref{weakly_small}
it is $O(1/\ln(1/\eta))$. 

The conventional Fokker-Planck operator occurs in the asymptotic
approximations of $\mc{L}_B$ for both Henyey-Greenstein and screened
Rutherford scattering. For Henyey-Greenstein scattering, $\mc{L}_{FP}$ is
dominated by a pseudodifferential operator $\mc{L}_{3/2}$, and for screened
Rutherford scattering, $\mc{L}_{FP}$ weakly dominates another
pseudodifferential operator $\mc{L}_{FP}\mc{L}_1$. In neither case can one
write the write the leading plus first-order approximations to $\mc{L}_B$ as
pure differential operators-additional pseudodifferential operators, with
unbounded spectra, must be included.

\section{Fokker-Planck cross section}
\red{Premiere partie redondante avec Ligou, on utilise \cite{morel_81}}
\begin{equation}
\bo\cdot\bn \Psi + \Sigma_a \Psi = \Gamma_B\Psi +Q
\end{equation}                                
with:
\begin{equation}
\Gamma_B \Psi = \int_{4\pi}
\(\Sigma_s(E',\mu_0)\Psi(\mu',E)-\Sigma_s(E,\mu_0)\Psi(\mu,E)\)d\bo_0
\label{gamma_b}
\end{equation}
where:
\begin{equation}
\mu_0 = \cos(\theta_0)
\end{equation}
and:
\begin{equation}
\mu' = \mu \cos(\theta_0) - \sqrt{1-\mu^2}\sin(\theta_0) \cos(\theta_0)
\label{mu_p}
\end{equation}
Performing a Taylor a series expansion about $\theta_0=0$ and retaining terms
up to $\theta_0^2$ gives:
\begin{equation}
\Sigma_s(E',\mu_0) \Psi(\mu',E) - \Sigma_s(E,\mu_0)\Psi(\mu,E) = I_1+I_2
\end{equation}
where:
\begin{equation}
I_1 = \Sigma_s(E,\mu_0) \(\frac{\partial}{\partial \mu}\Psi(\mu,E)
\(\frac{\partial \mu'}{\partial \theta_0}\theta_0 +\frac{1}{2}
\frac{\partial^2\mu'}{\partial \theta_0^2}\theta_0^2\) +\frac{1}{2}
\frac{\partial^2}{\partial \mu^2}\Psi(\mu,E) \(\frac{\partial \mu'}{\partial
\theta_0}\)^2 \theta_0^2 \)
\label{I_1}
\end{equation}
and:
\begin{equation}
I_2 = \frac{\partial}{\partial E}\Sigma_s(E,\mu_0) \Psi(\mu,E)
\(\frac{\partial E'}{\partial \theta_0}\theta_0 +
\frac{1}{2}\frac{\partial^2 E'}{\partial \theta_0^2}\theta_0^2\)+\frac{1}{2}
\frac{partial^2}{\partial E^2} \Sigma_s(E,\mu_0) \Psi(\mu,E) \(\frac{\partial
E'}{\partial \theta_0}\)^2 \theta_0^2
\label{I_2}
\end{equation}
Using \cref{mu_p} to evaluate the derivatives in \cref{I_1} gives:
\begin{equation}
I_1 = \Sigma_s(E,\mu_0) \theta_0^2 \(-\frac{\mu}{2}\frac{\partial
\Psi}{\partial \mu} + \frac{(1-\mu^2)}{2} (\cos(\varphi_0))^2 \frac{\partial^2
\Psi}{\partial \mu^2}\)
\label{I_1_bis}
\end{equation}      
Because $E=E_s(E',\theta_0)$ and $E'=E$ when $\theta_0=0$, it follows that:
\begin{equation}
\begin{split}
\frac{\partial E'}{\partial \theta_0}\theta_0 + \frac{1}{2}\frac{\partial^2
E'}{\partial \theta_0^2}\theta_0^2 +\hdots &= -\frac{\partial E_s}{\partial
\theta_0} \theta_0 - \frac{1}{2} \frac{\partial E_s}{\partial
\theta_0^2}\theta_0^2+\hdots\\
&= E-E_s(E,\theta_0) 
\end{split}
\end{equation}
Taking this into account, \cref{I_2} becomes:
\begin{equation}
I_2 = \frac{\partial}{\partial E}\Sigma_s(E,\mu_0) \Psi(\mu,E) (E-E_s) +
\frac{1}{2}\frac{\partial^2}{\partial E^2} \Sigma_s(E,\mu_0) \Psi(\mu,E)
(E-E_s)^2
\label{I_2_bis}
\end{equation}
Substituting \cref{I_1_bis,I_2_bis} into \cref{gamma_b} and integrating gives:
\begin{equation}
\Gamma_B \Psi \approx \Gamma_{FP} \Psi
\end{equation}
where:
\begin{equation}
\Gamma_{FP}\Psi = \frac{\alpha}{2} \frac{\partial}{\partial \mu} (1-\mu^2)
\frac{\partial}{\partial \mu} \Psi + \frac{\partial}{\partial E} \beta \Psi +
\frac{1}{2}\frac{\partial^2}{\partial^2}\gamma \Psi
\label{gamma_fp}
\end{equation}
and:
\begin{align}
\alpha &= 2\pi \int_{-1}^1 \Sigma_s(E,\mu_0) (1-\mu_0) d\mu_0 \label{alpha}\\
\beta &= \int_0^{\infty} \Sigma_s(E\rightarrow W') (E-E') dE' \label{beta}\\
\gamma &= \int_0^{\infty} \Sigma_s(E\rightarrow E') (E-E') dE' \label{gamma}
\end{align}
We refer to $\Gamma_{FP}$ as the Fokker-Planck scattering operator. The
function $\alpha$, $\beta$ and $\gamma$ are known as the momentum transfer,
stopping power, and means square stopping power, respectively. We refer to
them collectively as the Fokker-Planck functions. The equation that we seek to
solve:
\begin{equation}
\bo\cdot\bn\Psi +\Sigma_a \Psi = \Gamma_{FP} \Psi+Q
\end{equation}
If the angular flux is sufficiently smooth, the expansions given in \cref{I_1,I_2} 
are accurate for ``small'' values of $\theta_0$.
If the scattering is highly forward peaked, the integrand in \cref{gamma_b} is 
significantly nonzero over only a ``small'' region about
$\theta_0=0$. Thus, one would expect that with sufficiently forward that with
sufficiently forward-peaked scattering, the dominant contribution to the
Boltzmann integral would come from the range of $\theta_0$ values over which
the expansions are accurate and that $\Gamma_{FP}\Psi$ would thereby closely
approximate $\Gamma_B\Psi$. Experience has shown that good results for scalar
quantities (angle and energy integrated), such as energy and charge deposition
profiles, can be expected in charged-particle calculations. However, results
for detailed differential quantities are generally inadequate. The
Fokker-Planck equation is very useful in spite of this deficiency because it
is the scalar rather than differential quantities that are most often of
applied interest.

Let us define:
\begin{align}
\Gamma_{FP}^{\alpha} \Psi &= \frac{\alpha}{2} \frac{\partial}{\partial \mu}
(1-\mu^2) \frac{\partial}{\partial \mu}\Psi \label{gamma_alpha}\\
\Gamma_{FP}^e\Psi &=\frac{\partial}{\partial E}\beta\Psi +
\frac{1}{2}\frac{\partial^2}{\partial E^2} \gamma \Psi \label{gamma_e}
\end{align}
We note that $\Gamma_{FP}^{\alpha}$ causes particles to redistribute in
direction without energy change, while $\Gamma_ {FP}^e$ causes particles to
redistribute particle in energy without directional change. This would suggest
that $\Gamma_{FP}^{\alpha}$ should be approximated with a cross section of the
form:
\begin{equation}
\Sigma_s(E'\rightarrow E,\mu_0) = \Sigma^{\alpha}(E,\mu_0) \delta(E'-E)
\end{equation}
while $\Gamma_{FP}^e$ should be approximated by a cross section of the form:
\begin{equation}
\Sigma_s(E'\rightarrow E,\mu_0) = \Sigma^e(E'\rightarrow E) \frac{1}{2\pi}
\delta(\mu_0-1)
\end{equation}
Additional justification for this decoupled approximation is found in the fact
that the Fokker-Planck scattering operator given in \cref{gamma_fp}
can also be derived assuming a composite decoupled cross section:
\begin{equation}
\Sigma_s(E'\rightarrow E,\mu_0) = \Sigma^{\alpha}(E,\mu_0) \delta(E'-E) +
+\Sigma^e(E'\rightarrow E) \frac{1}{2\pi} \delta(\mu_0-1)
\end{equation}
As one would expect, the derivation is very similar to that for the elastic
(coupled) cross section. Hence, we only give a brief outline of it.

The Boltzmann scattering operators corresponding to $\Sigma^{\alpha}$ and
$\Sigma^e$ are given by:
\begin{align}
& \Gamma_B^{\alpha} = \int_{4\pi} \Sigma^{\alpha}(E,\mu_0)
\(\Psi(\mu')-\Psi(\mu)\) d\bo' \label{gamma_b_alpha}\\
& \Gamma_B^e = \int_{0}^{\infty} \(\Sigma^e(E'\rightarrow E) \Psi(E') -
\Sigma^e(E\rightarrow E') \Psi(E)\)dE' \label{gamma_b_e}
\end{align}
Expanding the integrand in \cref{gamma_b_alpha} about $\theta_0=0$,
where $\theta_0$ keeping terms up to $\theta_0^2$ and integrating gives:
\begin{equation}
\Gamma_B^{\alpha} = \Gamma_{FP}^{\alpha}\Psi
\end{equation}
with:
\begin{equation}
\alpha = 2\pi \int_{-1}^{1} \Sigma^{\alpha} (E,\mu_0) (1-\mu_0) d\mu_0
\end{equation}
Expanding the integrand in \cref{gamma_b_e} about $\tau=0$, where
$\tau=E'-E$, keeping terms up to $\tau^2$, and integrating gives:
\begin{equation}
\Gamma_B^e\Psi =\Gamma_{FP}^e\Psi
\end{equation}
with:
\begin{align}
&\beta = \int_0^{\infty} \Sigma^e(E\rightarrow E') (E-E') dE'\\
&\gamma = \int_0^{\infty}  \Sigma^e(E\rightarrow E') (E-E')^2 dE'
\end{align}
An argument, completely analogous to that presented for the coupled case, can
be used to show $\Gamma_B^{\alpha}$ and $\Gamma_{FP}^{\alpha}$ better
approximate one another as $\Sigma^{\alpha}$ becomes increasingly peaked about
$\mu_0=1$. For similar reasons, $\Gamma_B^e$ and $\Gamma_{FP}^e$ better
approximate one another as $\Sigma^e$ becomes increasingly peaked about
$E'=E$. 

\subsection{Coefficients for $\Sigma^{\alpha}$}
Because $\Sigma^{\alpha}$ does not change particle energy, it corresponds to a
within-group cross section. Thus, we need only to define the Legendre
coefficients, $\{\Sigma_l^{\alpha}\}_{l=0}^{N-1}$. This is fairly
straightforward to do once two fundamental properties of $\Gamma_B^{\alpha}$
and $\Gamma_{FP}^{\alpha}$ are noted. Let $\{P_l(\mu)\}_{l=}^{\infty}$ denote
the Legendre polynomials. By expanding $\Sigma^{\alpha}$ in such polynomials
and invoking the addition theorem for the spherical harmonics, the following
well-known result can be demonstrated:
\begin{equation}
\Gamma_B^{\alpha}P_l(\mu) = (\Sigma_l^{\alpha}-\Sigma_0^{\alpha}) P_l(\mu_0)
\label{gamma_b_alpha_p}
\end{equation}
with:
\begin{align}
&l=0,\hdots,\infty\\
&\Sigma_l^{\alpha} = 2\pi \int_{-1}^1 \Sigma^{\alpha} (E,\mu_0) P_l(\mu_0)
d\mu_0
\end{align}
Using standard recurrence relations for the Legendre polynomials, it is not
difficult to show that:
\begin{equation}
\Gamma_{FP}^{\alpha} P_l(\mu) = \frac{\alpha}{2} l(l+1) P_l(\mu)
\label{gamma_fp_alpha_p}
\end{equation}
If the angular flux is expressible as a polynomial of arbitrary degree $L$:
\begin{equation}
\Psi(\mu) = \sum_{l=0}^L \frac{2l+1}{4\pi} \phi_l P_l(\mu)
\end{equation}
then it follows from \cref{gamma_b_alpha_p,gamma_fp_alpha_p} that we can define 
$\Sigma^{\alpha}$ so that:
\begin{equation}
\Gamma_B^{\alpha}\Psi=\Gamma_{FP}^{\alpha}\Psi
\end{equation}
by setting:
\begin{equation}
\Sigma_0^{\alpha}-\Sigma_l^{\alpha} = \frac{\alpha}{2}l(l+1)
\label{sigma_m_sigma}
\end{equation}
with $l=1,\hdots,L$.\\
Choosing $\Sigma_L=0$ so as to minimize the resulting value for
$\Sigma_0^{\alpha}$, \cref{sigma_m_sigma} gives:
\begin{equation}
\Sigma^{\alpha}(E,\mu_0) = \sum_{l=0}^L \frac{2l+1}{4\pi} \Sigma_l^{\alpha}
P_l(\mu_0)
\end{equation}
where:
\begin{equation}
\Sigma_l^{\alpha} = \frac{\alpha}{2}\(L(L+1)-l(l+1)\),\ \  l=0,\hdots,L
\end{equation}
Thus, with an expansion degree of $L$ for $\Sigma^{\alpha}$, we see that
$\Gamma_B^{\alpha}$ and $\Gamma_{FP}^{\alpha}$ are completely equivalent when
operating on polynomials of degree $L$ or less. With appropriate quadrature
sets and expansion orders, the $S_n$ representation for $\Gamma_{FP}^{\alpha}$
is equivalent to that obtained by interpolating the discrete angular flux
values with a polynomial and operating on that polynomial with
$\Gamma_{FP}^{\alpha}$.

It is instructive to consider the behavior of $\Sigma^{\alpha}$ in the limit
as the degree of the expansion is increased. Previous analysis show that
$\Sigma^{\alpha}$ better approximated $\Gamma_B^{\alpha}$ as it becomes
increasingly forward-peaked. In addition, it should obviously have the correct
momentum transfer. Thus, we first take note that regardless of expansion order
the momentum transfer of $\Sigma^{\alpha}$ is exact:
\begin{equation}
\begin{split}
2\pi \int_{-1}^1 \Sigma^{\alpha}(E,\mu_0) (1-\mu_0) d\mu_0 &=
\Sigma_0-\Sigma_1\\
&=\frac{\alpha}{2} L(L+1) - \frac{\alpha}{2} (L(L+1)-2)\\
&=\alpha
\end{split}
\end{equation}
The average cosine of the scattering angle can be expressed as follows:
\begin{equation}
\begin{split}
\bar{\mu}_0 &= \frac{\Sigma_1^{\alpha}}{\Sigma_0^{\alpha}}\\
&=\frac{L(L+1)-2}{L(L+1)}
\end{split}
\end{equation}
It is easily seen that as $L$ increased, $\bar{\mu}_0$ goes to unity. Thus,
$\Sigma^{\alpha}$ becomes increasingly forward-peaked as $L$ increases. The
total magnitude of $\Sigma^{\alpha}$ becomes unbounded in the same limit:
\begin{equation}
\Sigma_0^{\alpha} = \frac{\alpha}{2} L (L+1)
\end{equation}
Expressing the momentum transfer as follows:
\begin{equation}
\alpha = \Sigma_0^{\alpha} (1-\bar{\mu}_0)
\end{equation}
it is seen that $\Sigma_0^{\alpha}$ must increase without bound if the
momentum transfer is to remain constant as $\Sigma^{\alpha}$ becomes
increasingly forward-peaked. This show that $\Gamma_{FP}^{\alpha}$ corresponds
to a type of continuous-deflection approximation. The $\Gamma_{FP}^{\alpha}$
operator effectively causes particles to scatter continuously while incurring
a differential deflection in each scattering event. The net result is that
particles continuously deflect with the mean deflection per unit pathlength
given by the momentum transfer.

\red{Vu qu'on fait que du 1 groupe, pas la peine de regarder aux autres
termes, meme development dans \cite{morel_89}}

\red{Partie prise de  \cite{morel_96}}
In this section, we take the first step in the development of our approximate
BFP equation by replacing the angular Fokker-Planck operator with a Boltzmann
operator. The angular Fokker-Planck operator is replaced with the scattering
operator $B_{\alpha}$:
\begin{equation}
B_{\alpha} = \int_{4\pi} \Sigma_{\alpha} (E,\mu_0)
\Psi(\bo',E)d\bo'-\Sigma_{\alpha}\Psi
\end{equation}
where:
\begin{equation}
\Sigma_{\alpha}(E,\mu_0) = \frac{\alpha(E)}{1-\mu_s} \frac{1}{2\pi}
\delta(\mu_0-\mu_s)
\end{equation}
and where $\mu_s$, which we refer to as the Fokker-Planck scattering angle, is
a variable parameter. Note that there is no energy loss associated with
$B_{\alpha}$ and further that there is only one polar angle of scatter. The
cosine of this scattering angle is equal to $\mu_s$. To avoid confusion
between the smooth-component Boltzmann operator and $B_{\alpha}$, we refer to
all quantities associated with the smooth-component Boltzmann operator as
``smooth-component'' quantities and all quantities associated with
$B_{\alpha}$ as ``continuous-scattering'' quantities. For instance, we refer
to $\Sigma_{\alpha}$ as the continuous-scattering cross section and to
$\Sigma_t$ as the smooth-component total cross section.

We now demonstrate that $B_{\alpha}$ converges to the continuous-scattering
operator int the limit as $\mu_s\rightarrow 1$. In particular, we show that
any fixed number of eigenvalues and eigenfunctions of the
continuous-scattering operator can be preserved to an arbitrary degree of
accuracy as a function of $\mu_s$. Expanding the scattering cross section in
Legendre polynomials, applying the spherical-harmonics addition theorem, and
then operating on the function $Y_l^m$, we find that:
\begin{equation}
B_{\alpha} Y_l^M = \(\Sigma_{\alpha}^l-\Sigma_{\alpha}^0\)Y_l^m
\end{equation}
where:
\begin{equation}
\Sigma_{\alpha}^0 = \frac{\alpha}{1-\mu_s}
\label{sigma_a_0}
\end{equation}                            
and:
\begin{equation}
\Sigma_{\alpha}^l = \frac{\alpha}{1-\mu_s} P_l(\mu_s)
\end{equation}
From \cref{sigma_a_0}, it follows that the spherical-harmonic
functions are eigenfunctions of $B_{\alpha}$.
The associated Legendre function $P_l^m$ satisfies Legendre's associated
differential equation:
\begin{equation}
\frac{\partial}{\partial \mu} \((1-\mu^2)\frac{\partial}{\partial \mu}P_l^m\)
+ \(l(l+1) - \frac{m^2}{1-\mu^2}\)P_l^m=0
\label{fp_op_eig}
\end{equation}
Applying the Fokker-Planck angular operator to $Y_l^m$ and using \cref{fp_op_eig}, 
we find that:
\begin{equation}
F_{\alpha}Y_l^m = -\frac{\alpha}{2} l(l+1) Y_l^m
\label{F_a_y_l_m}
\end{equation}
where:
\begin{equation}
F_{\alpha} \Psi = \frac{\alpha}{2}\(\frac{\partial}{\partial \mu} \((1-\mu^2)
\frac{\partial \Psi}{\partial \mu}\) +
\frac{1}{1-\mu^2}\frac{\partial^2\Psi}{\partial \varphi^2}\)
\end{equation}
It follows from \cref{F_a_y_l_m} that the spherical-harmonic
functions are eigenfunctions of the Fokker-Planck angular operator. Thus, the
spherical-harmonics functions are the eigenfunctions of both $B_{\alpha}$ and
the angular Fokker-Planck operator.
To show that $B_{\alpha}$ becomes equivalent to $F_{\alpha}$ in the limit as
$\mu_s\rightarrow 1$, we expand the eigenvalues of $B_{\alpha}$ in a
first-order Taylor series about $\mu_s=1$:
\begin{equation}
\begin{split}
\Sigma_{\alpha}^l - \Sigma_{\alpha}^0 &= \frac{\alpha}{1-\mu_s}
(P_l(\mu_s)-1)\\
&=\frac{\alpha}{1-\mu_s} \(P_l(1)+\frac{\partial}{\partial x}P_l(x)\Big|_{x=1}
(\mu_s-1)-1\)
\end{split}
\label{sig_comp}
\end{equation}
Since:
\begin{align}
P_l(1) &= 1\\
\frac{\partial}{\partial x} P_l(x)\Big|_{x=1} = \frac{l(l+1)}{2}
\end{align}
\Cref{sig_comp} becomes:
\begin{equation}
\Sigma_{\alpha}^l - \Sigma_{\alpha}^0 = -\frac{\alpha}{2}l(l+1)
\end{equation}
We find that the eigenvalues of $B_{\alpha}$ converge to the eigenvalues of
$F_{\alpha}$ in the limit as $\mu_s\rightarrow 1$. Thus, these two operators
become equivalent in this limit.

The error in each eigenvalue monotonically decreases as $\mu_s$ is increased,
but for any fixed value of $\mu_s$, the error monotonically increases with
increasing eigenvalue index. Thus, while $B_{\alpha}$ does in fact converge to
$F_{\alpha}$ as $\mu_s\rightarrow 1$, it does so nonuniformly. Thus, for any
fixed value of $\mu_s$, the high-order eigenvalues of $F_{\alpha}$ are grossly
underestimated by $B_{\alpha}$. Fortunately, this is error in the high-order
eigenvalues is usually unimportant. To explain why this is so, we first we
first consider the following time-dependent transport equation:
\begin{equation}
\frac{\partial}{\partial t}\Psi(t,\mu) = \frac{\alpha}{2}
\frac{\partial}{\partial \mu} \((1-\mu^2)\frac{\partial \Psi}{\partial \mu}\)
\end{equation}
with initial condition:
\begin{equation}
\Psi(0,\mu) = \frac{1}{2\pi} \delta(\mu-1)
\label{init_cond}
\end{equation}
The solution to \cref{init_cond} is:
\begin{equation}
\Psi = \frac{1}{2\pi} \sum_{l=0}^{\infty} \exp(-\lambda_l t)P_l(\mu)
\label{exp}
\end{equation}
where $\lambda_l$ is the eigenvalue of $F_{\alpha}$ with Legendre index $l$.
Note from \cref{exp} that the $l^{th}$ Legendre moment of the
angular flux solution is exponentially attenuated in proportion to
$\lambda_l$. Since these eigenvalues become unbounded for large $l$, it
follows that the high-order flux moments will be ``very small'' for all
nonzero values of $t$. A gross underestimate of $\lambda_l$ will still result
in a very small flux moment if the approximate of eigenvalue is large
relative to $1/t$. Thus, for sufficiently large times, an accurate solution
can be obtained even though the high-order eigenvalues are grossly
underestimated. 

This example illustrates the following basic idea. In problems with highly
forward-peaked scattering, the high-order flux moments are much more rapidly
attenuated than the low-order moments as the particle distribution evolves in
space and time. Thus, errors in the eigenvalues are large relative to the
temporal and spatial scale lengths associated with a given problem.


\section{Galerkin}
\red{Analyse tiree de \cite{pautz_fp}}
The Fokker-Planck equation is not an asymptotic limit of the transport
equation unless the scattering kernel is such that:
\begin{equation}
\frac{\la\bar{\(1-\mu_0\)^2}\ra}{\la\(1-\bar{\mu}_0\)\ra}\rightarrow 0
\end{equation}
as $\bar{\mu}_0 \rightarrow 1$.

\begin{equation}
\begin{split}
\phi_{l,m}(\br) &= \int_{4\pi} d\bo' \Psi(\br,\bo') Y_l^{m,*}\\
&=(D \Psi)_{l,m}
\end{split}
\label{D2M}
\end{equation}
where $D$ is the directions-to-moments operator.
\begin{equation}
\begin{split}
\Psi(\br,\bo) &= \sum_{l=0}^{\infty} \sum_{m=-l}^l \(\frac{2l+1}{4\pi}\)
Y_l^m(\bo)\phi_{l,m}(\br)\\
&= M\phi(\bo)
\end{split}
\label{M2D}
\end{equation}
where $M$ is the moments-to-directions operator. \Cref{D2M,M2D} define the 
transformations from $\phi$ to $\Psi$ (``moments to
directions'') and from $\Psi$ to $\Phi$ (``directions to moments''). By
combining \cref{D2M,M2D}, we obtain the following requirement:
\begin{equation}
(I-MD)\Psi = 0
\end{equation}
$M=D^{-1}$ for analytic transport (\red{pris a l'interieur de la derivation de
la forme analytique})

Let us define:
\begin{align}
\Sigma_a &= \hat{\Sigma}_a\\
\Sigma_{s,l} &= \frac{\hat{\Sigma}_{s,l}(\br)}{\epsilon}\\
\Sigma_{s,l}(\br) &=\Sigma_{s,0}(\br) \(1-\frac{l(l+1)}{2}A(\br)\epsilon +
O(\gamma)\)
\end{align}
\begin{equation}
\begin{split}
&\bo_k\cdot\bn\Psi_k(\br) + \(\Sigma_a(\br)+\Sigma_{s,0}(\br)\) \Psi_k(\br)
=\\
& \sum_{l=0}^{N-1} \sum_{m=-l}^l \(\frac{2l+1}{4\pi}\)Y_l^m(\bo_k) \phi_{l,m}(\br)
\frac{\hat{\Sigma}_{s,0}(\br)}{\epsilon}\(1-\frac{l(l+1)}{2}A(\br)\epsilon +
O(\gamma)\) + q(\br,bo_k)
\end{split}
\label{discr_eps}
\end{equation}
where:
\begin{align}
\Psi_k(\br) &= \Psi(\br,\bo_k)\\
\phi_{l,m}(\br) &= \sum_{k=1}^K w_k Y_l^{m,*}(\bo_k) \Psi_k(\br) \label{phi_p}
\end{align}
Here, the $w_k$ and $\bo_k$ are the quadrature weights and directions,
respectively, of a quadrature set of order $N$. In level-symmetric quadrature
sets, $K=N$ in 1D, $K=\frac{N(N+2)}{2}$ in 2D and $K=N(N+2)$ in 3D. Note that
in the standard discrete ordinates treatment, the scattering order \cref{discr_eps} 
is truncated at $N-1$. Manipulation of \cref{discr_eps} yields:
\begin{equation}
\begin{split}
&\bo_k\cdot\bn \Psi_k(\br) +\hat{\Sigma}_a(\br)\Psi_k(\br)\\
&\frac{\hat{\Sigma}_{s,0}(\br)}{\epsilon} \(\Psi_k(\br)-\sum_{l=0}^{N-1}
\sum_{m=-l}^m \(\frac{2l+1}{4\pi}\) Y_l^m(\bo_k) \phi_{l,m}^{\br}\)\\
&=-\frac{\(\Sigma_{tr}(\br)-\hat{\Sigma}_a(\br)\)}{2} \sum_{l=0}^{N-1}
\sum_{m=-l}^{l}\(\frac{2l+1}{4\pi}\) l(l+1) Y_l^m(\bo_k) \phi_l^m(\br)+\\
& q(\br,\bo_k) + O\(\frac{\gamma}{\epsilon}\) 
\end{split}
\label{discr_eps_manip}
\end{equation}
We insert the ansatz:
\begin{align}
\Psi &= \Psi^{(0)} + \epsilon \Psi^{(1)} + \epsilon^2\Psi^{(2)}+\hdots\\
\phi_{l,m} &= \phi_{l,m}^{(0)} + \epsilon \phi_{l,m}^{(1)} + \epsilon^2
\phi_{l,m}^{(2)}+\hdots
\end{align}
into \cref{discr_eps} and consider terms of $O(1)$ to find:
\begin{equation}
\begin{split}
\phi_{l,m}^{(0)}(\br) &= \sum_{k=1}^K w_k Y_l^{m,*}(\bo_k) \Psi_k^(0)(\br)\\ 
&= (D_N\Psi^{(0)})_{l,m}
\end{split}
\label{D2M_disc}
\end{equation}
We insert the ansatz into \cref{discr_eps_manip} and consider terms
of $O(\epsilon^{-1})$ to find:
\begin{equation}
\begin{split}
\Psi_k^{(0)}(\br) &=\sum_{l=0}^{N-1} \sum_{m=-l}^l \(\frac{2l+1}{4\pi}\)
Y_l^m(\bo_k) \phi_{l,m}^{(0)}(\br)\\
&= (M_N\phi^{(0)})_k
\end{split}
\label{M2D_disc}
\end{equation}
where there is no $O(\gamma)$ term, since it is easily shown that
$\gamma\rightarrow 0$ as $\epsilon\rightarrow 0$,i.e., that there are no
$O(1)$ components in $\gamma$. \Cref{D2M_disc,M2D_disc} may be combined to give:
\begin{align}
&(I-M_ND_N)\Psi^{(0)} = 0 \label{identity_a}\\
&(I-D_NM_N)\phi^{(0)} = 0 \label{identity_b}
\end{align}
i.e., $\Psi^{(0)}$ must be in the null space of $I-M_N D_N$ and $\phi^{(0)}$
must be in the null space of $I-D_N M_N$. Since $\phi$ is always constructed
from $\Psi$ according to \cref{phi_p}, \cref{identity_b}
is equivalent to:
\begin{equation}
D_N(I-M_N D_N)\Psi^{(0)} = 0
\end{equation}
which is always satisfied if \cref{identity_a} is satisfied.
Therefore, we need only to show that equation \cref{identity_a} is satisfied
in order to satisfy \cref{D2M_disc,M2D_disc}.

A sufficient (although not strictly necessary) condition for satisfying
\cref{identity_a} is that $M_ND_N=I$. This will be always be true if
$M_N$ and $D_N$ are inverse of each other (as in analytic transport), in which
case there are as many moments in the scattering expansion as there are
discrete angles. It may or may not be true of there are fewer moments than
angles. (These assertions follow directly from the dimensions of the
matrices.) In one-dimensional slab and spherical geometry, it will be true if
and only if the quadrature set exactly integrates polynomials of degree
$2N-2$, as is the case with the Gauss-Legendre set. In standard
multidimensional implementations, there are generally more discrete angles
than scattering moments, so in these cases $M_N D_N \neq I$. If $M_N
D_N\approx I$, then \cref{identity_b} will be satisfied only if some
other (generally nonphysical) constraints are met. (Specifically, $\Psi^{(0)}$
must be constrained to live in the null space of $I-M_N D_N$.) If equation
\cref{identity_b} is not satisfied, the asymptotic ansatz is not valid. In
such a case, there is no $O(1)$ solution to \cref{discr_eps,phi_p}. Later, we will 
describe a relatively simple alteration of the
standard discrete ordinates method that will ensure that $M_N D_N = I$.

If we assume that $M_N D_N=I$, then the $O(\epsilon)$ terms in \cref{phi_p} yields:
\begin{equation}
\begin{split}
\phi_{l,m}^{(1)}(\br) &= \sum_{k=1}^K w_k Y_{l}^{m,*} (\bo_k)
\Psi_k^{(1)}(\br)\\
&=(D_N\Psi^{(1)})_{l,m}
\end{split}
\end{equation}

The $O(1)$ terms in \cref{discr_eps_manip} yield:
\begin{equation}
\begin{split}
&\bo_k\cdot \bn \Psi_k^{(0)}(\br) + \hat{\Sigma}_a(\br)\\
&\hat{\Sigma}_{s,0}(\br) \(\Psi_k^{(1)} (\br) - \sum_{n=0}^{N-1}
\sum_{m=-l}^{l}\(\frac{2l+1}{4\pi}\) Y_l^m (\bo_k)\phi_{l,m}^{(1)}(\br)\)\\
&=-\frac{\Sigma_{tr}(\br)-\hat{\Sigma}_a(\br)}{2} \sum_{l=0}^{N-1}
\sum_{m=-l}^l \(\frac{2l+1}{4\pi}\) l(l+1) Y_l^m(\bo_k)
\phi_{l,m}^{(0)}(\br)\\
&+ q(\br,\bo_k) + O\(\frac{\gamma}{\epsilon}\)
\end{split}
\label{O_1_terms}
\end{equation}
The scattering term on the left side of \cref{O_1_terms} will disappear only if:
\begin{equation}
\Psi_k^{(1),*}(\br) = \Psi_k^{(1)}(\br)
\label{psi_1_star_old}
\end{equation}
where:
\begin{equation}
\begin{split}
\Psi_k^{(1),*} &= \sum_{l=0}^{N-1}\sum_{m=-l}^l \(\)
\end{split}
\label{psi_1_star}
\end{equation}   
This is satisfied under the assumption that $M_N D_N$ is the identity.
\Cref{O_1_terms} yields:
\begin{equation}
\begin{split}
&\bo_k \cdot \bn \Psi_k^{(0)}+\hat{\Sigma}_a(\br) \Psi_k^{(0)}(\br)\\
&=\frac{\Sigma_{tr}(\br)-\hat{\Sigma}_a(\br)}{2}
\(\(\frac{\partial}{\partial \mu}(1-\mu^2)\frac{\partial}{\partial \mu}+
\(\frac{1}{1-\mu^2}\)\frac{\partial^2}{\partial \varphi^2}\)\tilde{\Psi}^(0)
(\br,\bo)\)_{\bo=\bo_k}\\
&+ q(\br,\bo_k) + O\(\frac{\gamma}{\epsilon}\)
\end{split}
\label{last_pautz}
\end{equation}
where $k=1,\hdots,K$ and $\tilde{\Psi}^{(0)}(\br,\bo)$ is an interpolant
through the points $\{\bo,_k,\Psi_k^{(0)}(\br,\bo)\}$. Thus, if
$O\(\frac{\gamma}{\epsilon}\) \rightarrow 0$ as $\epsilon \rightarrow 0$,
\cref{last_pautz} is a pseudo-spectral discretization of the angular
variable in the exact Fokker-Planck equation. (Pseudo-spectral methods use
collocation to determine coefficients in global function expansions.) In
one-dimensional slab and spherical geometry, $\tilde{\Psi}^{(0)}(\br,\bo)$ is
the $(N-1)$-order polynomial interpolant through the points
$\{\bo_k,\Psi_k^{(0)}(\br)\}$. The definition in multidimensional geometry
will be described in the following discussion.

The foregoing discussion indicated that the transformation from discrete
values to angular moments and back to discrete values to angular moments and
back to discrete values should be the identity. If \cref{phi_p,psi_1_star} define 
the discrete-to-moments and moments-to-discrete transformations, then we will 
not have the identity unless the quadrature set is Gauss-Legendre in one-dimensional 
slab or spherical geometry. Given a different quadrature set of multidimensional 
geometry then, the $S_N$ method may not limit to a discretization of the 
Fokker-Planck equation unless \cref{phi_p} and/or \cref{psi_1_star} is replaced.

Morel\cite{galerkin_morel} reached the same conclusion via a completely
different analysis and offered suggestions for replacing the offending
equation(s). The simplest suggestion in one-dimensional slab and spherical
geometry is to use $\tilde{\phi}_{l,0}$ the exact moments of the $(N-1)$-order
polynomial, $\tilde{\Psi}$, that goes through the points
$\{\bo_k,\Psi_k(\br)\}$; thereby redefining $D_N$ to be $M_N^{-1}$. Morel
labeled this the ``Galerkin'' quadrature set, since he derived it by means of
a Galerkin weighting method. The use of the exact moments causes
\cref{psi_1_star_old} to be satisfied regardless of quadrature set, and
\cref{last_pautz} then follows.

In multidimensional geometries, the Galerkin quadrature has a more complex
definition. Recall that there are fewer moments than discrete angles in
standard multidimensional implementations of the discrete ordinates method.
To satisfy \cref{psi_1_star_old} in all circumstances, we must first
increase the number of spherical harmonics in our flux expansion by using
harmonics of higher orders. Morel\cite{galerkin_morel} and Reed\cite{reed}
proposed suitable spherical harmonics interpolation space for multidimensional
geometries. For two-dimensional geometries, the following interpolation space
is suggested:
\begin{equation}
Y_l^m = \left\{
\begin{aligned}
&0\leq m \leq n, & \textrm{if }0\leq l\leq N-1 \\
&0< l\textrm{ odd }\leq N, & \textrm{if }l=N
\end{aligned}
\right.
\end{equation}
The interpolation space suggested for three dimensions is:
\begin{equation}
Y_l^m = \left\{
\begin{aligned}
&-l\leq m \leq l, & \textrm{if }0\leq l \leq N-1\\
&-l\leq m <0 \textrm{ and } 0<l\textrm{ odd }\leq l, & \textrm{if }l=N\\
&-l\leq m\textrm{ even }<0,& \textrm{l=N+1}
\end{aligned}
\right.
\end{equation}
The Galerkin quadrature is then defined by adjusting the limits of the
summations in \cref{psi_1_star} to augment $M_N$ and then redefining
$D_N=M_N^{-1}$. As in the one-dimensional case, \cref{psi_1_star_old} will be 
satisfied regardless of the discrete angle set when the Galerkin treatment is 
used, and \cref{last_pautz} then follows, where $\tilde{\Psi}^{(0)}(\br,\bo)$ 
is now defined as the spherical harmonic interpolant through the points 
$\{\bo_k,\Psi_k^{(0)}(\br)\}$.

We note that the use of the Galerkin quadrature allows the selection of a
greater variety of discrete angle sets in \cref{discr_eps} since the
corresponding quadrature weights (if they are defined) are not actually used.

More can be said about the effects of using a non-Gaussian or non-Galerkin
quadrature to evaluate \cref{phi_p}. Let us define the scattering
ratio matrix $\bs{C}$ by:
\begin{equation}
\bs{C} = \frac{1}{\Sigma_t}\bs{D}\bs{M}\bs{\Sigma}
\end{equation}
where:
\begin{itemize}
\item $\bs{D}$ = discrete-to-moments matrix
\item $\bs{M}$ = moments-to-discrete matrix
\item $\bs{\Sigma}$ = diagonal matrix whose entries are the scattering
coefficients $\{\Sigma_{s,l}\}$ in the order and frequency corresponding to
their respective moments in the other matrices.
\end{itemize}
If exact integrals are used, then $\bs{D}=\bs{M}^{-1}$, and $\bs{C}$ will be a
diagonal matrix whose entries are the scattering ratios
$\frac{\Sigma_{s,l}}{\Sigma_t}$. In a nonmultiplying medium, each diagonal
term (and hence each eigenvalue of $\bs{C}$) will be nonnegative and will not
exceed unity. If, however, inexact integrations are used, not only will
$\bs{C}$ differ from $\frac{\bs{\Sigma}}{\Sigma_t}$, but there is also the
possibility of introducing one or more eigenvalues whose absolute values
exceed unity. This is physically equivalent to artificially introducing
multiplication into the medium. Depending on the amount of leakage present,
\cref{discr_eps,phi_p} may not have a steady-state
solution as $\epsilon\rightarrow 0$. This inconsistency is clearly
unacceptable when a steady-state solution is known to exist.

\red{Aussi dand \cite{galerkin_morel}}
We note that while the condition $D_N=M_N^{-1}$ is certainly sufficient for
obtaining the correct Fokker-Planck limit, it is not strictly necessary. We
need only to satisfy \cref{identity_a}, i.e., that $\tilde{\Psi}$ be
in the null space of $I-M_ND_N$. If $M_N D_N\neq I$, then certain angular
eigenmodes cannot be present in a stable solution, i.e., $\tilde{\Psi}$ must
be in a restricted subspace of the domain of $M_N D_N$. It is entirely
possible that a contrived selection of boundary conditions and sources could
result in a solution that does not contain any of the unstable modes; however,
this would suggest that physically realistic boundary conditions and
sources could result in a solution that does not contain any of the unstable
modes. Alternatively, one could filter out the unstable mode components of the
scattering source; this would stabilize the solution, but this would yield a
different solution than that obtained when exact integrations are used. Our
recommendation is to avoid these complications altogether by simply using the
exact inverse of $M_N$.\\

\red{Pris de \cite{galerkin_morel}}\\
We begin by considering the interpolatory trial space angular flux
representation on which this method is based:
\begin{equation}
\Psi(\mu) = \sum_{m=1}^N \Psi_m B_m(\mu)
\label{psi_b}
\end{equation}
Methods for generating the interpolatory basis function appearing in 
\cref{psi_b} are discussed in \cite{galerkin_morel}. The scattering source
corresponding to this representation is given by:
\begin{equation}
S(\mu) = \int_{4\pi} \Sigma_s(\mu_0)\Psi(\mu')d\bo'
\label{S_mu}
\end{equation}
where:
\begin{equation}
\mu_0 = \mu'\mu+\sqrt{\(1-\mu^{',2}\)\(1-\mu^2\)} \cos\(\varphi\)
\end{equation}
Expanding the cross section in Legendre polynomials, we obtain:
\begin{equation}
\Sigma_s(\mu_0) = \sum_{l=0}^{\infty} \frac{2l+1}{2}\Sigma_{s,l} P_l(\mu_0)
\label{sigma_s_mu}
\end{equation}
where:
\begin{equation}
\Sigma_{s,l} = \int_{-1}^{1} \Sigma_s(\mu_0)P_l(\mu_0) d\mu_0
\end{equation}
Substituting \cref{sigma_s_mu} into \cref{S_mu}, using the
spherical harmonics addition theorem to express $P_l(\mu_0)$ in terms of
$\mu'$ and $\mu$ and completing the integrations over $\mu$ and $\varphi$, we
get:
\begin{align}
&S(\mu) = \sum_{l=0}^{\infty} \frac{2l+1}{2} \xi_l P_l(\mu) \label{S_mu_2}\\
&\xi_l = \Sigma_{s,l} \phi_l \label{xi_l_s}\\
&\phi_l = \int_{-1}^1 \Psi(\mu)P_l(\mu)d\mu \label{phi_l_int}
\end{align}
Let $\tilde{S}(\mu)$ denote the trial approximation to $S(\mu)$ that we seek:
\begin{equation}
\tilde{S}(\mu) = \sum_{m=1}^N \tilde{S}_m B_m(\mu)
\label{tilde_s_mu}
\end{equation}
As previously stated, the Galerkin method requires that the residual
associated with the trial space approximation be orthogonal to the weighting
space. The residual associated with \cref{tilde_s_mu} is given by:
\begin{equation}
R(\mu) = \tilde{S}(\mu) - S(\mu)
\label{R_mu}
\end{equation}
The weighting space associated with our method is the space of global
polynomials of degree $N-1$ or less form a basis for this space, it follows
that the residual must satisfy:
\begin{equation}
\int_{-1}^1 R(\mu) P_l(\mu) d\mu = 0.0,\ \ l=0,\hdots,N-1
\label{int_r_mu}
\end{equation}
Substituting from \cref{S_mu_2,tilde_s_mu,R_mu}) into \cref{int_r_mu}, 
performing the integration, rearranging the resulting equation, we obtain:
\begin{align}
&\tilde{\xi}_l = \xi_l,\ \ l=0,\hdots,N-1 \label{xi_vs_xi}\\
&\tilde{\xi}_l = \int_{-1}^1 \tilde{S}(\mu) P_l(\mu) d\mu\\
&\xi_l = \int_{-1}^1 S(\mu) P_l(\mu) d\mu
\end{align}
The central theme of our method becomes clear on examination of 
\cref{xi_vs_xi}: the discrete scattering source values are chosen such that
the interpolatory representation for that scattering source has the same
Legendre moments through degree $N-1$ as the exact scattering source
calculated with the interpolatory representation for the angular flux. Thus,
the Galerkin formalism, when applied with an appropriate weighting space,
yields a very straightforward approximation. One highly desirable property of
this approximation is that only the cross-section moments through degree $N-1$
are required. This property is directly due to our choice of weighting space.
Specifically, this property is obtained because all elements of the weighting
space are orthogonal to the Legendre polynomials of degree $N$ or greater.

The Galerkin scattering matrix can be conveniently represented in terms of
three matrices: the discrete-to-moment matrix, the cross-section matrix, and
the moment-to-discrete matrix. The discrete-to-moment matrix $D$ maps a vector
of discrete angular flux values to a corresponding vector of Legendre flux
moments. These flux moments are calculated with the interpolatory flux
representation defined by the discrete angular flux vector:
\begin{equation}
\bs{\phi} = \bs{D\Psi}
\end{equation}
where:
\begin{align}
&\bs{\phi} = (\phi_0,\phi_1,\hdots,\phi_{N-1})\\
&\bs{\Psi} = (\Psi_1,\Psi_2,\hdots,\Psi_{N-1})\\
&\bs{D}_{l,m} = \int_{-1}^1 B_m(\mu) P_l(\mu)d\mu \label{bs_D}
\end{align}
Note that \cref{bs_D} follows directly from \cref{psi_b,phi_l_int}:
\begin{equation}
\phi_l = \sum_{m=1}^{N} \psi_m \int_{-1}^1 B_m(\mu) P_l(\mu)d\mu
\end{equation}
The cross-section matrix maps the vector of flux moments to the vector of
scattering source moments:
\begin{equation}
\begin{split}
\bs{\xi} &= \bs{\Sigma \phi}\\
&= \bs{\Sigma D}\Psi
\label{bs_xi}
\end{split}
\end{equation}
where:
\begin{align}
&\bs{\xi} = (\xi_0,\xi_1,\hdots,\xi_{N-1})\\
&\bs{\Sigma} = diag(\Sigma_{s,0},\Sigma_{s,1},\hdots,\Sigma_{s,N-1})
\label{bs_sigma}
\end{align}
Note that \cref{bs_xi} follows directly from \cref{tilde_s_mu}. The 
moment-to-discrete matrix $M$ maps a vector of
Legendre moments to a corresponding vector of discrete angular flux values.
The trial space element obtained by interpolating the discrete flux values has
the Legendre moments defined by the moment vector. Thus, the
moment-to-discrete matrix matrix is inverse of the discrete-to-moment matrix:
\begin{equation}
\bs{M} = \bs{D}^{-1}
\end{equation}
Having defined the three constituent matrices, we proceed with the
construction of the Galerkin inscatter matrix. First, we note that the
vector of Legendre moments corresponding to the trial space element obtained
by interpolating the discrete angular flux values is given by:
\begin{equation}
\bs{\Phi} = \bs{D}\psi
\end{equation}
The Legendre moments of the scattering source calculated with the trial space
element and the exact scattering cross section are given by:
\begin{equation}
\bs{\xi} = \bs{\Sigma D}\Psi
\label{bs_xi_2}
\end{equation}
Our Galerkin approximation requires that the trial space element obtained by
interpolating the discrete scattering source value have the same moments as
those given by \cref{bs_xi_2}. Mathematically stated, the discrete
scattering source vector $\bs{S}$ must satisfy:
\begin{equation}
\bs{DS} = \bs{\Sigma D}\psi
\label{bs_ds}
\end{equation}
where:
\begin{equation}
\bs{S} = (S_1,S_2,\hdots,S_N)
\end{equation}
Solving \cref{bs_ds} for $\bs{S}$, we obtain the desired
relationship:
\begin{equation}
\bs{S} = \bs{M\Sigma D}\psi
\label{bs_s}
\end{equation}
The matrix $\bs{\Sigma}$ is the scattering matrix obtained with the spherical
harmonic approximation of degree $N-1$ (the $P_{N-1}$ approximation). Since
$\bs{M}$ is the inverse of $\bs{D}$, it follows that the Galerkin scattering
matrix $\bs{S}$ is equivalent to the $P_{N-1}$ scattering matrix $\bs{\Sigma}$
under the similarity transformation defined by the discrete-to-moments and
moment-to-discrete matrices. Furthermore, if a global polynomial trial space
is used, it follows from \cref{bs_s} that our Galerkin scattering
source is identical to the spherical harmonic scattering source. Thus, our
Galerkin quadrature method can be characterized as a type of generalized
spherical harmonics scattering source approximation. We fell that this
characterization leads to considerable insight into the basic nature of our
method.

It is easily shown that delta function scattering of the following type is
treated exactly with the Galerkin method:
\begin{equation}
\Sigma_s(\mu_0) = \delta(\mu_0-1)
\end{equation}
We begin by noting that the cross-section matrix defined by \cref{bs_sigma} 
is the identity matrix. Substituting the identity matrix for the cross-section 
matrix in \cref{bs_ds}, we obtain the following expression:
\begin{equation}
\bs{S} = \bs{MD}\psi
\label{bs_s2}
\end{equation}
Recalling that $\bs{M}$ is the inverse of $\bs{D}$, we find that \cref{bs_s2} gives:
\begin{equation}
\bs{S} = \psi
\end{equation}
which is the exact result. Exactly treating this type of scattering has two
important ramifications for charged-particle transport. First, in
charged-particle transport, cross sections of the following form arise:
\begin{equation}
\Sigma^{k\rightarrow g}(\mu_0) = \Sigma^{k\rightarrow g} \delta(\mu_0-1)
\end{equation}
where $\Sigma^{k\rightarrow g}(\mu_0)$ denotes the differential cross section
associated with a transfer from group $k$ to group $g$. Thus, it is essential
that delta function scattering be treated exactly. Second, for relativistic
electron scattering, the extended transport correction reduces within-group
scattering cross sections by two orders of magnitude or more. Without this
reduction, the dominance ratio are so close to unity in practical calculations
that source iteration has a prohibitively slow convergence rate. Thus, from
the viewpoint of computational efficiency, it is essential that the extended
transport correction be applied. Because the Galerkin method treats delta
function  scattering exactly, an extended transport correction of the
within-group cross-section moments leaves the solution
invariant\cite{morel_79}. This is a highly desirable property since it allows
us to dramatically reduce within-group scattering cross sections without any
loss of accuracy.

The scattering source representation given by \cref{bs_s2} is
algebraically equivalent to the standard $S_n$ expression, provided that the
$S_n$ calculation is performed with an $N$-point quadrature set and a
cross-section expansion of degree $N-1$. To demonstrate this, we first express
\cref{bs_s2} in summation form:
\begin{align}
&\phi_l = \sum_{m=1}^N D_{l,m} \Psi_m, & l=0,\hdots,N-1 \label{37_a}\\
&\xi_l = \Sigma_{s,l} \phi_l, & l=0,\hdots,N-1 \label{37_b}\\
&S_m = \sum_{l=0}^{N-1} M_{m,l}\xi_l, & m=1,\hdots,N \label{37_c}
\end{align}
The standard $S_n$ method is given by:
\begin{align}
&\phi_l = \sum_{m=1}^N \Psi_m P_l(\mu_m) w_m, & l=0,\hdots,N-1 \label{38_a}\\
&\xi_l = \Sigma_{s,l} \Phi_l, & l=0,\hdots, N-1 \label{38_b} \\
&S_m = \sum_{l=0}^{N-1}\xi_l \frac{2l+1}{2}P_l(\mu_m), & m=1,\hdots,N
\label{38_c}
\end{align}
By inspecting \cref{37_a,37_b,37_c,38_a,38_b,38_c}, it is clear that the
standard $S_n$ method can be obtained from \cref{37_a,37_b,37_c} simply by
redefining the matrices $D$ and $M$ as follows:
\begin{align}
& \bs{D}_{l,m} = P_l(\mu_m) w_m \label{39_a}\\
& \bs{D}_{m,l} = \frac{2l+1}{2} P_l(\mu_m) \label{39_b}
\end{align}
Since the same matrix multiplications are performed in both the standard
$S_n$ and Galerkin quadrature methods, the computational work required by
these methods is identical.

We have thus far compared the Galerkin and $S_n$ methods under the assumption
that the $S_n$ method is used with an $N$-pont quadrature set and a $P_{N-1}$
cross-section expansion. Before considering more general cases, we must first
discuss the concept os a cross-section ``expansion'' as it relates to the
Galerkin and $S_n$ approximations. Our Galerkin method with an $N$-point flux
representation is said to require cross -section expansions of degree $N-1$.
This is true insofar as the cross-section mometns through degree $N-$ are used
but, as previously discussed, we do not actually use the Legendre expansion of
degree $N-1$ accurately represents the cross section. Conversely, the standard
$S_n$ method actually uses the Legendre expansion to represent the cross
section. Thus, with the Galerkin method, one need not be concerned with the
accuracy of the Legendre expansion, but rather meed only be concerned with the
accuracy with which $\tilde{S}(\mu)$ approximates $S(\mu)$. This is a very
desirable property when the cross section is highly forward-peaked, and,
represent the main advantage of the Galerkin method relative to the standard
$S_n$ method. For instance, a delta function cross-section expansion of finite
degree is never converged, yet the scattering source for such a cross section
is treatedd exactly with the Galerkin method.

If the $S_n$ method is used with an expansion of degree >$N-1$, the Galerkin
method obviously require less computational work, but most importantly, we
argue that in general one cannot expect greater accuracy with the standard
$S_n$ method simply because more cross-section moments are being usede. On the
other hand, if the scattering is isotropic, it is clearly a disadvantage for
the Galerkin method if one must use $N-1$ expansion coefficients with all but
the first being zeros. Fortunately, if the cross-section moments, are nonzero
only through degree $L$ with $L<N-1$, the full matrix expression for the
scattering source need not to be used. Specifically, only the first $L+1$ rows
of the $\bs{D}$ matrix and the first $L+1$ columns of the $\bs{M}$ matrix need
to be retained:
\begin{align}
&\phi_l = \sum_{m=1}^N \bs{D}_{l,m} \Psi_m, & l=0,\hdots,L \label{40_a}\\
&\xi_l = \Sigma_{s,l}\phi_l,& l=0,\hdots,L \label{40_b}\\
&S_m = \sum_{l=0}^L \bs{M}_{m,l} \xi_l,& m=1,\hdots,N \label{40_c}
\end{align}
\Cref{40_a,40_b,40_c} are algebraically equivalent to the standard $S_n$
method with a cross-section expansion of degree $L$. Thus, the galerkin and
standrad $S_n$ methods require the same computational work, provided that the
cross-section expansion order used in the $S_n$ method is $\leq N-1$. The
standard $S_n$ method, a quadrature set simply consists of cosines and weights
that are used to construc the $\bs{M}$ and $\bs{D}$ matrices. In the Galerkin
method, these matrices are defined in terms of an angular flux interpolation
scheme. Thus, an $N$-pont Galerkin quadrature set consists of the quadrature
points, $\{\mu_m\}_{m=1}^N$, and the matrices $\bs{M}$ and $\bs{D}$, with the
restrictions that $\bs{D}$ must give the Legendre moments corresponding to a
specific interpolation of the angula flux values at the quadrature points, and
$\bs{M}$ must be the inverse of $\bs{D}$. As previously discussed, there is a
standard quadrature set, called the companion set, that corresponds to each
Galerkin quadrature set. This set consists of the Galerkin quadrature points
and the unique set of weights that give and exact integration of the Galerkin
trial space elements. It is easily seen from \cref{37_a} that these weights
compose the first row of the $\bs{D}$ matrix:
\begin{equation}
\begin{split}
\phi_0 &= \sum_{m=1}^N \bs{D}_{0,m}\\
&=\sum_{m=1}^N \psi_m w_m
\end{split}
\end{equation}
Standard $S_n$ codes generallly require weights that sum to unity. The
definitions that we have weights that sum to 2. To obtain matrices that are
consistent with weights that sum to unity, simply scale the $\bs{D}$ matrix by
$\frac{1}{2}$ and the $\bs{M}$ matrix by 2.

It is obviously of interest to ask if there are any Galerkin sets that are
equivalent to standard sets in the sense that each of the Galerkin sets gives
a scattering matrix identical to that obtained with the standard $S_n$ method
and the respective companion quadrature set. A Galerkin set that has this
property is said to be ``companion-equivalent''. There is at least one family
of Galerkin quadrature sets that has this property: the sets generated with
global polynomial interpolation at the Gauss quadrature cosines. These sets
have the standard full-range Gauss quadrature sets as their companion sets.
Thus, we refer to them as the full-range Galerkin Gauss sets. It is not
difficult to demonstrate that these sets are companion-equivalent. For
instance, it is clear that \cref{39_b} must be the expression for the $\bs{M}$
matrix whenever global polynomaial interpolation is used. Thus, we need only
demonstrate that \cref{39_a} is correct for the $\bs{D}$ matrix. We begin by
using the standard Gauss quadrature formula to evaluate \cref{bs_D}:
\begin{equation}
\bs{D}_{l,m} = \sum_{k=1}^N B_m(\mu_k) P_l(\mu_k) w_k
\label{gal_m_last}
\end{equation}
All of the interpolatory basis functions are polynomials of degree $N-1$.
Since the Legendre polynomials appearing in \cref{gal_m_last} are of degree
$N-1$ or less, the integrand in \cref{gal_m_last} must be a polynomial of
degree $2N-2$ or less. The $N$-point Gauss quadrature set is exact for all
polynomials of degree $2N-1$ or less. Therefore, \cref{gal_m_last} is exact.
As can be seen from \cref{bs_D}, the interpolatory basis functions are defined
such that the $m^{th}$ expansion coefficient is equal to the angular flux
evaluated at the $m^{th}$ interpolation cosine. By successively evaluating
\cref{bs_D} at eaxh cosine, one finds that the interpolatory basis functions
must satisfy:
\begin{equation}
B_m(\mu_k) = \left\{
\begin{aligned}
&1 & \textrm{for }m=k\\
&0 &\textrm{otherwise}
\end{aligned}
\right.
\label{llast}
\end{equation}
Substituting from \cref{llast} into \cref{gal_m_last}, we obtain:
\begin{equation}
\bs{D}_{l,m} = P_l(\mu_m)w_m
\label{lllast}
\end{equation}
\Cref{lllast} is the standard $S-n$ expression for the $\bs{D}$ matrix. Thus,
we have proved that the full-range Galerkin Gauss quadrature sets are
companion-equivalent.

There are no companion-equivalent Galerkin sets based on global polynomial
interpolation other than the Gauss sets. This follows from the fact that no
other symmetric $N$-point quadrature sets can integrat the polynomials of
degree $2N-2$ or less that appear in \cref{gal_m_last}. The sets with highest
integration accuracy after the Gauss sets are the Lobatto sets, which only
integrate polynomials of degree $2N-3$ or less. It seems highly unlikely that
there any Galerkin sets based on nonglobal polynomial interpolation that are
companion-equivalent, but we cannot prove it. Such an $N$-point Galerkin set
would have to have a companion set that integrates products of the non-global
polynomial trial space functions and global polynomials of degree $N-1$ or
less. In addition, the standard $S_n$ expression for the $\bs{M}$ matrix,
which is generally correct only for global polynomial trial spaces, would also
have to be correct for the nonglobal polynomial trial space. It seems highly
unlikely that all of these requirements can be met.

A Galerkin set differs from its companion set only if a cross-section
expansion of sufficiently high order is being used. For instance, with
isotropic scattering, every Galerkin set and its companion set are equivalent.
Most standard $S_n$ quadrature sets are designed to integrate either global of
half-range polynomials. Half-range sets can also integrate certain global
polynomials because global polynomials always form a subspace of any
half-range polynomials, the higher is the degree of the cross-section
expansion for which that standard set is equivalent to its corresponding
Galerkin set. For instance, the standard $N$-poin Lobatto quadrature set,
which integrates global polynomials of degree $2N-2$ or less, is equivalent to
the Galerkin $N$-point Lobatto set (generated with global polynomial
interpolation at the standard Lobatto points) for cross-section expansion of
degree $N-2$ or less. The standard double-Gauss N-point set, which integrates
global polynomials of degree $N-1$, is equivalent to the Galerkin double-Gauss
N-point set (generated with half-range polynomial interpolation at the
standard double-Gauss points) for cross-section expansions of degree $N/2-1$
or less. Even though a Galerkin set and its companion set may not be
rigorously equivalent, they may nonetheless give very similar results even
with highly anisotropic expansions. For instance, if one is comparing a
Galerkin doouble-Gauss $N$-point set with its companion set using a highly
anisotropic cross-section expansion of degree $N-1$ in a problem for which the
flux moments of degree $N/2$ and greater are very small, the two quadratures
will give almost identical results provided that the companion scattering
matrix is stable (i.e., all eigenvalues must have magnitudes less than or
equal to unity). This follows from the fact that these sets differ only with
respect to the treatment of the moments of degree $N/2$ or greater, and if
these moments are small, the differences in the respective soltuions must
similarly be small. However, if the companion scattering matrix is unstable,
anisotropic components of the angular flux solution that should be small can
be made arbitraily large, and arbitrarily large errors can occur.

In closing this section, we consider the effect of Galerkin quadrature on the
positivity of the scattering source. It is well known that highly truncated
Legendre expansions often give rise to negatie scattering source values along
directions where small but positive values should be obtained. we have alerady
shown that when Galerkin quadrature is used, poor convergence of the
cross-section expansion does not necessarily imply a porr scattering source
representation. For instance, all finite expansions for the delta function
$\delta(\mu_0-1)$ are poorly converged and negative over large segments of the
angualat domain, yet this type of scattering is treated exactly with Galerkin
quadrature. This suggests that Galerkin quadrature might give significantly
more positive scattering sources than the standard quadrature method when
highly truncated cross-section expansions are used. In general, we find that
the Galerkin quadrature method is somewhat less susceptible to negativity
problems than the standard method, but it is definitely not a solution for
such difficulties. This follows for two reasons. First and foremost, the
Galerkin method is based on a global preservation schem that gives rise to a
trial space scattering source $\tilde{S}(\mu)$ that represents a type of
generalized least-squares fit ot the ``exact'' scattering source $S(\mu)$.
Such fits do not guarantee global positivity, but they do tend to be both
positive and accurate at points wher the scattering source is ``large''.
Negativities generally onlu occur at points wher the scattering source is
small relative to the maximum value that occurs. As a result, we find that
good angle-integrated results can be expected with the Galerkin quadrature
method regardless of negativities in the truncated cross-section expansion,
but differential or half-range quantities may be inaccurate. Second, with
polynomial trial spaces, negativities in the scattering source are often due
solely to negativities in the the trial space angulat flux representation
itself. For instance, it can be deduced from \cref{S_mu_2,xi_l_s,phi_l_int}
that the exact scattering source resulting from a global polynomials of degree
$N-1$ is a global polynomial of degree $N-1$ or less. Thus, for such trial
spaces, the trial spcae scattering source and the exact scattering source are
identical. This implies that the trial space scattering source must be
positive whenever the trial space angulat flux is positive. Unfortunately,
positive flux values ate the interpolatory trial space representation. Since a
standard Gauss quadrature is equivalent to a Galerkin quadrature with a global
polynomial trial space, it follows that the standard $N$-point Gauss
scattering source with expansions of degree $N-1$ must be positive if the
global polynomial interpolating the discrete angular flux values is positive.
However, experience indicates that negativities often arise with $N$-point
Gauss quadratures and cross-section expansions of degree $N-1$ in both neutral
and charged-particle transport calculations. Thus, it would appear that
positivity of the Galerkin scattering source is unlikely to be achieved
without the use of a positive trial space representation (i.e., a
representation that is positive whenever the discrete angular flux values are
positive). Piecewise-constant trial spaces exhibit such positivity, but
initial results with such spaces have been disappointing. Out experiance leads
us to a suspect that a general solution to the negativity problem cannot be
achieved with a Galerkin method based on a weighting space of global
functions.
