\section{Fokker-Planck cross section}
%\red{Premiere partie redondante avec Ligou, on utilise \cite{morel_81}}
%\begin{equation}
%\bo\cdot\bn \Psi + \Sigma_a \Psi = \Gamma_B\Psi +Q
%\end{equation}                                
%with:
%\begin{equation}
%\Gamma_B \Psi = \int_{4\pi}
%\(\Sigma_s(E',\mu_0)\Psi(\mu',E)-\Sigma_s(E,\mu_0)\Psi(\mu,E)\)d\bo_0
%\label{gamma_b}
%\end{equation}
%where:
%\begin{equation}
%\mu_0 = \cos(\theta_0)
%\end{equation}
%and:
%\begin{equation}
%\mu' = \mu \cos(\theta_0) - \sqrt{1-\mu^2}\sin(\theta_0) \cos(\theta_0)
%\label{mu_p}
%\end{equation}
%Performing a Taylor a series expansion about $\theta_0=0$ and retaining terms
%up to $\theta_0^2$ gives:
%\begin{equation}
%\Sigma_s(E',\mu_0) \Psi(\mu',E) - \Sigma_s(E,\mu_0)\Psi(\mu,E) = I_1+I_2
%\end{equation}
%where:
%\begin{equation}
%I_1 = \Sigma_s(E,\mu_0) \(\frac{\partial}{\partial \mu}\Psi(\mu,E)
%\(\frac{\partial \mu'}{\partial \theta_0}\theta_0 +\frac{1}{2}
%\frac{\partial^2\mu'}{\partial \theta_0^2}\theta_0^2\) +\frac{1}{2}
%\frac{\partial^2}{\partial \mu^2}\Psi(\mu,E) \(\frac{\partial \mu'}{\partial
%\theta_0}\)^2 \theta_0^2 \)
%\label{I_1}
%\end{equation}
%and:
%\begin{equation}
%I_2 = \frac{\partial}{\partial E}\Sigma_s(E,\mu_0) \Psi(\mu,E)
%\(\frac{\partial E'}{\partial \theta_0}\theta_0 +
%\frac{1}{2}\frac{\partial^2 E'}{\partial \theta_0^2}\theta_0^2\)+\frac{1}{2}
%\frac{partial^2}{\partial E^2} \Sigma_s(E,\mu_0) \Psi(\mu,E) \(\frac{\partial
%E'}{\partial \theta_0}\)^2 \theta_0^2
%\label{I_2}
%\end{equation}
%Using \cref{mu_p} to evaluate the derivatives in \cref{I_1} gives:
%\begin{equation}
%I_1 = \Sigma_s(E,\mu_0) \theta_0^2 \(-\frac{\mu}{2}\frac{\partial
%\Psi}{\partial \mu} + \frac{(1-\mu^2)}{2} (\cos(\varphi_0))^2 \frac{\partial^2
%\Psi}{\partial \mu^2}\)
%\label{I_1_bis}
%\end{equation}      
%Because $E=E_s(E',\theta_0)$ and $E'=E$ when $\theta_0=0$, it follows that:
%\begin{equation}
%\begin{split}
%\frac{\partial E'}{\partial \theta_0}\theta_0 + \frac{1}{2}\frac{\partial^2
%E'}{\partial \theta_0^2}\theta_0^2 +\hdots &= -\frac{\partial E_s}{\partial
%\theta_0} \theta_0 - \frac{1}{2} \frac{\partial E_s}{\partial
%\theta_0^2}\theta_0^2+\hdots\\
%&= E-E_s(E,\theta_0) 
%\end{split}
%\end{equation}
%Taking this into account, \cref{I_2} becomes:
%\begin{equation}
%I_2 = \frac{\partial}{\partial E}\Sigma_s(E,\mu_0) \Psi(\mu,E) (E-E_s) +
%\frac{1}{2}\frac{\partial^2}{\partial E^2} \Sigma_s(E,\mu_0) \Psi(\mu,E)
%(E-E_s)^2
%\label{I_2_bis}
%\end{equation}
%Substituting \cref{I_1_bis,I_2_bis} into \cref{gamma_b} and integrating gives:
%\begin{equation}
%\Gamma_B \Psi \approx \Gamma_{FP} \Psi
%\end{equation}
%where:
%\begin{equation}
%\Gamma_{FP}\Psi = \frac{\alpha}{2} \frac{\partial}{\partial \mu} (1-\mu^2)
%\frac{\partial}{\partial \mu} \Psi + \frac{\partial}{\partial E} \beta \Psi +
%\frac{1}{2}\frac{\partial^2}{\partial^2}\gamma \Psi
%\label{gamma_fp}
%\end{equation}
%and:
%\begin{align}
%\alpha &= 2\pi \int_{-1}^1 \Sigma_s(E,\mu_0) (1-\mu_0) d\mu_0 \label{alpha}\\
%\beta &= \int_0^{\infty} \Sigma_s(E\rightarrow E') (E-E') dE' \label{beta}\\
%\gamma &= \int_0^{\infty} \Sigma_s(E\rightarrow E') (E-E')^2 dE' \label{gamma}
%\end{align}



We recall \cref{bfp}:
\begin{equation}
\begin{split}
&\bo\cdot\Psi+(\Sigma_s'+\Sigma_a)\Psi=\sum_{l=0}^{\infty}\sum_{m=-l}^l
Y_l^m(\bo)\int_E^{E/\alpha}\Sigma_{s,l}(E'\rightarrow
E)\phi_{l,m}(E')dE'+\\
&\frac{\partial}{\partial E}S''\Psi+T''\[\frac{\partial}{\partial
\mu}(1-\mu^2) \frac{\partial}{\partial
\mu}+\frac{1}{(1-\mu^2)}\frac{\partial^2}{\partial\varphi^2}\]\Psi+Q
\end{split}
\end{equation}

We refer to $\mc{L}_{FP}$ as the Fokker-Planck scattering operator. The
function $\alpha=2T$, $S$ and $R$ are known as the momentum transfer,
stopping power, and means square stopping power, respectively. We refer to
them collectively as the Fokker-Planck functions. The equation that we seek to
solve:
\begin{equation}
\bo\cdot\bn\Psi +\Sigma_a \Psi = \mc{L}_{FP} \Psi+Q
\end{equation}
%If the angular flux is sufficiently smooth, the expansions given in \cref{I_1,I_2} 
%are accurate for ``small'' values of $\theta_0$.
%If the scattering is highly forward peaked, the integrand in \cref{gamma_b} is 
%significantly nonzero over only a ``small'' region about
%$\theta_0=0$. 
If the angular flux is sufficiently smooth and the scattering is higly forward
peaked, one would expect that with sufficiently forward that with
sufficiently forward-peaked scattering, the dominant contribution to the
Boltzmann integral would come from the range of $\theta_0$ values over which
the expansions are accurate and that $\mc{L}_{FP}\Psi$ would thereby closely
approximate $\mc{L}_B\Psi$. Experience has shown that good results for scalar
quantities (angle and energy integrated), such as energy and charge deposition
profiles, can be expected in charged-particle calculations. However, results
for detailed differential quantities are generally inadequate. The
Fokker-Planck equation is very useful in spite of this deficiency because it
is the scalar rather than differential quantities that are most often of
applied interest.

Let us define:
\begin{align}
\mc{L}_{FP}^{\alpha} \Psi &= \frac{\alpha}{2} \frac{\partial}{\partial \mu}
(1-\mu^2) \frac{\partial}{\partial \mu}\Psi \label{gamma_alpha}\\
\mc{L}_{FP}^e\Psi &=\frac{\partial}{\partial E}S\Psi +
\frac{1}{2}\frac{\partial^2}{\partial E^2} T \Psi \label{gamma_e}
\end{align}
We note that $\mc{L}_{FP}^{\alpha}$ causes particles to redistribute in
direction without energy change, while $\mc{L}_ {FP}^e$ causes particles to
redistribute particle in energy without directional change. This would suggest
that $\mc{L}_{FP}^{\alpha}$ should be approximated with a cross section of the
form:
\begin{equation}
\Sigma_s(E'\rightarrow E,\mu_0) = \Sigma_s^{\alpha}(E,\mu_0) \delta(E'-E)
\end{equation}
while $\Gamma_{FP}^e$ should be approximated by a cross section of the form:
\begin{equation}
\Sigma_s(E'\rightarrow E,\mu_0) = \Sigma_s^e(E'\rightarrow E) \frac{1}{2\pi}
\delta(\mu_0-1)
\end{equation}
Additional justification for this decoupled approximation is found in the fact
that the Fokker-Planck scattering operator
can also be derived assuming a composite decoupled cross section:
\begin{equation}
\Sigma_s(\mu_0,E'\rightarrow E) = \Sigma_s^{\alpha}(\mu_0,E) \delta(E'-E) 
+\Sigma_s^e(E'\rightarrow E) \frac{1}{2\pi} \delta(\mu_0-1)
\end{equation}
As one would expect, the derivation is very similar to that for the elastic
(coupled) cross section. Hence, we only give a brief outline of it.

The Boltzmann scattering operators corresponding to $\Sigma_s^{\alpha}$ and
$\Sigma_s^e$ are given by:
\begin{align}
& \mc{L}_B^{\alpha} = \int_{4\pi} \Sigma_s^{\alpha}(E,\mu_0)
\(\Psi(\mu')-\Psi(\mu)\) d\bo' \label{gamma_b_alpha}\\
& \mc{L}_B^e = \int_{0}^{\infty} \(\Sigma_s^e(E'\rightarrow E) \Psi(E') -
\Sigma_s^e(E\rightarrow E') \Psi(E)\)dE' \label{gamma_b_e}
\end{align}
Expanding the integrand in \cref{gamma_b_alpha} about $\theta_0=0$,
where $\theta_0$ keeping terms up to $\theta_0^2$ and integrating gives:
\begin{equation}
\mc{L}_B^{\alpha} = \mc{L}_{FP}^{\alpha}\Psi
\end{equation}
with:
\begin{equation}
\alpha = 2\pi \int_{-1}^{1} \Sigma_s^{\alpha} (E,\mu_0) (1-\mu_0) d\mu_0
\end{equation}
Expanding the integrand in \cref{gamma_b_e} about $\tau=0$, where
$\tau=E'-E$, keeping terms up to $\tau^2$, and integrating gives:
\begin{equation}
\mc{L}_B^e\Psi =\mc{L}_{FP}^e\Psi
\end{equation}
with:
\begin{align}
&S = 2\pi \int_0^{\infty}\int_{-1}^1 \Sigma_s^e(\mu_0,E\rightarrow E')
(E-E')d\mu_0 dE'\\
&R = 2\pi\int_0^{\infty}\int_{-1}^1  \Sigma_s^e(E\rightarrow E') (E-E')^2
d\mu_0 dE'
\end{align}
An argument, completely analogous to that presented for the coupled case, can
be used to show $\mc{L}_B^{\alpha}$ and $\mc{L}_{FP}^{\alpha}$ better
approximate one another as $\Sigma^{\alpha}$ becomes increasingly peaked about
$\mu_0=1$. For similar reasons, $\mc{L}_B^e$ and $\mc{L}_{FP}^e$ better
approximate one another as $\Sigma_s^e$ becomes increasingly peaked about
$E'=E$. 

\subsection{Coefficients for $\Sigma^{\alpha}$}
Because $\Sigma_s^{\alpha}$ does not change particle energy, it corresponds to a
within-group cross section. Thus, we need only to define the Legendre
coefficients, $\{\Sigma_{s,l}^{\alpha}\}_{l=0}^{N-1}$. This is fairly
straightforward to do once two fundamental properties of $\mc{L}_B^{\alpha}$
and $\mc{L}_{FP}^{\alpha}$ are noted. Let $\{P_l(\mu)\}_{l=0}^{\infty}$ denote
the Legendre polynomials. By expanding $\Sigma_s^{\alpha}$ in such polynomials
and invoking the addition theorem for the spherical harmonics, the following
well-known result can be demonstrated:
\begin{equation}
\mc{L}_B^{\alpha}P_l(\mu) = (\Sigma_{s,l}^{\alpha}-\Sigma_{s,0}^{\alpha}) P_l(\mu_0)
\label{gamma_b_alpha_p}
\end{equation}
with:
\begin{align}
&l=0,\hdots,\infty\\
&\Sigma_{s,l}^{\alpha} = 2\pi \int_{-1}^1 \Sigma_s^{\alpha} (E,\mu_0) P_l(\mu_0)
d\mu_0
\end{align}
Using standard recurrence relations for the Legendre polynomials, it is not
difficult to show that:
\begin{equation}
\mc{L}_{FP}^{\alpha} P_l(\mu) = \frac{\alpha}{2} l(l+1) P_l(\mu)
\label{gamma_fp_alpha_p}
\end{equation}
If the angular flux is expressible as a polynomial of arbitrary degree $L$:
\begin{equation}
\Psi(\mu) = \sum_{l=0}^L \frac{2l+1}{4\pi} \phi_l P_l(\mu)
\end{equation}
then it follows from \cref{gamma_b_alpha_p,gamma_fp_alpha_p} that we can define 
$\Sigma_s^{\alpha}$ so that:
\begin{equation}
\mc{L}_B^{\alpha}\Psi=\mc{L}_{FP}^{\alpha}\Psi
\end{equation}
by setting:
\begin{equation}
\Sigma_{s,0}^{\alpha}-\Sigma_{s,l}^{\alpha} = \frac{\alpha}{2}l(l+1)
\label{sigma_m_sigma}
\end{equation}
with $l=1,\hdots,L$.\\
Choosing $\Sigma_{s,L}=0$ so as to minimize the resulting value for
$\Sigma_{s,0}^{\alpha}$, \cref{sigma_m_sigma} gives:
\begin{equation}
\Sigma_s^{\alpha}(E,\mu_0) = \sum_{l=0}^L \frac{2l+1}{4\pi} \Sigma_{s,l}^{\alpha}
P_l(\mu_0)
\end{equation}
where:
\begin{equation}
\Sigma_{s,l}^{\alpha} = \frac{\alpha}{2}\(L(L+1)-l(l+1)\),\ \  l=0,\hdots,L
\end{equation}
Thus, with an expansion degree of $L$ for $\Sigma_s^{\alpha}$, we see that
$\mc{L}_B^{\alpha}$ and $\mc{L}_{FP}^{\alpha}$ are completely equivalent when
operating on polynomials of degree $L$ or less. With appropriate quadrature
sets and expansion orders, the $S_n$ representation for $\mc{L}_{FP}^{\alpha}$
is equivalent to that obtained by interpolating the discrete angular flux
values with a polynomial and operating on that polynomial with
$\mc{L}_{FP}^{\alpha}$.

It is instructive to consider the behavior of $\Sigma_s^{\alpha}$ in the limit
as the degree of the expansion is increased. Previous analysis show that
$\Sigma_s^{\alpha}$ better approximated $\mc{L}_B^{\alpha}$ as it becomes
increasingly forward-peaked. In addition, it should obviously have the correct
momentum transfer. Thus, we first take note that regardless of expansion order
the momentum transfer of $\Sigma_s^{\alpha}$ is exact:
\begin{equation}
\begin{split}
2\pi \int_{-1}^1 \Sigma_s^{\alpha}(E,\mu_0) (1-\mu_0) d\mu_0 &=
\Sigma_{s,0}-\Sigma_{s,1}\\
&=\frac{\alpha}{2} L(L+1) - \frac{\alpha}{2} (L(L+1)-2)\\
&=\alpha
\end{split}
\end{equation}
The average cosine of the scattering angle can be expressed as follows:
\begin{equation}
\begin{split}
\bar{\mu}_0 &= \frac{\Sigma_{s,1}^{\alpha}}{\Sigma_{s,0}^{\alpha}}\\
&=\frac{L(L+1)-2}{L(L+1)}
\end{split}
\end{equation}
It is easily seen that as $L$ increased, $\bar{\mu}_0$ goes to unity. Thus,
$\Sigma_s^{\alpha}$ becomes increasingly forward-peaked as $L$ increases. The
total magnitude of $\Sigma_s^{\alpha}$ becomes unbounded in the same limit:
\begin{equation}
\Sigma_{s,0}^{\alpha} = \frac{\alpha}{2} L (L+1)
\end{equation}
Expressing the momentum transfer as follows:
\begin{equation}
\alpha = \Sigma_{s,0}^{\alpha} (1-\bar{\mu}_0)
\end{equation}
it is seen that $\Sigma_{s,0}^{\alpha}$ must increase without bound if the
momentum transfer is to remain constant as $\Sigma_s^{\alpha}$ becomes
increasingly forward-peaked. This show that $\mc{L}_{FP}^{\alpha}$ corresponds
to a type of continuous-deflection approximation. The $\mc{L}_{FP}^{\alpha}$
operator effectively causes particles to scatter continuously while incurring
a differential deflection in each scattering event. The net result is that
particles continuously deflect with the mean deflection per unit pathlength
given by the momentum transfer.

\red{Vu qu'on fait que du 1 groupe, pas la peine de regarder aux autres
termes, meme development dans \cite{morel_89}}

\red{Partie prise de  \cite{morel_96}}
%In this section, we take the first step in the development of our approximate
%BFP equation by replacing the angular Fokker-Planck operator with a Boltzmann
%operator. The angular Fokker-Planck operator is replaced with the scattering
%operator $B_{\alpha}$:
%\begin{equation}
%B_{\alpha} = \int_{4\pi} \Sigma_{\alpha} (E,\mu_0)
%\Psi(\bo',E)d\bo'-\Sigma_{\alpha}\Psi
%\end{equation}
%where:
%\begin{equation}
%\Sigma_{\alpha}(E,\mu_0) = \frac{\alpha(E)}{1-\mu_s} \frac{1}{2\pi}
%\delta(\mu_0-\mu_s)
%\end{equation}
%and where $\mu_s$, which we refer to as the Fokker-Planck scattering angle, is
%a variable parameter. Note that there is no energy loss associated with
%$B_{\alpha}$ and further that there is only one polar angle of scatter. The
%cosine of this scattering angle is equal to $\mu_s$. To avoid confusion
%between the smooth-component Boltzmann operator and $B_{\alpha}$, we refer to
%all quantities associated with the smooth-component Boltzmann operator as
%``smooth-component'' quantities and all quantities associated with
%$B_{\alpha}$ as ``continuous-scattering'' quantities. For instance, we refer
%to $\Sigma_{\alpha}$ as the continuous-scattering cross section and to
%$\Sigma_t$ as the smooth-component total cross section.
%
%We now demonstrate that $B_{\alpha}$ converges to the continuous-scattering
%operator int the limit as $\mu_s\rightarrow 1$. In particular, we show that
%any fixed number of eigenvalues and eigenfunctions of the
%continuous-scattering operator can be preserved to an arbitrary degree of
%accuracy as a function of $\mu_s$. Expanding the scattering cross section in
%Legendre polynomials, applying the spherical-harmonics addition theorem, and
%then operating on the function $Y_l^m$, we find that:
%\begin{equation}
%B_{\alpha} Y_l^M = \(\Sigma_{\alpha}^l-\Sigma_{\alpha}^0\)Y_l^m
%\end{equation}
%where:
%\begin{equation}
%\Sigma_{\alpha}^0 = \frac{\alpha}{1-\mu_s}
%\label{sigma_a_0}
%\end{equation}                            
%and:
%\begin{equation}
%\Sigma_{\alpha}^l = \frac{\alpha}{1-\mu_s} P_l(\mu_s)
%\end{equation}
%From \cref{sigma_a_0}, it follows that the spherical-harmonic
%functions are eigenfunctions of $B_{\alpha}$.
%The associated Legendre function $P_l^m$ satisfies Legendre's associated
%differential equation:
%\begin{equation}
%\frac{\partial}{\partial \mu} \((1-\mu^2)\frac{\partial}{\partial \mu}P_l^m\)
%+ \(l(l+1) - \frac{m^2}{1-\mu^2}\)P_l^m=0
%\label{fp_op_eig}
%\end{equation}
Applying the Fokker-Planck angular operator to $Y_l^m$ and using
\cref{eigenvalue}, we find that:
\begin{equation}
\mc{L}_{FP}^{\alpha}Y_l^m = -\frac{\alpha}{2} l(l+1) Y_l^m
\label{F_a_y_l_m}
\end{equation}
where:
\begin{equation}
\mc{L}_{FP}^{\alpha} \Psi = \frac{\alpha}{2}\(\frac{\partial}{\partial \mu} 
\((1-\mu^2) \frac{\partial \Psi}{\partial \mu}\) +
\frac{1}{1-\mu^2}\frac{\partial^2\Psi}{\partial \varphi^2}\)
\end{equation}
It follows from \cref{F_a_y_l_m} that the spherical-harmonic
functions are eigenfunctions of the Fokker-Planck angular operator. Thus, the
spherical-harmonics functions are the eigenfunctions of both $\mc{L}_B^{\alpha}$ and
the angular Fokker-Planck operator.
To show that $\mc{L}_B^{\alpha}$ becomes equivalent to $\mc{L}_{FP}^{\alpha}$ in 
the limit as $\mu_s\rightarrow 1$, we expand the eigenvalues of $\mc{L}_B^{\alpha}$ 
in a first-order Taylor series about $\mu_s=1$:
\begin{equation}
\begin{split}
\Sigma_{s,l}^{\alpha} - \Sigma_{s,0}^{\alpha} &= \frac{\alpha}{1-\mu_s}
(P_l(\mu_s)-1)\\
&=\frac{\alpha}{1-\mu_s} \(P_l(1)+\frac{\partial}{\partial x}P_l(x)\Big|_{x=1}
(\mu_s-1)-1\)
\end{split}
\label{sig_comp}
\end{equation}
Using \cref{p_l,p_l_p}, \cref{sig_comp} becomes:
\begin{equation}
\Sigma_{s,l}^{\alpha} - \Sigma_{s,0}^{\alpha} = -\frac{\alpha}{2}l(l+1)
\end{equation}
We find that the eigenvalues of $B_{\alpha}$ converge to the eigenvalues of
$F_{\alpha}$ in the limit as $\mu_s\rightarrow 1$. Thus, these two operators
become equivalent in this limit.

The error in each eigenvalue monotonically decreases as $\mu_s$ is increased,
but for any fixed value of $\mu_s$, the error monotonically increases with
increasing eigenvalue index. Thus, while $B_{\alpha}$ does in fact converge to
$\mc{L}_{FP}^{\alpha}$ as $\mu_s\rightarrow 1$, it does so nonuniformly. Thus, for any
fixed value of $\mu_s$, the high-order eigenvalues of $\mc{L}_{FP}^{\alpha}$ are 
grossly underestimated by $\mc{L}_B^{\alpha}$. Fortunately, this is error in the 
high-order eigenvalues is usually unimportant. To explain why this is so, we first we
first consider the following time-dependent transport equation:
\begin{equation}
\frac{\partial}{\partial t}\Psi(t,\mu) = \frac{\alpha}{2}
\frac{\partial}{\partial \mu} \((1-\mu^2)\frac{\partial \Psi}{\partial \mu}\)
\end{equation}
with initial condition:
\begin{equation}
\Psi(0,\mu) = \frac{1}{2\pi} \delta(\mu-1)
\label{init_cond}
\end{equation}
The solution to \cref{init_cond} is:
\begin{equation}
\Psi = \frac{1}{2\pi} \sum_{l=0}^{\infty} \exp(-\lambda_l t)P_l(\mu)
\label{exp}
\end{equation}
where $\lambda_l$ is the eigenvalue of $\mc{L}_{FP}^{\alpha}$ with Legendre index $l$.
Note from \cref{exp} that the $l^{th}$ Legendre moment of the
angular flux solution is exponentially attenuated in proportion to
$\lambda_l$. Since these eigenvalues become unbounded for large $l$, it
follows that the high-order flux moments will be ``very small'' for all
nonzero values of $t$. A gross underestimate of $\lambda_l$ will still result
in a very small flux moment if the approximate of eigenvalue is large
relative to $1/t$. Thus, for sufficiently large times, an accurate solution
can be obtained even though the high-order eigenvalues are grossly
underestimated. 

This example illustrates the following basic idea. In problems with highly
forward-peaked scattering, the high-order flux moments are much more rapidly
attenuated than the low-order moments as the particle distribution evolves in
space and time. Thus, errors in the eigenvalues are large relative to the
temporal and spatial scale lengths associated with a given problem.

