\section{Fokker-Planck cross section}
We recall the BFP equation \cref{bfp}:
\begin{equation}
\begin{split}
&\bo\cdot\psi+(\Sigma_s'+\Sigma_a)\psi=\sum_{l=0}^{\infty}\sum_{m=-l}^l
Y_l^m(\bo)\int_E^{E/\alpha}\Sigma_{s,l}'(E'\rightarrow
E)\phi_{l,m}(E')dE'+\\
&\frac{\partial(S\psi)}{\partial E}+\frac{\alpha}{2}\(\frac{\partial}{\partial
\mu}(1-\mu^2) \frac{\partial}{\partial
\mu}+\frac{1}{(1-\mu^2)}\frac{\partial^2}{\partial\varphi^2}\)\psi+Q
\end{split}
\end{equation}
where we used $\alpha = 2T$.
Let us define:
\begin{align}
\mc{L}_{FP}^{\alpha} \psi &= \frac{\alpha}{2} \frac{\partial}{\partial \mu}
(1-\mu^2) \frac{\partial}{\partial \mu}\psi \label{gamma_alpha}\\
\mc{L}_{FP}^e\psi &=\frac{\partial}{\partial E}S\psi +
\frac{1}{2}\frac{\partial^2}{\partial E^2} R \psi \label{gamma_e}
\end{align}
We note that $\mc{L}_{FP}^{\alpha}$ causes particles to redistribute in
direction without energy change, while $\mc{L}_ {FP}^e$ causes particles to
redistribute particle in energy without directional change. This would suggest
that $\mc{L}_{FP}^{\alpha}$ should be approximated with a cross section of the
form:
\begin{equation}
\Sigma_s(E'\rightarrow E,\mu_0) = \Sigma_s^{\alpha}(E,\mu_0) \delta(E'-E)
\end{equation}
while $\Gamma_{FP}^e$ should be approximated by a cross section of the form:
\begin{equation}
\Sigma_s(E'\rightarrow E,\mu_0) = \Sigma_s^e(E'\rightarrow E) \frac{1}{2\pi}
\delta(\mu_0-1)
\end{equation}
Additional justification for this decoupled approximation is found in the fact
that the Fokker-Planck scattering operator
can also be derived assuming a composite decoupled cross section:
\begin{equation}
\Sigma_s(\mu_0,E'\rightarrow E) = \Sigma_s^{\alpha}(\mu_0,E) \delta(E'-E) 
+\Sigma_s^e(E'\rightarrow E) \frac{1}{2\pi} \delta(\mu_0-1)
\end{equation}
As one would expect, the derivation is very similar to that for the elastic
(coupled) cross section. Hence, we only give a brief outline of it.

The Boltzmann scattering operators corresponding to $\Sigma_s^{\alpha}$ and
$\Sigma_s^e$ are given by:
\begin{align}
& \mc{L}_B^{\alpha} = \int_{4\pi} \Sigma_s^{\alpha}(E,\mu_0)
\(\psi(\mu')-\psi(\mu)\) d\bo' \label{gamma_b_alpha}\\
& \mc{L}_B^e = \int_{0}^{\infty} \(\Sigma_s^e(E'\rightarrow E) \psi(E') -
\Sigma_s^e(E\rightarrow E') \psi(E)\)dE' \label{gamma_b_e}
\end{align}
Expanding the integrand in \cref{gamma_b_alpha} about $\theta_0=0$,
where $\theta_0$ keeping terms up to $\theta_0^2$ and integrating gives:
\begin{equation}
\mc{L}_B^{\alpha} = \mc{L}_{FP}^{\alpha}\psi
\end{equation}
with:
\begin{equation}
\alpha = 2\pi \int_{-1}^{1} \Sigma_s^{\alpha} (E,\mu_0) (1-\mu_0) d\mu_0
\end{equation}
Expanding the integrand in \cref{gamma_b_e} about $\tau=0$, where
$\tau=E'-E$, keeping terms up to $\tau^2$, and integrating gives:
\begin{equation}
\mc{L}_B^e\psi =\mc{L}_{FP}^e\psi
\end{equation}
An argument, completely analogous to that presented for the coupled case, can
be used to show $\mc{L}_B^{\alpha}$ and $\mc{L}_{FP}^{\alpha}$ better
approximate one another as $\Sigma^{\alpha}$ becomes increasingly peaked about
$\mu_0=1$. For similar reasons, $\mc{L}_B^e$ and $\mc{L}_{FP}^e$ better
approximate one another as $\Sigma_s^e$ becomes increasingly peaked about
$E'=E$. 

\subsection{Legendre polynomial expansion of $\Sigma_s^{\alpha}$}
Now we want to find the Legendre polynomial expansion of $\Sigma_s^{\alpha}$
as it has been done in \cite{morel_81,morel_89,morel_96}.
Because $\Sigma_s^{\alpha}$ does not change particle energy, it corresponds to a
within-group cross section. We have already shown that:
\begin{align}
\mc{L}_B^{\alpha}Y_l^m(\bo) &= (\Sigma_{s,l}^{\alpha}-\Sigma_{s,0}^{\alpha})
Y_l^m(\bo) \label{gamma_b_alpha_p}\\
\mc{L}_{FP}^{\alpha} Y_l^m(\bo) &= -\frac{\alpha}{2} l(l+1) Y_l^m(\bo) 
\label{gamma_fp_alpha_p}
\end{align}
with:
\begin{equation}
\Sigma_{s,l}^{\alpha} = 2\pi \int_{-1}^1 \Sigma_s^{\alpha} (\mu_0) P_l(\mu_0)
d\mu_0
\end{equation}
If the angular flux is expressible as a polynomial of arbitrary degree $L$:
\begin{equation}
\psi(\bo) = \sum_{l=0}^L \sum_{m=-l}^l \phi_{l,m} Y_l^m(\bo)
\end{equation}
then it follows from \cref{gamma_b_alpha_p,gamma_fp_alpha_p} that we can define 
$\Sigma_s^{\alpha}$ so that:
\begin{equation}
\mc{L}_B^{\alpha}\psi=\mc{L}_{FP}^{\alpha}\psi
\end{equation}
by setting:
\begin{equation}
\Sigma_{s,l}^{\alpha}-\Sigma_{s,0}^{\alpha} = -\frac{\alpha}{2}l(l+1)
\label{sigma_m_sigma}
\end{equation}
with $l=1,\hdots,L$.\\
Another method to show that $\mc{L}_B^{\alpha}$ and $\mc{L}_{FP}^{\alpha}$
become equivalent in the limit as $\mu_s\rightarrow 1$, consists of expanding the 
eigenvalues of $\mc{L}_B^{\alpha}$ in a first-order Taylor series about $\mu_s=1$:
\begin{equation}
\begin{split}
\Sigma_{s,l}^{\alpha} - \Sigma_{s,0}^{\alpha} &= \frac{\alpha}{1-\mu_s}
(P_l(\mu_s)-1)\\
&=\frac{\alpha}{1-\mu_s} \(P_l(1)+\frac{\partial}{\partial x}P_l(x)\Big|_{x=1}
(\mu_s-1)-1\)
\end{split}
\label{sig_comp}
\end{equation}
Using \cref{p_l,p_l_p}, \cref{sig_comp} becomes:
\begin{equation}
\Sigma_{s,l}^{\alpha} - \Sigma_{s,0}^{\alpha} = -\frac{\alpha}{2}l(l+1)
\end{equation}
We find that the eigenvalues of $B_{\alpha}$ converge to the eigenvalues of
$F_{\alpha}$ in the limit as $\mu_s\rightarrow 1$. Thus, these two operators
become equivalent in this limit.
Choosing $\Sigma_{s,L}=0$ so as to minimize the resulting value for
$\Sigma_{s,0}^{\alpha}$, \cref{sigma_m_sigma} gives:
\begin{equation}
\Sigma_s^{\alpha}(\mu_0) = \sum_{l=0}^L \frac{2l+1}{4\pi} \Sigma_{s,l}^{\alpha}
P_l(\mu_0)
\end{equation}
where:
\begin{equation}
\Sigma_{s,l}^{\alpha} = \frac{\alpha}{2}\(L(L+1)-l(l+1)\),\ \  l=0,\hdots,L
\end{equation}
Thus, with appropriate quadrature sets and expansion orders, the $S_n$
representation of $\mc{L}_{FP}^{\alpha}$ is equivalent to that obtained by
interpolating the discrete angular flux values with a polynomial and operating 
on that polynomial with $\mc{L}_{FP}^{\alpha}$.

It is interesting to consider the behavior of $\Sigma_s^{\alpha}$ in the limit
as the degree of the expansion is increased. First, we should note that regardless 
of expansion order the momentum transfer of $\Sigma_s^{\alpha}$ is exact:
\begin{equation}
\begin{split}
2\pi \int_{-1}^1 \Sigma_s^{\alpha}(\mu_0) (1-\mu_0) d\mu_0 &=
\Sigma_{s,0}-\Sigma_{s,1}\\
&=\frac{\alpha}{2} L(L+1) - \frac{\alpha}{2} (L(L+1)-2)\\
&=\alpha
\end{split}
\end{equation}
The average cosine of the scattering angle can be expressed as follows:
\begin{equation}
\begin{split}
\bar{\mu}_0 &= \frac{\Sigma_{s,1}^{\alpha}}{\Sigma_{s,0}^{\alpha}}\\
&=\frac{L(L+1)-2}{L(L+1)}
\end{split}
\end{equation}
It is easily seen that as $L$ increased, $\bar{\mu}_0$ goes to unity. Thus,
$\Sigma_s^{\alpha}$ becomes increasingly forward-peaked as $L$ increases. The
total magnitude of $\Sigma_s^{\alpha}$ becomes unbounded in the same limit:
\begin{equation}
\Sigma_{s,0}^{\alpha} = \frac{\alpha}{2} L (L+1)
\end{equation}
Expressing the momentum transfer as follows:
\begin{equation}
\alpha = \Sigma_{s,0}^{\alpha} (1-\bar{\mu}_0)
\end{equation}
it is seen that $\Sigma_{s,0}^{\alpha}$ must increase without bound if the
momentum transfer is to remain constant as $\Sigma_s^{\alpha}$ becomes
increasingly forward-peaked. This shows that $\mc{L}_{FP}^{\alpha}$ corresponds
to a type of continuous-deflection approximation. The particles are continuously 
deflected with the mean deflection per unit pathlength given by the momentum transfer.

The error made by using  $\mc{L}_B$ instead of $\mc{L}_{FP}$ in each eigenvalue 
monotonically decreases as $\mu_s$ is increased, but for any fixed value of $\mu_s$, 
the error monotonically increases with increasing eigenvalue
index\cite{morel_96}. Thus, while $\mc{L}_B^{\alpha}$ does in fact converge to
$\mc{L}_{FP}^{\alpha}$ as $\mu_s\rightarrow 1$, it does so nonuniformly. Thus, for any
fixed value of $\mu_s$, the high-order eigenvalues of $\mc{L}_{FP}^{\alpha}$ are 
grossly underestimated by $\mc{L}_B^{\alpha}$. Fortunately, this is error in the 
high-order eigenvalues is usually unimportant\cite{morel_96}.
