\section{Fokker-Planck cross section}
In this section, we derive the Fokker-Planck scattering cross section such
that the Fokker-Planck operator can be approximated by the Boltzmann operator.  
First we recall the BFP equation \cref{bfp}:
\begin{equation}
\begin{split}
&\bo\cdot\psi+(\Sigma_{s,reg}+\Sigma_a)\psi=\sum_{l=0}^{\infty}\sum_{m=-l}^l
Y_l^m(\bo)\int_E^{E/\alpha}\Sigma_{s,l,reg}(E'\rightarrow
E)\phi_{l,m}(E')dE'+\\
&\frac{\partial(S\psi)}{\partial E}+\frac{\alpha}{2}\(\frac{\partial}{\partial
\mu}(1-\mu^2) \frac{\partial}{\partial
\mu}+\frac{1}{(1-\mu^2)}\frac{\partial^2}{\partial\varphi^2}\)\psi+Q
\end{split}
\end{equation}
where we used $\alpha = 2T$.
Let us define:
\begin{align}
\mc{L}_{FP}^{\alpha} \psi &= \frac{\alpha}{2} \frac{\partial}{\partial \mu}
(1-\mu^2) \frac{\partial}{\partial \mu}\psi \label{gamma_alpha}\\
\mc{L}_{FP}^e\psi &=\frac{\partial}{\partial E}S\psi +
\frac{1}{2}\frac{\partial^2}{\partial E^2} R \psi \label{gamma_e}
\end{align}
We note that $\mc{L}_{FP}^{\alpha}$ causes particles to redistribute in
direction without energy change, while $\mc{L}_ {FP}^e$ causes particles to
redistribute particles in energy without directional change. Therefore,
$\mc{L}_{FP}^{\alpha}$ can be approximated using the following cross section:
\begin{equation}
\Sigma_s(\mu_0,E'\rightarrow E) = \Sigma_s^{\alpha}(\mu_0,E) \delta(E'-E)
\end{equation}
where $\Sigma_{s}^{\alpha} (\mu_0,E) = \frac{\alpha(E)}{1\mu_s} \frac{1}{2\pi}
\delta(\mu_0-\mu_s)$ and $\mu_s$ is a parameter; while $\Gamma_{FP}^e$ 
should be approximated by a cross section of the form:
\begin{equation}
\Sigma_s(\mu_0,E'\rightarrow E) = \Sigma_s^e(E'\rightarrow E) \frac{1}{2\pi}
\delta(\mu_0-1)
\end{equation}
More justifications of this approximation can be found in the fact 
that the Fokker-Planck scattering operator can also be derived assuming 
\cite{morel_81}:
\begin{equation}
\Sigma_s(\mu_0,E'\rightarrow E) = \Sigma_s^{\alpha}(\mu_0,E) \delta(E'-E) 
+\Sigma_s^e(E'\rightarrow E) \frac{1}{2\pi} \delta(\mu_0-1)
\end{equation}

\subsection{Legendre polynomial expansion of $\Sigma_s^{\alpha}$}
Now we expressed the Legendre polynomial expansion of $\Sigma_s^{\alpha}$
as it has been done in \cite{morel_89,morel_81,morel_96}. We will focus on
$\mc{L}_{FP}^{\alpha}$ since this research we solve the energy-integrated
Boltzmann equation. Therefore, the $\mc{L}_{FP}^e$ operator does not appear in
the equation that we want to solve. Because $\Sigma_s^{\alpha}$ does not change 
particle energy, it corresponds to a within-group cross section. We define:
\begin{align}
  \mc{L}_B^{\alpha} &= \Sigma_s^{\alpha} \mc{L}_B\\
  \mc{L}_{FP}^{\alpha} &= \frac{\alpha}{2}\mc{L}_{FP}
\end{align}
and thus:
\begin{align}
\mc{L}_B^{\alpha}Y_l^m(\bo) &= (\Sigma_{s,l}^{\alpha}-\Sigma_{s,0}^{\alpha})
Y_l^m(\bo) \label{gamma_b_alpha_p}\\
\mc{L}_{FP}^{\alpha} Y_l^m(\bo) &= -\frac{\alpha}{2} l(l+1) Y_l^m(\bo) 
\label{gamma_fp_alpha_p}
\end{align}
Using \cref{ang_flux,gamma_b_alpha_p,gamma_fp_alpha_p}, we can define 
$\Sigma_s^{\alpha}$ such that:
\begin{equation}
\mc{L}_B^{\alpha}\psi=\mc{L}_{FP}^{\alpha}\psi
\end{equation}
by setting:
\begin{equation}
\Sigma_{s,l}^{\alpha}-\Sigma_{s,0}^{\alpha} = -\frac{\alpha}{2}l(l+1)
\label{sigma_m_sigma}
\end{equation}
with $l=1,\hdots,L$. Choosing $\Sigma_{s,L}^{\alpha}=0$ to minimize 
$\Sigma_{s,0}^{\alpha}$ and \cref{sigma_m_sigma} becomes:
\begin{equation}
\Sigma_{s,l}^{\alpha} = \frac{\alpha}{2}\(L(L+1)-l(l+1)\),\ \  l=0,\hdots,L
\end{equation}
Using appropriate quadrature sets and expansion orders, the $S_n$
representation of $\mc{L}_{FP}^{\alpha}$ is equivalent to the one obtained by
interpolating the discrete angular flux values with a polynomial and operating 
on that polynomial.

Now we look at the behavior of $\Sigma_s^{\alpha}$ when the degree of the expansion 
is increased. First, we should note that the momentum transfer of
$\Sigma_s^{\alpha}$ is exact for any expansion order:
\begin{equation}
\begin{split}
2\pi \int_{-1}^1 \Sigma_s^{\alpha}(\mu_0) (1-\mu_0) d\mu_0 &=
\Sigma_{s,0}-\Sigma_{s,1}\\
&=\frac{\alpha}{2} L(L+1) - \frac{\alpha}{2} (L(L+1)-2)\\
&=\alpha
\end{split}
\end{equation}
With the previous relationships, the average cosine of the scattering angle
becomes:
\begin{equation}
\begin{split}
\bar{\mu}_0 &= \frac{\Sigma_{s,1}^{\alpha}}{\Sigma_{s,0}^{\alpha}}\\
&=\frac{L(L+1)-2}{L(L+1)}
\end{split}
\end{equation}
It is easily seen that when $L$ increases, $\bar{\mu}_0$ goes to one and
$\Sigma_s^{\alpha}$ becomes increasingly forward-peaked. The
total magnitude of $\Sigma_s^{\alpha}$ becomes unlimited when $L$ goes to
$\infty$:
\begin{equation}
\Sigma_{s,0}^{\alpha} = \frac{\alpha}{2} L (L+1)
\end{equation}
This shows that $\mc{L}_{FP}^{\alpha}$ corresponds to a continuous-deflection 
interaction. The particles are continuously 
deflected with the mean deflection per unit pathlength given by the momentum transfer.

$\mc{L}_{B}^{\alpha}$ converges to $\mc{L}_{FP}^{\alpha}$ when $\mu_s$ tends
to one but it does not converge uniformly. For any fixed value of $\mu_s$, 
the high-order eigenvalues of $\mc{L}_{FP}^{\alpha}$ are grossly underestimated 
by $\mc{L}_B^{\alpha}$. Fortunately, this is error in the high-order eigenvalues 
is usually unimportant \cite{morel_96}.
