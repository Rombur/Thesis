\section{Galerkin}
\red{Analyse tiree de \cite{pautz_fp} Dire que l'on travaille que sur 1 seul
groups}
The Fokker-Planck equation is not an asymptotic limit of the transport
equation unless the scattering kernel is such that:
\begin{equation}
\frac{\la\overline{\(1-\mu_0\)^2}\ra}{\la\(1-\bar{\mu}_0\)\ra}\rightarrow 0
\end{equation}
as $\bar{\mu}_0 \rightarrow 1$.

\begin{equation}
\begin{split}
\phi_{l,m}(\br) &= \int_{4\pi} d\bo' \Psi(\br,\bo') Y_l^{m,*}\\
&=(D \Psi)_{l,m}
\end{split}
\label{D2M}
\end{equation}
where $D$ is the directions-to-moments operator.
\begin{equation}
\begin{split}
\Psi(\br,\bo) &= \sum_{l=0}^{\infty} \sum_{m=-l}^l Y_l^m(\bo)\phi_{l,m}(\br)\\
&= M\phi(\bo)
\end{split}
\label{M2D}
\end{equation}
where $M$ is the moments-to-directions operator. \Cref{D2M,M2D} define the 
transformations from $\phi$ to $\Psi$ (``moments to
directions'') and from $\Psi$ to $\Phi$ (``directions to moments''). By
combining \cref{D2M,M2D}, we obtain the following requirement:
\begin{equation}
(I-MD)\Psi = 0
\end{equation}
$M=D^{-1}$ for analytic transport (\red{pris a l'interieur de la derivation de
la forme analytique})

Let us define:
\begin{align}
\Sigma_a &= \hat{\Sigma}_a\\
\Sigma_{s,l} &= \frac{\hat{\Sigma}_{s,l}(\br)}{\epsilon}\\
\Sigma_{s,l}(\br) &=\Sigma_{s,0}(\br) \(1-\frac{l(l+1)}{2}A(\br)\epsilon +
O(\gamma)\)
\end{align}
\begin{equation}
\begin{split}
&\bo_k\cdot\bn\Psi_k(\br) + \(\Sigma_a(\br)+\Sigma_{s,0}(\br)\) \Psi_k(\br)
=\\
& \sum_{l=0}^{N-1} \sum_{m=-l}^l Y_l^m(\bo_k) \phi_{l,m}(\br)
\frac{\hat{\Sigma}_{s,0}(\br)}{\epsilon}\(1-\frac{l(l+1)}{2}A(\br)\epsilon +
O(\gamma)\) + q(\br,bo_k)
\end{split}
\label{discr_eps}
\end{equation}
where:
\begin{align}
\Psi_k(\br) &= \Psi(\br,\bo_k)\\
\phi_{l,m}(\br) &= \sum_{k=1}^K w_k Y_l^{m,*}(\bo_k) \Psi_k(\br) \label{phi_p}
\end{align}
Here, the $w_k$ and $\bo_k$ are the quadrature weights and directions,
respectively, of a quadrature set of order $N$. In level-symmetric quadrature
sets, $K=N$ in 1D, $K=\frac{N(N+2)}{2}$ in 2D and $K=N(N+2)$ in 3D. Note that
in the standard discrete ordinates treatment, the scattering order \cref{discr_eps} 
is truncated at $N-1$. Manipulation of \cref{discr_eps} yields:
\begin{equation}
\begin{split}
&\bo_k\cdot\bn \Psi_k(\br) +\hat{\Sigma}_a(\br)\Psi_k(\br)\\
&\frac{\hat{\Sigma}_{s,0}(\br)}{\epsilon} \(\Psi_k(\br)-\sum_{l=0}^{N-1}
\sum_{m=-l}^m  Y_l^m(\bo_k) \phi_{l,m}^{\br}\)\\
&=-\frac{\(\Sigma_{tr}(\br)-\hat{\Sigma}_a(\br)\)}{2} \sum_{l=0}^{N-1}
\sum_{m=-l}^{l} l(l+1) Y_l^m(\bo_k) \phi_l^m(\br)+\\
& q(\br,\bo_k) + O\(\frac{\gamma}{\epsilon}\) 
\end{split}
\label{discr_eps_manip}
\end{equation}
We insert the ansatz:
\begin{align}
\Psi &= \Psi^{(0)} + \epsilon \Psi^{(1)} + \epsilon^2\Psi^{(2)}+\hdots\\
\phi_{l,m} &= \phi_{l,m}^{(0)} + \epsilon \phi_{l,m}^{(1)} + \epsilon^2
\phi_{l,m}^{(2)}+\hdots
\end{align}
into \cref{discr_eps} and consider terms of $O(1)$ to find:
\begin{equation}
\begin{split}
\phi_{l,m}^{(0)}(\br) &= \sum_{k=1}^K w_k Y_l^{m,*}(\bo_k) \Psi_k^(0)(\br)\\ 
&= (D_N\Psi^{(0)})_{l,m}
\end{split}
\label{D2M_disc}
\end{equation}
We insert the ansatz into \cref{discr_eps_manip} and consider terms
of $O(\epsilon^{-1})$ to find:
\begin{equation}
\begin{split}
\Psi_k^{(0)}(\br) &=\sum_{l=0}^{N-1} \sum_{m=-l}^l Y_l^m(\bo_k) 
\phi_{l,m}^{(0)}(\br)\\
&= (M_N\phi^{(0)})_k
\end{split}
\label{M2D_disc}
\end{equation}
where there is no $O(\gamma)$ term, since it is easily shown that
$\gamma\rightarrow 0$ as $\epsilon\rightarrow 0$, i.e., that there are no
$O(1)$ components in $\gamma$. \Cref{D2M_disc,M2D_disc} may be combined to give:
\begin{align}
&(I-M_ND_N)\Psi^{(0)} = 0 \label{identity_a}\\
&(I-D_NM_N)\phi^{(0)} = 0 \label{identity_b}
\end{align}
i.e., $\Psi^{(0)}$ must be in the null space of $I-M_N D_N$ and $\phi^{(0)}$
must be in the null space of $I-D_N M_N$. Since $\phi$ is always constructed
from $\Psi$ according to \cref{phi_p}, \cref{identity_b}
is equivalent to:
\begin{equation}
D_N(I-M_N D_N)\Psi^{(0)} = 0
\end{equation}
which is always satisfied if \cref{identity_a} is satisfied.
Therefore, we need only to show that equation \cref{identity_a} is satisfied
in order to satisfy \cref{D2M_disc,M2D_disc}.

A sufficient (although not strictly necessary) condition for satisfying
\cref{identity_a} is that $M_ND_N=I$. This will be always be true if
$M_N$ and $D_N$ are inverse of each other (as in analytic transport), in which
case there are as many moments in the scattering expansion as there are
discrete angles. It may or may not be true of there are fewer moments than
angles. (These assertions follow directly from the dimensions of the
matrices.) In one-dimensional slab and spherical geometry, it will be true if
and only if the quadrature set exactly integrates polynomials of degree
$2N-2$, as is the case with the Gauss-Legendre set. In standard
multidimensional implementations, there are generally more discrete angles
than scattering moments, so in these cases $M_N D_N \neq I$. If $M_N
D_N\approx I$, then \cref{identity_b} will be satisfied only if some
other (generally nonphysical) constraints are met. (Specifically, $\Psi^{(0)}$
must be constrained to live in the null space of $I-M_N D_N$.) If equation
\cref{identity_b} is not satisfied, the asymptotic ansatz is not valid. In
such a case, there is no $O(1)$ solution to \cref{discr_eps,phi_p}. Later, we will 
describe a relatively simple alteration of the
standard discrete ordinates method that will ensure that $M_N D_N = I$.

If we assume that $M_N D_N=I$, then the $O(\epsilon)$ terms in \cref{phi_p} yields:
\begin{equation}
\begin{split}
\phi_{l,m}^{(1)}(\br) &= \sum_{k=1}^K w_k Y_{l}^{m,*} (\bo_k)
\Psi_k^{(1)}(\br)\\
&=(D_N\Psi^{(1)})_{l,m}
\end{split}
\end{equation}

The $O(1)$ terms in \cref{discr_eps_manip} yield:
\begin{equation}
\begin{split}
&\bo_k\cdot \bn \Psi_k^{(0)}(\br) + \hat{\Sigma}_a(\br)\Psi_k(\br)\\
&\hat{\Sigma}_{s,0}(\br) \(\Psi_k^{(1)} (\br) - \sum_{n=0}^{N-1}
\sum_{m=-l}^{l}\(\frac{2l+1}{4\pi}\) Y_l^m (\bo_k)\phi_{l,m}^{(1)}(\br)\)\\
&=-\frac{\Sigma_{tr}(\br)-\hat{\Sigma}_a(\br)}{2} \sum_{l=0}^{N-1}
\sum_{m=-l}^l \(\frac{2l+1}{4\pi}\) l(l+1) Y_l^m(\bo_k)
\phi_{l,m}^{(0)}(\br)\\
&+ q(\br,\bo_k) + O\(\frac{\gamma}{\epsilon}\)
\end{split}
\label{O_1_terms}
\end{equation}
The scattering term on the left side of \cref{O_1_terms} will disappear only if:
\begin{equation}
\Psi_k^{(1),*}(\br) = \Psi_k^{(1)}(\br)
\label{psi_1_star_old}
\end{equation}
where:
\begin{equation}
\begin{split}
\Psi_k^{(1),*} &= \sum_{l=0}^{N-1}\sum_{m=-l}^l \(\)
\end{split}
\label{psi_1_star}
\end{equation}   
This is satisfied under the assumption that $M_N D_N$ is the identity.
\Cref{O_1_terms} yields:
\begin{equation}
\begin{split}
&\bo_k \cdot \bn \Psi_k^{(0)}+\hat{\Sigma}_a(\br) \Psi_k^{(0)}(\br)\\
&=\frac{\Sigma_{tr}(\br)-\hat{\Sigma}_a(\br)}{2}
\(\(\frac{\partial}{\partial \mu}(1-\mu^2)\frac{\partial}{\partial \mu}+
\(\frac{1}{1-\mu^2}\)\frac{\partial^2}{\partial \varphi^2}\)\tilde{\Psi}^{(0)}
(\br,\bo)\)_{\bo=\bo_k}\\
&+ q(\br,\bo_k) + O\(\frac{\gamma}{\epsilon}\)
\end{split}
\label{last_pautz}
\end{equation}
where $k=1,\hdots,K$ and $\tilde{\Psi}^{(0)}(\br,\bo)$ is an interpolant
through the points $\{\bo,_k,\Psi_k^{(0)}(\br,\bo)\}$. Thus, if
$O\(\frac{\gamma}{\epsilon}\) \rightarrow 0$ as $\epsilon \rightarrow 0$,
\cref{last_pautz} is a pseudo-spectral discretization of the angular
variable in the exact Fokker-Planck equation. (Pseudo-spectral methods use
collocation to determine coefficients in global function expansions.) In
one-dimensional slab and spherical geometry, $\tilde{\Psi}^{(0)}(\br,\bo)$ is
the $(N-1)$-order polynomial interpolant through the points
$\{\bo_k,\Psi_k^{(0)}(\br)\}$. The definition in multidimensional geometry
will be described in the following discussion.

The foregoing discussion indicated that the transformation from discrete
values to angular moments and back to discrete values to angular moments and
back to discrete values should be the identity. If \cref{phi_p,psi_1_star} define 
the discrete-to-moments and moments-to-discrete transformations, then we will 
not have the identity unless the quadrature set is Gauss-Legendre in one-dimensional 
slab or spherical geometry. Given a different quadrature set of multidimensional 
geometry then, the $S_N$ method may not limit to a discretization of the 
Fokker-Planck equation unless \cref{phi_p} and/or \cref{psi_1_star} is replaced.

Morel\cite{galerkin_morel} reached the same conclusion via a completely
different analysis and offered suggestions for replacing the offending
equation(s). The simplest suggestion in one-dimensional slab and spherical
geometry is to use $\tilde{\phi}_{l,0}$ the exact moments of the $(N-1)$-order
polynomial, $\tilde{\Psi}$, that goes through the points
$\{\bo_k,\Psi_k(\br)\}$; thereby redefining $D_N$ to be $M_N^{-1}$. Morel
labeled this the ``Galerkin'' quadrature set, since he derived it by means of
a Galerkin weighting method. The use of the exact moments causes
\cref{psi_1_star_old} to be satisfied regardless of quadrature set, and
\cref{last_pautz} then follows.

In multidimensional geometries, the Galerkin quadrature has a more complex
definition. Recall that there are fewer moments than discrete angles in
standard multidimensional implementations of the discrete ordinates method.
To satisfy \cref{psi_1_star_old} in all circumstances, we must first
increase the number of spherical harmonics in our flux expansion by using
harmonics of higher orders. Morel\cite{galerkin_morel} and Reed\cite{reed}
proposed suitable spherical harmonics interpolation space for multidimensional
geometries. For two-dimensional geometries, the following interpolation space
is suggested:
\begin{equation}
Y_l^m = \left\{
\begin{aligned}
&0\leq m \leq n, & \textrm{if }0\leq l\leq N-1 \\
&0< l\textrm{ odd }\leq N, & \textrm{if }l=N
\end{aligned}
\right.
\end{equation}
The interpolation space suggested for three dimensions is:
\begin{equation}
Y_l^m = \left\{
\begin{aligned}
&-l\leq m \leq l, & \textrm{if }0\leq l \leq N-1\\
&-l\leq m <0 \textrm{ and } 0<l\textrm{ odd }\leq l, & \textrm{if }l=N\\
&-l\leq m\textrm{ even }<0,& \textrm{l=N+1}
\end{aligned}
\right.
\end{equation}
\red{The selection rules for more general quadratures (triangular and
product) are given in \cite{galerkin_sanchez}.}

The Galerkin quadrature is then defined by adjusting the limits of the
summations in \cref{psi_1_star} to augment $M_N$ and then redefining
$D_N=M_N^{-1}$. As in the one-dimensional case, \cref{psi_1_star_old} will be 
satisfied regardless of the discrete angle set when the Galerkin treatment is 
used, and \cref{last_pautz} then follows, where $\tilde{\Psi}^{(0)}(\br,\bo)$ 
is now defined as the spherical harmonic interpolant through the points 
$\{\bo_k,\Psi_k^{(0)}(\br)\}$.

We note that the use of the Galerkin quadrature allows the selection of a
greater variety of discrete angle sets in \cref{discr_eps} since the
corresponding quadrature weights (if they are defined) are not actually used.

More can be said about the effects of using a non-Gaussian or non-Galerkin
quadrature to evaluate \cref{phi_p}. Let us define the scattering
ratio matrix $\bs{C}$ by:
\begin{equation}
\bs{C} = \frac{1}{\Sigma_t}\bs{D}\bs{M}\bs{\Sigma}
\end{equation}
where:
\begin{itemize}
\item $\bs{D}$ = discrete-to-moments matrix
\item $\bs{M}$ = moments-to-discrete matrix
\item $\bs{\Sigma}$ = diagonal matrix whose entries are the scattering
coefficients $\{\Sigma_{s,l}\}$ in the order and frequency corresponding to
their respective moments in the other matrices.
\end{itemize}
If exact integrals are used, then $\bs{D}=\bs{M}^{-1}$, and $\bs{C}$ will be a
diagonal matrix whose entries are the scattering ratios
$\frac{\Sigma_{s,l}}{\Sigma_t}$. In a nonmultiplying medium, each diagonal
term (and hence each eigenvalue of $\bs{C}$) will be nonnegative and will not
exceed unity. If, however, inexact integrations are used, not only will
$\bs{C}$ differ from $\frac{\bs{\Sigma}}{\Sigma_t}$, but there is also the
possibility of introducing one or more eigenvalues whose absolute values
exceed unity. This is physically equivalent to artificially introducing
multiplication into the medium. Depending on the amount of leakage present,
\cref{discr_eps,phi_p} may not have a steady-state
solution as $\epsilon\rightarrow 0$. This inconsistency is clearly
unacceptable when a steady-state solution is known to exist.

\red{Aussi dand \cite{galerkin_morel}}
We note that while the condition $D_N=M_N^{-1}$ is certainly sufficient for
obtaining the correct Fokker-Planck limit, it is not strictly necessary. We
need only to satisfy \cref{identity_a}, i.e., that $\tilde{\Psi}$ be
in the null space of $I-M_ND_N$. If $M_N D_N\neq I$, then certain angular
eigenmodes cannot be present in a stable solution, i.e., $\tilde{\Psi}$ must
be in a restricted subspace of the domain of $M_N D_N$. It is entirely
possible that a contrived selection of boundary conditions and sources could
result in a solution that does not contain any of the unstable modes; however,
this would suggest that physically realistic boundary conditions and
sources could result in a solution that does not contain any of the unstable
modes. Alternatively, one could filter out the unstable mode components of the
scattering source; this would stabilize the solution, but this would yield a
different solution than that obtained when exact integrations are used. Our
recommendation is to avoid these complications altogether by simply using the
exact inverse of $M_N$.\\

\red{Pris de \cite{galerkin_morel} 1D equation}\\
We begin by considering the interpolatory trial space angular flux
representation on which this method is based:
\begin{equation}
\Psi(\mu) = \sum_{m=1}^N \Psi_m B_m(\mu)
\label{psi_b}
\end{equation}
Methods for generating the interpolatory basis function appearing in 
\cref{psi_b} are discussed in \cite{galerkin_morel}. The scattering source
corresponding to this representation is given by:
\begin{equation}
S(\mu) = \int_{4\pi} \Sigma_s(\mu_0)\Psi(\mu')d\bo'
\label{S_mu}
\end{equation}
where:
\begin{equation}
\mu_0 = \mu'\mu+\sqrt{\(1-\mu^{'2}\)\(1-\mu^2\)} \cos\(\varphi\)
\end{equation}
Expanding the cross section in Legendre polynomials, we obtain:
\begin{equation}
\Sigma_s(\mu_0) = \sum_{l=0}^{\infty} \frac{2l+1}{2}\Sigma_{s,l} P_l(\mu_0)
\label{sigma_s_mu}
\end{equation}
where:
\begin{equation}
\Sigma_{s,l} = \int_{-1}^{1} \Sigma_s(\mu_0)P_l(\mu_0) d\mu_0
\end{equation}
Substituting \cref{sigma_s_mu} into \cref{S_mu}, using the
spherical harmonics addition theorem to express $P_l(\mu_0)$ in terms of
$\mu'$ and $\mu$ and completing the integrations over $\mu$ and $\varphi$, we
get:
\begin{align}
&S(\mu) = \sum_{l=0}^{\infty} \frac{2l+1}{2} \xi_l P_l(\mu) \label{S_mu_2}\\
&\xi_l = \Sigma_{s,l} \phi_l \label{xi_l_s}\\
&\phi_l = \int_{-1}^1 \Psi(\mu)P_l(\mu)d\mu \label{phi_l_int}
\end{align}
Let $\tilde{S}(\mu)$ denote the trial approximation to $S(\mu)$ that we seek:
\begin{equation}
\tilde{S}(\mu) = \sum_{m=1}^N \tilde{S}_m B_m(\mu)
\label{tilde_s_mu}
\end{equation}
As previously stated, the Galerkin method requires that the residual
associated with the trial space approximation be orthogonal to the weighting
space. The residual associated with \cref{tilde_s_mu} is given by:
\begin{equation}
Res(\mu) = \tilde{S}(\mu) - S(\mu)
\label{R_mu}
\end{equation}
The weighting space associated with our method is the space of global
polynomials of degree $N-1$ or less form a basis for this space, it follows
that the residual must satisfy:
\begin{equation}
\int_{-1}^1 Res(\mu) P_l(\mu) d\mu = 0.0,\ \ l=0,\hdots,N-1
\label{int_r_mu}
\end{equation}
Substituting from \cref{S_mu_2,tilde_s_mu,R_mu}) into \cref{int_r_mu}, 
performing the integration, rearranging the resulting equation, we obtain:
\begin{align}
&\tilde{\xi}_l = \xi_l,\ \ l=0,\hdots,N-1 \label{xi_vs_xi}\\
&\tilde{\xi}_l = \int_{-1}^1 \tilde{S}(\mu) P_l(\mu) d\mu\\
&\xi_l = \int_{-1}^1 S(\mu) P_l(\mu) d\mu
\end{align}
The central theme of our method becomes clear on examination of 
\cref{xi_vs_xi}: the discrete scattering source values are chosen such that
the interpolatory representation for that scattering source has the same
Legendre moments through degree $N-1$ as the exact scattering source
calculated with the interpolatory representation for the angular flux. Thus,
the Galerkin formalism, when applied with an appropriate weighting space,
yields a very straightforward approximation. One highly desirable property of
this approximation is that only the cross-section moments through degree $N-1$
are required. This property is directly due to our choice of weighting space.
Specifically, this property is obtained because all elements of the weighting
space are orthogonal to the Legendre polynomials of degree $N$ or greater.

The Galerkin scattering matrix can be conveniently represented in terms of
three matrices: the discrete-to-moment matrix, the cross-section matrix, and
the moment-to-discrete matrix. The discrete-to-moment matrix $D$ maps a vector
of discrete angular flux values to a corresponding vector of Legendre flux
moments. These flux moments are calculated with the interpolatory flux
representation defined by the discrete angular flux vector:
\begin{equation}
\bs{\phi} = \bs{D\Psi}
\end{equation}
where:
\begin{align}
&\bs{\phi} = (\phi_0,\phi_1,\hdots,\phi_{N-1})\\
&\bs{\Psi} = (\Psi_1,\Psi_2,\hdots,\Psi_{N-1})\\
&\bs{D}_{l,m} = \int_{-1}^1 B_m(\mu) P_l(\mu)d\mu \label{bs_D}
\end{align}
Note that \cref{bs_D} follows directly from \cref{psi_b,phi_l_int}:
\begin{equation}
\phi_l = \sum_{m=1}^{N} \psi_m \int_{-1}^1 B_m(\mu) P_l(\mu)d\mu
\end{equation}
The cross-section matrix maps the vector of flux moments to the vector of
scattering source moments:
\begin{equation}
\begin{split}
\bs{\xi} &= \bs{\Sigma \phi}\\
&= \bs{\Sigma D}\Psi
\label{bs_xi}
\end{split}
\end{equation}
where:
\begin{align}
&\bs{\xi} = (\xi_0,\xi_1,\hdots,\xi_{N-1})\\
&\bs{\Sigma} = diag(\Sigma_{s,0},\Sigma_{s,1},\hdots,\Sigma_{s,N-1})
\label{bs_sigma}
\end{align}
Note that \cref{bs_xi} follows directly from \cref{tilde_s_mu}. The 
moment-to-discrete matrix $M$ maps a vector of
Legendre moments to a corresponding vector of discrete angular flux values.
The trial space element obtained by interpolating the discrete flux values has
the Legendre moments defined by the moment vector. Thus, the
moment-to-discrete matrix matrix is inverse of the discrete-to-moment matrix:
\begin{equation}
\bs{M} = \bs{D}^{-1}
\end{equation}
Having defined the three constituent matrices, we proceed with the
construction of the Galerkin inscatter matrix. First, we note that the
vector of Legendre moments corresponding to the trial space element obtained
by interpolating the discrete angular flux values is given by:
\begin{equation}
\bs{\Phi} = \bs{D}\psi
\end{equation}
The Legendre moments of the scattering source calculated with the trial space
element and the exact scattering cross section are given by:
\begin{equation}
\bs{\xi} = \bs{\Sigma D}\Psi
\label{bs_xi_2}
\end{equation}
Our Galerkin approximation requires that the trial space element obtained by
interpolating the discrete scattering source value have the same moments as
those given by \cref{bs_xi_2}. Mathematically stated, the discrete
scattering source vector $\bs{S}$ must satisfy:
\begin{equation}
\bs{DS} = \bs{\Sigma D}\psi
\label{bs_ds}
\end{equation}
where:
\begin{equation}
\bs{S} = (S_1,S_2,\hdots,S_N)
\end{equation}
Solving \cref{bs_ds} for $\bs{S}$, we obtain the desired
relationship:
\begin{equation}
\bs{S} = \bs{M\Sigma D}\psi
\label{bs_s}
\end{equation}
The matrix $\bs{\Sigma}$ is the scattering matrix obtained with the spherical
harmonic approximation of degree $N-1$ (the $P_{N-1}$ approximation). Since
$\bs{M}$ is the inverse of $\bs{D}$, it follows that the Galerkin scattering
matrix $\bs{S}$ is equivalent to the $P_{N-1}$ scattering matrix $\bs{\Sigma}$
under the similarity transformation defined by the discrete-to-moments and
moment-to-discrete matrices. Furthermore, if a global polynomial trial space
is used, it follows from \cref{bs_s} that our Galerkin scattering
source is identical to the spherical harmonic scattering source. Thus, our
Galerkin quadrature method can be characterized as a type of generalized
spherical harmonics scattering source approximation. We fell that this
characterization leads to considerable insight into the basic nature of our
method.

It is easily shown that delta function scattering of the following type is
treated exactly with the Galerkin method:
\begin{equation}
\Sigma_s(\mu_0) = \delta(\mu_0-1)
\end{equation}
We begin by noting that the cross-section matrix defined by \cref{bs_sigma} 
is the identity matrix. Substituting the identity matrix for the cross-section 
matrix in \cref{bs_ds}, we obtain the following expression:
\begin{equation}
\bs{S} = \bs{MD}\psi
\label{bs_s2}
\end{equation}
Recalling that $\bs{M}$ is the inverse of $\bs{D}$, we find that \cref{bs_s2} gives:
\begin{equation}
\bs{S} = \psi
\end{equation}
which is the exact result. Exactly treating this type of scattering has two
important ramifications for charged-particle transport. First, in
charged-particle transport, cross sections of the following form arise:
\begin{equation}
\Sigma^{k\rightarrow g}(\mu_0) = \Sigma^{k\rightarrow g} \delta(\mu_0-1)
\end{equation}
where $\Sigma^{k\rightarrow g}(\mu_0)$ denotes the differential cross section
associated with a transfer from group $k$ to group $g$. Thus, it is essential
that delta function scattering be treated exactly. Second, for relativistic
electron scattering, the extended transport correction reduces within-group
scattering cross sections by two orders of magnitude or more. Without this
reduction, the dominance ratio are so close to unity in practical calculations
that source iteration has a prohibitively slow convergence rate. Thus, from
the viewpoint of computational efficiency, it is essential that the extended
transport correction be applied. Because the Galerkin method treats delta
function  scattering exactly, an extended transport correction of the
within-group cross-section moments leaves the solution
invariant\cite{morel_79}. This is a highly desirable property since it allows
us to dramatically reduce within-group scattering cross sections without any
loss of accuracy.

The scattering source representation given by \cref{bs_s2} is
algebraically equivalent to the standard $S_n$ expression, provided that the
$S_n$ calculation is performed with an $N$-point quadrature set and a
cross-section expansion of degree $N-1$. To demonstrate this, we first express
\cref{bs_s2} in summation form:
\begin{align}
&\phi_l = \sum_{m=1}^N D_{l,m} \Psi_m, & l=0,\hdots,N-1 \label{37_a}\\
&\xi_l = \Sigma_{s,l} \phi_l, & l=0,\hdots,N-1 \label{37_b}\\
&S_m = \sum_{l=0}^{N-1} M_{m,l}\xi_l, & m=1,\hdots,N \label{37_c}
\end{align}
The standard $S_n$ method is given by:
\begin{align}
&\phi_l = \sum_{m=1}^N \Psi_m P_l(\mu_m) w_m, & l=0,\hdots,N-1 \label{38_a}\\
&\xi_l = \Sigma_{s,l} \Phi_l, & l=0,\hdots, N-1 \label{38_b} \\
&S_m = \sum_{l=0}^{N-1}\xi_l \frac{2l+1}{2}P_l(\mu_m), & m=1,\hdots,N
\label{38_c}
\end{align}
By inspecting \cref{37_a,37_b,37_c,38_a,38_b,38_c}, it is clear that the
standard $S_n$ method can be obtained from \cref{37_a,37_b,37_c} simply by
redefining the matrices $D$ and $M$ as follows:
\begin{align}
& \bs{D}_{l,m} = P_l(\mu_m) w_m \label{39_a}\\
& \bs{M}_{m,l} = \frac{2l+1}{2} P_l(\mu_m) \label{39_b}
\end{align}
Since the same matrix multiplications are performed in both the standard
$S_n$ and Galerkin quadrature methods, the computational work required by
these methods is identical.

We have thus far compared the Galerkin and $S_n$ methods under the assumption
that the $S_n$ method is used with an $N$-pont quadrature set and a $P_{N-1}$
cross-section expansion. Before considering more general cases, we must first
discuss the concept os a cross-section ``expansion'' as it relates to the
Galerkin and $S_n$ approximations. Our Galerkin method with an $N$-point flux
representation is said to require cross -section expansions of degree $N-1$.
This is true insofar as the cross-section mometns through degree $N-$ are used
but, as previously discussed, we do not actually use the Legendre expansion of
degree $N-1$ accurately represents the cross section. Conversely, the standard
$S_n$ method actually uses the Legendre expansion to represent the cross
section. Thus, with the Galerkin method, one need not be concerned with the
accuracy of the Legendre expansion, but rather meed only be concerned with the
accuracy with which $\tilde{S}(\mu)$ approximates $S(\mu)$. This is a very
desirable property when the cross section is highly forward-peaked, and,
represent the main advantage of the Galerkin method relative to the standard
$S_n$ method. For instance, a delta function cross-section expansion of finite
degree is never converged, yet the scattering source for such a cross section
is treatedd exactly with the Galerkin method.

If the $S_n$ method is used with an expansion of degree >$N-1$, the Galerkin
method obviously require less computational work, but most importantly, we
argue that in general one cannot expect greater accuracy with the standard
$S_n$ method simply because more cross-section moments are being usede. On the
other hand, if the scattering is isotropic, it is clearly a disadvantage for
the Galerkin method if one must use $N-1$ expansion coefficients with all but
the first being zeros. Fortunately, if the cross-section moments, are nonzero
only through degree $L$ with $L<N-1$, the full matrix expression for the
scattering source need not to be used. Specifically, only the first $L+1$ rows
of the $\bs{D}$ matrix and the first $L+1$ columns of the $\bs{M}$ matrix need
to be retained:
\begin{align}
&\phi_l = \sum_{m=1}^N \bs{D}_{l,m} \Psi_m, & l=0,\hdots,L \label{40_a}\\
&\xi_l = \Sigma_{s,l}\phi_l,& l=0,\hdots,L \label{40_b}\\
&S_m = \sum_{l=0}^L \bs{M}_{m,l} \xi_l,& m=1,\hdots,N \label{40_c}
\end{align}
\Cref{40_a,40_b,40_c} are algebraically equivalent to the standard $S_n$
method with a cross-section expansion of degree $L$. Thus, the galerkin and
standrad $S_n$ methods require the same computational work, provided that the
cross-section expansion order used in the $S_n$ method is $\leq N-1$. The
standard $S_n$ method, a quadrature set simply consists of cosines and weights
that are used to construc the $\bs{M}$ and $\bs{D}$ matrices. In the Galerkin
method, these matrices are defined in terms of an angular flux interpolation
scheme. Thus, an $N$-pont Galerkin quadrature set consists of the quadrature
points, $\{\mu_m\}_{m=1}^N$, and the matrices $\bs{M}$ and $\bs{D}$, with the
restrictions that $\bs{D}$ must give the Legendre moments corresponding to a
specific interpolation of the angula flux values at the quadrature points, and
$\bs{M}$ must be the inverse of $\bs{D}$. As previously discussed, there is a
standard quadrature set, called the companion set, that corresponds to each
Galerkin quadrature set. This set consists of the Galerkin quadrature points
and the unique set of weights that give and exact integration of the Galerkin
trial space elements. It is easily seen from \cref{37_a} that these weights
compose the first row of the $\bs{D}$ matrix:
\begin{equation}
\begin{split}
\phi_0 &= \sum_{m=1}^N \bs{D}_{0,m}\\
&=\sum_{m=1}^N \psi_m w_m
\end{split}
\end{equation}
Standard $S_n$ codes generallly require weights that sum to unity. The
definitions that we have weights that sum to 2. To obtain matrices that are
consistent with weights that sum to unity, simply scale the $\bs{D}$ matrix by
$\frac{1}{2}$ and the $\bs{M}$ matrix by 2.

It is obviously of interest to ask if there are any Galerkin sets that are
equivalent to standard sets in the sense that each of the Galerkin sets gives
a scattering matrix identical to that obtained with the standard $S_n$ method
and the respective companion quadrature set. A Galerkin set that has this
property is said to be ``companion-equivalent''. There is at least one family
of Galerkin quadrature sets that has this property: the sets generated with
global polynomial interpolation at the Gauss quadrature cosines. These sets
have the standard full-range Gauss quadrature sets as their companion sets.
Thus, we refer to them as the full-range Galerkin Gauss sets. It is not
difficult to demonstrate that these sets are companion-equivalent. For
instance, it is clear that \cref{39_b} must be the expression for the $\bs{M}$
matrix whenever global polynomaial interpolation is used. Thus, we need only
demonstrate that \cref{39_a} is correct for the $\bs{D}$ matrix. We begin by
using the standard Gauss quadrature formula to evaluate \cref{bs_D}:
\begin{equation}
\bs{D}_{l,m} = \sum_{k=1}^N B_m(\mu_k) P_l(\mu_k) w_k
\label{gal_m_last}
\end{equation}
All of the interpolatory basis functions are polynomials of degree $N-1$.
Since the Legendre polynomials appearing in \cref{gal_m_last} are of degree
$N-1$ or less, the integrand in \cref{gal_m_last} must be a polynomial of
degree $2N-2$ or less. The $N$-point Gauss quadrature set is exact for all
polynomials of degree $2N-1$ or less. Therefore, \cref{gal_m_last} is exact.
As can be seen from \cref{bs_D}, the interpolatory basis functions are defined
such that the $m^{th}$ expansion coefficient is equal to the angular flux
evaluated at the $m^{th}$ interpolation cosine. By successively evaluating
\cref{bs_D} at eaxh cosine, one finds that the interpolatory basis functions
must satisfy:
\begin{equation}
B_m(\mu_k) = \left\{
\begin{aligned}
&1 & \textrm{for }m=k\\
&0 &\textrm{otherwise}
\end{aligned}
\right.
\label{llast}
\end{equation}
Substituting from \cref{llast} into \cref{gal_m_last}, we obtain:
\begin{equation}
\bs{D}_{l,m} = P_l(\mu_m)w_m
\label{lllast}
\end{equation}
\Cref{lllast} is the standard $S-n$ expression for the $\bs{D}$ matrix. Thus,
we have proved that the full-range Galerkin Gauss quadrature sets are
companion-equivalent.

There are no companion-equivalent Galerkin sets based on global polynomial
interpolation other than the Gauss sets. This follows from the fact that no
other symmetric $N$-point quadrature sets can integrat the polynomials of
degree $2N-2$ or less that appear in \cref{gal_m_last}. The sets with highest
integration accuracy after the Gauss sets are the Lobatto sets, which only
integrate polynomials of degree $2N-3$ or less. It seems highly unlikely that
there any Galerkin sets based on nonglobal polynomial interpolation that are
companion-equivalent, but we cannot prove it. Such an $N$-point Galerkin set
would have to have a companion set that integrates products of the non-global
polynomial trial space functions and global polynomials of degree $N-1$ or
less. In addition, the standard $S_n$ expression for the $\bs{M}$ matrix,
which is generally correct only for global polynomial trial spaces, would also
have to be correct for the nonglobal polynomial trial space. It seems highly
unlikely that all of these requirements can be met.

A Galerkin set differs from its companion set only if a cross-section
expansion of sufficiently high order is being used. For instance, with
isotropic scattering, every Galerkin set and its companion set are equivalent.
Most standard $S_n$ quadrature sets are designed to integrate either global of
half-range polynomials. Half-range sets can also integrate certain global
polynomials because global polynomials always form a subspace of any
half-range polynomials, the higher is the degree of the cross-section
expansion for which that standard set is equivalent to its corresponding
Galerkin set. For instance, the standard $N$-poin Lobatto quadrature set,
which integrates global polynomials of degree $2N-2$ or less, is equivalent to
the Galerkin $N$-point Lobatto set (generated with global polynomial
interpolation at the standard Lobatto points) for cross-section expansion of
degree $N-2$ or less. The standard double-Gauss N-point set, which integrates
global polynomials of degree $N-1$, is equivalent to the Galerkin double-Gauss
N-point set (generated with half-range polynomial interpolation at the
standard double-Gauss points) for cross-section expansions of degree $N/2-1$
or less. Even though a Galerkin set and its companion set may not be
rigorously equivalent, they may nonetheless give very similar results even
with highly anisotropic expansions. For instance, if one is comparing a
Galerkin doouble-Gauss $N$-point set with its companion set using a highly
anisotropic cross-section expansion of degree $N-1$ in a problem for which the
flux moments of degree $N/2$ and greater are very small, the two quadratures
will give almost identical results provided that the companion scattering
matrix is stable (i.e., all eigenvalues must have magnitudes less than or
equal to unity). This follows from the fact that these sets differ only with
respect to the treatment of the moments of degree $N/2$ or greater, and if
these moments are small, the differences in the respective soltuions must
similarly be small. However, if the companion scattering matrix is unstable,
anisotropic components of the angular flux solution that should be small can
be made arbitraily large, and arbitrarily large errors can occur.

In closing this section, we consider the effect of Galerkin quadrature on the
positivity of the scattering source. It is well known that highly truncated
Legendre expansions often give rise to negatie scattering source values along
directions where small but positive values should be obtained. we have alerady
shown that when Galerkin quadrature is used, poor convergence of the
cross-section expansion does not necessarily imply a porr scattering source
representation. For instance, all finite expansions for the delta function
$\delta(\mu_0-1)$ are poorly converged and negative over large segments of the
angualat domain, yet this type of scattering is treated exactly with Galerkin
quadrature. This suggests that Galerkin quadrature might give significantly
more positive scattering sources than the standard quadrature method when
highly truncated cross-section expansions are used. In general, we find that
the Galerkin quadrature method is somewhat less susceptible to negativity
problems than the standard method, but it is definitely not a solution for
such difficulties. This follows for two reasons. First and foremost, the
Galerkin method is based on a global preservation schem that gives rise to a
trial space scattering source $\tilde{S}(\mu)$ that represents a type of
generalized least-squares fit ot the ``exact'' scattering source $S(\mu)$.
Such fits do not guarantee global positivity, but they do tend to be both
positive and accurate at points wher the scattering source is ``large''.
Negativities generally onlu occur at points wher the scattering source is
small relative to the maximum value that occurs. As a result, we find that
good angle-integrated results can be expected with the Galerkin quadrature
method regardless of negativities in the truncated cross-section expansion,
but differential or half-range quantities may be inaccurate. Second, with
polynomial trial spaces, negativities in the scattering source are often due
solely to negativities in the the trial space angulat flux representation
itself. For instance, it can be deduced from \cref{S_mu_2,xi_l_s,phi_l_int}
that the exact scattering source resulting from a global polynomials of degree
$N-1$ is a global polynomial of degree $N-1$ or less. Thus, for such trial
spaces, the trial spcae scattering source and the exact scattering source are
identical. This implies that the trial space scattering source must be
positive whenever the trial space angulat flux is positive. Unfortunately,
positive flux values ate the interpolatory trial space representation. Since a
standard Gauss quadrature is equivalent to a Galerkin quadrature with a global
polynomial trial space, it follows that the standard $N$-point Gauss
scattering source with expansions of degree $N-1$ must be positive if the
global polynomial interpolating the discrete angular flux values is positive.
However, experience indicates that negativities often arise with $N$-point
Gauss quadratures and cross-section expansions of degree $N-1$ in both neutral
and charged-particle transport calculations. Thus, it would appear that
positivity of the Galerkin scattering source is unlikely to be achieved
without the use of a positive trial space representation (i.e., a
representation that is positive whenever the discrete angular flux values are
positive). Piecewise-constant trial spaces exhibit such positivity, but
initial results with such spaces have been disappointing. Out experiance leads
us to a suspect that a general solution to the negativity problem cannot be
achieved with a Galerkin method based on a weighting space of global
functions.
