\section{Galerkin quadratures}
Until now, we have not explained which quadrature should be used. We have only
said that we need an appropriate quadrature but we did not explain what we
needed. In this section, we will introduce the Galerkin quadrature. Morel
first introduced them in \cite{galerkin_morel} but here we
will introduce them as it was done in \cite{pautz_fp}. 

First, we start by recalling the definition of $\psi$ and $\phi_{l,m}$:
\begin{equation}
\begin{split}
\phi_{l,m}(\br) &= \int_{4\pi} d\bo' \psi(\br,\bo') Y_l^{m,*}\\
&=(D \psi)_{l,m}
\end{split}
\label{D2M}
\end{equation}
where $D$ is the directions-to-moments operator. We also have:
\begin{equation}
\begin{split}
\psi(\br,\bo) &= \sum_{l=0}^{\infty} \sum_{m=-l}^l Y_l^m(\bo)\phi_{l,m}(\br)\\
&= M\phi(\bo)
\end{split}
\label{M2D}
\end{equation}
where $M$ is the moments-to-directions operator. By combining \cref{D2M,M2D}, 
we obtain:
\begin{equation}
(I-MD)\Psi = 0
\label{identity_alpha}
\end{equation}
since for analytic transport $M=D^{-1}$.

Now, we define:
\begin{align}
  \epsilon &= 1 - \la \bar{\mu_0}\ra \label{eps_pautz} \\
    \gamma &= \la\overline{(1-\mu_0)^2}\ra \label{gamma_pautz}\\
    A(\br) &= \frac{1-\bar{\mu_0}(\br)}{\epsilon}\label{A_pautz}\\
  \Sigma_a &= \hat{\Sigma}_a \label{hat_sig_a_0}\\
\Sigma_{s,l} &= \frac{\hat{\Sigma}_{s,l}(\br)}{\epsilon} \label{hat_sig_a_1}\\
\Sigma_{s,l}(\br) &=\Sigma_{s,0}(\br) \(1-\frac{l(l+1)}{2}A(\br)\epsilon +
  O(\gamma)\) \label{hat_sig_a_2}
\end{align}
where $\la X \ra$ is a typical value of $X$. Using \crefrange{eps_pautz}{hat_sig_a_2} 
in \cref{transport_sn}, we get:
\begin{equation}
\begin{split}
&\bo_k\cdot\bn\psi_k(\br) + \(\Sigma_a(\br)+\Sigma_{s,0}(\br)\) \psi_k(\br)
=\\
& \sum_{l=0}^{N-1} \sum_{m=-l}^l Y_l^m(\bo_k) \phi_{l,m}(\br)
\frac{\hat{\Sigma}_{s,0}(\br)}{\epsilon}\(1-\frac{l(l+1)}{2}A(\br)\epsilon +
O(\gamma)\) + Q(\br,\bo_k)
\end{split}
\label{discr_eps}
\end{equation}
where:
\begin{align}
\psi_k(\br) &= \psi(\br,\bo_k)\\
\phi_{l,m}(\br) &= \sum_{k=1}^K w_k Y_l^{m,*}(\bo_k) \psi_k(\br) \label{phi_p}
\end{align}
Here, the $w_k$ and $\bo_k$ are the quadrature weights and directions of a 
quadrature set of order $N$. For triangular quadrature sets, $K=N$ in 1D, 
$K=\frac{N(N+2)}{2}$ in 2D and $K=N(N+2)$ in 3D. \Cref{discr_eps} yields:
\begin{equation}
\begin{split}
&\bo_k\cdot\bn \psi_k(\br) +\hat{\Sigma}_a(\br)\psi_k(\br)+
\frac{\hat{\Sigma}_{s,0}(\br)}{\epsilon} \(\psi_k(\br)-\sum_{l=0}^{N-1}
\sum_{m=-l}^m  Y_l^m(\bo_k) \phi_{l,m}\)\\
&=-\frac{\(\Sigma_{tr}(\br)-\hat{\Sigma}_a(\br)\)}{2} \sum_{l=0}^{N-1}
\sum_{m=-l}^{l} l(l+1) Y_l^m(\bo_k) \phi_l^m(\br)+
 Q(\br,\bo_k) + O\(\frac{\gamma}{\epsilon}\) 
\end{split}
\label{discr_eps_manip}
\end{equation}
We insert the ansatz:
\begin{align}
\psi &= \psi^{(0)} + \epsilon \psi^{(1)} + \epsilon^2\psi^{(2)}+\hdots\\
\phi_{l,m} &= \phi_{l,m}^{(0)} + \epsilon \phi_{l,m}^{(1)} + \epsilon^2
\phi_{l,m}^{(2)}+\hdots
\end{align}
into \cref{discr_eps}. The terms of $O(1)$ give:
\begin{equation}
\begin{split}
\phi_{l,m}^{(0)}(\br) &= \sum_{k=1}^K w_k Y_l^{m,*}(\bo_k) \psi_k^{(0)}(\br)\\ 
&= (D_N\psi^{(0)})_{l,m}
\end{split}
\label{D2M_disc}
\end{equation}
Now we insert the ansatz into \cref{discr_eps_manip} and we look at the terms
of $O(\epsilon^{-1})$:
\begin{equation}
\begin{split}
\psi_k^{(0)}(\br) &=\sum_{l=0}^{N-1} \sum_{m=-l}^l Y_l^m(\bo_k) 
\phi_{l,m}^{(0)}(\br)\\
&= (M_N\phi^{(0)})_k
\end{split}
\label{M2D_disc}
\end{equation}
there is no $O(\gamma)$ term, since $\gamma\rightarrow 0$ when 
$\epsilon\rightarrow 0$, i.e., that there are no
$O(1)$ components in $\gamma$. \Cref{D2M_disc,M2D_disc} may be combined to give:
\begin{align}
&(I-M_ND_N)\psi^{(0)} = 0 \label{identity_a}\\
&(I-D_NM_N)\phi^{(0)} = 0 \label{identity_b}
\end{align}
This means that $\psi^{(0)}$, respectively $\phi^{(0)}$, must be in the kernel
of $I-M_N D_N$, respectively $I-D_N M_N$. Using \cref{phi_p},
\cref{identity_b} becomes:
\begin{equation}
D_N(I-M_N D_N)\Psi^{(0)} = 0
\end{equation}
which is always satisfied if \cref{identity_a} is satisfied.
Therefore, if \cref{identity_a} is satisfied, \cref{D2M_disc,M2D_disc} are
automatically satisfied.

A sufficient condition to satisfy \cref{identity_a} is that $M_N D_N =I$.
This is of course true if $D_N=M_N^{-1}$ like in analytic transport. Obviously
for $M_N$ and $D_N$ to be inverse of each other, the matrices have to be square. 
Thus, the number of moments in the scattering expansion must be equal to number 
of discrete angles. In one-dimension, $D_N = M_N^{-1}$ is satisfied if the 
quadrature set integrates exactly any polynomials of degree $2N-2$, like the 
Gauss-Legendre quadrature does. In multidimensional problems, the standard 
quadrature sets use more discrete angles than there are scattering moments. 
Therefore, $M_N$ and $D_N$ are rectangular matrices and they cannot be inverse 
of each other. In this case, \cref{identity_a} can only be satisfied if 
$\Psi^{(0)}$ is in the kernel of $(I-M_ND_N)$. This can be achieved only if 
$\Psi^{(0)}$ satisfies nonphysical constraints \cite{pautz_fp}. If 
\cref{identity_a} is not satisfied, the asymptotic ansatz is not valid and 
there is no $O(1)$ solution to \cref{discr_eps,phi_p}.\\
If we assume that $M_N D_N=I$, then the $O(\epsilon)$ terms in \cref{phi_p}
give:
\begin{equation}
\begin{split}
\phi_{l,m}^{(1)}(\br) &= \sum_{k=1}^K w_k Y_{l}^{m,*} (\bo_k)
\psi_k^{(1)}(\br)\\
&=(D_N\Psi^{(1)})_{l,m}
\end{split}
\end{equation}
In \cref{discr_eps_manip}, the $O(1)$ terms give:
\begin{equation}
\begin{split}
&\bo_k\cdot \bn \psi_k^{(0)}(\br) + \hat{\Sigma}_a(\br)\psi_k(\br) +
\hat{\Sigma}_{s,0}(\br) \(\psi_k^{(1)} (\br) - \sum_{n=0}^{N-1}
\sum_{m=-l}^{l} Y_l^m (\bo_k)\phi_{l,m}^{(1)}(\br)\)\\
&=-\frac{\Sigma_{tr}(\br)-\hat{\Sigma}_a(\br)}{2} \sum_{l=0}^{N-1}
\sum_{m=-l}^l  l(l+1) Y_l^m(\bo_k)
\phi_{l,m}^{(0)}(\br)+ Q(\br,\bo_k) + O\(\frac{\gamma}{\epsilon}\)
\end{split}
\label{O_1_terms}
\end{equation}
Now, we want the scattering term on the left side of \cref{O_1_terms} to
disappear to keep only $\psi_k^{(0)}$ and $\phi_{l,m}^{(0)}$. This is possible
only if:
\begin{equation}
\psi_k^{(1),*}(\br) = \psi_k^{(1)}(\br)
\label{psi_1_star_old}
\end{equation}
where:
\begin{equation}
\begin{split}
\psi_k^{(1),*} &= \sum_{l=0}^{N-1}\sum_{m=-l}^l Y_l^m(\bo_k)
\phi_{l,m}^{(1)}(\br)\\
&= (M_N\phi^{(1)})_k\\
&=(M_N D_N \psi^{(1)})_k
\end{split}
\label{psi_1_star}
\end{equation}   
which is satisfied because of our previous assumption that $M_N D_N = I$.
\Cref{O_1_terms} yields:
\begin{equation}
\begin{split}
  &\bo_k \cdot \bn \psi_k^{(0)}(\br)+\hat{\Sigma}_a(\br) \psi_k^{(0)}(\br) =
  \frac{\Sigma_{tr}(\br)-\hat{\Sigma}_a(\br)}{2}
  \(\(\frac{\partial}{\partial \mu}(1-\mu^2)\frac{\partial}{\partial \mu}+
  \(\frac{1}{1-\mu^2}\)\frac{\partial^2}{\partial \varphi^2}\)\tilde{\Psi}^{(0)}
  (\br,\bo)\)_{\bo=\bo_k}\\
  &+ Q(\br,\bo_k) + O\(\frac{\gamma}{\epsilon}\)
\end{split}
\label{last_pautz}
\end{equation}
where $k=1,\hdots,K$ and $\tilde{\psi}^{(0)}(\br,\bo)$ is an interpolant
through the points $P_k=\{\bo_k,\Psi_k^{(0)}(\br,\bo)\}$. In one-dimensional 
slab and spherical geometry, $\tilde{\Psi}^{(0)}(\br,\bo)$ is the $(N-1)$-order 
polynomial interpolant through the points $\{\bo_k,\Psi_k^{(0)}(\br)\}$.
``In one-dimensional geometry, $\tilde{\Psi}^{(0)}(\br,\bo)$ is the $(N-1)-$order
polynomial interpolant through the points $P_k$. In multidimensional
geometries, $\tilde{\psi}^{(0)}(\br,\bo)$ is the spherical harmonic
interpolant through the points $P_k$ \cite{pautz_fp}.'' A quadrature which 
satisfies the relation $D_N = M_N^{-1}$ is called a ``Galerkin'' quadrature 
because Morel derived it by using a Galerkin weighting method \cite{galerkin_morel}.

In \cite{galerkin_morel}, Morel made the following suggestions to find the 
correct limit using the $S_N$:
\begin{description}
\item [One-dimensional geometry:] The Gauss-Legendre quadrature set is the
  only quadrature set which is a Galerkin quadrature.
\item [Multidimensional geometry:] The standard quadrature sets have fewer 
  moments than discrete. Therefore to satisfy \cref{psi_1_star}, spherical
  harmonics of higher order need to be added to increase the number of moments
  up to the number of angular flux. Morel in \cite{galerkin_morel} and Reed in 
  \cite{reed} proposed an heuristic algorithm choice to choose the spherical 
  harmonics for multidimensional geometries. For two-dimensional triangular 
  quadrature sets, the spherical harmonics to build $M$ are the following:
  \begin{equation}
    Y_l^m = \left\{
      \begin{aligned}
        &0\leq m \leq n, & \textrm{if }0\leq l\leq N-1 \\
        &0< l\textrm{ odd }\leq N, & \textrm{if }l=N
      \end{aligned}
      \right.
  \end{equation}
  For three dimensional triangular quadrature sets, the spherical harmonics
  are:
  \begin{equation}
    Y_l^m = \left\{
      \begin{aligned}
        &-l\leq m \leq l, & \textrm{if }0\leq l \leq N-1\\
        &-l\leq m <0 \textrm{ and } 0<l\textrm{ odd }\leq l, & \textrm{if }l=N\\
        &-l\leq m\textrm{ even }<0,& \textrm{l=N+1}
      \end{aligned}
      \right.
  \end{equation}
  General necessary conditions and explanations of the heuristic rules above
  have been analyzed in \cite{galerkin_sanchez}.
\end{description}

It is very important to use a Galerkin quadrature for charged particle
transport. Using a standard quadrature may lead to an unphysical solution. To
show it, we need to define the scattering ratio $\bs{C}$ by:
\begin{equation}
\bs{C} = \frac{1}{\Sigma_t}\bs{D}\bs{M}\bs{\Sigma}
\end{equation}
where:
\begin{itemize}
\item $\bs{D}$ is discrete-to-moments matrix
\item $\bs{M}$ is moments-to-discrete matrix
\item $\bs{\Sigma}$ is the scattering matrix containing the moments of the
  scattering cross sections on its diagonal.
\end{itemize}
$\bs{C}$ is a diagonal matrix whose entries are the scattering ratios
$c_l=\frac{\Sigma_{s,l}}{\Sigma_t} \leq 1$. The fact that $\bs{MD}=\bs{I}$
assures a one-to-one relation between the angular flux and the flux moments;
furthermore, the orthogonality of all spherical harmonic functions used in the
angular flux representation is preserved. However, if a standard quadrature is
employed, then $\bs{D} = \bs{M}^T \bs{W}$, where $\bs{W}$ is the diagonal
matrix of weights, and inexact integration occurs for the set of spherical
harmonics than span $\bs{M}$. Thus, $\bs{C}$ will differ from
$\frac{\bs{\Sigma}}{\Sigma_t}$ and its eigenvalue could exceed unity. This is
equivalent to numerically adding multiplication into the medium and, depending
on the amount of leakage present in the geometrical configuration, the
steady-state transport equation may not reach a steady state solution
\cite{pautz_fp}.

A very important property of the Galerkin quadratures is that they treat the
delta function scattering exactly. This is very important for two reasons. 
The first reason is that in charged-particle transport, some cross sections 
have the following form:
\begin{equation}
\Sigma^{k\rightarrow g}(\mu_0) = \Sigma^{k\rightarrow g} \delta(\mu_0-1)
\end{equation}
where $\Sigma^{k\rightarrow g}(\mu_0)$ is the differential cross section
associated with a transfer from group $k$ to group $g$. Thus, it is essential
that delta function scattering be treated exactly. The second reason, which is
more important for us since we work with energy-integrated equation, is that for 
electron scattering, the extended transport correction can be used to reduce
the within-group scattering cross sections by two orders of magnitude or more 
\cite{morel_79}. Without the extended transport correction, the high order of 
the flux moments are very important and DSA does not accelerate the
convergence of the solver. Because the Galerkin method treats delta
function  scattering exactly, the extended transport correction does not
modify the solution of the problem \cite{morel_79}. This is very interesting 
since it allows to reduce significantly the higher moments of the scattering
cross-section expansion without losing of accuracy. Showing this property is
quite simple given what we have done before. We know that in the analytic case
we have:
\begin{equation}
\int_{4\pi} \delta(\mu_0-1) \psi(\bo')\ d\bo' = \psi(\bo)
\end{equation}
When the scattering source is discretized, we have $\bs{\Sigma}=\bs{I}$.
Therefore, it is obvious that:
\begin{equation}
  \bs{D\Sigma M}\psi = \psi
\end{equation}

To really understand the Galerkin quadrature, it is interesting to recall 
the development made in \cite{galerkin_morel}. First, we expand the
one-dimensional angular flux on an interpolatory trial space:
\begin{equation}
\Psi(\mu) = \sum_{m=1}^N \Psi_m B_m(\mu)
\label{psi_b}
\end{equation}
Methods for generating the interpolatory basis function can be found in 
\cite{galerkin_morel}. Next, we expand the scattering source on the Legendre
polynomials:
\begin{equation}
\mc{S}(\mu) = \sum_{l=0}^{\infty} \frac{2l+1}{2} \xi_l P_l(\mu)
\label{S_mu_2}
\end{equation}
where:
\begin{equation}
  \xi_l = \int_{-1}^1 \mc{S}(\mu) P_l(\mu) d\mu
\end{equation}
Now, we use the interpolatory trial space to approximate $S(\mu)$:
\begin{equation}
\tilde{\mc{S}}(\mu) = \sum_{m=1}^N \tilde{\mc{S}}_m B_m(\mu)
\label{tilde_s_mu}
\end{equation}
Since a Galerkin method is used, the residual associated with the trial space 
approximation must be orthogonal to the weighting
space. The residual associated with \cref{tilde_s_mu} is given by:
\begin{equation}
Res(\mu) = \tilde{\mc{S}}(\mu) - \mc{S}(\mu)
\label{R_mu}
\end{equation}
Since the Legendre polynomials form a basis the weighting space, we
orthogonalize against the Legendre polynomials:
\begin{equation}
\int_{-1}^1 Res(\mu) P_l(\mu) d\mu = 0.0,\ \ l=0,\hdots,N-1
\label{int_r_mu}
\end{equation}
\Cref{int_r_mu} is satisfied if:
\begin{align}
&\tilde{\xi}_l = \xi_l,\ \ l=0,\hdots,N-1 \label{xi_vs_xi}\\
&\tilde{\xi}_l = \int_{-1}^1 \tilde{\mc{S}}(\mu) P_l(\mu) d\mu
\end{align}
The main idea of the Galerkin quadrature can be seen on \cref{xi_vs_xi}: ``the
discrete scattering source values are chosen such that the interpolatory
representation for that scattering source has the same Legendre moments
through degree $N-1$ as the exact scattering source calculated with the
interpolatory representation for the angular flux \cite{galerkin_morel}''. 
Because all the elements of the weighting space can be expressed as a linear
combination of Legendre polynomials $P_l$ with $l<N$ and $\int_{-1}^{1}
P_l(x)P_m(x)\ dx=0$ for $l\neq m$, only the first $N-1$ cross-section moments
are needed.

Now, we compare the cross-section ``expansion'' for the Galerkin method and
the $S_n$ method. Both the Galerkin method and the standard $S_n$ method use
the Legendre expansion of the cross section. However, whereas the $S_n$ method
actually uses the Legendre expansion, the ``Galerkin'' scattering source is
fully consistent with the exact cross section. It is not
important whether or not the Legendre cross-section expansion represent
accurately the scattering cross section. What is important is that
$\tilde{S}(\mu)$ is an accurate approximation of $S(\mu)$. This is a very
important property for charged particle transport. For example, the delta
function expansion of finite degree is never converged, that is a problem if a 
$S_n$ method is used. However, like we saw before the scattering source computed 
by the Galerkin method is treated exactly.

If the scattering is isotropic, the $S_n$ method appears superior to the
Galerkin, since the Galerkin requires the number of flux moments to be equal
to the number of angular flux. Fortunately, if the higher order cross-sections
moments are zero, only the first rows of $\bs{D}$ and the first columns of
$\bs{M}$ need to be kept.

A Galerkin quadrature set and its companion quadrature sets are different only
if the expansion order of the scattering cross section is high enough. When the 
scattering is isotropic, the Galerkin quadrature set is the same than its
companion set. Even when the scattering is highly anisotropic, the results
given by the Galerkin set and its companion set can be very close. This is due
to the fact the high order moments of the scattering cross section are often,
but not always, very small.
