\chapter{\uppercase{Conclusions and future work}}
\section{Conclusions}
In this dissertation, we first recalled the development of the
Boltzmann-Fokker-Planck equation and the limitations of the Fokker-Planck operator.
The Boltzmann-Fokker-Planck equation was introduced for charged particle transport
because the scattering kernel is so forward-peaked that a standard Legendre
expansion of the scattering kernel would require hundreds of terms. We also
shown that the Fokker-Planck operator is an asymptotic limit of the Boltzmann 
operator when the scattering is forward-peaked and that the energy transfer 
during a collision tends to zero. In the Boltzmann-Fokker-Planck equation, the 
Fokker-Planck operator is used to model 
the highly forward-peaked scattering collisions whereas the Boltzmann operator 
is used for the wide angle scattering collisions. The Fokker-Planck operator 
simplifies the calculation of the transport equation but it is valid only if 
the kernel is peaked enough. For instance, realistic screened Rutherford 
cross sections are not peaked enough. Then, we introduced the Fokker-Planck 
cross sections which mimic the Fokker-Planck operator when using the Boltzmann
operator. Since Fokker-Planck cross sections are the most forward-peaked cross
section (the extended transport correction renders the delta scattering equivalent 
to pure absorption), we used them for our tests. Finally, we introduced the Galerkin 
quadratures. Galerkin quadratures are capital to have correct results when 
the scattering is highly anisotropic. Galerkin quadratures are equivalent
to standard quadratures when the scattering is weakly anisotropic but when the
scattering is highly anisotropic, using a standard quadrature can introduce
non physical solutions.

After that, we introduced the angular multigrid methods to speed up
the calculation in highly anisotropic medium. When the medium is highly
anisotropic, the Diffusion Synthetic Acceleration (DSA) is not effective
anymore because it cannot speed up the convergence of high order of the 
flux moments. The initial work on this topic was done by Morel and Manteuffel. 
They developed an one dimensional angular multigrid method that they used to
accelerate the convergence of Source Iteration (SI). This angular multigrid 
method uses an $S_{n/2}$ sweep to correct the $S_n$ sweep. The $S_{n/2}$ 
correction is itself corrected by a $S_{n/4}$ correction, etc. until the 
$S_4$ correction is corrected by $P1$ equations. They shown that when 
Fokker-Planck cross sections are used the spectral radius of the new 
method is bounded by 0.6 whereas the spectral radius of DSA can become 
arbitrary close to unity. Pautz et al. generalized the angular multigrid 
method to multidimensional geometries. In this case, the successive correction
used an $S_2$ sweep before DSA. Unfortunately, the generalized method was 
unstable. To stabilize it, the corrections need to be filter with a 
diffusion equation. The effect of this diffusive filter is take out the high 
frequencies of the correction. This method is better than DSA but when the
anisotropy increases, the spectral radius can become arbitrary close to one.
In this research, we recast the angular multigrid method for
multidimensional geometries without filtering as a preconditioner for a Krylov
subspace solver. The new method is always more effective and efficient than 
DSA and is more effective as the anisotropy increases but unlike the one 
dimensional method, the number of iterations does not saturate with 
increasing anisotropy.

We also extended the Modified Interior Penalty (MIP) DSA
developed for triangular cells to arbitrary polygonal meshes using the
PieceWise Linear Discontinuous (PWLD) finite elements. Arbitrary polygonal
meshes can potentially decrease the number of unknowns to mesh a domain and
they simplifies adaptive mesh refinement by suppressing hanging nodes. Then, 
we compared the Algebraic MultiGrid (AMG) preconditioner with the more 
common Symmetric Gauss-Seidel (SGS) preconditioner. AMG was shown to be much 
more efficient than SGS. Among the different AMG methods tested the AGMG code 
was the fastest and about 20 times faster than CG used without preconditioning.
