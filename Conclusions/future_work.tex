\section{Future work}
There are several ways to continue this work:
\begin{description}
  \item[Extension for medical applications:] Modern radiotherapy uses Intensity 
    Modulated Radiation Therapy (IMRT) as one of the methods to
    treat cancer. IMRT uses photons to ionize the water present in the cells
    to form free radicals. These free radicals will damage the DNA of the
    cancerous cells causing them to die. IMRT allows to have several beams with 
    different intensity  profiles. To optimize the intensity profile, it is 
    very common to divide the beams in small beamlets of constant intensity. 
    In real application, the number of beamlets is around a few thousands. 
    The optimization of the position and the intensity of all these beamlets
    is a very complex problem and a lot of objective functions and 
    constraints have been proposed \cite{minima,dose-volume,complexity,math}.
    Due to the large number of variables, the number of dose calculation is
    very high and any speed up of this calculation can significantly decrease
    the time needed for the optimization to complete.
    To be able to compute the dose in a human body, the existing code needs 
    to be extended to handle three dimensional geometries. BiLinear 
    Discontinuous finite elements should be replaced by TriLinear Discontinuous 
    finite elements. PWLD finite elements in 3D are described in \cite{pwld_3d}. 
    The code should also be able to use the multigroup cross sections generated by 
    CEPXS \cite{cepxs}. CEPXS is a code generating multigroup Coupled 
    Electron-Photon cross sections (XS) in order to \cite{cepxs}:
    \begin{itemize}
      \item generate coupled electron-photon cross sections which can be used by
        standard discrete ordinates codes.
      \item model the same physical interactions as Version 2.1 of the
        Integrated-TIGER-Series (ITS) code package.
    \end{itemize}
  \item[AMG for DSA on massively parallel computers:] Like a lot of other
    solvers, AMG algorithms have been developed for massively parallel
    computers \cite{amg_multicore,amg_parallel} and for GPU \cite{nvidia}.
    While developing an AMG method for massively parallel computer, two steps
    of the algorithm must be designed carefully: the coarsening step and the
    smoothing step. In \cite{mis}, the authors explore different coarsening
    methods including an uncoupled algorithm. This uncoupled coarsening method
    coarses the grid without communication between the processor. The problem
    is that the coarsening depends on the domain partitioning. For the
    partitioning to be independent of the domain partitioning, communications
    between the processors are required. Another delicate part of AMG is the
    smoothing step. It is very important that the smoother scales well. Some
    of the smoothers used in today's supercomputers will not scale properly
    with the next generation of supercomputers \cite{amg_parallel}. Given all
    these conditions, it will be important to verify that AMG are still the
    most effective preconditioners on a massively parallel computer.
  \item[Convergence study of AGMG for MIP and development of AMG for MIP:] A 
    more theoretical study of AGMG, which is the most effective preconditioner, 
    is needed. The convergence property of AGMG have been studied for non-singular 
    symmetric M-matrices with non-negative row sum. MIP produces SPD matrices
    but they are not M-matrices due to presence of positive off-diagonal
    entries. Therefore, there is no theoretical background for the convergence 
    MIP using AGMG which is, in this case, an heuristic method. Studying the 
    convergence of AGMG for MIP could lead in a new AMG algorithm or an adaptation 
    of AGMG for MIP.
  \item[Comparison of different AMG methods:] It would be of great interest 
    to compare more AMG schemes, for instance using the ones of hypre 
    \cite{hypre_guide}. In this research, most of the parameters kept their 
    default values but a more extensive study of the effectiveness of AMG 
    would require to tune each method. The choice of the DSA is very important 
    because the choice of the discretization has a huge impact on the properties 
    of the discretized system. Most of the theory for AMG algorithms has been 
    developed for M-matrix and thus, it might be interesting to use a DSA
    which produces such matrices.
\end{description}
