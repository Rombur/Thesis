\section{Cross sections}
\subsection{Physics of the interactions}
The principal interactions that electrons and photons can have, are
\cite{cepxs} :
\begin{description}
\item [electron-to-electron :] collisional scattering, knock-on production,
radiative scattering, elastic scattering and Auger production following impact
ionization.
\item [electron-to-photon :] bremsstrahlung production and fluorescence
production following impact ionization.
\item [photon-to-photon :] incoherent scattering, coherent scattering (Thomson 
scattering and Raleigh scattering) and fluorescence production following
photoionization.
\item [photon-to-electron :] photoelectric production, Compton electron
production, pair production and Auger production following photoionization.
\item [photon-to-positron :] pair production.
\item [positron-positron :] collisional scattering, radiative scattering and
elastic scattering.
\item [positron-to-photon :] bremsstrahlung production, fluorescence
production following impact ionization and annihilation radiation.
\item [positron-to-electron :] knock-on production and Auger production
following impact ionization.
\end{description}

\begin{description}
\item [Thomson scattering :] an electron, assumed to be free, oscillates
classically in response to the electric vector of a passing electromagnetic
wave. The oscillating electron promptly emits photons of the same frequency as
the incident wave. The net effect of Thomson scattering is the redirection of
some incident photons with no transfer of energy to the medium. Thomson
scattering represents the low-energy limit of Compton scattering, as the
incident photon energy approaches zero \cite{radiation}.
\item [Raleigh scattering :] the scattering results from the combined,
coherent action of an atom as a whole. The scattering angle is usually very
small. There is no appreciable loss of energy by the photon to the atom,
which, however, does ``recoil" enough to conserve momentum \cite{radiation}.
\item [photo-electric effect :] a photon undergoes an interaction with an
absorber atom in which the photon disappears. An energetic photoelectron is
ejected from one of the bound shells of the atom. For gamma rays of
sufficient energy, the most probable origin of the photoelectron is the $K$
shell of the atom. The photoelectron appears with an energy given by :
\begin{equation}
E_{e^-} = h\nu -E_b
\end{equation}
where $E_b$ represents the binding energy of the photoelectron in its original
shell. The interaction leaves an ionized absorber atom with a vacancy in one
of its bound shells. This hole is quickly filled through the rearrangement of
electrons from other shells of the atom. Therefore, one or more X-ray photons
may also be generated. In some fraction of the cases, the emission of Auger
electrons may substitute for the X-ray. The photoelectric process is the
predominant mode of interaction for photons of relatively low energy. The
process is more important for high atomic number $Z$.The probability of
photoelectric absorption $\tau$ depends on $Z$ and $h\nu$ as :
\begin{equation}
\tau \propto \frac{Z^n}{\(h\nu\)^3}
\end{equation}
where $n$ varies between 4 and 5. The photoelectric interaction is most likely
to occur if the energy of the incident photon is just greater than the binding
energy of the electron with which it interacts.
\item [fluorescence :] fluorescence is a process in which some of the energy
of a photon is used to create a second photon of less energy.
\item [Compton effect :]  a Compton interaction is one in which only a portion
of the energy is absorbed and a photon is produced with reduced energy. This
photon leaves the site of the interaction in a different direction from that
of the original photon. If a photon of energy $h\nu$ and momentum
$\frac{h\nu}{c}$ is a incident on a stationary, free electron, after the
collision, the photon is scattered at an angle $\theta$ with energy $h\nu'$.
The Compton shift is given by :
\begin{equation}
\Delta \lambda = \lambda'-\lambda = \frac{h}{mc} \(1-\cos \theta\)
\end{equation}
The Klein-Nishina formal gives the differential cross section of photons
scattered from a single free electron :
\begin{equation}
\frac{d \sigma}{d\bo} = \frac{k_0^2 e^4}{2 m_e^2
c^4}\(\frac{\nu'}{\nu}\)^2\(\frac{\nu}{\nu'}+\frac{\nu'}{\nu}-\sin^e \theta\)
m^e sr^{-1}
\end{equation}
where $\frac{d\Sigma}{d\bo}$ is the differential scattering cross sections,
$k_0 = \frac{1}{4\pi \epsilon_0}$, $\epsilon_0$ is the vacuum permittivity,
$e$ is the elementary charge, $m_e$ is the mass of an electron and $c$ is the
speed of light in vacuum \cite{radiation}.
\end{description}

\subsection{CEPXS and CEPXS-BFP}
Inelastic interactions, collisional and radiative scattering, can be divided
into two classes : ``catastrophic" interactions that result in large-energy
losses and ``soft" interactions that result in small-energy losses. p25 CEPXS
