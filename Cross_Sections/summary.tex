\subsection{Summary}

\subsubsection{Elastic scattering}
The cross section data of Riley, et. al. is used at non-relativistic energies
($<256 keV$). The Mott cross section with Moliere screening is used at
relativistic energies ($>256keV$). The extended transport correction is applied
to these elastic-scattering cross sections to make them amenable to
representation by a low-order Legendre expansion.

\subsubsection{Inelastic scattering}
The M\o ller cross section is used for large-energy loss collisions. For other
collisions, the restricted CSD approximation is used. In this approximation,
the restricted collisional stopping power, $S^C$, specifies the energy loss
the electron per pathlength due to small-energy loss collisions. In CEPXS,
$S^C$ is calculated as the difference between the total collisional stopping
power, $\mathcal{S}^C$, and that portion of $\mathcal{S}^C$ that is due to
large-energy loss collisions. The total collisional stopping power is
tabulated at discrete energies for all elements in the set of electron data
called DATAPAC. For electron energies greater than 10keV, these collisional
stopping powers (which are derived from Bethe theory) are employed. CEPXS
employs an extrapolation of the stopping power for those elections with energy
less than 10 keV

\subsubsection{Knock-on production}
The M\o ller cross section is used to determine the production of knock-on
electrons with energies down to the cutoff energy. Knock-on production and
impact ionization are not correlated in CEPXS.

\subsubsection{Radiative energy loss and bremsstrahlung photon production}
The bremsstrahlung cross section is based on a formulation by Berger et al.,
involving Born-approximation cross sections. This
cross section is used to describe both bremsstrahlung photon production and
the slowing down of an electron by radiative events, the restricted CSD
approximation is used. In this approximation, the restricted radiative
stopping power, $S^B$, specifies the energy loss of the electron per
pathlength due to radiative emission in which the energy loss is small. In
CEPXS, $S^B$ is calculated as the difference between the total radiative
stopping power, $\mathcal{S}^B$, and that portion of $\mathcal{S}^B$ that is
due to large-energy loss radiative events. The total radiative stopping power
is tabulated at discrete energies for all elements in DATAPAC.\\
Bremsstrahlung photons are produced with energies greater than or equal to the
cutoff energy. 

\subsubsection{Impact ionization}
The Gryzinski impact ionization cross sections are used for the K,L1,L2,L3 and
M shells. With the NO-PCODE option, impact ionization is limited to the
K-shell and the Kolbenstvedt cross section is used. Following impact
ionization, a cascade of fluorescence photons and Auger electrons will be
produced. 

\subsubsection{Compton incoherent scattering and electron production}
The Klein-Nishina cross section is used for Compton incoherent scattering and
electron production. 

\subsubsection{Photoelectric absorption and photoelectron production}
The Biggs-Lighthill cross sections are used for photoelectric absorption. When
using the NO-PCODE option, photoelectrons emerge with either the energy of the
photon or with kinetic energy equal to that of the photon less the K-shell
binding energy. The angular distributions devised by Fischer and Sauter are
used for the emitted electrons. The relaxation cascade that accompanies
photoionization is the same as that which accompanies impact ionization.

\subsubsection{Pair interaction and the production of charged secondaries}
The Biggs-Lighthill cross sections are used for the absorption of photons by
pair interactions. The energy distribution of the charged secondaries produced
by a pair interaction are obtained from Bethe-Heitler theory. Since
Bremsstrahlung production and pair production are inverse processes, the same
angular distribution is used for both bremsstrahlung photons and the
secondaries produced by pair interactions.\\
While positrons can be treated as a separate species of particles in CEPXS,
the default option is that positrons are assumed to be electrons. This
assumption has only a minor effect on the spatial profile of the energy
deposition. Namely, annihilation radiation is produced at the site of the pair 
interaction rather than where a positron is absorbed. However, the
assumption that a positron is identical to a electron has a major impact on
the charge deposition profile. Under this assumption, two electronic charges
are removed from the medium at the site  of the pair interaction. In an actual
pair interaction, no charge removal occurs at the site of the interaction.

