\section{Modified Interior Penalty on arbitrary polygonal cells} \label{sec_mip}
First, let us recall the derivation of MIP (see \cite{mip}). MIP is based on 
the Interior Penalty (IP) form of the diffusion equation
\cite{ip,mip} (weakly imposed boundary conditions are applied to each grid
cell and the test functions are averaged over cells). The continuous equations
are :
\begin{align}
  -\bn\mathrm{D} \bn \phi_0 +\Sigma_a \phi_0 &= Q_0 & \textrm{ for }\br \in
  \mc{D}\\
  \frac{1}{4}\phi_0 - \frac{1}{2} \mathrm{D} &=0 & \textrm{ for }\br \in
  \partial \mc{D}^d\\
  -\mathrm{D}\partial_n \phi_0 &= J^{inc} & \textrm{ for }\br \in \partial
  \mc{D}^r
\end{align}
where $\phi_0$ is the scalar flux and $\mathrm{D}$ is the diffusion
coefficient. Applying the IP technique, we obtain:
\begin{equation}
  b_{IP}(\phi_0,\phi_0^*) = l_{IP}(\phi_0^*)
\end{equation}
\begin{equation}
  \begin{split}
    b_{IP}(\phi_0,\phi_0^*)=& \(\Sigma_a \phi_0,\phi_0^*\)_{\mc{D}} + 
    (\mathrm{D} \bn \phi_0, \bn \phi_0^*)_{\mc{D}} + \(\kappa_e^{IP} \llb
    \phi_0\rrb, \llb \phi_0^*\rrb\)_{E_h^i}\\
    &+ \(\llb \phi_0 \rrb,\ldb \mathrm{D} \partial_n \phi_0^*\rdb\)_{E_h^i} +
    \(\ldb \mathrm{D} \partial_n \phi_0\rdb,\llb \phi_0^* \rrb\)_{E_h^i} + 
    \(\kappa_e^{IP} \phi_0, \phi_0^*\)_{\partial \mc{D}^d}\\
    & -\frac{1}{2} \(\phi_0,\mathrm{D}\partial_n,\phi_0^*\)_{\partial \mc{D}^d}
    -\frac{1}{2}\(\mathrm{D}\partial_n\phi_0,\phi_0\)_{\partial \mc{D}^d}
  \end{split}
  \label{b_ip}
\end{equation}
and:
\begin{equation}
  l_{IP}(\phi_0^*) = \(Q_0,\phi_0^*\)_{\mc{D}} +
  \(J^{inc},\phi_0^*\)_{\partial \mc{D}^r}
\end{equation}
where $\phi_0$ and $\phi_0^*$ $\in W_{\mc{D}}^h$ where the finite
dimensional polynomial space is $W_{\mc{D}}^h = \{\phi_0 \in L^2(\mc{D});
\phi_0|_K \in V_p(K), \forall K \in \mathbb{T}_h\}$ with $V_p(K)$ is the space
of polynomials of degree up to $p$ on element $K$, $(f,g)_{\mc{D}} = 
\sum_{K\in \mathbb{T}_h} \(f,g\)_K$, 
$(f,g)_K = \int_K fg\ d\br$, $(f,g)_{E_h^i}=\sum_{e\in E_h^i}(f,g)_e$, 
$(f,g)_e = \int_e fg\ ds$, $Q_0 = \Sigma_{s,0} \delta \phi$, 
$J^{inc} = \sum_{\bo_m\cdot\bs{n}_b >0} w_m |\bo_m \cdot \bs{n}_b| \delta
\psi_m$, $\mathbb{T}_h$ is the mesh used to discretize the spatial domain
$\mc{D}$ into nonoverlapping elements $K$, $E_h^i$ is the set of interior
edges, $\partial \mc{D}^d$ is the boundary of
the domain with Dirichlet condition, $\partial \mc{D}^r$ is the boundary of
the domain with reflective condition, $\Sigma_a$ is the absorption 
cross section, D is the diffusion coefficient, $\bs{n}_b$ is the outward
normal unit vector, $\partial_n = \bs{n}_e\cdot \bn$ where $\bs{n}_e$ is the 
normal unit vector associated with a given edge $e$ (on the boundary
$\bs{n}_e = \bs{n}_b$), $\llb\phi_0\rrb = \phi_0^+ - \phi_0^-$ is the 
jump of $\phi_0$ at the interface between two elements, $\ldb\phi_0\rdb = 
\frac{\phi_0^+ + \phi_0^-}{2}$ is the mean of $\phi$ at the interface between 
two elements, $\phi_0^{\pm}=\lim_{s\rightarrow 0^{\pm}}\phi_0(\bs{r}+s\bs{n}_e)$, 
$Q_0$ represents the volumetric source term due to the successive 
error in the scattering term, and $J^{inc}$ is the incoming partial current.
The penalty parameter $\kappa_{e}^{IP}$ is given by:
\begin{equation}
  \kappa_e^{IP} = \left\{
    \begin{aligned}
      & \frac{c\(p^+\)}{2} \frac{\mathrm{D}^+}{h_{\bot}^+}+\frac{c\(p^-\)}{2}
      \frac{\mathrm{D}^-}{h_{\bot}^-}& \textrm{ on interior edges, i.e., }
      e\in E_h^i\\
      & c(p)\frac{\mathrm{D}}{h_{\bot}}& \textrm{ on boundary edges, i.e., }
      e\in \partial \mc{D}^d
    \end{aligned}
  \right.
\end{equation}
where $c(p)$ is given by $c(p)=2p(p+1)$, $p$ is the polynomial order ($p=1$ in
this research) and $h_{\bot}$ is the length of the cell in the direction
orthogonal to the edge $e$. On triangular cells, $h_{\bot}$ equals
$\frac{2A}{L_e}$ where $A$ is the area of the triangle and $L_e$ is the length
of the edge $e$.\\
This discretization of DSA is SPD. Unfortunately, the authors in \cite{mip} 
found that IP yields 
an unstable DSA scheme when the cells are large compared to the mean-free-path. 
This phenomenon is due to the fact that in optically thick medium, the ratios 
$\frac{D^{\pm}}{h^{\pm}}$ are very small. Therefore, the penalty coefficient 
is small and the method is unstable.\\
This led the authors of \cite{mip} to develop another discretization of DSA: 
the Diffusion Conforming Form (DCF). This discretization starts from 
the one-group $S_n$ transport equation with isotropic source and scattering:
\begin{equation}
  \bo_d\cdot\bn \psi_d(\br) + \Sigma_t(\br) \psi_d = \frac{1}{4\pi}
  \Sigma_s(\br) \phi_0(\br) + \frac{1}{4\pi}Q(\br)
  \label{1g_iso}
\end{equation}
The variational form of this equation is:
\begin{equation}
  b(\psi,\psi^*) = l(\psi^*)
\end{equation}
with:
\begin{equation}
  b(\psi,\psi^*) = a(\psi,\psi^*)  - \sum_{e\in \partial \mc{D}^r}
  \sum_{\bo_d\cdot \bs{n}_b<0} 4\pi w_m \la \psi_{d'},\psi_d^*\ra_e -
  \(\Sigma_s \phi_0,\phi_0^*\)_{\mc{D}}
\end{equation}
\begin{equation}
  \begin{split}
    a(\psi,\psi^*) =& \sum_{d=1}^M 4 \pi w_d \(\(\bo_d\cdot \bn +\Sigma_t\)
    \psi_d,\psi_d^*\)_{\mc{D}} + \sum_{d=1}^M 4\pi w_d \la \llb \psi_m\rrb,
    \psi_m^{*,+}\ra_{E_h^i}\\
    &+\sum_{e\in \partial \mc{D}^r} \sum_{\bo_d\cdot \bs{n}_b<0} 4 \pi\la 
    \psi_d,\psi_d^*\ra_e 
  \end{split}
\end{equation}
\begin{equation}
  l(\psi^*) = (Q,\phi^*)_\mc{D} + \sum_{e\in\partial \mc{D}^d} \sum_{\bo_d
    \cdot \bs{n}_b<0} 4\pi w_m \la \psi_m^{inc},\psi_m^*\ra_e
\end{equation}
where $\la f,g, \ra_e = \int_e |\bo_d \cdot \bs{n}_e| fg\ ds$, $\la f,g
\ra_{E_h^i} = \sum_{e\in E_h^i} \la f,g \ra_e$, and $\psi$ and $\psi^*$ 
$\in W_{\mc{D}}^h$ where the finite dimensional polynomial space is 
$W_{\mc{D}}^h = \{\psi \in L^2(\mc{D});
\psi|_K \in V_p(K), \forall K \in \mathbb{T}_h\}$ with $V_p(K)$ is the space
of polynomials of degree up to $p$ on element $K$, $(f,g)_{\mc{D}} = 
\sum_{K\in \mathbb{T}_h} \(f,g\)_K$.\\
The operator $a$ is constituted of the streaming term, the interaction term 
and the upwind terms. This is the operator that is inverted during a transport 
sweep. The operator
$b$ contains $a$, the scattering term and the reflective boundary
conditions. This operator is inverted upon convergence of SI or the Krylov solver.\\
The discretized SI at iteration $\ell$ can be written as:
\begin{equation}
  a(\psi^{(\ell+1/2)},\psi^*) = l(\psi^*) + (\Sigma_s
  \phi_0^{(\ell)},\phi_0^*)_{\mc{D}} + \sum_{e\in \mc{D}^r} \sum_{\bo_d \cdot
  \bs{n}_b < 0} 4\pi w_d \la \psi_{d'}^{(\ell)},\psi_d^*\ra_e
  \label{si_l_b}
\end{equation}
\begin{equation}
  \phi^{(\ell)} = \sum_{d=1}^M w_d \psi_d^{(\ell)}.
  \label{si_l_a}
\end{equation}
If we note the converged angular and scalar fluxes $\psi^c$ and $\phi_0^c$, we get:
\begin{equation}
  a(\psi^{c},\psi^*) = l(\psi^*) + (\Sigma_s
  \phi_0^{c},\phi_0^*)_{\mc{D}} + \sum_{e\in \mc{D}^r} \sum_{\bo_d \cdot
  \bs{n}_b < 0} 4\pi w_d \la \psi_{d'}^{c},\psi_d^*\ra_e
  \label{si_c_b}
\end{equation}
\begin{equation}
  \phi_0^{c} = \sum_{d=1}^M w_d \psi_d^{c}.
  \label{si_c_a}
\end{equation}
Subtracting \cref{si_l_a}, respectively \cref{si_l_b}, to \cref{si_c_a},
respectively \cref{si_c_b}, we obtain an angular error equation:
\begin{equation}
  a(\epsilon^{(\ell+1/2)},\psi^*) = \(\Sigma_s \mc{E}^{(\ell)},\phi_0^*\)_{\mc{D}} +
  \sum_{e\in \partial \mc{D}^r} \sum_{\bo_d \cdot \bs{n}_b<0} 4\pi w_d \la
  \epsilon_{d'}^{(\ell)},\psi_m^*\ra_e
  \label{si_f}
\end{equation}
and:
\begin{equation}
  \mc{E}^{(\ell)} = \sum_{d=1}^M w_d \epsilon_d^{(\ell)}
\end{equation}
where the angular error and the scalar error are given by:
\begin{align}
  & \epsilon^{(\ell)} = \psi^c-\psi^{(\ell)}\\
  & \mc{E}^{(\ell)} = \phi_0^c-\phi_0^{(\ell)}.
\end{align}           
It is important to note that the linear form $l$ has disappeared from
\cref{si_f} and thus, the external volumetric source and the incident
Dirichlet boundary conditions have disappeared. We now introduce:
\begin{align}
  & \delta \psi^{(\ell)} = \psi^{(\ell+1/2)}-\psi^{(\ell)} = \epsilon^{(c)}-
  \epsilon^{(l+1/2)}\\
  & \delta \phi_0^{(\ell)} = \phi_0^{(\ell+1/2)} - \phi_0^{(\ell)} = \mc{E}^{(\ell)} -
  \mc{E}^{(\ell+1/2)}.
\end{align}
Finally, we get the final form of the transport equation for the error:
\begin{equation}
  b(\epsilon^{(\ell+1/2)},\psi^*) = \(\Sigma_s \delta
  \phi_0^{(\ell)},\phi_0^*\)_{\mc{D}} + \sum_{e\in \partial \mc{D}^r} \sum_{\bo_d
  \cdot \bs{n}_b<0} 4\pi w_d \la \delta \psi_{d'}^{(\ell)},\psi_d^*\ra_e.
  \label{tsa}
\end{equation}
Note that solving \cref{tsa} would give the exact correction needed to
obtain the converged transport solution:
\begin{align}
  &\psi^c = \psi^{(\ell+1/2)} + \epsilon^{(\ell+1/2)}\\
  &\phi_0^c = \phi_0^{(\ell+1/2)} + \mc{E}^{(\ell+1/2)}
\end{align}
but solving \cref{tsa} is as difficult as solving the transport equation.
Therefore, we will replace the transport operator in \cref{tsa} by a diffusion 
operator instead. The solution of this diffusion operator will be denoted by 
a $\tilde{}$ symbol:
\begin{align}
  & \psi^{(\ell+1)} = \psi^{(\ell+1/2)} + \tilde{\epsilon}^{(\ell+1/2)}\\
  & \phi^{(\ell+1)} = \phi^{(\ell+1/2)} + \tilde{\mc{E}}^{(\ell+1/2)}
\end{align}
If we assume that the primal and dual angular fluxes
are linearly anisotropic (diffusion approximation) and we assume Fick's law to
be valid:
\begin{align}
  & \bs{J} = -\mathrm{D}\bn \phi_0\\
  & \bs{J}^* = \mathrm{D}\bn \phi_0^*,
\end{align}
we then obtain:
\begin{align}
  \tilde{\epsilon}^{(\ell+1/2)} &= \frac{1}{4\pi} (\phi_0-3\mathrm{D} \bn
  \phi_0 \cdot \bo_d) \label{eps_tilde}\\
  \tilde{\psi}_d^* &= \frac{1}{4\pi} (\phi_0^*+3\mathrm{D} \bo \phi_0^*\cdot
  \bo_d) \label{psi_tilde}
\end{align}
Substituting \cref{eps_tilde,psi_tilde} in \cref{tsa}, we obtain a
synthetic discontinuous finite elements diffusion operator in which the
scalar flux $\phi_0$ is the only unknown. Using:
\begin{align}
  & \sum_{d=1}^M w_d = 4 \pi \label{omega_0}\\
  & \sum_{d=1}^M w_d \bo_d = 0 \label{omega_1}\\
  & \sum_{d=1}^M w_d \bo_d \cdot \bo_d = \frac{4\pi}{3}\bs{I},
  \label{omega_2}
\end{align}
we obtain:
\begin{equation}
  \sum_{d=1}^M 4 \pi w_d \(\Sigma_t \tilde{\epsilon}_d^{(\ell+1/2)},
  \tilde{\psi}_d^*\)_{\mc{D}} = \(\Sigma_t \phi_0,\phi_0^*\)_{\mc{D}} -
  \(3\Sigma_t \mathrm{D} \bn \phi_0,\mathrm{D} \bn \phi_0^*\)_{\mc{D}}
\end{equation}
\begin{equation}
  \(\Sigma_s \tilde{\mc{E}}^{(\ell+1/2)},\tilde{\phi}^*\)_{\mc{D}} =
  \(\Sigma_s \phi_0,\phi_0^*\)_{\mc{D}}.
\end{equation}
If we define:
\begin{equation}
  \mathrm{D} = \frac{1}{3\Sigma_t}
\end{equation}
we get:
\begin{equation}
  \begin{split}
    & \sum_{d=1}^M 4 \pi w_d \(\Sigma_t \tilde{\epsilon}_d^{(\ell+1/2)},
    \tilde{\psi}_d^*\)_{\mc{D}} + \(\Sigma_s \tilde{\mc{E}}^{(\ell+1/2)},
    \tilde{\phi}^*\)_{\mc{D}} = \(\Sigma_a \phi_0,\phi_0^*\)_{\mc{D}} -\\
    &\(\bn \phi_0, \mathrm{D} \bn \phi_0^*\)_{\mc{D}}.
  \end{split}
\end{equation}
Now, we analyze the streaming term:
\begin{equation}
  \begin{split}
    \sum_{d=1}^M 4 \pi w_d \(\bo_d \cdot \bn \tilde{\epsilon}_d^{(\ell+1/2)},
    \tilde{\psi}_d^*\)_{\mc{D}}  =& \(\bn \phi_0,\mathrm{D} \bn
    \phi_0^*\)_{\mc{D}} - \(\bn \cdot \mathrm{D} \bn
    \phi_0,\phi_0^*\)_{\mc{D}}\\
    =& \(\bn \phi_0, \mathrm{D} \bn \phi_0^*\)_{\mc{D}} - \(\bn \cdot \bn
    \phi_0,\phi_0^*\)_{\mc{D}}\\
    =& \(\bn \phi_0,\mathrm{D} \bn \phi_0^*\)_{\mc{D}} + \(\mathrm{D} \bn
    \phi_0,\bn \phi_0^*\)_{\mc{D}}\\
     & + \(\mathrm{D} \bn \phi_0^+ \cdot \bs{n}_e, 
    \phi_0^{*,+}\)_{E_h^i}\\ 
     & - \(\mathrm{D} \bo \phi_0^{-} \cdot \bs{n}_e,
    \phi_0^{*,-}\)_{E_h^i}\\
     & - \(\mathrm{D}\bn \phi_0 \cdot \bs{n_e},\phi_0^*\)_{\partial \mc{D}}
  \end{split}
\end{equation}
where integration by part was performed and we used:
\begin{equation}
  \sum_{d=1}^M w_d \bo_d \cdot \bo_d \cdot \bo_d = 0.
\end{equation}
Manipulating the interior edge terms give successively :
\begin{equation}
  \begin{split}
    \sum_{d=1}^M 4 \pi w_d \la \llb \tilde{\epsilon}_d^{(\ell+1/2)}\rrb,
    \tilde{\psi}_d^{*,+}\ra_{E_h^i} =& \sum_{e \in E_h^i} \sum_{d=1}^M 4 \pi
    w_d |\bo_d \cdot \bs{n}_e| \(\llb \tilde{\epsilon}_m^{(\ell+1/2)}\rrb,
    \tilde{\psi}_d^{*,+}\)_e\\
    =& \sum_{e \in E_h^i} \Bigg(\sum_{\bo_d \cdot \bs{n}_e>0} \frac{w_d}{4\pi}
    |\bo_d \cdot \bs{n}_e| \big(\llb\phi_0\rrb - 3\\ 
    &\llb \mathrm{D} \bn \phi_0\rrb \cdot \bo_d, 
    \phi_0^{*,+} + 3 \mathrm{D} \bn  \phi_0^{*+} \cdot \bo_d\big)_e -\\
    &\sum_{\bo_d \cdot \bs{n}_e<0} \frac{w_d}{4\pi} |\bo_d \cdot
    \bs{n}_e| \bigg(\llb \phi_0 \rrb - 3 \llb \mathrm{D}\bn \phi_0 \rrb\cdot
    \bo_d,\\
    &\phi_0^{*,-} +3 \mathrm{D} \bn \phi_0^{*,-} \cdot \bo_d \bigg)_e\Bigg)\\
    = & \sum_{e\in E_h^i} \sum_{\bo_d \cdot \bs{n}_e>0} \frac{w_d}{4\pi}
    |\bo_d \cdot \bs{n}_e| \(\(\llb \phi_0 \rrb - 3 \llb \mathrm{D} \bn 
    \phi_0 \rrb \cdot\right.\right.\\
    &\left. \left. \bo_d,\phi_0^{*,+} + 3 \mathrm{D} \bn \phi_0^{*,+}\cdot \bo_d\)_e -
    \(\llb \phi_0 \rrb + 3\right.\right.\\ 
    &\left.\left. \llb \mathrm{D} \bn \phi_0 \rrb \cdot \bo_d,
    \phi_0^{*,-} - 3 \mathrm{D} \bn \phi_0^{*,-} \cdot \bo_d\)_e\)\\
    =&\frac{1}{4} \(\llb \phi_0\rrb,\llb \phi_0^* \rrb\)_{E_h^i} + \(\llb
    \phi_0 \rrb,\ldb \mathrm{D} \bn \phi_0^* \cdot \bs{n}\rdb \)_{E_h^i}\\
    &-\(\llb \mathrm{D} \bn \phi_0 \cdot \bs{n}\rrb,\ldb \phi_0^*
    \rdb\)_{E_h^i} -\frac{9}{16} \(\llb \mathrm{D} \bn \phi_0\rrb,\right.\\
    &\left.  \llb \mathrm{D} \bn \phi_0^*\rrb\)_{E_h^i}
    - \frac{9}{16} \( \llb \mathrm{D} \bn \phi_0 \cdot \bs{n}\rrb,\right.\\
    &\left.\llb \mathrm{D} \bn \phi_0^* \cdot \bs{n}\rrb\)_{E_h^i}
  \end{split}
\end{equation}
where we employed the following properties of the angular quadrature:
\begin{align}
  & \sum_{\bo_d \cdot \bs{n}>0} w_d |\bo_d \cdot \bs{n}| \approx \pi \\
  & \sum_{\bo_d\cdot \bs{n}>0} w_d |\bo_d \cdot \bs{n}| \bo_d \approx
  \frac{2\pi}{3} \bs{n}\\
  & \sum_{\bo_d \cdot \bs{n}>0} w_d |\bo_d\cdot \bs{n}|\bo_m\cdot \bo_d
  \approx \frac{\pi}{4}\(\bs{I}+\bs{n}\bs{n}\)
\end{align}
where $\bs{n}\bs{n}$ is a matrix. Even if these properties cannot be strictly
satisfied, numerical results show that the error is negligible.
Finally, we obtain:
\begin{equation}
  b_{DCF}(\phi_0,\phi_0^*) = l_{DCF}(\phi_0^*)
\end{equation}
with:
\begin{equation}
  \begin{split}
    b_{DCF}(\phi_0,\phi_0^*)=& \(\Sigma_a \phi_0,\phi_0^*\)_{\mc{D}} +
    \(\mathrm{D}\bn \phi_0,\mathrm{D}\bn \phi_0\)_{\mc{D}} + \frac{1}{4} 
    \(\llb\phi_0\rrb,\llb\phi_0^*\rrb\)_{E_h^i}\\
    & + \(\llb \phi_0\rrb, \ldb\mathrm{D}\partial_n 
    \phi_0^*\rdb\)_{E_h^i} + \(\ldb\mathrm{D} \partial_n \phi_0\rdb, \llb
    \phi_0^*\rrb\)_{E_h^i}\\
    & + \frac{1}{4} \(\phi_0,\phi_0^*\)_{\partial \mc{D}^d} -\frac{1}{2}
    \(\phi_0,\mathrm{D} \partial_n \phi_0^*\)_{\partial \mc{D}^d}\\
    & - \frac{1}{2} \(\mathrm{D} \partial_n \phi_0,\phi_0\)_{\partial
    \mc{D}^d}-\frac{9}{16} \(\llb\mathrm{D}\bn\phi_0\rrb,\llb\mathrm{D} \bn
    \phi_0^*\rrb\)_{E_h^i}\\ 
    &- \frac{9}{16}\(\llb\mathrm{D}\partial_n \phi_0\rrb,\llb 
    \mathrm{D}\partial_n \phi_0^*\rrb \)_{E_h^i}-\frac{9}{16} 
    \(\mathrm{D}\bn \phi_0,\mathrm{D}\bn \phi_0^*\)_{\partial \mc{D}^d}\\
    & - \frac{9}{16} \(\mathrm{D}\partial_n
    \phi_0,\mathrm{D} \partial_n \phi_0^*\)_{\partial \mc{D}^d}
    -\frac{9}{4}\(\mathrm{D}\partial_n \phi_0,\mathrm{D}\partial_n
    \phi_0^*\)_{\partial \mc{D}^r}
  \end{split}
\end{equation}
\begin{equation}
  \begin{split}
    l_{DCF}(\phi_0^*) =& \(Q_0,\phi_0^*\)_{\mc{D}}+\(
     J^{inc},\phi_0^*\)_{\partial \mc{D}^r} - \(\bs{\mc{Y}},\mathrm{D}\bn
    \phi_0^*\)_{\partial \mc{D}^r}\\ 
    &+ 2 \(\bs{\mc{Y}}^{inc} \cdot \bs{n}, \mathrm{D} \partial_n 
    \phi_0^*\)_{\partial \mc{D}^r}
  \end{split}
\end{equation}
where $\bs{\mc{Y}}^{inc} = - \sum_{\bo_d \cdot \bs{n}_b>0} 3 w_d \bo_d |\bo_d
\cdot \bs{n}_b| \delta \psi_d^{(\ell)}$.\\
DCF is symmetric but not positive definite positive. DCF is unstable for cell size
in between one and four mean-free-paths but it is stable and very efficient 
for optically thick medium. In this case, $\bn \phi_0 \approx 0$ and 
$\partial_n \phi_0 \approx 0$ and the terms $\(\mathrm{D} \bo
\phi_0^{\pm},\mathrm{D}\bn\phi_0^{*,\pm}\)$ and $\(\mathrm{D} \partial_n
\phi_0^{\pm},\mathrm{D}\partial_n \phi_0^{\pm}\)$ are negligible.
In this limit, $b_{DCF}$ becomes:
\begin{equation}
  \begin{split}
    b_{DCF}(\phi_0,\phi_0^*)=& \(\Sigma_a \phi_0,\phi_0^*\)_{\mc{D}} +
    \(\mathrm{D}\bn \phi_0,\mathrm{D}\bn \phi_0\)_{\mc{D}}
    +\frac{1}{4}\(\llb\phi_0\rrb,\llb\phi_0^*\rrb\)_{E_h^i}\\
    &+ \(\llb \phi_0\rrb, \ldb\mathrm{D}\partial_n \phi_0^*\rdb\)_{E_h^i} +
    \(\ldb\mathrm{D} \partial_n \phi_0\rdb, \llb \phi_0^*\rrb\)_{E_h^i}
    + \frac{1}{4} \(\phi_0,\phi_0^*\)_{\partial \mc{D}^d}\\ 
    &-\frac{1}{2} \(\phi_0,\mathrm{D} \partial_n \phi_0^*\)_{\partial \mc{D}^d} -
    \frac{1}{2} \(\mathrm{D} \partial_n \phi_0,\phi_0\)_{\partial
    \mc{D}^d}
  \end{split}
\end{equation}
which is exactly the same equation than \cref{b_ip} if 
$\kappa_e^{IP}=\frac{1}{4}$.\\
MIP is obtained by replacing the penalty coefficient, $\kappa_e^{IP}$, 
by $\kappa_e^{MIP}=\max \(\kappa_e^{IP},\frac{1}{4}\)$ in \cref{b_ip}.
This ensures that MIP will converge in optically thick medium since the
penalty coefficient will never be less than $\frac{1}{4}$.

DSA gives only a correction for the scalar flux but by assuming that the angular 
dependence satisfies a diffusion expansion, the angular correction can be 
computed using \cref{eps_tilde}. 
This correction can be used when some of the boundary 
conditions are periodic or reflective.

If PWLD finite elements are used instead of LD finite elements, the weak form 
of MIP is not modified. However when the cells are not triangular, there is no 
simple way to compute $h_{\bot}$. To simplify this, we assume that the polygonal 
cells are not too far from being regular polygonal 
cells. In such cases, if the cell has an even number of edges, the orthogonal 
length equals two times the apothem, i.e. two times the segment between the 
midpoint of a side of the polygon and the center of this polygon 
$\(\textrm{apothem}=2\times \frac{\textrm{area}}{\textrm{perimeter}}\)$. If 
the cell has an odd number of edges, the orthogonal length is given by the 
apothem plus the circumradius, i.e. the radius of the circle circumscribed to 
the polygon $\(\textrm{circumradius}=\sqrt{\frac{2\times \textrm{area}}{V
\sin\(\frac{2\pi}{V}\)}}\)$. Therefore, $h_{\bot}$ is given by:
\begin{table}[H]
\begin{center}
\caption{Orthogonal length of the cell for different cells}
\begin{tabular}{|c|c|c|c|c|}
\hline
Number of edges & 3 & 4 & $> 4$ and even & $> 4$ and odd \\
\hline
$h_{\bot}$ & $2 \times \frac{\textrm{area}}{L_e}$ &
$\frac{\textrm{area}}{L_e}$ & $4\times
\frac{\textrm{area}}{\textrm{perimeter}}$ & $2 \times
\frac{\textrm{area}}{\textrm{perimeter}}+\sqrt{\frac{2\times
\textrm{area}}{V\sin\(\frac{2\pi}{V}\)}}$\\
\hline
\end{tabular}
\end{center}
\end{table}
