\section{Results} \label{sec_res}
In this section, we show two Fourier analyses of MIP: one where the $S_n$ order is
varied and one where the aspect ratio is varied. We also compare different
methods to solve MIP: congugate gradient (CG), conjugate gradient
preconditioned with symmetric Gauss-Seidel (PCG-SGS), conjugate gradient
preconditioned with ML using uncoupled aggregation (PCG-MLU),
conjugate gradient preconditioned with ML using MIS aggregation (PCG-MLM),
and AGMG. The options used for ML can be found in the Appendix. Unless
otherwise specified, PWLD finite elements are used in this section.
\subsection{Fourier Analyses}
Analysing Source Iteration accelerated with DSA is often performed using
Fourier analysis \cite{larsen_dsa,consistent_p1}. When a Fourier analysis is
performed, the error is decomposed into different modes and by inspecting the 
damping of the different error modes, the effectiveness of the DSA scheme can 
be studied. The largest damping factor is the spectral radius of the method. 
The smaller the spectral radius is, the faster the scheme converges. If the 
spectral radius is greater than one, the method is unstable. 
\subsubsection{$S_n$ order varied}
This Fourier analysis was carried on a square cell, using a
Gauss-Legendre-Chebyshev (GLC) quadrature. The medium is homogeneous, the scattering
ratio $c=0.9999$ and periodic boundary conditions are used. The $x-$axis is the mesh
size in mean free path and the $y-$axis is the spectral radius. On Figure
\ref{fig_fa_op}, there are four curves corresponding to different $S_n$ order: 
$S_2$, $S_4$, $S_8$, and $S_{16}$.
\begin{figure}[H]
  \centering
  \includegraphics[width=0.5\textwidth]{./Dsa/sn_order_9999}
  \caption{Fourier analysis as a function of the mesh optical thickness,
  homogeneous infinite medium case}
  \label{fig_fa_op}
\end{figure}
MIP is stable for every cell size. The spectral radius is always less than
0.5, except for $S_2$ where it peaks at about 0.7.
\subsubsection{Aspect ratio varied}
For this Fourier analysis, we use a $S_{16}$ GLC quadrature, a homogeneous
medium, $c=0.9999$ and periodic boundary conditions. The $x-$axis is the mesh
size in mean free path in the $x$ direction and the $y-$axis is the spectral
radius. On Figure \ref{fig_fa_ar}, there are five curves corresponding to five 
different aspect ratio: $\frac{Y}{X}=\frac{1}{16}$, $\frac{Y}{X}=\frac{1}{4}$, 
$\frac{Y}{X}=1$, $\frac{Y}{X}=4$, $\frac{Y}{X}=16$, and $\frac{Y}{X}=100$. 
\begin{figure}[H]
  \centering
  \includegraphics[width=0.5\textwidth]{./Dsa/aspect_ratio_9999_2}
  \caption{Fourier analysis as a function of the mesh optical thickness,
  homogeneous infinite medium case for different aspect ratios}
  \label{fig_fa_ar}
\end{figure}
MIP is stable for every aspect ratio and the maximum of the spectral radius
peaks at about 0.5. However, we noted that when both $c$ 
approaches one and the aspect ratio is large, MIP can become ill-conditioned. 
In Chapter \ref{anmg_chapter}, MIP was used for problems with $c=1$, without any issues 
because the cells were square (aspect ratio is one).

\subsection{Homogeneous medium}
Next, we compare different solvers for MIP on a homogeneous medium, 100cm $\times$
100cm, $\Sigma_t=1$cm$^{-1}$ and $\Sigma_s=0.999$cm$^{-1}$, with vacuum boundary 
conditions and a source of intensity 1cm$^{-3}$s$^{-1}$. We use a $S_8$
Gauss-Legendre-Chebyshev quadrature, a Source Iteration solver with relative
tolerance of $10^{-8}$ and a relative tolerance for MIP of $10^{-10}$.
\begin{description}
  \item[Quadrilateral cells:] the mesh is composed of 49236 quadrilateral cells
    that corresponds to 197052 degrees of freedom.
  \item[Polygonal cells:] the mesh is composed of 45204 triangles, 823 
    quadrilaterals, 4978 pentagons, 4155 hexagons, 725 heptagons, and 24 
    octagons, for a total of 55909 cells and 193991 degrees of freedom. This 
    example will allow us to test MIP and the different preconditioners on a 
    mesh composed of different types of cell.
\end{description}
The meshes and the solutions of these two problems are given on Figure
\ref{fig_meshes_phi}:
\begin{figure}[H]
\centering    
\subfloat[Quadrilateral cells]{\includegraphics[width=0.45\textwidth]
  {./Dsa/big_homog_quad_crop}}
\subfloat[Polygonal cells]{\includegraphics[width=0.45\textwidth]
  {./Dsa/big_homog_poly_crop}}
\caption{Meshes and scalar fluxes}
\label{fig_meshes_phi}
\end{figure}

In Table \ref{tab_1}, the different solvers, used on the quadrilateral cells, 
are compared:
\begin{table}[H]
\begin{center}
\caption{Comparison of different preconditioners for quadrilateral cells}
\begin{tabular}{|c|c|c|c|c|c|c|}
\hline
& No-DSA & CG & PCG-SGS &  PCG-MLU & PCG-MLM & AGMG\\
\hline
SI iter   & 7311    & 24      & 24       & 24      & 24      & 24      \\
Prec (s)  & NA      & NA      & 0.171358 & 1.8255  & 9.56078 & 0.332   \\
MIP (s)   & NA      & 1095.7  & 1311.76  & 192.622 & 197.632 & 29.9727 \\
CG iter   & NA      & 56649   & 17332    & 630     & 604     & 578     \\
Total (s) & 39176.7 & 1264.98 & 1477.95  & 363.202 & 367.841 & 194.568 \\
\hline
\end{tabular}
\label{tab_1}
\end{center}
\end{table}
In this Table, SI iter is the number of Source Iteration iterations 
needed to solve the problem, Prec is the time in seconds needed to
initialize the preconditioner used by CG, MIP is the total time in
seconds spent solving DSA during the calculation, CG iter is the total number 
of CG iterations used to solve MIP, and Total is the time in
seconds needed to solve the problem.

Using MIP decreases significantly the number of SI iterations and the
calculation time as expected. Using PCG-SGS decreases by a factor of three 
of the number of CG iterations compared to CG but the time needed to solve 
MIP is greater. This is because each PCG-SGS iteration is much slower
  than one unpreconditioned CG iteration. SGS requires basically two
  triangular solves. It is unclear why these simple solves would be so costly
  in CPU time so as to actually increase the total solver time while the
 number of CG iterations has been divided by three. With ML, the number 
of CG iterations is reduced by a factor 
of 50 and the MIP calculation time is reduced by a factor three compared to 
CG. AGMG is by far the most efficient solver, the number of CG iterations is 
slightly lower than PCG-ML but the MIP calculation is 20 times faster than CG.

The different solvers, used on the polygonal cells, are compared in Table
\ref{tab_2}:
\begin{table}[H]
\begin{center}
\caption{Comparison of different preconditioners for polygonal cells using SI}
\begin{tabular}{|c|c|c|c|c|c|c|}
\hline
& No-DSA & CG & PCG-SGS & PCG-MLU & PCG-MLM & AGMG\\
\hline
SI iter   & 7311    & 23      & 23      & 23      & 23      & 23      \\
Prec (s)  & NA      & NA      & 0.06388 & 1.73379 & 8.0426  & 0.388   \\
MIP (s)   & NA      & 877.861 & 1263.31 & 198.63  & 191.989 & 31.242  \\
CG iter   & NA      & 46262   & 16712   & 652     & 603     & 555     \\
Total (s) & 42666.7 & 1060.53 & 1447.53 & 382.275 & 384.422 & 216.946 \\
\hline
\end{tabular}
\label{tab_2}
\end{center}
\end{table}
We see that using different types of cells in the same mesh does not affect
the performance of MIP or of the preconditioners. 

In the next test, the problem is exactly the same as the previous one using
polygonal cells but the SI solver is replaced by GMRES. The comparison is done
in Table \ref{tab_3}:
\begin{table}[H]
\begin{center}
\caption{Comparison of different preconditioners for polygonal cells using
GMRES}
\begin{tabular}{|c|c|c|c|c|c|c|}
\hline
& No-DSA & CG & PCG-SGS & PCG-MLU & PCG-MLM & AGMG\\
\hline
GMRES iter & 266     & 12      & 12        & 12      & 12      & 12 \\
Prec (s)   & NA      & NA      & 0.0675611 & 1.56115 & 7.89327 & 0.0331 \\
MIP (s)    & NA      & 546.56  & 770.244   & 126.723 & 120.68  & 22.3754 \\
CG iter    & NA      & 28653   & 10274     & 407     & 390     & 351 \\
Total (s)  & 1549.17 & 675.319 & 898.149   & 261.121 & 261.937 & 162.47 \\
\hline
\end{tabular}
\label{tab_3}
\end{center}
\end{table}
The conclusions are the same as in the SI case. The performance of the
preconditioners is not affected by the change of solver.

\subsection{Heterogeneous medium}
In this example, we use a heterogeneous medium composed of 184 triangles, 3720
quadrilaterals and 2791 regular hexagons of side 0.05$cm$ for a total of 6695 
cells and 32178 degrees of freedom. The domain is 5.28275$cm$ by 4.6$cm$. 
Reflective boundary conditions are used. The quadrature is a $S_{16}$ 
Gauss-Legendre-Chebyshev quadrature. The SI solver has a relative tolerance of 
$10^{-8}$ and the relative tolerance for MIP is $10^{-10}$. The domain is 
composed of three zones (see Figure \ref{fig_zone_hex}):
\begin{figure}[H]
  \centering
  \includegraphics[width=0.5\textwidth]{./Dsa/source_crop}
  \caption{Zones of the domain discretized by triangles, rectangles, and
  hexagons}
  \label{fig_zone_hex}
\end{figure}
The properties of the different zones are:
\begin{description}
  \item[Green zone:] $\Sigma_t =1.5$cm$^{-1}$, $\Sigma_s = 1.44$cm$^{-1}$, source$ =
    1$cm$^{-3}$s$^{-1}$
  \item[Red zone:] $\Sigma_t = 1$cm$^{-1}$, $\Sigma_s = 0.9$cm$^{-1}$, no source
  \item[Blue zone:] $\Sigma_t = 1$cm$^{-1}$, $\Sigma_s = 0.3$cm$^{-1}$, no source
\end{description}
The different solvers are compared in Table \ref{tab_4}:
\begin{table}[H]
  \begin{center}
    \caption{Comparison of different preconditioners for a heterogeneous medium}
    \begin{tabular}{|c|c|c|c|c|c|c|}
      \hline
      & No-DSA & CG & PCG-SGS & PCG-MLU & PCG-MLM & AGMG\\
      \hline
      SI iter   & 122      & 18      & 18       & 18       & 18      & 18 \\
      Prec (s)  & NA       & NA      & 0.016149 & 0.336215 & 1.36803 & 0.065 \\
      MIP (s)   & NA       & 60.2031 & 123.05   & 31.7048  & 30.8669 & 2.80108\\
      CG iter   & NA       & 12016   & 6764     & 423      & 391     & 248 \\
      Total (s) & 413.274  & 131.297 & 188.586  & 101.888  & 103.734 & 71.5392\\
      \hline
    \end{tabular}
    \label{tab_4}
  \end{center}
\end{table}
We can see that the comments made for the homogeneous tests are
still valid. MIP is effective even with heterogeneous medium and AGMG is
still the fastest method. It is interesting to note that, contrary to 
  the homogeneous tests where the number of CG iterations remained similar for all 
  algebraic multigrid preconditioners, for this heterogeneous test AGMG requires
  significantly fewer iterations than PCG-MLU and PCG-MLM. This difference may
  be due to the fact that ML was first designed to be used for
  continuous finite elements discretization and that we are using
discontinuous finite elements.

The cross sections of the different zones were taken from \cite{mip}. In the
next test, they are modified to make the problem more challenging:
\begin{description}
  \item[Green zone:] $\Sigma_t =1.5$cm$^{-1}$, $\Sigma_s = 1.499$cm$^{-1}$, source$ =
    1$cm$^{-3}$s$^{-1}$
  \item[Red zone:] $\Sigma_t = 1$cm$^{-1}$, $\Sigma_s = 0.999$cm$^{-1}$, no source
  \item[Blue zone:] $\Sigma_t = 1$cm$^{-1}$, $\Sigma_s = 0.3$cm$^{-1}$, no source
\end{description}
The different solvers are compared in Table \ref{tab_5}:
\begin{table}[H]
  \begin{center}
    \caption{Comparison of different preconditioners for a highly diffusive 
    heterogeneous medium}
    \begin{tabular}{|c|c|c|c|c|c|c|}
      \hline
      & No-DSA & CG & PCG-SGS & PCG-MLU & PCG-MLM & AGMG\\
      \hline
      SI iter   & 278     & 17      & 17        & 17       & 17      & 17 \\
      Prec (s)  & NA      & NA      & 0.0160661 & 0.368768 & 1.41632 & 0.07 \\
      MIP (s)   & NA      & 58.422  & 126.93    & 33.2225  & 31.3045 & 2.924 \\
      CG iter   & NA      & 12214   & 6679      & 415      & 386     & 248 \\
      Total (s) & 910.566 & 120.889 & 190.413   & 99.7524  & 97.4666 & 70.6424 \\
      \hline
    \end{tabular}
    \label{tab_5}
  \end{center}
\end{table}
Making the problem more diffusive increases the interest of using DSA but it
does not change the behavior of the preconditioners.

\subsection{AMR mesh}
In this example from \cite{mip}, the domain is $10cm\times 10cm$. The left and bottom
boundaries are reflective whereas the right and the top boundaries are vacuum. 
There are 10720 cells: 10482 quadrilaterals, 236 pentagons,
and 2 hexagons for a total of 43120 degrees of freedom. 
As in the previous example, the domain is composed of three zones (see Figure
\ref{fig_zone_amr}):
\begin{figure}[H]
  \centering
  \includegraphics[width=0.6\textwidth]{./Dsa/zone_amr}
  \caption{Zones of the AMR mesh}
  \label{fig_zone_amr}
\end{figure}
where:
\begin{description}
  \item[Green zone:] $\Sigma_t=1.5cm^{-1}$, $\Sigma_s=1.44cm^{-1}$,
    source=$1cm^{-3}s^{-1}$
  \item[Red zone:] $\Sigma_t=1cm^{-1}$, $\Sigma_s=0.9cm^{-1}$, no source
  \item[Blue zone:] $\Sigma_t=1cm^{-1}$, $\Sigma_s=0.3cm^{-1}$, no source
\end{description}
The distribution of cells is given on Figure \ref{fig_distr}:
\begin{figure}[H]
  \centering
  \includegraphics[width=0.6\textwidth]{./Dsa/polygon_amr}
  \caption{Polygons distribution}
  \label{fig_distr}
\end{figure}
where:
\begin{description}
  \item[Blue cells] are quadrilaterals.
  \item[Green cells] are pentagons.
  \item[Red cells] are hexagons.
\end{description}
This mesh is typical of a mesh obtained after one level of adaptive mesh
refinement (the cells at the interface of different zones have been refined
once). We see that instead of introducing hanging nodes, we have introduced
pentagons and hexagons in the mesh.
A $S_{16}$ GLC quadrature is employed. The tolerance on SI is $10^{-8}$ and
the tolerance on the CG solvers is $10^{-10}$.
The different solvers are compared in Table \ref{tab_6}:
\begin{table}[H]
  \caption{Comparison of preconditioners on an AMR mesh}
  \begin{center}
    \begin{tabular}{|c|c|c|c|c|c|c|}
      \hline
       & No-DSA & CG & PCG-SGS & PCG-MLU & PCG-MLM & AGMG \\
      \hline
      SI iter    & 184     & 19      & 19       & 19      & 19       & 19 \\
      Prec (s)   & NA      & NA      & 0.043463 & 0.358002 & 1.19301 & 0.0111\\
      MIP (s)    & NA      & 48.1908 & 81.0992  & 25.2699 & 25.0699  & 
      2.56198\\
      CG iter    & NA      & 11300   & 4734     & 361     & 361      & 264 \\
      Total (s)  & 802.985 & 138.825 & 172.423  & 116.018 & 116.517  &
      94.1963\\
      \hline
    \end{tabular}
    \label{tab_6}
  \end{center}
\end{table}
As expected, the results are similar to our previous test.

\subsection{Rectangular cells}
As mentioned previously in this Chapter, AMG can have difficulties when the
aspect ratio of rectangular cells is high. Moreover, when the aspect ratio is
high and the scattering ratio is close to one, MIP becomes ill conditioned. 
In the next four examples, the domain is square $100cm \times 100cm$ with vacuum 
boundaries. There are 10000 cells and we use BLD finite elements for the first
three runs and PWLD finite elements for the last one; 
there are 40000 degrees of freedom. The relative tolerance on SI is $10^{-8}$ 
and the relative tolerance on CG is $10^{-10}$. We use a $S_{8}$ GLC quadrature, 
$\Sigma_t = 1 cm^{-1}$, and $\Sigma_s = 0.999 cm^{-1}$. The source is
$1n/(cm^3s)$. In the first test, the domain is discretized by 100 cells along
$x$ and 100 cells along $y$. In the second test, the domain is discretized by
250 cells along $x$ and 40 cells along $y$. Therefore, the aspect ratio is 6.25
for the second test. In the last two tests, the domain is discretized by 1000
cells along $x$ and 10 cells along $y$ (the aspect ratio is 100).

We also compared the effect of the size of the coarsest grid on the convergence.
In the following Tables (Table \ref{bld_ar_1}, Table \ref{bld_ar_6}, Table 
\ref{bld_ar_100}, and Table \ref{pwld_ar_100}), we 
compare No-DSA, CG, and AGMG defined previously with:
\begin{description}
  \item[PCG-MLU-D:] conjugate gradient preconditioned with ML
    using uncoupled coarsening with a coarsest grid of size less 
    or equal of 128 (default value).
  \item[PCG-MLU-M:] conjugate gradient preconditioned with ML
    using uncoupled coarsening with a coarsest grid of size less 
    or equal of 400 (same value than AGMG).
  \item[PCG-MLM-D:] conjugate gradient preconditioned with ML
    using MIS coarsening with a coarsest grid of size less or 
    equal of 128 (default value).
  \item[PCG-MLM-M:] conjugate gradient preconditioned with ML
    using MIS coarsening with a coarsest grid of size less or 
    equal of 400 (same value than AGMG).
\end{description}
\begin{landscape}
  \begin{center}
    \begin{table}[H]
      \caption{Comparison of preconditioners on rectangular mesh with an aspect ratio of 1.0 with BLD finite elements}
      \begin{tabular}{|c|c|c|c|c|c|c|c|c|}
        \hline
        & No-DSA & CG & PCG-SGS & PCG-MLU-D & PCG-MLU-M & PCG-MLM-D & PCG-MLM-M & AGMG \\
        \hline
        SI iter    & 7311    & 21      & 21        & 21       & 21       & 21      & 21      & 21      \\
        Prec (s)   & NA      & NA      & 0.0167298 & 0.345833 & 0.392942 & 1.06411 & 1.2699  & 0.057   \\
        MIP (s)    & NA      & 39.4564 & 90.393    & 23.9443  & 28.3395  & 23.5808 & 28.5564 & 2.76611 \\
        CG iter    & NA      & 8584    & 4911      & 377      & 377      & 376     & 376     & 265     \\
        Total (s)  & 8254.46 & 66.8963 & 119.492   & 48.8737  & 56.932   & 48.5779 & 59.0295 & 31.6606 \\
        \hline
      \end{tabular}
      \label{bld_ar_1}      
    \end{table}
    \begin{table}[H]
    \caption{Comparison of preconditioners on rectangular mesh with an aspect ratio of 6.25 with BLD finite elements}  
      \begin{tabular}{|c|c|c|c|c|c|c|c|c|}
        \hline
        & No-DSA & CG & PCG-SGS & PCG-MLU-D & PCG-MLU-M & PCG-MLM-D & PCG-MLM-M & AGMG \\
        \hline
        SI iter    & 7311    & 23      & 23        & 23      & 23       & 23      & 23       & 23      \\
        Prec (s)   & NA      & NA      & 0.0205462 & 0.32853 & 0.381816 & 1.02377 & 1.111999 & 0.671   \\
        MIP (s)    & NA      & 87.8195 & 178.535   & 74.1533 & 89.1188  & 73.0178 & 80.5592  & 4.20011 \\
        CG iter    & NA      & 18985   & 9820      & 1219    & 1219     & 1206    & 1206     & 171     \\
        Total (s)  & 7311.49 & 117.768 & 210.944   & 100.424 & 120.757  & 101.374 & 110.913  & 32.857  \\
        \hline
      \end{tabular}
      \label{bld_ar_6}      
    \end{table}
    \begin{table}[H]
    \caption{Comparison of preconditioners on rectangular mesh with an aspect ratio of 100 with BLD finite elements}  
      \begin{tabular}{|c|c|c|c|c|c|c|c|c|}
        \hline
        & No-DSA & CG & PCG-SGS & PCG-MLU-D & PCG-MLU-M & PCG-MLM-D & PCG-MLM-M & AGMG \\
        \hline
        SI iter    & 7304    & 24      & 24       & 24       & 24       & 24      & 24      & 24      \\
        Prec (s)   & NA      & NA      & 0.049191 & 0.651429 & 0.451487 & 1.15778 & 1.1675  & 0.383   \\
        MIP (s)    & NA      & 361.306 & 656.815  & 945.56   & 986.592  & 917.151 & 917.481 & 3.74259 \\
        CG iter    & NA      & 82613   & 38263    & 14128    & 14215    & 13810   & 13824   & 177     \\
        Total (s)  & 8581.25 & 400.116 & 698.613  & 984.163  & 1027.1   & 956.392 & 957.611 & 43.6815 \\
        \hline
      \end{tabular}
      \label{bld_ar_100}      
    \end{table}
    \begin{table}[H]
    \caption{Comparison of preconditioners on rectangular mesh with an aspect ratio of 100 with PWLD finite elements}  
      \begin{tabular}{|c|c|c|c|c|c|c|c|c|}
        \hline
        & No-DSA & CG & PCG-SGS & PCG-MLU-D & PCG-MLU-M & PCG-MLM-D & PCG-MLM-M & AGMG \\
        \hline
        SI iter    & 7304   & 24      & 24        & 24       & 24       & 24      & 24      & 24      \\
        Prec (s)   & NA     & NA      & 0.0164239 & 0.362463 & 0.778893 & 1.03128 & 1.66949 & 0.052   \\
        MIP (s)    & NA     & 372.227 & 742.902   & 941.06   & 925.247  & 922.258 & 964.044 & 6.93176 \\
        CG iter    & NA     & 84802   & 43466     & 14180    & 14179    & 13896   & 13894   & 821     \\
        Total (s)  & 9035.6 & 414.301 & 784.77    & 985.796  & 969.318  & 966.77  & 1010.44 & 44.7032 \\
        \hline
      \end{tabular}
      \label{pwld_ar_100}      
    \end{table}
  \end{center}
\end{landscape}
As predicted, inverting MIP requires a lot more CG iterations when the aspect
ratio increases. The only exception is for AGMG which requires fewer iterations
when the aspect ratio is 6.25 and 100 than when it is 1.0. As expected, PCG-MLU and
PCG-MLM are much more affected by the aspect ratio than CG and PCG-SGS.
The number of CG iterations for CG and PCG-SGS is multiplied by two
when the aspect ratio is increased from 1.0 to 6.25, whereas it is multiplied by a 
little more than three for PCG-MLU and PCG-MLM. When the aspect ratio is 100, PCG-MLU 
and PCG-MLM are the slowest methods. This is not totally unexpected since these examples 
have been designed to test the limitations of algebraic multigrid preconditioners. It is 
interesting to note that changing the size of the coarsest grid for PCG-MLU and PCG-MLM 
does not affect the number of CG iterations. Even if AGMG does not seems to be to affected 
by the change in the aspect ratio when BLD finite elements are employed, using PWLD finite 
elements dramatically increases the number of CG iterations when the aspect ratio is 100. 
AGMG is the only method which is significantly impacted by the change of
finite element type; however, it stays by far the fastest method to solve the MIP equations.
