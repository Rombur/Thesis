\chapter{Introduction}
\section{Transport equation}
\begin{itemize}
\item $\mu_0 = \bo'\cdot \bo$ is the cosine of the polar angle
\item $\bo$ is a unit vector in the flight direction
\item $\mu = \cos(\theta)$, where $\theta$ is a polar angle
\end{itemize}
The transport equation is given by:
\begin{equation}
\bo\cdot \bn \Psi(\br,\bo,E) + \Sigma_t(\br,E)\Psi(\br,\bo,E) =
\int_0^{\infty}dE'\int_{4\pi}d\bo'\ 
\Sigma_s(\br,\bo'\cdot\bo,E'\rightarrow E)\Psi(\br,\bo',E')+Q(\br,\bo,E)
\label{transport_p}
\end{equation}
Here $\Psi(E,\bo)$ is defined by:
\begin{equation}
\Psi(E,\bo) = vf(E,\bo)
\end{equation}
where $v$ is the particle speed, $\Sigma_t(E)$ is the total cross-section
given by:
\begin{equation}
\Sigma_t(E) = \Sigma_a(E)+\Sigma_s(E)
\end{equation}
and $Q(E,\bo)$ represents any source of particles. We are not concern about
boundary conditions.

The in-scattering term can be represented as a sum over its a surface harmonic
components. To this end, we expand the scattering kernel in Legendre
polynomials according to:
\begin{equation}
\Sigma_s(\br,\bo'\cdot\bo,E'\rightarrow E)=\sum_{l=0}^{\infty}
\Sigma_{s,l}(\br,E'\rightarrow E)P_l(\bo'\cdot\bo)
\end{equation}
The orthogonality of the Legendre polynomials allows us to write the expansion
coefficients as:
\begin{align}
&\Sigma_{s,l}(\br,E'\rightarrow E) = \int_{-1}^1 d\mu_0 P_l(\mu_0) 
\Sigma_s(\br,\mu_0,E'\rightarrow E)\\
&\Sigma_s(\br,\mu_0,E'\rightarrow E) = \sum_{l=0}^{\infty}
\Sigma_{s,l}(\br,E\rightarrow E) P_l(\mu_0)
\end{align}
Further, we expand the solution $\Psi(\br,\bo,E)$ in surface harmonics according
to:
\begin{equation}
\Psi(\br,\bo,E) = \sum_{l=0}^{\infty}\sum_{m=-l}^l \phi_{l,m}(\br,E) Y_l^m(\bo)
\label{ang_flux}
\end{equation}
The expansion coefficients $\phi_{l,m}$ are given by:
\begin{equation}
\phi_{l,m}(\br,E) = \int_{4\pi} d\bo\ Y_{l}^{m,*}(\bo) \Psi(\br,\bo,E)
\label{moments}
\end{equation}
A property of the surface harmonics is that they satisfy the partial equation
given by:
\begin{equation}
\[\frac{\partial}{\partial\mu}(1-\mu^2)\frac{\partial}{\partial
\mu}+\(\frac{1}{1-\mu^2}\)\frac{\partial^2}{\partial \varphi}+l(l+1)\]Y_l^m(\bo)=0
\label{eigenvalue}
\end{equation}
Using the addition theorem:
\begin{equation}
P_l(\bo\cdot\bo') = \frac{1}{2\pi} P_l(\mu_0) = \frac{2}{2l+1} \sum_{m=-l}^l 
Y_l^m(\bo)\cdot
Y_l^{m,*}(\bo')
\end{equation}
we get:
\begin{equation}
\bo\cdot\bn \Psi(\br,\bo,E) +
\Sigma_t(\br,E)\Psi(\br,\bo,E)=\sum_{l=0}^{\infty}\sum_{m=-l}^{l} Y_l^m(\bo) 
\int_0^{\infty} dE'\ \Sigma_{s,l}(\br,E'\rightarrow E)\Psi_{l,m}(\br,E') +
Q(\br,\bo,E)
\end{equation}
or equivalently:
\begin{equation}
\begin{split}
&\bo\cdot\bn \Psi(\br,\bo,E) +
\(\Sigma_a(\br,E)+\Sigma_s(\br,E)\)\Psi(\br,\bo,E) =\\
&\sum_{l=0}^{\infty}\sum_{m=-l}^l Y_l^m(\bo) \int_0^{\infty}dE'\
\Psi_{l,m}(\br,E') \int_{-1}^1 d\mu_0\ P_l(\mu_0)
\Sigma_{s}(\br,\mu_0,E'\rightarrow E)+Q(\br,\bo,E)
\end{split}
\end{equation}
