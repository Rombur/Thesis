\section{Implementation}
Janus is composed of the following classes:
\begin{description}
  \item[PARAMETERS:] In this class, the different parameters such as, the 
    type of solver for transport equation (SI or a Krylov method), the type 
    of solver for MIP, the convergence tolerance, the boundary conditions, 
    the intensity of the sources, the cross sections, etc, are read from an
    input file and stored. If Fokker-Planck cross sections are used, they 
    are computed here. The different cross sections for the angular
    multigrid are computed by this class and the extended transport correction 
    is applied.
  \item[TRIANGULATION:] In this class, the geometry, the material IDs, and the
    source IDs are read. Two different input files can be read. When the mesh
    uses rectangular cells, the abscissae then the ordinates have to be given 
    in increasing value. After that, the materials IDs and the source IDs are
    read. For instance, if the domain, $1cm \times 1cm$, is decomposed 
    in four identical cells, the input file looks like:
    \begin{verbatim}
    rectangle
    2 2      // (number of x-divisions, number of y-divisions)
    0. 0.5 1.0
    0. 0.5 1.0
    0 0 0 0  // (material IDs)
    0 0 0 0  // (source IDs)
    \end{verbatim}
    The other acceptable type of input file is used for polygonal cells. In 
    that case, the number of edges of a cell is given first, followed by the 
    coordinates of each vertex, the material IDs and finally the source IDs.
    The vertices must be ordered in an anti-clockwise order but there is no need
    to order the cell or for two successive cells to be adjacent in the mesh. 
    For instance, a possible input file would be:
    \begin{verbatim}
    polygon
    4   // (number of cells)
    3 1. 0. 1. 1. 0.5 0.5 0 0
    5 0. 0. 0.5 0. 0.5 0.5 0.5 1. 0. 1. 0 0
    3 0.5 0. 1. 0. 0.5 0.5 0 0
    3 0.5 0.5 1. 1. 0.5 1. 0 0
    \end{verbatim}
    This class assumes that the domain is rectangular. After reading the
    geometry, the EDGE objects are created. Before an edge is created, it must
    be checked that the edge does not already exist. To do so, the coordinates
    of the two vertices of the candidate edge are compared with the
    coordinates of the vertices of a subset of the existing edge. This subset
    corresponds to the smallest subset of edges having an abscissa,
    respectively an ordinate, of one of their vertices equals to an abscissa,
    respectively an ordinate, of one of the vertices of the candidate edge.
  \item[EDGE:] This class contains the coordinates of the vertices associated
    to the edge, the global and local IDs of the edge, the IDs of the cells 
    associated to the edge, the type of cell (interior or on the boundary and 
    the type of boundary: vacuum, isotropic or most normal direction of the
    quadrature), the two normal vectors associated to the two cells, the 
    length of the edge, etc.
  \item[FINITE\_ELEMENT:] This class is the base class for BLD and PWLD.
    It contains all the matrices needed for the DFE discretization of the transport
    equation and of the diffusion equation.
  \item[BLD:] This class derives from FINITE\_ELEMENT and builds the bilinear
    finite elements.
  \item[PWLD:] This class derives from FINITE\_ELEMENT and builds the
    piecewise linear finite elements.
  \item[QUADRATURE:] This class is the base class for both the GLC and LS classes. 
    QUADRATURE builds the discrete-to-moment and moment-to-discrete matrices and
    stores the different directions used by the quadrature. The
    directions on the first octant are computed in GLC and LS. Then, they are
    deployed over the other octants. After that, the spherical harmonics
    are computed and evaluated at the given directions. When a Galerkin
    quadrature is used, selection rules are employed and the discrete-to-moment
    matrix is computed by inverting the moment-to-discrete matrix. Otherwise,
    the discrete-to-moment matrix is obtained by transposing the moment-to-discrete
    matrix and by multiplying it by the weights of the quadrature.
  \item[GLC:] This class derives from QUADRATURE and computes the weights and
    the directions used by the Gauss-Legendre-Chebyshev triangular quadrature.
  \item[LS:] This class derives from QUADRATURE and computes the weights and
    the directions used by the Level-Symmetric quadrature up to $S_{24}$.
  \item[CELL:] This class stores the ID of cell, a vector of pointers to the
    edges of cell, the intensity of the source in the cell, the material
    properties ($\Sigma_s$, $\Sigma_t$, and the diffusion coefficient), 
    the FINITE\_ELEMENT associated with the cell, etc. The orthogonal lengths are
    calculated by this class.
  \item[DOF\_HANDLER:] This class builds the \emph{mesh} by creating all the 
    CELL objects and the  FINITE\_ELEMENT objects associated with them. It is 
    the DOF\_HANDLER object that computes the sweep
    ordering for all the directions. First, the edges on the boundary with a
    known incoming flux are put in a list \emph{incoming\_edges}. The sweep
    ordering will continue as long as this list is not empty. The first edge 
    in the list is popped and the associated cell, which has not been
    accepted in the sweep order, is found. Then, we loop over the edges of 
    the cell to determine which ones are
    associated to an outgoing flux and which ones are associated to an
    incoming flux. The cell is accepted if all the
    edges which are not ``outgoing'' are in \emph{incoming\_edges}. If the cell is
    rejected, the edge is pushed back at the end of the list. If the cell is
    accepted, the edges of the cell which are incoming are removed from
    \emph{incoming\_edges}. The others edges are pushed back at the end of the
    list except if they are on the boundary of the domain.
  \item[TRANSPORT\_OPERATOR:] The calculation is performed in this class. 
    It derives from the Epetra\_Operator of Trilinos.
    TRANSPORT\_OPERATOR handles the angular multigrid by calling itself
    recursively and restricting and projecting the flux moments on the
    different ``angular'' grids. It is also in this class that the scattering 
    source is computed and the sweeps are  performed. 
  \item[MIP:] This class builds and solves MIP. The first time that
    \emph{Solve()} is called, the left hand-side is built and stored using a 
    compressed row storage format (CRS). Then, the problem is
    solved by CG, PCG-SSOR, PCG-MLU, PCG-MLM, or AGMG. If AGMG is employed, 
    there is an extra step to convert the right hand-side to Fortran data type 
    and the result back to an Epetra\_MultiVector.
  \item[TRANSPORT\_SOLVER:] This class builds the PARAMETERS object, the
    TRIANGULATION object, the QUADRATURE object(s), the DOF\_HANDLER object,
    and the initial TRANSPORT\_OPERATOR object. SI and the Krylov solvers are
    called in \emph{Solve()}. The final result and the mesh are written in a file
    by this class.
  \item[EXCEPTION:] This class handles the exceptions that can be thrown.
\end{description}
