\chapter{\uppercase{Janus}}
\label{janus_chapter}
\section{Introduction}
In this Chapter, we detail the implementation of the transport code developed
in this research, Janus. Janus is a two-dimensional one-group $S_n$ transport
solver. It uses arbitrary polygonal meshes and implements an angular multigrid
preconditioner for highly anisotropic medium. The ASCII output file generated
by Janus can be converted to a silo file \cite{silo}, using an other C++ code,
Apollo. This output file can be read by VisIt \cite{visit}. A python code, 
Diana, can be used to generate the mesh or to convert a mesh generated by Triangle
\cite{triangle} into a mesh readable by Janus. Another python code, Mercury,
can be used to generate input file for Janus. Mercury can help writing an
input by checking that all the data required by Janus are present and that
they are written in the right order. 

Janus is documented using Doxygen \cite{doxygen}. It is built upon Trilinos
10.4 \cite{trilinos} and GSL (GNU Scientific Library) \cite{gsl} and uses
Autoconf, Automake, and Autotest \cite{autoconf,automake}. Git \cite{git} 
is used for revision control. 

Janus, Diana, and Mercury can be cloned at 
\emph{\hbox{git://gitorious.org/transport/janus.git}} 

Apollo can be cloned at \emph{\hbox{git://gitorious.org/transport/plot.git}}

