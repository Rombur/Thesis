\section{The $S_n$ Transport Equations}
Given an angular quadrature set $\{\bo_d,w_d\}_{1\leq d \leq M}$, the
one-group $_n$ transport equation with isotropic source and scattering is:
\begin{equation}
  \(\bo_d \cdot \bn +\Sigma_t (\br)\)\psi_d(\br) = \frac{1}{4\pi}\Sigma_s
  (\br)\phi(\br) + \frac{1}{4\pi} S(\br),\ \textrm{ for } \br \in \mc{D},\
  1\leq d \leq M,
  \label{one_speed}
\end{equation}
with $\psi_d(\br)=\psi(\br,\bo_d)$ the angular flux at position $\br$ in
direction $\bo_d$, $\Sigma_t$, and $\Sigma_s$ the total and scattering cross
section, respectively, and $\mc{D}$ the spatial domain. The scalar flux is
defined and evaluated as follows:
\begin{equation}
  \psi(\br) \equiv \int_{4\pi} \psi(\br,\bo)\ d\bo \simeq \sum_{d=1}^M w_d
  \psi_d\(\br\).
  \label{flux_moments}
\end{equation}
The system of equations is closed assuming incoming boundary conditions on
(with $\partial \mc{D} = \partial \mc{D}^{d} \cap \partial \mc{D}^r$):
\begin{equation}
  \psi_d (\br_b) = \left\{
  \begin{aligned}
    &\psi_d^{inc}(\br_b), &\br_b \in \partial \mc{D}^{d,-}=\{\partial \mc{D}^d
    \textrm{ such that } \bo_d \cdot \bs{n}_b <0\} \\
    &\psi_{d^{\prime}}(\br_b), & \br_b \in \partial \mc{D}^{r,-}_d = \{
    \partial \mc{D}^r \textrm{ such that } \bo_d \cdot \bs{n} < 0\}
  \end{aligned}
  \right. ,
\end{equation}
where $\bs{n}_b = \bs{n}(\bs{r}_b)$ is the outward unit normal vector on the
boundary. The reflecting direction of $\bo_d$ at a point $\br_b$ on the
boundary is given by:
\begin{equation}
  \bo_{d^{\prime}} = \bo_d - 2\(\bo_d \cdot \bs{n}_b\)\bs{n}_b.
\end{equation}
We assume the angular quadrature set satisfies the following two conditions
for any outward unit normal vector on the reflecting boundary $\partial
\mc{D}^r$:
\begin{itemize}
  \item $\forall d=1,\hdots,M$, the reflected direction $\bo_{d^{\prime}}$ is
    also in the quadrature set (which is simple to obtain for rectangular
    geometries);
  \item the weights of the incident and reflected directions are equal, i.e.,
    $w_d=w_{d^{\prime}}$
\end{itemize}

For the time being, we assume that no reflective boundaries exist $\(\partial
\mc{D}^r = 0\)$. Then, \cref{one_speed} can be written in a compact form using
operators:
\begin{align}
  &\bs{L}\Psi = \bs{M\Sigma} \Phi + S = q, \label{transport_op}\\
  &\Phi = \bs{D} \Psi, \label{phi_D_psi}
\end{align}
where $\Psi$ is the vector of angular fluxes, $\Phi$ the vector flux moments
(with isotropic scattering, the only moment required is the scalar flux), $q$
is the total (scattering+external) source, $\bs{L}$ is the streaming operator,
$\bs{M}$ is the moment-to-direction operator, and $\bs{D}$ is the
direction-to-moment operator.
$\bs{L}=diag(\bs{L}_1,\hdots,\bs{L}_d,\hdots,\bs{L}_M)$ is a diagonal
operator; given the total source $q$, one can solve independently for the
resulting angular fluxed in all directions. \Cref{transport_op,phi_D_psi} can
be re-arranged in terms of the scalar flux only:
\begin{equation}
  \Phi = \bs{D L}^{-1} \(\bs{M \Sigma}\Phi+S\).
\end{equation}
The action of $\bs{D}\bs{L}^{-1}$ is often referred to as a \emph{transport sweep}
because for any direction $\bo_d$, the action of $\bs{L}_d^{-1}$ can be
obtained by traversing the mesh (i.e.,sweeping) in the direction of flow,
successively inverting $\bs{L}_D$ in each set of downstream cells. Thus, one need 
only to solve a small linear system of equations, cell by cell. The order in which 
the elements are solved constitutes the graph of the sweep. We have thus far 
considered only situations where
the graph does not present some dependencies (cycles). Note that cycles in the
sweep graph can also appear due to reflective boundary conditions. These graph
dependencies can either be lagged within the iterative procedure of the
solution vector consisting of the scalar flux is augmented by the angular flux
unknowns that cause the cycles. We will explain these details in the paragraph
related to iterative techniques, but first we generalize our operator
notations for situations where we need to keep in the solution vector both the
flux moments and some angular fluxes due to dependencies in the graph (non
convex meshed and/or reflective boundaries).

If the graph of the sweep presents dependencies, we practically break the
transport sweep on these boundaries and introduce the notion of significant
angular fluxes. In this situation, we define a matrix $\bs{N}$ that extracts
from the entire angular flux vector all out-going angular fluxes on the
boundaries causing a dependency in the graph of the sweep, i.e., the
significant angular flux vector is given by:
\begin{equation}
  \Psi_{SAF} = \bs{N} \Psi,
\end{equation}
$\bs{N}$ is simply an operator that extracts from the entire angular flux vector,
the values required to break the graph dependencies. We then split the loss
and streaming operator $\bs{L}$ into two parts:
\begin{equation}
  \bs{L} = \underline{\bs{L}} - \overline{\bs{L}},
\end{equation}
where $\underline{\bs{L}}$ is the lower block triangular matrix (which can be
inverted during a transport sweeps) and $\overline{\bs{L}}$ is the strictly
upper triangular block, causing the dependencies in the sweep
($\overline{\bs{L}}$ only contains integrals along some incoming edges of a
cell). Note that $\bs{N}^T\bs{N}$ is a diagonal operator that contains one only
for angular flux values that are labeled as ``significant''. Then, we have:
\begin{equation}
  \overline{\bs{L}} \Psi = \overline{\bs{L}} \bs{N}^T\bs{N}\Psi =
  \overline{\bs{L}}\bs{N}^T \Psi_{SAF},
\end{equation}
and can, therefore, recast the transport equation as:
\begin{align}
  &\underline{\bs{L}}\Psi =\overline{\bs{L}}\bs{N}^T \Psi_{SAF} + \bs{M\Sigma}
  \Phi + S,\\
  &\Phi = \bs{D}\Psi.
\end{align}
