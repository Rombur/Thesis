\section{Solution Techniques}
\subsection{Unaccelerated Procedures}
\Cref{transport_op,phi_D_psi} can be solved using the Source Iteration (SI)
method (a stationary iterative technique also known as Richardson iteration).
The SI technique at the $l^{th}$ iteration is given by:
\begin{equation}
  \Phi^{(l+1)} = \bs{DL}^{-1} \(\bs{M \Sigma}\Phi^{(l)} + S\).
\end{equation}
Alternatively, a subspace Krylov method (usually GMRES) can be employed to
solve the following transport system of equations:
\begin{equation}
  \(\bs{I}-\bs{DL}^{-1} \bs{M\Sigma}\)\Phi = \bs{DL}^{-1}S.
\end{equation}
Both the SI and the GMRES approaches require transport sweeps (the action of
$\bs{L}^{-1}$ is required in both procedures).

When the scattering ratio $c=\frac{\Sigma_s}{\Sigma_t}$ tends to one in
optically thick domains, the number of SI and GMRES can become large. Fourier
analyses (for continuous, i.e., undiscretized transport) confirmed that SI
attenuates rapidly error modes associated with high frequencies (transport
dominated modes) while leaving almost unaffected low-frequency error modes
(diffusion dominated modes) \cite{dsa_ref}. To accelerate the convergence, a DSA
preconditioner is thus needed; in addition, some level of consistency is
necessary between the spatial discretization of the transport operator and
than of the diffusion operator. In Chapter \ref{mip_chapter}, we will adapt
the MIP discontinuous finite element discretization of the diffusion equation
for arbitrary polygonal grids is and employ MIP as a DSA preconditioner.

For completeness, we provide the SI and GMRES solution techniques in the case
graph dependencies are present. During one SI iteration, the scalar flux and
the angular significant flux are updated as follows:
\begin{equation}
  \begin{bmatrix}
    \Phi\\
    \Psi_{SAF}
  \end{bmatrix}^{(l+1)}
  = 
  \begin{bmatrix}
    \bs{D}\underline{\bs{L}}^{-1} \bs{M\Sigma} & \bs{D}\underline{\bs{L}}^{-1}
    \overline{\bs{L}}^{-1}\bs{N}^T\\
    \bs{N}\underline{\bs{L}}^{-1} \bs{M\Sigma} & \bs{N}\underline{\bs{L}}^{-1}
    \overline{\bs{L}}^{-1}\bs{N}^T
  \end{bmatrix}
  \begin{bmatrix}
    \Phi\\
    \Psi_{SAF}
  \end{bmatrix}^{(l)}
  +
  \begin{bmatrix}
    \bs{D}\\
    \bs{N}
  \end{bmatrix}
  \underline{\bs{L}}^{-1} S.
  \label{si_matrix_notation}
\end{equation}
\Cref{si_matrix_notation} is simply coded by appending the $\Psi_{SAF}$ to the
scalar flux unknowns (after a transport sweep, the operator $\bs{D}$ is
applied to yield the newest scalar flux whereas the operator $\bs{N}$ is
applied to update the significant angular flux). Note that when
$\overline{\bs{L}}=0$ (i.e., no dependencies in the sweep), we obtain the
standard SI formula:
\begin{equation}
  \Phi^{(l+1)} = \bs{DL}^{-1} \(\bs{M\Sigma}\Phi^{(l)} + S\).
  \label{matrix_form_si}
\end{equation}
From the SI formula of \cref{si_matrix_notation}, it follows that the linear
system for a GMRES-based transport solves is simply:
\begin{equation}
  \(\bs{I}-\bs{T}\)x=b,
\end{equation}
with:
\begin{align}
  &\bs{T} = 
  \begin{bmatrix}
    \bs{D}\underline{\bs{L}}^{-1}\bs{M\Sigma} &
    \bs{D}\underline{\bs{L}}^{-1}\overline{\bs{L}}\bs{N}^T\\
    \bs{N}\underline{\bs{L}}^{-1}\bs{M\Sigma} & \bs{N}
    \underline{\bs{L}}^{-1} \overline{\bs{L}}\bs{N}^T
  \end{bmatrix}                                      
  ,\\
  &x =
  \begin{bmatrix}
    \Phi\\
    \Psi_{SAF}
  \end{bmatrix}
  ,\\
  &b=
  \begin{bmatrix}
    \bs{D}\\
    \bs{N}
  \end{bmatrix}
  \underline{\bs{L}}^{-1}S.
\end{align}

\subsection{Synthetic Acceleration and Preconditioning}
Ignoring graph dependencies for simplicity of the presentation, the transport
equation is:
\begin{equation}
  \(\bs{L}-\bs{M\Sigma D}\)\Psi = S.
  \label{te_1}
\end{equation}
It is often computationally effective write the above linear system as (sweep
preconditioning):
\begin{equation}
  \(\bs{I}-\bs{L}^{-1}\bs{M\Sigma D}\) \Psi = \bs{L}^{-1}S
\end{equation}
because $\bs{L}$ is easier to invert than $(\bs{L}-\bs{M\Sigma D})$. Therefore,
an iterated scheme (the SI technique) yields formally:
\begin{equation}
  \Psi^{(\ell+1)}=\bs{L}^{-1}\(\bs{M\Sigma D}\Psi^{(\ell)}+S\).
\end{equation}
The error equation is:
\begin{equation}
  \begin{split}
    \Psi-\Psi^{(\ell+1)} &= \bs{L}^{-1} \bs{M\Sigma D}\(\Psi - \Psi^{(\ell)}\)\\
                      &= \bs{L}^{-1}\bs{M\Sigma D}\(\Psi-\Psi^{(\ell+1)} +
    \Psi^{(\ell+1)}-\Psi^{(\ell)}\),
  \end{split}
\end{equation}
that is, the transport equation satisfied by the angular error
$\epsilon^{(\ell+1)}=\Psi-\Psi^{(\ell+1)}$ is:
\begin{equation}
  \(\bs{L}-\bs{M\Sigma D}\)\epsilon^{(\ell+1)} = \bs{M\Sigma
  D}\(\Psi^{(\ell+1)}-\Psi^{(\ell)}\).
  \label{ang_err_eq}
\end{equation}
This equation is of the same form as \cref{te_1} (where the source term is now
the scattering due to the difference in successive flux iterates) and,
therefore, is just as difficult to solve. However, solving it would provide
the exact additive term required to obtain the exact solution:
\begin{equation}
  \Psi = \Psi^{(\ell+1)} + \epsilon^{(\ell+1)}
\end{equation}
Since the diffusion error modes are not efficiently attenuated by the above SI
process, it is natural to seek a low-order error equation. Taking the $0^{th}$
and $1^{st}$ angular moment moment of \cref{ang_err_eq}, one obtains a
diffusion equation for the scalar $\varepsilon$:
\begin{equation}
  \bs{A}\epsilon^{(\ell+1)} = \Sigma \(\Phi^{(\ell+1)}-\Phi^{(\ell)}\)
\end{equation}
where $\bs{A}$ is the diffusion operator. However, the scalar correction
$\epsilon^{(\ell+1)}$, when added to the previous iterate of the scalar flux
$\Phi^{(\ell+1)}$, will not yield the exact scalar flux solution because the
low-order error equation is not strictly identical to the transport error
equation. However, it is expected that significant speedup can be achieved in
the iterative solution technique that can now be described as follows:
\begin{enumerate}
  \item Perform a transport sweep and obtain the scalar flux after that sweep:
    \begin{equation}
      \Phi^{(\ell+1/2)} = \bs{DL}^{-1}\(\bs{M\Sigma}\Phi^{(\ell)}+S\)
    \end{equation}
  \item Solve for the diffusion error equation corrective addition:
    \begin{equation}
      \bs{A}\epsilon^{(\ell+1/2)}=\bs{\Sigma}\(\Phi^{(\ell+1/2)}-\Phi^{(\ell)}\)
    \end{equation}
  \item Obtain a new estimate of the scalar flux for the next transport sweep:
    \begin{equation}
      \Phi^{(\ell+1)} = \Phi^{(\ell+1/2)}+\epsilon^{(\ell+1/2)}
    \end{equation}
\end{enumerate}
When the process is recast in a Krylov (GMRES) solver, one obtains the
following preconditioned GMRES solve:
\begin{equation}
  \(\bs{I}+\bs{A}^{-1}\bs{\Sigma}\)\(\bs{I}-\bs{DL}^{-1}\bs{M}\bs{\Sigma}\)
  \Phi = \(\bs{I}+\bs{A}^{-1}\bs{\Sigma}\)\bs{DL}^{-1}S.
\end{equation}
As seen in \cite{consistent_p1}, DSA requires some spatial consistency to
converge. Moreover, we also ignored the effect of anisotropic scattering. The
discussion of these aspects is left for Chapter \ref{anmg_chapter} where DSA's
ineffectiveness in such situations is discussed.
