\chapter*{ABSTRACT}
\addcontentsline{toc}{part}{ABSTRACT}
\pagestyle{plain}
\pagenumbering{roman}
\setcounter{page}{2}
\indent In this dissertation, advanced numerical methods for highly forward
peaked scattering deterministic calculations are devised, implemented, and
assessed. Since electrons interact with the surrounding environment through
Coulomb interactions, the scattering kernel is highly forward-peaked. 
This bears the consequence that, with standard preconditioning, the standard Legendre
expansion of the scattering kernel requires too many terms for the discretized
equation to be solved efficiently using a deterministic method. The Diffusion
Synthetic Acceleration (DSA), usually used to speed up the calculation when
the scattering is weakly anisotropic, is inefficient for electron transport.
This led Morel and Manteuffel to develop an one-dimensional angular multigrid
(ANMG) which has proved to be very effective when the scattering is highly
anisotropic. Later, Pautz et al. generalized this scheme to multidimensional
geometries, but this method had to be stabilized by a diffusive filter that
degrades the overall convergence of the iterative scheme. In this dissertation, we
recast the multidimensional angular multigrid method without the filter as a
preconditioner for a Krylov solver. This new method is stable independently of
the anisotropy of the scattering and is increasingly more effective and
efficient as the anisotropy increases compared to DSA preconditioning wrapped
inside a Krylov solver. At the coarsest level of ANMG, a DSA step is needed. In
this research, we use the Modified Interior Penalty (MIP) DSA. This DSA was
shown to be always stable on triangular cells with isotropic scattering. 
Because this DSA discretization leads to symmetric definite-positive matrices, 
it is usually solved using a conjugate gradient preconditioned (CG) by SSOR but
here, we show that algebraic multigrid methods are vastly superior than more common
CG preconditioners such as SSOR.

Another important part of this dissertation is dedicated to transport 
equation and diffusion solves on arbitrary polygonal meshes. The advantages of
polygonal cells are that the number of unknowns needed to mesh a domain can be
decreased and that adaptive mesh refinement implementation is simplified:
rather than handling hanging nodes, the adapted computational mesh includes 
different types of polygons. Numerical examples are presented for arbitrary
quadrilateral and polygonal grids.
\pagebreak{}
