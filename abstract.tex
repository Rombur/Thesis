\chapter*{ABSTRACT}
\addcontentsline{toc}{part}{ABSTRACT}
\pagestyle{plain}
\pagenumbering{roman}
\setcounter{page}{2}
\indent In this thesis, we focus on methods to speed up photon-electron transport
calculations. Since electrons interact with the surrounding environment 
through Coulomb interaction, the scattering kernel is very forward peaked. 
This has two consequences: first, the standard Legendre expansion of the 
scattering kernel requires too many terms for the equation to be solved 
efficiently. In this work, we introduce the Boltzmann-Fokker-Planck 
equation which uses a Boltzmann operator for large angle scattering 
collisions and a Fokker-Planck operator for the forward peaked scattering 
collisions. We show that the Fokker-Planck operator is an asymptotic 
limit of the Boltzmann operator when the scattering is forward peaked. 
Limitations of the Fokker-Planck operator are discussed and Fokker-Planck 
cross sections are introduced. Then, the importance of the Galerkin 
quadratures is explained. The second consequence of the forward peaked 
scattering is that Diffusion Synthetic Acceleration (DSA) does not speed 
up the calculation like it does when the scattering is weakly anisotropic. 
In the past, Morel and Manteuffel developed an one dimensional angular 
multigrid which proved to be very effective when the scattering is 
highly anisotropic. Later, Pautz et al. generalized this scheme to 
multidimensional geometries but this method had to be stabilized by 
a diffusive filter that degrades the spectral radius: when the anisotropy 
increases the spectral radius goes to unity. In this work, we recast the 
multidimensional angular multigrid method without the filter (ANMG) as 
a preconditioner for a Krylov solver. This new method is always stable 
and effective. It is increasingly more effective and efficient as the 
anisotropy increases compared to DSA. We also adapt the Modified Interior 
Penalty (MIP) DSA developed for triangles to arbitrary polygonal meshes 
using the PieceWise Linear Discontinuous finite elements. The advantages 
of polygonal cells are that the number of unknowns to mesh a domain can 
be decreased and adaptive mesh refinement is simplified. This DSA is used 
at the coarsest level of ANMG. To further speed up the calculation, we 
compared different preconditioners for MIP. We show that algebraic multigrid 
methods are much better than more common preconditioners like symmetric
Gauss-Seidel.
\pagebreak{}
