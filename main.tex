\documentclass{report}
\usepackage{amsmath}
\usepackage{array}
\usepackage{color}
\usepackage{graphicx}
\usepackage{float} %utiliser H pour forcer a mettre l'image ou on veut
\usepackage{lscape} %utilisation du mode paysage
\usepackage{mathbbol} % permet d'avoir le vrai symbol pour les reels grace a mathbb
\usepackage{enumerate}
\usepackage{marvosym}
\usepackage{moreverb} % permet d'utiliser verbatimtab : conservation la tabulation
\usepackage{url}


\setlength {\textwidth}{16cm}
\setlength {\textheight}{21cm}
\setlength {\oddsidemargin}{0cm}
\setlength{\headsep}{5pt} 

\newcommand\bn{\boldsymbol{\nabla}}
\newcommand\bo{\boldsymbol{\Omega}}
\newcommand\br{\mathbf{r}}
\newcommand\la{\left\langle}
\newcommand\ra{\right\rangle}
\newcommand\bs{\boldsymbol}
\newcommand\red{\textcolor{red}}
\newcommand\mc{\mathcal}

\renewcommand{\(}{\left(}
\renewcommand{\)}{\right)}
\renewcommand{\[}{\left[}
\renewcommand{\]}{\right]}


\begin{document}
\title{Acceleration techniques for coupled electron-photon transport}
\author{Bruno Turcksin} 
\date{}
\maketitle

\tableofcontents

% Boltzmann, Boltzmann-CSD and BFP equations
\section{Transport equation}
Inelastic interactions, collisional and radiative scattering, can be divided
into two classes : ``catastrophic" interactions that result in large-energy
losses and ``soft" interactions that result in small-energy losses. p25 CEPXS


% bibliography
\bibliographystyle{unsrt}
\bibliography{biblio}
% include all the references
%\nocite{*}

\end{document}

